\documentclass[oneside, reqno, 12pt]{amsart}
  \usepackage{amssymb}
  \usepackage{amsthm}
  \usepackage{amsmath}
  \usepackage{amsxtra}
  \usepackage{mathrsfs}
  \usepackage{comment}
  \usepackage{graphicx}
  %\usepackage[all]{xy}
  \usepackage{placeins}
  \usepackage{multicol}
  \usepackage{bbm}
  \usepackage{stmaryrd}
  \usepackage{enumerate}
  \usepackage{mathdots}
  \usepackage{cancel}
  \usepackage{empheq}
  \usepackage{hyperref}
  \usepackage[most]{tcolorbox}
  \usepackage{epsdice}
  \usepackage{booktabs}

  \usepackage{calc}
  \usepackage{tikz}
  \usetikzlibrary{arrows,calc,automata,shadows,backgrounds,positioning,intersections,fadings,decorations,shapes,matrix}
  \usepackage{tikz-cd}

  %\usepackage{fullpage}

  \setlength{\hfuzz}{4pt}

  \newtheorem{thm}{Theorem}[subsection]
  \newtheorem{lem}[thm]{Lemma}
  \newtheorem{cor}[thm]{Corollary}
  \newtheorem{conj}[thm]{Conjecture}
  \newtheorem{prop}[thm]{Proposition}
  \newtheorem{alg}[thm]{Algorithm}
  \theoremstyle{definition}
  \newtheorem{defn} [thm] {Definition}
  \newtheorem{claim}[thm]{Claim}
  \newtheorem{example} [thm] {Example}
  \newtheorem{rem} [thm] {Remark}

  \newenvironment{problab}[1]
  {\noindent\textbf{Problem #1}.}
  {\vskip 6pt}

  \newenvironment{exolab}[1]
  {\noindent\textbf{Exercise #1}.}
  {\vskip 6pt}

  \theoremstyle{remark}
  \newtheorem*{soln}{Solution}

  \newenvironment{enumalph}
  {\begin{enumerate}\renewcommand{\labelenumi}{\textnormal{(\alph{enumi})}}}
  {\end{enumerate}}

  \newenvironment{enumroman}
  {\begin{enumerate}\renewcommand{\labelenumi}{\textnormal{(\roman{enumi})}}}
  {\end{enumerate}}

  \newcommand{\C}{\mathbb C}
  \newcommand{\D}{\mathbb D}
  \newcommand{\F}{\mathbb F}
  \newcommand{\Q}{\mathbb Q}
  \newcommand{\R}{\mathbb R}
  \newcommand{\Z}{\mathbb Z}
  \renewcommand{\S}{\mathbb S}
  \newcommand{\T}{\mathbb T}
  \renewcommand{\P}{\mathcal P}
  \renewcommand{\O}{\mathcal O}
  \newcommand{\M}{\rm M}
  \newcommand{\A}{\mathbb A}
  \newcommand{\V}{\mathbb V}
  \newcommand{\I}{\mathbb I}
  \newcommand{\m}{\mathfrak m}
  \newcommand{\wt}{\widetilde}

  \newcommand{\PP}{\mathbb P}


  \newcommand{\gl}{\mathfrak gl}
  \renewcommand{\sl}{\mathfrak sl}
  \newcommand{\s}{\mathfrak s}

  \newcommand{\Cov}{\text{Cov}}
  \newcommand{\charac}{\text{char}}

  \DeclareMathOperator{\Div}{Div}
  \DeclareMathOperator{\ddiv}{div}
  \DeclareMathOperator{\Pic}{Pic}
  \DeclareMathOperator{\res}{res}
  \DeclareMathOperator{\lcm}{lcm}
  \DeclareMathOperator{\img}{img}
  \DeclareMathOperator{\ann}{ann}
  \DeclareMathOperator{\Tor}{Tor}
  \DeclareMathOperator{\tr}{tr}
  \DeclareMathOperator{\Hom}{Hom}
  \DeclareMathOperator{\End}{End}
  \DeclareMathOperator{\Aut}{Aut}
  \DeclareMathOperator{\Gal}{Gal}
  \DeclareMathOperator{\Spl}{Spl}
  \DeclareMathOperator{\sgn}{sgn}
  \DeclareMathOperator{\ord}{ord}
  \DeclareMathOperator{\ad}{ad}
  \DeclareMathOperator{\GL}{GL}
  \DeclareMathOperator{\SL}{SL}
  \DeclareMathOperator{\Sp}{Sp}
  \DeclareMathOperator{\Ob}{Ob}
  \DeclareMathOperator{\nul}{Nul}
  \DeclareMathOperator{\col}{Col}
  \DeclareMathOperator{\row}{Row}

  \DeclareMathOperator{\id}{id}
  \DeclareMathOperator{\Frac}{Frac}

  \newcommand{\Cs}{\mathscr C}
  \newcommand{\Ds}{\mathscr D}
  \newcommand{\Fs}{\mathscr F}
  \newcommand{\Gs}{\mathscr G}
  \newcommand{\Hs}{\mathscr H}
  \renewcommand{\I}{\mathbb I}
  \newcommand{\mm}[1]{{\color{blue} \sf MM: [#1]}}
  \newcommand{\todo}[1]{{\color{red} \sf TODO: [#1]}}
  \newcommand{\Belyi}{Bely\u{\i} }
  \renewcommand{\thesection}{\Roman{section}}

  \tikzset{commutative diagrams/.cd, arrow style = tikz, diagrams = {>=latex}}
  \tikzset{>=latex}

  \everymath{\displaystyle}
  \newcommand{\defi}[1]{\textbf{#1}}
  \setcounter{secnumdepth}{3}
\title{Solvable Notes}
\author{Michael Musty}
\begin{document}
  \newtcolorbox{details}{breakable, title=Details, colframe=red}
  \newtcolorbox{notes}{breakable, title=Notes, colframe=gray}
\maketitle
\tableofcontents
\section{\Belyi Maps}{
  \todo{transitive permutation triples}
  \todo{function fields}
  \todo{the solvable story and the analogy with number fields}
  \todo{iterated quadratic extensions of function fields}
  \todo{naming convention so we can refer to specific examples}
}
\section{Galois Groups and Coverings}{
  \todo{describe how the iterative procedure leads us to a group theory question}
  \todo{covering spaces via Rotman}
}
\section{An Algorithm to Produce Permutation Triples}{
  \subsection{Intro}{
    Let $\sigma$ be a transitive permutation triple \todo{link to previous section where this is defined}.
    Recall that this means
    \begin{align*}
      \sigma = \left(\sigma_0, \sigma_1, \sigma_\infty\right)\in S_d^3,
      \quad
      \sigma_\infty\sigma_1\sigma_0 = 1,
    \end{align*}
    and $\sigma$ generates a transitive subgroup of $S_d$.
    As is described in \todo{link},
    $\sigma$ corresponds to a \Belyi map $\phi: X\to\PP^1$ of degree $d$.
    Given $\sigma$,
    we would like to describe all (Galois) \Belyi maps
    $(\widetilde{X},\widetilde{\phi})$ that are degree $m$ covers
    of the given \Belyi map $(X,\phi)$.
    We will only be concerned with the particular case $m=2$,
    but for now we remain in the more general setting.
    Put another way, $(\widetilde{X},\widetilde{\phi})$ makes the diagrams
    \begin{center}
      \begin{tikzcd}
        \widetilde{X}\arrow{dd}[swap]{\widetilde{\phi}}\arrow{dr}{\psi}\\
        &X\arrow{dl}{\phi}\\
        \PP^1
      \end{tikzcd}
      \quad
      \quad
      \quad
      \begin{tikzcd}
        K(\widetilde{X})\arrow[-]{dd}[swap]{}\arrow[-]{dr}{m}\\
        &K(X)\arrow[-]{dl}{d}\\
        K(\PP^1)
      \end{tikzcd}
    \end{center}
    commute.
    According to Galois theory we have the following
    (upside down)
    diagram.
    \begin{center}
      \begin{tikzcd}
        \Gal(K(\widetilde{X})/K(\widetilde{X}))\arrow[-]{dd}[swap]{}\arrow[-]{dr}{m}\\
        &\Gal(K(\widetilde{X})/K(X))\arrow[-]{dl}{d}\\
        \Gal(K(\widetilde{X})/K(\PP^1))
      \end{tikzcd}
    \end{center}
    Let $G,\widetilde{G}, H$ denote the Galois groups
    \begin{align*}
      G &= \Gal(K(X)/K(\PP^1))\\
      \widetilde{G} &= \Gal(K(\widetilde{X})/K(\PP^1))\\
      H &= \Gal(K(\widetilde{X})/K(X)).
    \end{align*}
    The above situation can be organized into an exact sequence of groups
    \begin{center}
      \begin{tikzcd}
        1\arrow{r}&H\arrow{r}{\iota}&\widetilde{G}\arrow{r}{f}&G\arrow{r}&1.
      \end{tikzcd}
    \end{center}
    Now recall from section \todo{link}
    that the groups $\widetilde{G}$ and $G$ are
    the monodromy groups for $\widetilde{X}$ and $X$ respectively.
    Thus,
    there is an action of $\widetilde{G}$
    on the $md$ sheets of the covering $\widetilde{X}\to\PP^1$
    and there is an action of $G$ on the $d$ sheets of the covering $X\to\PP^1$.
    Identifying $H$ with its image $\iota(H)$ in $\widetilde{G}$,
    we also have an action of $H$ on the $md$ sheets of the covering $\widetilde{X}\to\PP^1$.
    To summarize,
    we can view $H\trianglelefteq\widetilde{G}\leq S_{md}$ and $G\leq S_d$.
    The groups $\widetilde{G}$ and $G$ are transitive since
    we only consider connected covers.
    \par
    Moreover, the action of $\widetilde{G}$ on the $md$ sheets is required to
    factor through the diagram
    \begin{center}
      \begin{tikzcd}
        \widetilde{X}\arrow{dd}[swap]{\widetilde{\phi}}\arrow{dr}{\psi}\\
        &X\arrow{dl}{\phi}\\
        \PP^1
      \end{tikzcd}.
    \end{center}
    To explain,
    the map $\psi:\widetilde{X}\to X$ is a $m$-sheeted cover
    and identifies the $md$ sheets of $\widetilde{X}$ into $d$ blocks.
    From the exact sequence above,
    we have the diagram
    \begin{center}
      \begin{tikzcd}
        \wt{G}\arrow{rr}{f}\arrow{dr}[swap]{\pi}&&G\\
        &G/H\arrow{ur}[swap]{\hat{f}}
      \end{tikzcd}
    \end{center}
    with $\hat{f}$ an isomorphism.
    Let $\wt{g}\in f^{-1}(g)$ for some $g\in G$.
    According to the above diagrams,
    $\wt{G}$ must have a well-defined
    action on the $d$ blocks
    realizing the element $\wt{g}$
    as a permutation in $S_d$.
    \mm{more explanation?}
    Moreover,
    the action of $\wt{g}$ on blocks
    must agree with the action of $g$
    on $\{1,2,\dots,d\}$.
    That is,
    we must have
    \begin{align*}
      \hat{f}(\pi(\wt{g})) = f(\wt{g}) = g.
    \end{align*}
    \mm{
      do we need to prove anything here?
      $G$-equivariant bijection?
    }
  }
  \subsection{Degree $4$}{
    \begin{example}
      \todo{$m=2$, $d=4$, the Galois story}
      Up to isomorphism,
      there is only one degree $2$
      \Belyi map $\phi:X\to\PP^1$ corresponding to the
      permutation triple
      \begin{align*}
        \sigma = \left( (1\,2),(1\,2),(1)(2) \right).
      \end{align*}
      From $\sigma$ (and the above discussion)
      we would like to find all $\wt{\sigma}$
      corresponding to degree $m=2$ covers of $X$
      that are Galois \Belyi maps.
      \begin{notes}
        that is,
        $\wt{X}\to\PP^1$
        is degree 4 with monodromy
        $\wt{G}$ having order 4.
      \end{notes}
      That is,
      $\wt{\sigma}$ correspond to \Belyi maps $(\wt{X},\wt{\phi})$
      and monodromy groups $\wt{G}$ making the diagrams
      \begin{center}
        \begin{tikzcd}
          \widetilde{X}\arrow{dd}[swap]{\widetilde{\phi}}\arrow{dr}{\psi}\\
          &X\arrow{dl}{\phi}\\
          \PP^1
        \end{tikzcd}
        \quad
        \quad
        \quad
        \begin{tikzcd}
          1\arrow[-]{dd}[swap]{4}\arrow[-]{dr}{2}\\
          &H\arrow[-]{dl}{2}\\
          \wt{G}
        \end{tikzcd}
      \end{center}
      commute.
      Since $m=2$,
      $H$ is an order 2 subgroup of $S_4$
      that identifies 4 sheets to 2 sheets.
      Thus $H = \langle\tau\rangle$ with
      $\tau = (1\,2)(3\,4)$, $\tau = (1\,3)(2\,4)$, or $\tau = (1\,4)(2\,3)$.
      We choose $\tau = (1\,3)(2\,4)$ so that the identification
      of sheets corresponds to reducing the sheets modulo $2$
      (labeling them $1,2,3,4$).
      This identifies the $4$ sheets into $2$ blocks
      $\left\{\boxed{1\,3},\boxed{2\,4}\right\}$.
      \par
      Moreover, by assuming $\wt{X}$ Galois,
      we get
      $H\trianglelefteq\wt{G}$.
      Also, $G = \langle \sigma \rangle = S_2\cong\Z/2\Z$.
      Thus, we are looking for all permutation triples
      $\wt{\sigma}\in S_4^3$ that generate a (transitive) order $4$ monodromy group
      $\wt{G}\leq S_4$ sitting in the exact sequence
      \begin{center}
        \begin{tikzcd}
          1\arrow{r}&\langle\tau\rangle\arrow{r}{\iota}&\wt{G}\arrow[swap]{d}{\pi}\arrow{r}{f}&\langle\sigma\rangle\arrow{r}&1\\
          &&\wt{G}/\langle\tau\rangle\arrow[swap]{ur}{\hat{f}}
        \end{tikzcd}.
      \end{center}
      \par
      Now we have the combinatorial task of pulling back
      $\sigma$
      under the map $f$.
      \begin{notes}
        One thing that makes me nervous is
        that we do not know the group $\wt{G}$ a priori.
        If we have $\wt{G}$ in hand
        (from a transitive group database or by some other means)
        with $\langle\tau\rangle\trianglelefteq\wt{G}$,
        then pulling back $\sigma$ under $f$
        is nothing more than finding all $\wt{\sigma}$
        that map to $\sigma$ in the quotient
        (once this quotient is identified with
        $\langle\sigma\rangle$).
        \mm{
          When we don't have $\wt{G}$,
          how do we prove that
          identifying sheets via $\tau$
          yields all possible $\wt{\sigma}$?
        }
      \end{notes}
      The first requirement for any element in a preimage of $f$
      is that it has a well-defined action on the set of blocks
      $\left\{\boxed{1\,3},\boxed{2\,4}\right\}$
      obtained from $\tau$.
      For instance,
      $(1\,2\,3\,4)$
      does not have a well-defined
      action on these blocks.
      The only other requirement
      is that the action on blocks agrees
      with the permutation we are pulling back.
      \mm{more explanation?}
      \par
      In our example, $\sigma_\infty = (1)(2)$.
      Then $f^{-1}(\sigma_\infty)$
      is precisely the $\wt{\sigma_\infty}\in S_4$
      with a well-defined action on
      $B = \left\{\boxed{1\,3},\boxed{2\,4}\right\}$
      such that the action of $\wt{\sigma_\infty}$
      on $B$ agrees with the action
      of $\sigma_\infty$ on $\{1,2\}$.
      Thus
      \begin{align*}
        f^{-1}(\sigma_\infty) = \{(1)(3)(2)(4),(1\,3)(2\,4)\}
        = \{(1)(2)(3)(4),(1\,3)(2\,4)\}.
      \end{align*}
      For each $\wt{\sigma_\infty}\in f^{-1}(\sigma_\infty)$
      we can indeed check that the action
      of $\wt{\sigma_\infty}$
      on blocks is given by
      $\left(\boxed{1\,3}\right)\left(\boxed{2\,4}\right)$
      (i.e. $\wt{\sigma_\infty}$ fixes each block).
      Similarly,
      for $\sigma_0 = \sigma_1 = (1\,2)$ we have that
      \begin{align*}
        f^{-1}(\sigma_0) = f^{-1}(\sigma_1) &=
        \{(1\,2\,3\,4), (1\,2)(3\,4), (1\,4\,2\,3), (1\,4)(2\,3)\}.
      \end{align*}
      That is,
      each $\wt{\sigma_i}\in f^{-1}(\sigma_i)$ for $i\in\{0,1\}$
      acts on blocks by
      $\left(\boxed{1\,3}\,\boxed{2\,4}\right)$.
      \par
      Thus we have the following possibilities for the
      preimages of
      $\sigma_0=\sigma_1=(1\,2)$ and $\sigma_\infty = (1)(2)$
      under $f$:
      \begin{align*}
        \begin{matrix}
          (1\,2\,3\,4), (1\,2)(3\,4),\\
          (1\,4\,2\,3), (1\,4)(2\,3)
        \end{matrix}
        &\mapsto
        (1\,2)\\
        \begin{matrix}
          (1\,3)(2\,4), (1)(2)(3)(4)
        \end{matrix}
        &\mapsto
        (1)(2)
      \end{align*}
      and we can sort through these possibilities to
      see which ones correspond to degree $4$ (Galois) \Belyi maps
      according to \todo{link}.
      If $\wt{\sigma_\infty} = (1)(2)(3)(4)$,
      then the condition that
      \begin{align*}
        \wt{\sigma_\infty}\wt{\sigma_1}\wt{\sigma_0} = (1)(2)(3)(4)
      \end{align*}
      allows for only
      the single possibility
      $\wt{\sigma_0} = (1\,2\,3\,4)$
      and $\wt{\sigma_1} = (1\,4\,2\,3)$
      (up to simultaneous conjugation).
      \mm{
        should we prove that simultaneous conjugation
        does not mess up our convention of identifying
        sheets mod 2?
      }
      In this case $\wt{G}\cong\Z/4\Z$.
      If $\wt{\sigma_\infty} = \tau$,
      then there are two possibilities up to simultaneous conjugation.
      One possibility is
      \begin{align*}
        \wt{\sigma} = \left( (1\,2\,3\,4), (1\,2\,3\,4), (1\,3)(2\,4) \right)
      \end{align*}
      with $\wt{G}\cong\Z/4\Z$.
      The other possibility is
      \begin{align*}
        \wt{\sigma} = \left( (1\,2)(3\,4), (1\,4)(3\,2), (1\,3)(2\,4) \right)
      \end{align*}
      with $\wt{G}\cong\Z/2\Z\times\Z/2\Z$.
      \par
      In summary,
      there are $3$ permutation triples $\wt{\sigma}$
      (up to isomorphism) corresponding to degree $4$
      Galois \Belyi maps that factor through the degree $2$
      \Belyi map corresponding to $\sigma$.
      They are as follows:
      \todo{make a table?}
      \begin{align*}
        \Big( (1\,2\,3\,4), (1\,4\,2\,3), (1)(2)(3)(4) \Big),
        \Big( (1\,2\,3\,4), (1\,2\,3\,4), (1\,3)(2\,4) \Big),
        \Big( (1\,2)(3\,4), (1\,4)(3\,2), (1\,3)(2\,4) \Big).
      \end{align*}
    \end{example}
  }
  \subsection{Degree $8$ Galois}{
    \begin{example}
      \todo{Galois degree 2 covers of 4T1-441}
      Let
      \begin{align*}
        \sigma = (\sigma_0,\sigma_1,\sigma_\infty)
        = \Big( (1\,2\,3\,4), (1\,4\,2\,3), (1)(2)(3)(4) \Big).
      \end{align*}
      We organize $f^{-1}(\sigma)$ in the following table.
      \begin{center}
        \begin{tabular}{ccc}
          \toprule
          $f^{-1}\Big( (1\,2\,3\,4)\Big)$
          & $f^{-1}\Big( (1\,4\,2\,3) \Big)$
          & $f^{-1}\Big( (1)(2)(3)(4)\Big)$\\
          \midrule
          $(1\,2\,3\,4)(5\,6\,7\,8)$
          & $(1\,8\,3\,2)(4\,7\,6\,5)$
          & $(1)(2)(3)(4)(5)(6)(7)(8)$ \\
          $(1\,6\,7\,4)(2\,3\,8\,5)$
          & $(1\,4\,3\,6)(2\,5\,8\,7)$
          & $(1\,5)(2\,6)(3\,7)(4\,8)$\\
          $(1\,6\,3\,4)(2\,7\,8\,5)$
          & $(1\,4\,7\,2)(3\,6\,5\,8)$ &\\
          $(1\,6\,7\,8)(2\,3\,4\,5)$
          & $(1\,4\,7\,6)(2\,5\,8\,3)$ &\\
          $(1\,2\,7\,4)(3\,8\,5\,6)$
          & $(1\,8\,3\,6)(2\,5\,4\,7)$ &\\
          $(1\,2\,3\,8)(4\,5\,6\,7)$
          & $(1\,4\,3\,2)(5\,8\,7\,6)$ &\\
          $(1\,2\,7\,8)(3\,4\,5\,6)$
          & $(1\,8\,7\,2)(3\,6\,5\,4)$ &\\
          $(1\,6\,3\,8)(2\,7\,4\,5)$
          & $(1\,8\,7\,6)(2\,5\,4\,3)$ &\\
          $(1\,6\,7\,8\,5\,2\,3\,4)$
          & $(1\,8\,7\,2\,5\,4\,3\,6)$ &\\
          $(1\,2\,3\,8\,5\,6\,7\,4)$
          & $(1\,4\,7\,6\,5\,8\,3\,2)$ &\\
          $(1\,2\,7\,8\,5\,6\,3\,4)$
          & $(1\,8\,3\,6\,5\,4\,7\,2)$ &\\
          $(1\,6\,3\,8\,5\,2\,7\,4)$
          & $(1\,8\,3\,2\,5\,4\,7\,6)$ &\\
          $(1\,2\,3\,4\,5\,6\,7\,8)$
          & $(1\,4\,7\,2\,5\,8\,3\,6)$ &\\
          $(1\,6\,7\,4\,5\,2\,3\,8)$
          & $(1\,8\,7\,6\,5\,4\,3\,2)$ &\\
          $(1\,6\,3\,4\,5\,2\,7\,8)$
          & $(1\,4\,3\,6\,5\,8\,7\,2)$ &\\
          $(1\,2\,7\,4\,5\,6\,3\,8)$
          & $(1\,4\,3\,2\,5\,8\,7\,6)$ &\\
          \bottomrule
        \end{tabular}
      \end{center}
      These possibilities combine to yield $512=16^2\cdot 2$
      possible $\wt{\sigma}$.
      Of these $512$ possibilities,
      only $32$ have $\wt{\sigma_\infty}\wt{\sigma_1}\wt{\sigma_0} = \id_{S_8}$.
      Of these $32$,
      only $24$ generate a transitive order $8$
      subgroup of $S_8$.
      \mm{include some of these lists?}
      Finally,
      if we identify triples that are simultaneously conjugate
      or in the same orbit under the $S_3$ action on triples,
      we get the following list of $3$ possibilities for $\wt{\sigma}$
      organized by $\wt{G}$:
      \begin{center}
        \begin{tabular}{c}
          \toprule
          $\wt{G}\cong\Z/8\Z$ \\
          \midrule
          $\Big( (1\,6\,7\,8\,5\,2\,3\,4), (1\,4\,3\,2\,5\,8\,7\,6), \id_{S_8}\Big)$\\
          $\Big( (1\,6\,7\,8\,5\,2\,3\,4), (1\,8\,3\,6\,5\,4\,7\,2), (1\,5)(2\,6)(3\,7)(4\,8)\Big)$\\
          \toprule
          $\wt{G}\cong\Z/4\Z\times\Z/2\Z$\\
          \midrule
          $\Big( (1\,2\,3\,4)(5\,6\,7\,8), (1\,8\,3\,6)(2\,5\,4\,7), (1\,5)(2\,6)(3\,7)(4\,8)\Big)$\\
          \bottomrule
        \end{tabular}.
      \end{center}
    \end{example}
    \begin{example}
      \todo{Galois degree 2 covers of 4T1-442}
      Let
      \begin{align*}
        \sigma = (\sigma_0,\sigma_1,\sigma_\infty)
        = \Big( (1\,2\,3\,4), (1\,2\,3\,4), (1\,3)(2\,4) \Big).
      \end{align*}
      We organize $f^{-1}(\sigma)$ in the following table.
      \begin{center}
        \begin{tabular}{ccc}
          \toprule
          $f^{-1}(\sigma_0)$ & $f^{-1}(\sigma_1)$ & $f^{-1}(\sigma_\infty)$\\
          \midrule
          $(1, 2, 3, 4)(5, 6, 7, 8)$ & $(1, 2, 3, 4)(5, 6, 7, 8)$ & $(1, 3)(2, 4)(5, 7)(6, 8)$\\
          $(1, 6, 7, 4)(2, 3, 8, 5)$ & $(1, 6, 7, 4)(2, 3, 8, 5)$ & $(1, 3)(2, 8)(4, 6)(5, 7)$\\
          $(1, 6, 3, 4)(2, 7, 8, 5)$ & $(1, 6, 3, 4)(2, 7, 8, 5)$ & $(1, 7)(2, 4)(3, 5)(6, 8)$\\
          $(1, 6, 7, 8)(2, 3, 4, 5)$ & $(1, 6, 7, 8)(2, 3, 4, 5)$ & $(1, 7)(2, 8)(3, 5)(4, 6)$\\
          $(1, 2, 7, 4)(3, 8, 5, 6)$ & $(1, 2, 7, 4)(3, 8, 5, 6)$ & $(1, 3, 5, 7)(2, 4, 6, 8)$\\
          $(1, 2, 3, 8)(4, 5, 6, 7)$ & $(1, 2, 3, 8)(4, 5, 6, 7)$ & $(1, 3, 5, 7)(2, 8, 6, 4)$\\
          $(1, 2, 7, 8)(3, 4, 5, 6)$ & $(1, 2, 7, 8)(3, 4, 5, 6)$ & $(1, 7, 5, 3)(2, 4, 6, 8)$\\
          $(1, 6, 3, 8)(2, 7, 4, 5)$ & $(1, 6, 3, 8)(2, 7, 4, 5)$ & $(1, 7, 5, 3)(2, 8, 6, 4)$\\
          $(1, 6, 7, 8, 5, 2, 3, 4)$ & $(1, 6, 7, 8, 5, 2, 3, 4)$ &\\
          $(1, 2, 3, 8, 5, 6, 7, 4)$ & $(1, 2, 3, 8, 5, 6, 7, 4)$ &\\
          $(1, 2, 7, 8, 5, 6, 3, 4)$ & $(1, 2, 7, 8, 5, 6, 3, 4)$ &\\
          $(1, 6, 3, 8, 5, 2, 7, 4)$ & $(1, 6, 3, 8, 5, 2, 7, 4)$ &\\
          $(1, 2, 3, 4, 5, 6, 7, 8)$ & $(1, 2, 3, 4, 5, 6, 7, 8)$ &\\
          $(1, 6, 7, 4, 5, 2, 3, 8)$ & $(1, 6, 7, 4, 5, 2, 3, 8)$ &\\
          $(1, 6, 3, 4, 5, 2, 7, 8)$ & $(1, 6, 3, 4, 5, 2, 7, 8)$ &\\
          $(1, 2, 7, 4, 5, 6, 3, 8)$ & $(1, 2, 7, 4, 5, 6, 3, 8)$ &\\
          \bottomrule
        \end{tabular}
      \end{center}
      These possibilities combine to yield $2048$ possible $\wt{\sigma}$.
      Of these, $128$ have
      $\wt{\sigma_\infty}\wt{\sigma_1}\wt{\sigma_0} = \id_{S_8}$.
      Of these, $120$ generate a transitive subgroup of $S_8$.
      Of these $120$ transitive groups, $24$ have order equal to $8$.
      Finally,
      after identifying triples up to simultaneous conjugation in $S_8$
      and the $S_3$ action on triples, we get the following
      list of $3$ possibilities for $\wt{\sigma}$ organized by $\wt{G}$:
      \begin{center}
        \begin{tabular}{c}
          \toprule
          $\wt{G}\cong\Z/8\Z$\\
          \midrule
          $\Big( (1\,6\,7\,8\,5\,2\,3\,4), (1\,6\,7\,8\,5\,2\,3\,4), (1\,3\,5\,7)(2\,8\,6\,4)\Big)$\\
          $\Big( (1\,6\,7\,8\,5\,2\,3\,4), (1\,2\,7\,4\,5\,6\,3\,8), (1\,7\,5\,3)(2\,4\,6\,8)\Big)$\\
          \toprule
          $\wt{G}\cong\Z/4\Z\times\Z/2\Z$\\
          \midrule
          $\Big( (1\,2\,3\,4)(5\,6\,7\,8), (1\,6\,3\,8)(2\,7\,4\,5), (1\,7)(2\,8)(3\,5)(4\,6)\Big)$\\
          \bottomrule
        \end{tabular}.
      \end{center}
      \mm{
        3 things to mention here\dots
        \begin{itemize}
          \item
            4T1-442 to 8T2-442 unramified.
          \item
            Notice that
            8T1-882 does not factor through
            4T1-442 even though it seems like it should be possible at first.
          \item
            We have identified a size 2 passport\dots
            should probably prove something about
            these always showing up and make sure
            attempts to optimize algorithm don't eliminate these\dots
        \end{itemize}
      }
    \end{example}
    \begin{example}
      \todo{Galois degree 2 covers of 4T2-222}
      Let
      \begin{align*}
        \sigma = (\sigma_0,\sigma_1,\sigma_\infty)
        = \Big( (1\,2\,3\,4), (1\,2\,3\,4), (1\,3)(2\,4) \Big).
      \end{align*}
      We organize $f^{-1}(\sigma)$ in the following table.
      \begin{center}
        \begin{tabular}{ccc}
          \toprule
          $f^{-1}(\sigma_0)$ & $f^{-1}(\sigma_1)$ & $f^{-1}(\sigma_\infty)$ \\
          \midrule
          $(1, 2)(3, 4)(5, 6)(7, 8)$ & $(1, 4)(2, 3)(5, 8)(6, 7)$ & $(1, 3)(2, 4)(5, 7)(6, 8)$ \\
          $(1, 2)(3, 8)(4, 7)(5, 6)$ & $(1, 4)(2, 7)(3, 6)(5, 8)$ & $(1, 3)(2, 8)(4, 6)(5, 7)$ \\
          $(1, 6)(2, 5)(3, 4)(7, 8)$ & $(1, 8)(2, 3)(4, 5)(6, 7)$ & $(1, 7)(2, 4)(3, 5)(6, 8)$ \\
          $(1, 6)(2, 5)(3, 8)(4, 7)$ & $(1, 8)(2, 7)(3, 6)(4, 5)$ & $(1, 7)(2, 8)(3, 5)(4, 6)$ \\
          $(1, 2, 5, 6)(3, 4, 7, 8)$ & $(1, 4, 5, 8)(2, 3, 6, 7)$ & $(1, 3, 5, 7)(2, 4, 6, 8)$ \\
          $(1, 2, 5, 6)(3, 8, 7, 4)$ & $(1, 4, 5, 8)(2, 7, 6, 3)$ & $(1, 3, 5, 7)(2, 8, 6, 4)$ \\
          $(1, 6, 5, 2)(3, 4, 7, 8)$ & $(1, 8, 5, 4)(2, 3, 6, 7)$ & $(1, 7, 5, 3)(2, 4, 6, 8)$ \\
          $(1, 6, 5, 2)(3, 8, 7, 4)$ & $(1, 8, 5, 4)(2, 7, 6, 3)$ & $(1, 7, 5, 3)(2, 8, 6, 4)$ \\
          \bottomrule
        \end{tabular}
      \end{center}
      These possibilities combine to yield $512$ possible $\wt{\sigma}$.
      Of these, $64$ have
      $\wt{\sigma_\infty}\wt{\sigma_1}\wt{\sigma_0} = \id_{S_8}$.
      Of these, $56$ generate a transitive subgroup of $S_8$
      and all $56$ have order $8$.
      Finally,
      after identifying triples up to simultaneous conjugation in
      $S_8$ and the $S_3$ action on triples, we get the following
      list of $3$ possibilities for $\wt{\sigma}$ organized by $\wt{G}$:
      \begin{center}
        \begin{tabular}{c}
          \toprule
          $\wt{G}\cong\Z/4\Z\times\Z/2\Z$\\
          \midrule
          $\Big( (1\,2)(3\,4)(5\,6)(7\,8), (1\,4\,5\,8)(2\,3\,6\,7), (1\,7\,5\,3)(2\,8\,6\,4)\Big)$\\
          \toprule
          $\wt{G}\cong D_8$\\
          \midrule
          $\Big( (1\,2)(3\,4)(5\,6)(7\,8), (1\,4)(2\,7)(3\,6)(5\,8), (1\,7\,5\,3)(2\,4\,6\,8)\Big)$\\
          \toprule
          $\wt{G}\cong Q_8$\\
          \midrule
          $\Big( (1\,2\,5\,6)(3\,4\,7\,8), (1\,4\,5\,8)(2\,3\,6\,7), (1\,3\,5\,7)(2\,8\,6\,4)\Big)$\\
          \bottomrule
        \end{tabular}
      \end{center}
    \end{example}
  }
  \subsection{The Non-Galois Story}{
    \begin{example}
      \todo{$m=2$, $d=4$, the non-Galois story}
      \mm{maybe do non-Galois in a different section\dots}
    \end{example}
    \begin{example}
      \todo{$m=3$\dots}
    \end{example}
  }
  \subsection{The Algorithm}{
    We now formally describe
    an algorithm to compute all
    Galois degree $2$ covers
    of a given \Belyi map.
    \begin{alg}[Brute Force]
    \end{alg}
    \begin{proof}
    \end{proof}
  }
  \subsection{Graph of Examples}{
    \begin{example}
      \todo{graph of examples}
    \end{example}
  }
}
\section{Orders in Function Fields}{
  \subsection{Intro}{
    Following \cite{dino},
    we will start with the general setting
    with data $(A,K,L)$
    where $A$ is a commutative domain,
    $K = \Frac(A)$,
    and $L$ a finite extension of $K$.
    Given such a triple $(A,K,L)$,
    we associate a subring $B\subset L$
    such that $L = \Frac(B)$.
    \todo{define order}
    \todo{define maximal order}
    \todo{define ring of integers}
  }
  \subsection{Example}{
  }
}
\section{Computing \Belyi\ Maps}{
  \subsection{Intro}{
    We begin by explaining a bit about the task
    of computing \Belyi\ maps in general.
    \todo{
      reference the various methods for computing maps in general
      and in specific cases too....
    }
    For a (much) more detailed account see \cite{SV}.
    \par
    \todo{Explain the special setting we are in}
  }
  \subsection{Explicit Example}{
    We now explain an explicit example.
    Let $K = \overline{\Q}$.
    Let $X_0 = \PP^1$ over $K$.
    Up to isomorphism,
    there is a unique degree $2$
    \Belyi map $\varphi:X_1\to X_0$.
    Let $K(X_0)$ be the function field of $X_0$
    which is $K(x_0)$ (the rational function field).
    To write down an equation for $X_1$,
    we must specify the ramification values
    (branch points)
    on $X_0$.
    We can do this by recording
    the ramification values in a divisor
    $D^0\in\Div(X_0)$.
    For the degree $2$ \Belyi map,
    we get to choose $2$ branch points on $X_0$,
    say $0$ and $\infty$
    so that $D^0 = 0 - \infty$.
    \par
    \mm{mention how this comes to us from previous section}
    \par
    Next,
    we find $f_0\in K(X_0)$
    such that the quadratic function field extension
    \begin{align*}
      K(X_1) := \frac{K(X_0)[x_1]}{(x_1^2-f_0)}
    \end{align*}
    corresponds to $\varphi$.
    \par
    \mm{write down explicitly the correspondence?}
    \par
    \mm{
      explain choices involved in choosing $f$...
      i.e. up to square
      and such that $\ddiv(f)$ has property...
    }
    \par
    We get that $f_0 = x_0$
    and $X_1$ is the projective closure of the
    affine curve defined by the equation
    $x_1^2-x_0$.
    On the function field side,
    we have the following diagram.
    \begin{center}
      \begin{tikzcd}
        &K(X_1) = \frac{K(x_0)[x_1]}{(x_1^2 - x_0)}\arrow[dash]{dl}\arrow[dash]{dd}\\
        R_1 = \frac{K[x_0,x_1]}{(x_1^2-x_0)}\arrow[dash]{dd}[swap]{}\\
        &K(X_0) = K(x_0)\arrow[dash]{dl}{}\\
        R_0 = K[x_0]
      \end{tikzcd}
    \end{center}
    Let $\frak{p}_0 = (x_0, x_1)$.
    We can see that we have the right ramification
    since
    \begin{align*}
      (x_0) &= \mathfrak{p}_0^2\\
      (1/x_0) &= \mathfrak{p}_0^{-2}
    \end{align*}
    and the discriminant of $R_1$ is $4x_0$.
    What are the primes above $(x_0-1)$?
    Well, $(x_0-1) = \frak{p}_1\frak{p}_2$
    where
    \begin{align*}
      \frak{p}_1 &= (x_0-1, x_1-1)\\
      \frak{p}_2 &= (x_0-1, x_1+1).
    \end{align*}
    Now...
    \par
    \mm{In summary,}
    \begin{center}
      \begin{tikzcd}
        &K(X_2) = \frac{K(X_1)[x_2]}{(x_2^2-f)}\arrow[dash]{dd}\arrow[dash]{dl}\\
        R_2\arrow[dash]{dd}\\
        &K(X_1) = \frac{K(x_0)[x_1]}{(x_1^2 - x_0)}\arrow[dash]{dl}\arrow[dash]{dd}\\
        R_1 = \frac{K[x_0,x_1]}{(x_1^2-x_0)}\arrow[dash]{dd}[swap]{}\\
        &K(X_0) = K(x_0)\arrow[dash]{dl}{}\\
        R_0 = K[x_0]
      \end{tikzcd}
    \end{center}
  }
}
\nocite{*}
\bibliographystyle{amsplain}
\bibliography{solvable}
\end{document}
