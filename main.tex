% arara: xelatex
% arara: bibtex
% arara: xelatex

\documentclass{dcthesis}

%This version of the thesis template has been updated for the February 2017 thesis guidelines provided by the School of Graduate and Advanced Studies by David Freund and Daryl DeFord.

%  Some good options are draft and singlespacing, for drafts, and *final*,
%for the final cut, and noheadings, for a printout without headings.  
%You can also use the copyright option to add a copyright.  Note:
%final will enforce a bunch of different options, like oneside, 12pt
%and doublespacing, as well as make the margins correct.   Draft, has
%larger margins and is appropriate for two sided printing.   


%%%%%%%%%%%%%%%%%%%%%%%%%%%
%%%%    IMPORTANT     %%%%%
%%%%%%%%%%%%%%%%%%%%%%%%%%%
%
%  Because dvips doesn't know/care about the page size of the dvi file
%  its working on, and because its so important that the margins of
%  your thesis are correct you have to make sure that you are using
%  the command dvi -t letter *thesisnamehere*
%
%  If you are using a terminal this is straightforward, but if you are
%  using a tex editing program it make take a little bit of searching
%  before you figure out how to make this work. 
%  
%  Alternatively, you might consider compiling using pdflatex, which
%  compiles straight to a PDF and doesn't have this problem.  

%%%%%%%%%%%%%%%%%%%%%%%%%%%%%%%%%%%%%%%%%%%%%%%%%%
% Some Defaults
%%%%%%%%%%%%%%%%%%%%%%%%%%%%%%%%%%%%%%%%%%%%%%%%%%
\committee[F. Jon Kull, Ph.D.]{}{}{}{}
\school{Dartmouth College}{Hanover, New Hampshire}
\degree{Doctor of Philosophy}
\field{Mathematics}

%%%%%%%%%%%%%%%%%%%%%%%%%%%%%%%%%%%%%%%%%%%%%%%%%%
% Additional Packages (add as desired)
%%%%%%%%%%%%%%%%%%%%%%%%%%%%%%%%%%%%%%%%%%%%%%%%%%
\usepackage{amsmath,amssymb,amsthm,amsxtra}
\usepackage{mathrsfs} 
\usepackage{lipsum}
\usepackage{xcolor}
\usepackage{tikz}
\usepackage{colonequals}
\usetikzlibrary{arrows,calc,automata,shadows,backgrounds,positioning,intersections,fadings,decorations.pathreplacing,shapes,snakes, matrix}
\usepackage{tikz-cd}
\tikzset{commutative diagrams/.cd, arrow style = tikz, diagrams = {>=latex}}
\tikzset{>=latex}

%\usepackage{index} %Uncomment if you would like to have an index. Compiling with an index takes more work than compiling without one. You will have to look up how to use the index package.

%%%%%%%%%%%%%%%%%%%%%%%%%%%%%%%%%%%%%%%%%%%%%%%%%%
%Formatting
%%%%%%%%%%%%%%%%%%%%%%%%%%%%%%%%%%%%%%%%%%%%%%%%%%
%  Change or add to as desired. 
%  These two commands make the first enumerations look like (a)
%  And the second level like (i).  
% \renewcommand{\labelenumi}{(\alph{enumi})}
% \renewcommand{\labelenumii}{(\roman{enumii})}
%  These commands make the section headings Boldface and not all
%  caps. It also removes the chapter numbers  
\renewcommand{\chaptermark}[1]{\markboth{{\sc #1}}{{\sc #1}}}
\renewcommand{\sectionmark}[1]{\markright{{\sc \thesection\ #1}}}
%  These commands have lowercase headings with chapter numbers. 
%\renewcommand{\chaptermark}[1]{markboth{#1}{}}
%\renewcommand{\sectionmark}[1]{\markright{\thesection\ #1}} 
\newcommand{\PP}{\mathbb P}
\newcommand{\CC}{\mathbb C}
\newcommand{\QQ}{\mathbb Q}
\newcommand{\ZZ}{\mathbb Z}
\renewcommand{\AA}{\mathbb A}
\newcommand{\defi}[1]{\textsf{#1}}
\newcommand{\jv}[1]{{\color{red} \sf JV: [#1]}}
\newcommand{\mm}[1]{{\color{blue} \sf MM: [#1]}}
\newcommand{\wt}[1]{\widetilde{#1}}
\newcommand{\QQal}{{\mathbb Q}^{\textup{al}}}
\newcommand{\QQab}{{\mathbb Q}^{\textup{ab}}}
\newcommand{\QQbar}{{\mathbb Q}^{\textup{al}}}
\newcommand{\kbar}{\overline{k}}
\newcommand{\Kbar}{\overline{K}}
\newcommand{\LL}{\mathscr L}

\DeclareMathOperator{\Div}{Div}
\DeclareMathOperator{\ddiv}{div}
\DeclareMathOperator{\sat}{sat}
\DeclareMathOperator{\ddeg}{deg}
\DeclareMathOperator{\ddim}{dim}
\DeclareMathOperator{\Lifts}{Lifts}
\DeclareMathOperator{\order}{order}

%%%%%%%%%%%%%%%%%%%%%%%%%%%%%%%%%%%%%%%%%%%%%%%%%%
% Theorem Declarations
%%%%%%%%%%%%%%%%%%%%%%%%%%%%%%%%%%%%%%%%%%%%%%%%%%
% A basic set of theorem declarations.  Add or remove as desired. 
% \newtheorem{prop}{Proposition}[chapter]
\newtheorem{prop}{Proposition}[section]
\newtheorem{theorem}[prop]{Theorem}
\newtheorem{lemma}[prop]{Lemma}
\newtheorem{corr}[prop]{Corollary}
\theoremstyle{definition}
\newtheorem{definition}[prop]{Definition}
\newtheorem{alg}[prop]{Algorithm}
\newtheorem{notation}[prop]{Notation}
\theoremstyle{remark}
\newtheorem{remark}[prop]{Remark}
\newtheorem{example}[prop]{Example}
\newtheorem*{claim}{Claim}
% numberwithin section for equations and figures
\numberwithin{equation}{section}
\numberwithin{figure}{section}
% \renewcommand{\thefigure}{\arabic{figure}}

%%%%%%%%%%%%%%%%%%%%%%%%%%%%%%%%%%%%%%%%%%%%%%%%%%
%Indices
%%%%%%%%%%%%%%%%%%%%%%%%%%%%%%%%%%%%%%%%%%%%%%%%%%
%This is for an index.  More work is needed for a notation index
%\makeindex

%%%%%%%%%%%%%%%%%%%%%%%%%%%%%%%%%%%%%%%%%%%%%%%%%%
% Macros
%%%%%%%%%%%%%%%%%%%%%%%%%%%%%%%%%%%%%%%%%%%%%%%%%%
% Add your math macros here

%%%%%%%%%%%%%%%%%%%%%%%%%%%%%%%%%%%%%%%%%%%%%%%%%%
% Title Page Information
%%%%%%%%%%%%%%%%%%%%%%%%%%%%%%%%%%%%%%%%%%%%%%%%%%
%  Your personal info goes here!
\title{2-Group Belyi Maps}
\author{Michael James Musty}
\date{\today}
\field{Mathematics}
\degree{Doctor of Philosophy}
%As an optional argument to \committee you can specify the dean of graduate studies.
\committee{John Voight}{Thomas Shemanske}{David Roberts}{Carl Pomerance}

\begin{document}

%%%%%%%%%%%%%%%%%%%%%%%%%%%%%%%%%%%%%%%%%%%%%%%%%%
%Front end of thesis
%%%%%%%%%%%%%%%%%%%%%%%%%%%%%%%%%%%%%%%%%%%%%%%%%%
\frontmatter

\maketitle

%%%%%%%%%%%%%%%%%%%%%%%%%%%%%%%%%%%%%%%%%%%%%%%%%%
%Abstract
%%%%%%%%%%%%%%%%%%%%%%%%%%%%%%%%%%%%%%%%%%%%%%%%%%
%NOTE:  Must be less than 350 words.  
\chapter*{Abstract}
%This is so the abstract appears in the ToC
\addcontentsline{toc}{section}{Abstract}
Write your abstract here.

\chapter*{Preface}
\addcontentsline{toc}{section}{Preface}
Preface and Acknowledgments go here!

%%%%%%%%%%%%%%%%%%%%%%%%%%%%%%%%%%%%%%%%%%%%%%%%%%
%Table of Contents
%%%%%%%%%%%%%%%%%%%%%%%%%%%%%%%%%%%%%%%%%%%%%%%%%%
\tableofcontents

%Add a list of tables
\listoftables

%Add a list of figures
\listoffigures

%%%%%%%%%%%%%%%%%%%%%%%%%%%%%%%%%%%%%%%%%%%%%%%%%%
%Main Portion of Thesis
%%%%%%%%%%%%%%%%%%%%%%%%%%%%%%%%%%%%%%%%%%%%%%%%%%
\mainmatter

%%%%%%%%%%%%%%%%%%%%%%%%%
%%  YOUR THESIS HERE!  %%
%%%%%%%%%%%%%%%%%%%%%%%%%

% \include{chapter1}

\chapter{Introduction}{\label{chapter:intro}
  \section{Belyi maps from a historical perspective}{
    In \cite{belyi}, G.V. Belyi proved that a Riemann surface
    $X$
    can be defined over a number field
    (when viewed as an algebraic curve over $\CC$)
    if and only if there exists a non-constant
    meromorphic function $\phi:X\to\PP^1_\CC$ unramified outside the set $\{0,1,\infty\}$.
    This result came to be known as Belyi's Theorem
    and the maps $\phi$ came to be known as Belyi maps (or Belyi functions).
    Although Belyi's Theorem has an elementary proof,
    it was a starting point for a great deal of modern research in the area.
    This work was largely spurred on by Grothendieck's
    \emph{Esquisse d'un programme}
    \cite{grothendieck}
    where he was impressed enough to write
    \begin{quote}
      \emph{jamais sans doute un r\'{e}sultat profond et d\'{e}routant ne fut
      d\'{e}montr\'{e} en si peu de lignes!}
      \newline
      never, without a doubt, was such a deep and disconcerting result proved in so few lines!
    \end{quote}
    An intriguing aspect of the theory of Belyi maps that arose from Grothendieck's
    work in the 1980s is the reformulation of these objects
    in a purely topological way.
    The preimage $\phi^{-1}([0,1])$ is a graph embedded on $X$,
    and Grothendieck developed axioms for embedded graphs in such a way
    that they coincided exactly with the category of Belyi maps.
    He called these graphs
    \emph{dessins d'enfants} or children's drawings.
    \par
    Even as a standalone theorem, Belyi's Theorem
    is a remarkable result in the mysterious way that it allows us to distinguish between
    algebraic and transcendental objects.
    However, the main interest in Belyi maps arises from Galois theory.
    The absolute Galois group of $\QQ$ acts on the set of Belyi maps
    via the defining equations.
    The induced action on the set of dessins
    \subsection[Inverse Galois theory]{Inverse Galois theory, Hurwitz families, and fields with few ramified primes}{
      \subsubsection{Inverse Galois theory}{
      }
      \subsubsection{Hurwitz families}{
      }
      \subsubsection{Number fields with few ramified primes}{
      }
    }
    \subsection[Dessins d'enfants]{Grothendieck's theory of dessins d'enfants}{
    }
  }
}
\chapter{Background}{\label{chapter:background}
  \section{Belyi maps}{\label{sec:belyimaps}
    \subsection{Algebraic curves and their function fields}{
    }
    \subsection{Riemann's existence theorem and covers of $\PP^1$}{
    }
    \subsection{Belyi's theorem}{
    }
    \subsection{Belyi maps and $G$-Belyi maps}{
      \label{subsec:belyimaps}
      \begin{prop}\label{prop:galoiscorrespondence}
        Galois correspondence of Belyi maps
      \end{prop}
      \begin{proof}
      \end{proof}
    }
  }
  \section{Group theory}{
    \subsection{Central group extensions and $H^2(G,A)$}{
      \begin{definition}
        \label{def:isoextentions}
      \end{definition}
    }
    \subsection{Holt's algorithm and \texttt{Magma} implementation}{
      \label{subsec:holtsalgorithm}
    }
    \subsection{Results on $2$-groups}{
    }
  }
  \section{Jacobians of curves}{
    \subsection{Abel-Jacobi and the construction over $\CC$}{
    }
    \subsection{Algebraic construction}{
    }
    \subsection{Riemann-Roch}{
    }
    \subsection{Torsion points and torsion fields}{
    }
  }
  \section{Galois representations}{
    \subsection{Representations of Galois groups of number fields}{
    }
    \subsection{Representations coming from geometry}{
    }
  }
}
\chapter{A database of $2$-group Belyi maps}{\label{chapter:database}
  In this chapter we describe an algorithm \mm{method?}
  to generate all $2$-group Belyi maps up to a given degree.
  We begin by defining this particular family of Belyi maps in Section \ref{sec:defs}.
  The algorithm is inductive in the degree.
  The base case in degree $1$ is discussed in Section \ref{sec:degree1}.
  We then move on to describe the inductive step of the algorithm
  which we describe in two parts.
  First we discuss the algorithm to enumerate the isomorphism classes
  using permutation triples in Section \ref{sec:isoclasses}.
  For a discussion on the relationship between permutation triples and Belyi maps
  see Chapter \ref{chapter:belyimaps}.
  Next we discuss the inductive step to produce Belyi curves and maps in Section \ref{sec:curvesandmaps}.
  In Section \ref{sec:runtime} we give a detailed description of the running time of the algorithm.
  Lastly,
  in Section \ref{sec:computations},
  we discuss the implementation and computations that we have carried out explicitly.
  \section{$2$-group Belyi maps}{\label{sec:defs}
    Recall the definition of a Belyi map in Section \ref{sec:belyimaps}.
    In this section we define a narrow our focus to a more specific family of Belyi maps
    which we now describe.
    \begin{definition}\label{def:galois}
      A degree $d$ Belyi map $\phi$ with monodromy group $G$
      is said to be \defi{Galois} if $\#G = d$.
    \end{definition}
    \begin{definition}\label{def:2groupbelyi}
      A \defi{$2$-group Belyi map} is a Galois Belyi map
      with monodromy group a $2$-group.
    \end{definition}
    \mm{some exposition}
  }
  \section{Degree $1$ Belyi maps}{\label{sec:degree1}
  }
  \pagebreak
  \section{An algorithm to enumerate isomorphism classes of $2$-group Belyi maps}{\label{sec:isoclasses}
    The algorithm we describe here is iterative.
    The degree $1$ case is discussed in Section \ref{sec:degree1}.
    We now set up some notation for the iteration.
    \begin{notation}\label{not:triples}
      First we suppose that we are given
      $\sigma$ a permutation triple corresponding
      to a $2$-group Belyi map $\phi:X\to\PP^1$.
      \begin{definition}
        \label{def:lift}
        We say that a permutation triple $\wt{\sigma}$
        is a \defi{degree $2$ lift} (or simply a \defi{lift})
        of a permutation triple $\sigma$ if
        there exists a short exact sequence of groups
        as in Figure \ref{fig:lifttriple}
        with $\iota(\ZZ/2\ZZ)$ contained in the center of
        $\langle\wt{\sigma}\rangle$.
      \end{definition}
      \begin{figure}[ht]\label{fig:lifttriple}
        \[
          \begin{tikzcd}[ampersand replacement=\&, row sep = small]
            1\arrow{r}\&\ZZ/2\ZZ\arrow{r}{\iota}\&\langle\widetilde{\sigma}\rangle\arrow{r}{\pi}\&\langle\sigma\rangle\arrow{r}\&1
          \end{tikzcd}
        \]
        \caption{$\wt{\sigma}$ a lift of $\sigma$}
      \end{figure}
      In Algorithm \ref{alg:triples} below
      we describe how to determine all lifts $\wt{\sigma}$ (up to isomorphism)
      of a given permutation triple $\sigma$.
      \begin{lemma}
        \label{lem:central}
        Let $\sigma$ be a permutation triple corresponding to a $2$-group Belyi map
        $\phi:X\to\PP^1$ and $\wt{\sigma}$ a lift of $\sigma$
        corresponding to a $2$-group Belyi map $\wt{\phi}:\wt{X}\to\PP^1$.
        Then there exists a permutation triple $\wt{\sigma}'$
        that is simultaneously conjugate to $\wt{\sigma}$ with
        $\iota(\langle\wt{\sigma}'\rangle)$ contained in the center of
        $\langle\sigma\rangle$.
      \end{lemma}
      \begin{proof}
      \end{proof}
      \begin{remark}
        \label{rmk:central}
        In light of Lemma \ref{lem:central},
        we can restrict our attention to central extensions of
        $\langle\sigma\rangle$
        in Definition \ref{def:lift}.
      \end{remark}
    \end{notation}
    \begin{alg}\label{alg:triples}
      Let the notation be as described above in \ref{not:triples}.
      \newline
      \textbf{Input}: $\sigma=(\sigma_0,\sigma_1,\sigma_\infty)\in S_d^3$ a permutation triple
      corresponding to a $2$-group Belyi map
      \newline
      \textbf{Output}: all lifts $\wt{\sigma}$
      of $\sigma$ up to simultaneous conjugation in $S_{2d}$
      sorted by passport
      \begin{enumerate}
        \item
          Let $G = \langle\sigma\rangle$
          and compute all central extensions $\wt{G}$
          sitting in the exact sequence in Figure \ref{fig:centralext}
          up to isomorphism (see Definition \ref{def:isoextentions}).
          For more information about the algorithms to do this
          see Section \ref{subsec:holtsalgorithm}.
          \begin{figure}[ht]
            \[
              \begin{tikzcd}[ampersand replacement=\&, row sep = small]
                1\arrow{r}\&\ZZ/2\ZZ\arrow{r}{\iota}\&\wt{G}\arrow{r}{\pi}\&G\arrow{r}\&1
              \end{tikzcd}
            \]
            \caption{$\wt{G}$ a (central) extension of $G$}
            \label{fig:centralext}
          \end{figure}
        \item
          For each extension $\wt{G}$ as in Figure \ref{fig:centralext}
          from the previous step
          we perform the following:
          \begin{enumerate}
            \item
              % For $s\in\{0,1,\infty\}$,
              % $\sigma_s$ has $2$ preimages under the map $\pi$
              % which yields $8$ triples $\wt{\sigma} = (\wt{\sigma}_0,\wt{\sigma}_1,\wt{\sigma}_\infty)$
              % with the property that $\pi(\wt{\sigma}_s) = \sigma_s$ for each $s$.
              Consider the set of triples
              \begin{equation}
                \label{eqn:lifts}
                \{
                  \wt{\sigma}\colonequals(\wt{\sigma}_0, \wt{\sigma}_1, \wt{\sigma}_\infty) :
                  \wt{\sigma}_s\in\pi^{-1}(\sigma_s)\text{ for }
                  s\in\{0,1,\infty\}
                \}
              \end{equation}
              and let $\Lifts(\sigma)$ denote the set of such $\wt{\sigma}$
              with the property that $\wt{\sigma}_\infty\wt{\sigma}_1\wt{\sigma}_0 = 1$
              and $\langle\wt{\sigma}\rangle = \wt{G}$.
            \item
              For each $\wt{\sigma}\in\Lifts(\sigma)$ compute
              $\order(\wt{\sigma})\colonequals(\order(\wt{\sigma}_0), \order(\wt{\sigma}_1), \order(\wt{\sigma}_\infty))\in\ZZ^3$
              and sort $\Lifts(\sigma)$ according to $\order(\wt{\sigma})$.
              Let
              \begin{equation}
                \label{eqn:liftsorders}
                \Lifts(\sigma,(a,b,c))\colonequals
                \{
                  \wt{\sigma}\in\Lifts(\sigma) :
                  \order(\wt{\sigma}) = (a,b,c)
                \}.
              \end{equation}
            \item
              For each set of triples $\Lifts(\sigma,(a,b,c))$
              remove simultaneously conjugate triples so that
              $\Lifts(\sigma,(a,b,c))$ has exactly one representative
              from each simultaneous conjugacy class.
              \mm{TODO: reword}
          \end{enumerate}
        \item
          Return the union of the sets $\Lifts(\sigma,(a,b,c))$
          ranging over all extensions as in Figure
          \ref{fig:centralext}
          and for each extension ranging over all orders
          $(a,b,c)$.
      \end{enumerate}
    \end{alg}
    \begin{proof}[Proof of correctness]
      The algorithms in Step 1 are addressed in Section \ref{subsec:holtsalgorithm}.
      Let $\phi:X\to\PP^1$ be the $2$-group Belyi map corresponding to $\sigma$.
      By Proposition \ref{prop:galoiscorrespondence},
      the groups obtained from Step 1 are precisely the
      groups that can occur as monodromy groups of degree $2$ covers of $X$.
      \mm{
        lemma in section about extensions to prove
        that two isomorphic extensions cannot produce
        nonisomorphic Belyi maps
        and that two nonisomorphic extensions cannot produce
        isomorphic Belyi maps
      }
      In Step 2 we restrict our attention to a single extension of $G$
      as in Figure \ref{fig:centralext}.
      When we pullback a triple $\sigma$ under the map $\pi$,
      there are $2^3=8$ preimages $\wt{\sigma}$.
      Of these $8$ preimages, exactly $4$ have the property that
      $\wt{\sigma}_\infty\wt{\sigma}_1\wt{\sigma}_0=1$.
      Of these $4$ triples, we only take those that generate $\wt{G}$
      and this makes up the set $\Lifts(\sigma)$.
      In Step 2(b),
      we are sorting $\Lifts(\sigma)$ by passport.
      Since $2$-group Belyi maps are Galois,
      the cycle structure of each $\wt{\sigma}_s\in\wt{\sigma}$
      is determined by the order of $\wt{\sigma}_s$
      so that sorting by order is the same as sorting by cycle structure.
      \begin{remark}\label{rmk:twouniquepassports}
        In fact,
        even though we do not need this for the algorithm,
        there are at most $2$ different passports that can occur
        in $\Lifts(\sigma)$.
        $2$ different passports occur when one of $\sigma_s\in\sigma$
        is the identity.
        If $\sigma$ does not contain an identity element,
        then all triples in $\Lifts(\sigma)$ have the same passport.
      \end{remark}
      At this point,
      we have constructed the sets $\Lifts(\sigma,(a,b,c))$.
      In light of Remark \ref{rmk:twouniquepassports},
      there are only $2$ possibilities:
      \begin{itemize}
        \item
          There is only one such set $\Lifts(\sigma, (a,b,c))$
          consisting of at most $4$ triples.
        \item
          There are $2$ sets $\Lifts(\sigma, (a,b,c))$ and $\Lifts(\sigma, (a',b',c'))$
          each consisting of at most $2$ triples.
      \end{itemize}
      Step 2(c) is to eliminate simultaneous conjugation in each
      set $\Lifts(\sigma, (a,b,c))$.
      After Step 2(c) is complete,
      the sets $\Lifts(\sigma, (a,b,c))$ contain exactly one permutation triple
      for each isomorphism class of $2$-group Belyi map with passport determined by
      $(a,b,c)$ and monodromy group $\wt{G}$ such that the diagram in
      Figure \ref{fig:liftbelyimap} commutes.
      \begin{figure}[ht]
        \[
          \begin{tikzcd}[ampersand replacement=\&]
            \widetilde{X}\arrow{dd}[swap]{\widetilde{\phi}}\arrow{dr}{2}\\
            \&X\arrow{dl}{\phi}\\
            \PP^1
          \end{tikzcd}
        \]
        \caption{
          The permutation triples $\wt{\sigma}$ constructed in
          Algorithm \ref{alg:triples}
          correspond to Belyi maps $\wt{\phi}:\wt{X}\to\PP^1$
          in the above diagram.
        }
        \label{fig:liftbelyimap}
      \end{figure}
      In Step 3 we collect together all sets $\Lifts(\sigma, (a,b,c))$
      as we range over all possible extensions in Step 1,
      and by the discussion for Step 2 yields the desired output.
    \end{proof}
    We
    \begin{example}\label{exm:lift}
      \mm{copypaste}
    \end{example}
  }
  \section{An algorithm to compute $2$-group Belyi curves and maps}{\label{sec:curvesandmaps}
    The algorithm we describe here is iterative.
    The degree $1$ case is discussed in Section \ref{sec:degree1}.
    We now set up some notation for the iteration.
    \begin{notation}\label{not:maps}
      First we suppose
      we are given the following data:
      \begin{itemize}
        \item
          $X\subset\PP^n_K$ defined over a number field $K$
          with coordinates $x_0,\dots,x_n$
          cut out by the equations $\{h_i=0\}_i$ with $h_i\in K[x_0,\dots,x_n]$
        \item
          $\phi:X\to\PP^1$ a $2$-group Belyi map of degree $d=2^n$
          given by $\phi([x_0:\dots:x_n])=[x_0:x_1]$
          with monodromy group $G = \langle\sigma\rangle$
          (necessarily a $2$-group)
          with $\sigma$ a permutation triple corresponding to $\phi$
        \item
          For $s\in\{0,1,\infty\}$ and $\tau$ a cycle of $\sigma_s\in\sigma$,
          denote the ramification point above $s$ corresponding to
          $\tau$ by $Q_{s,\tau}$
        \item
          $Y\subset\AA^n_K$ the affine patch of $X$ with $x_0\neq 0$
          with coordinates $(y_1,\dots,y_n)$ where $y_i = x_i/x_0$
          cut out by the equations $\{g_i=0\}_i$ with $g_i\in K[y_1,\dots,y_n]$
          so that $\phi:Y\to\AA^1$ is given by $\phi(y_1,\dots,y_n) = y_1$
        \item
          $\wt{\sigma}$ as in the output of Algorithm \ref{alg:triples}
          applied to the input $\sigma$
      \end{itemize}
      Algorithm \ref{alg:lift} below
      describes how to lift the degree $d$ Belyi map $\phi$
      to a degree $2d$ Belyi map $\wt{\phi}$ with ramification prescribed by $\wt{\sigma}$
      (also see Figure \ref{fig:lift}).
    \end{notation}
    \begin{figure}[ht]
      \label{fig:lift}
      \[
        \begin{tikzcd}[ampersand replacement=\&]
          \widetilde{X}\arrow{dd}[swap]{\widetilde{\phi}}\arrow{dr}{\psi}\\
          % \widetilde{X}\arrow{dd}[swap]{\widetilde{\phi}}\arrow{dr}{\text{degree 2}}\\
          \&X\arrow{dl}{\phi}\\
          \PP^1
        \end{tikzcd}
      \]
      \caption{
      Algorithm \ref{alg:lift} describes how to construct $\wt{\phi}$
        corresponding to a permutation triple $\wt{\sigma}$
        from a given $2$-group Belyi map $\phi$.
        % In Algorithm \ref{alg:lift} we construct $\wt{\phi}$
        % from a given $2$-group Belyi map $\phi$.
      }
    \end{figure}
    \begin{lemma}\label{lem:1dim}
      Let $D$ be a degree $0$ divisor on $X$.
      Then $\ddim\LL(D) \leq 1$.
    \end{lemma}
    \begin{proof}
      Suppose $\ddeg D = 0$, and
      Let $f,g\in\LL(D)\setminus\{0\}$.
      Write $D=D_0-D_\infty$ with $D_0,D_\infty\geq 0$.
      Since $f,g\in\LL(D)$, we have
      $\ddiv f,\ddiv g\geq D_0-D_\infty$.
      In particular,
      $f/g\in K^\times$.
    \end{proof}
    \begin{alg}\label{alg:lift}
      Let the notation be as described above in \ref{not:maps}.
      \newline
      \textbf{Input}: A $2$-group Belyi map $\phi:X\to\PP^1_K$
      and a permutation triple $\wt{\sigma}$
      \newline
      \textbf{Output}: A model (over $\QQal$) for the Belyi map
      $\wt{\phi}:\wt{X}\to\PP^1_K$
      with monodromy $\wt{\sigma}$
      \begin{enumerate}
        \item
          Let $R$ be the empty set of points on $X$.
          For each $s\in\{0,1,\infty\}$,
          If the order of $\sigma_s$ is strictly less than the order of $\wt{\sigma}_s$,
          then append the ramification points
          $\{Q_{s,\tau}\}_{\tau\in\sigma_s}$
          (the ramification points on $X$ above $s$ corresponding to the cycles of $\sigma_s$)
          to $R$.
        \item
          Let $D = \sum_{P}n_PP$ be a degree $0$ divisor on $X$ 
          with $n_P$ odd for every $P\in R$
          and $n_P = 0$ for $P\not\in R$.
          \mm{class group and base field}
        \item
          Compute $f\in \Kbar(X)^\times$ corresponding to a generator
          of the Riemann-Roch space $\LL(D)$.
        \item
          Write $f=a/b$ with $a,b\in \Kbar[y_1,\dots,y_n]$ and construct the ideal
          \[
            \wt{I}\colonequals\langle g_1,\dots,g_k, by_{n+1}^2-a\rangle
          \]
          in $\Kbar[y_1,\dots,y_n,y_{n+1}]$.
        \item
          Saturate $\wt{I}$ at $\langle b\rangle$ and denote this ideal by $\sat(\wt{I})$.
        \item
          Let $\wt{X}$ be the curve corresponding to $\sat(\wt{I})$ and
          $\wt{\phi}$ the map $(y_1,\dots,y_{n+1})\mapsto y_1$.
      \end{enumerate}
    \end{alg}
    \begin{proof}[Proof of correctness]
      By Algorithm \ref{alg:triples},
      there exists a $2$-group Belyi map $\wt{\phi}:\wt{X}\to\PP^1$
      with ramification according to $\wt{\sigma}$.
      Since $\wt{\phi}$ is Galois,
      the ramification behavior above each $s\in\{0,1,\infty\}$
      is constant
      (i.e. for a fixed $s$, all $Q_{s,\tau}$ are either unramified or
      ramified to order $2$).
      This ensures that the set $R$ constructed in Step $1$
      is precisely the set of ramification values of $\psi$
      (in Figure \ref{fig:lift}).
      Now that we have the ramification values,
      we can construct the new Belyi map and curve.
      We do this by extracting a square root in the function field.
      More precisely,
      again by Algorithm \ref{alg:triples},
      there exists $\wt{X}$
      with $\Kbar(\wt{X}) = \Kbar(X,\sqrt{f})$
      where $f\in\Kbar(X)^\times/\Kbar(X)^{\times 2}$
      and
      \begin{equation}\label{eqn:classgroup}
        \ddiv f= \sum_{Q_{s,\tau}\in R} Q_{s,\tau}+2D_\epsilon
        \in\frac{\Div^0(X)}{2\Div^0(X)}
      \end{equation}
    \end{proof}
    \begin{example}
    \end{example}
  }
  \section{Running time analysis}{\label{sec:runtime}
  }
  \section{Explicit computations}{\label{sec:computations}
  }
}

%%%%%%%%%%%%%%%%%%%%%%%%%%%%%%%%%%%%%%%%%%%%%%%%%%
%Backend of thesis (references and the like)
%%%%%%%%%%%%%%%%%%%%%%%%%%%%%%%%%%%%%%%%%%%%%%%%%%
\backmatter

%Add your bibliography filename and uncomment these lines
\bibliographystyle{amsplain}
\addcontentsline{toc}{chapter}{References}
\bibliography{references}

%Uncomment if you want an index
%\printindex

\end{document}
