% arara: xelatex
% arara: bibtex
% arara: xelatex
% arara: xelatex

\documentclass{dcthesis}

%This version of the thesis template has been updated for the February 2017 thesis guidelines provided by the School of Graduate and Advanced Studies by David Freund and Daryl DeFord.

%  Some good options are draft and singlespacing, for drafts, and *final*,
%for the final cut, and noheadings, for a printout without headings.  
%You can also use the copyright option to add a copyright.  Note:
%final will enforce a bunch of different options, like oneside, 12pt
%and doublespacing, as well as make the margins correct.   Draft, has
%larger margins and is appropriate for two sided printing.   


%%%%%%%%%%%%%%%%%%%%%%%%%%%
%%%%    IMPORTANT     %%%%%
%%%%%%%%%%%%%%%%%%%%%%%%%%%
%
%  Because dvips doesn't know/care about the page size of the dvi file
%  its working on, and because its so important that the margins of
%  your thesis are correct you have to make sure that you are using
%  the command dvi -t letter *thesisnamehere*
%
%  If you are using a terminal this is straightforward, but if you are
%  using a tex editing program it make take a little bit of searching
%  before you figure out how to make this work. 
%  
%  Alternatively, you might consider compiling using pdflatex, which
%  compiles straight to a PDF and doesn't have this problem.  

%%%%%%%%%%%%%%%%%%%%%%%%%%%%%%%%%%%%%%%%%%%%%%%%%%
% Some Defaults
%%%%%%%%%%%%%%%%%%%%%%%%%%%%%%%%%%%%%%%%%%%%%%%%%%
\committee[F. Jon Kull, Ph.D.]{}{}{}{}
\school{Dartmouth College}{Hanover, New Hampshire}
\degree{Doctor of Philosophy}
\field{Mathematics}

%%%%%%%%%%%%%%%%%%%%%%%%%%%%%%%%%%%%%%%%%%%%%%%%%%
% Additional Packages (add as desired)
%%%%%%%%%%%%%%%%%%%%%%%%%%%%%%%%%%%%%%%%%%%%%%%%%%
\usepackage{amsmath,amssymb,amsthm,amsxtra}
\usepackage{lipsum}
\usepackage{xcolor}
\usepackage{tikz}
\usepackage{colonequals}
\usetikzlibrary{arrows,calc,automata,shadows,backgrounds,positioning,intersections,fadings,decorations.pathreplacing,shapes,snakes, matrix}
\usepackage{tikz-cd}
\tikzset{commutative diagrams/.cd, arrow style = tikz, diagrams = {>=latex}}
\tikzset{>=latex}

%\usepackage{index} %Uncomment if you would like to have an index. Compiling with an index takes more work than compiling without one. You will have to look up how to use the index package.

%%%%%%%%%%%%%%%%%%%%%%%%%%%%%%%%%%%%%%%%%%%%%%%%%%
%Formatting
%%%%%%%%%%%%%%%%%%%%%%%%%%%%%%%%%%%%%%%%%%%%%%%%%%
%  Change or add to as desired. 
%  These two commands make the first enumerations look like (a)
%  And the second level like (i).  
% \renewcommand{\labelenumi}{(\alph{enumi})}
% \renewcommand{\labelenumii}{(\roman{enumii})}
%  These commands make the section headings Boldface and not all
%  caps. It also removes the chapter numbers  
\renewcommand{\chaptermark}[1]{\markboth{{\sc #1}}{{\sc #1}}}
\renewcommand{\sectionmark}[1]{\markright{{\sc \thesection\ #1}}}
%  These commands have lowercase headings with chapter numbers. 
%\renewcommand{\chaptermark}[1]{markboth{#1}{}}
%\renewcommand{\sectionmark}[1]{\markright{\thesection\ #1}} 
\newcommand{\PP}{\mathbb P}
\newcommand{\CC}{\mathbb C}
\newcommand{\QQ}{\mathbb Q}
\newcommand{\ZZ}{\mathbb Z}
\renewcommand{\AA}{\mathbb A}
\newcommand{\defi}[1]{\textsf{#1}}
\newcommand{\jv}[1]{{\color{red} \sf JV: [#1]}}
\newcommand{\mm}[1]{{\color{blue} \sf MM: [#1]}}
\newcommand{\wt}[1]{\widetilde{#1}}

\DeclareMathOperator{\Div}{Div}
\DeclareMathOperator{\ddiv}{div}

%%%%%%%%%%%%%%%%%%%%%%%%%%%%%%%%%%%%%%%%%%%%%%%%%%
% Theorem Declarations
%%%%%%%%%%%%%%%%%%%%%%%%%%%%%%%%%%%%%%%%%%%%%%%%%%
% A basic set of theorem declarations.  Add or remove as desired. 
\newtheorem{prop}{Proposition}[chapter]
\newtheorem{theorem}[prop]{Theorem}
\newtheorem{lemma}[prop]{Lemma}
\newtheorem{corr}[prop]{Corollary}
\theoremstyle{definition}
\newtheorem{definition}[prop]{Definition}
\newtheorem{alg}[prop]{Algorithm}
\theoremstyle{remark}
\newtheorem{remark}[prop]{Remark}
\newtheorem{example}[prop]{Example}
\newtheorem*{claim}{Claim}

%%%%%%%%%%%%%%%%%%%%%%%%%%%%%%%%%%%%%%%%%%%%%%%%%%
%Indices
%%%%%%%%%%%%%%%%%%%%%%%%%%%%%%%%%%%%%%%%%%%%%%%%%%
%This is for an index.  More work is needed for a notation index
%\makeindex

%%%%%%%%%%%%%%%%%%%%%%%%%%%%%%%%%%%%%%%%%%%%%%%%%%
% Macros
%%%%%%%%%%%%%%%%%%%%%%%%%%%%%%%%%%%%%%%%%%%%%%%%%%
% Add your math macros here

%%%%%%%%%%%%%%%%%%%%%%%%%%%%%%%%%%%%%%%%%%%%%%%%%%
% Title Page Information
%%%%%%%%%%%%%%%%%%%%%%%%%%%%%%%%%%%%%%%%%%%%%%%%%%
%  Your personal info goes here!
\title{2-Group Belyi Maps}
\author{Michael James Musty}
\date{\today}
\field{Mathematics}
\degree{Doctor of Philosophy}
%As an optional argument to \committee you can specify the dean of graduate studies.
\committee{John Voight}{Thomas Shemanske}{David Roberts}{Carl Pomerance}

\begin{document}

%%%%%%%%%%%%%%%%%%%%%%%%%%%%%%%%%%%%%%%%%%%%%%%%%%
%Front end of thesis
%%%%%%%%%%%%%%%%%%%%%%%%%%%%%%%%%%%%%%%%%%%%%%%%%%
\frontmatter

\maketitle

%%%%%%%%%%%%%%%%%%%%%%%%%%%%%%%%%%%%%%%%%%%%%%%%%%
%Abstract
%%%%%%%%%%%%%%%%%%%%%%%%%%%%%%%%%%%%%%%%%%%%%%%%%%
%NOTE:  Must be less than 350 words.  
\chapter*{Abstract}
%This is so the abstract appears in the ToC
\addcontentsline{toc}{section}{Abstract}
Write your abstract here.

\chapter*{Preface}
\addcontentsline{toc}{section}{Preface}
Preface and Acknowledgments go here!

%%%%%%%%%%%%%%%%%%%%%%%%%%%%%%%%%%%%%%%%%%%%%%%%%%
%Table of Contents
%%%%%%%%%%%%%%%%%%%%%%%%%%%%%%%%%%%%%%%%%%%%%%%%%%
\tableofcontents

%Add a list of tables
\listoftables

%Add a list of figures
\listoffigures

%%%%%%%%%%%%%%%%%%%%%%%%%%%%%%%%%%%%%%%%%%%%%%%%%%
%Main Portion of Thesis
%%%%%%%%%%%%%%%%%%%%%%%%%%%%%%%%%%%%%%%%%%%%%%%%%%
\mainmatter

%%%%%%%%%%%%%%%%%%%%%%%%%
%%  YOUR THESIS HERE!  %%
%%%%%%%%%%%%%%%%%%%%%%%%%

% \include{chapter1}

\chapter{Introduction}{\label{chapter:intro}
  \section{Belyi maps from a historical perspective}{
    \lipsum[1]
    \cite{serre}
    \subsection[Inverse Galois theory]{Inverse Galois theory, Hurwitz families, and fields with few ramified primes}{
      \lipsum[1]
      \subsubsection{Inverse Galois theory}{
        \lipsum[1]
      }
      \subsubsection{Hurwitz families}{
        \lipsum[1]
      }
      \subsubsection{Number fields with few ramified primes}{
        \lipsum[1]
      }
    }
    \subsection[Dessins d'enfants]{Grothendieck's theory of dessins d'enfants}{
      \lipsum[1]
    }
  }
  \section{Belyi maps from a modern perspective}{
    \lipsum[1]
    \subsection{Anabelian geometry}{
      \lipsum[1]
    }
  }
}
\chapter{Background}{\label{chapter:background}
  \section{Belyi maps}{\label{sec:belyimaps}
    \lipsum[1]
    \subsection{Algebraic curves and their function fields}{
      \lipsum[1]
    }
    \subsection{Riemann's existence theorem and covers of $\PP^1$}{
      \lipsum[1]
    }
    \subsection{Belyi's theorem}{
      \lipsum[1]
    }
    \subsection{Belyi maps and $G$-Belyi maps}{
      \lipsum[1]
    }
  }
  \section{Group theory}{
    \lipsum[1]
    \subsection{Central group extensions and $H^2(G,A)$}{
      \lipsum[1]
    }
    \subsection{Holt's algorithm and \texttt{Magma} implementation}{
      \lipsum[1]
    }
    \subsection{Results on $2$-groups}{
      \lipsum[1]
    }
  }
  \section{Jacobians of curves}{
    \lipsum[1]
    \subsection{Abel-Jacobi and the construction over $\CC$}{
      \lipsum[1]
    }
    \subsection{Algebraic construction}{
      \lipsum[1]
    }
    \subsection{Riemann-Roch}{
      \lipsum[1]
    }
    \subsection{Torsion points and torsion fields}{
      \lipsum[1]
    }
  }
  \section{Galois representations}{
    \lipsum[1]
    \subsection{Representations of Galois groups of number fields}{
      \lipsum[1]
    }
    \subsection{Representations coming from geometry}{
      \lipsum[1]
    }
  }
}
\chapter{A database of $2$-group Belyi maps}{\label{chapter:database}
  In this chapter we describe an algorithm \mm{method?}
  to generate all $2$-group Belyi maps up to a given degree.
  We begin by defining this particular family of Belyi maps in Section \ref{sec:defs}.
  The algorithm is inductive in the degree.
  The base case in degree $2$ is discussed in Section \ref{sec:degree2}.
  We then move on to describe the inductive step of the algorithm
  which we describe in two parts.
  First we discuss the algorithm to enumerate the isomorphism classes
  using permutation triples in Section \ref{sec:isoclasses}.
  For a discussion on the relationship between permutation triples and Belyi maps
  see Chapter \ref{chapter:belyimaps}.
  Next we discuss the inductive step to produce Belyi curves and maps in Section \ref{sec:curvesandmaps}.
  In Section \ref{sec:runtime} we give a detailed description of the running time of the algorithm.
  Lastly,
  in Section \ref{sec:computations},
  we discuss the implementation and computations that we have carried out explicitly.
  \section{$2$-group Belyi maps}{\label{sec:defs}
    Recall the definition of a Belyi map in Section \ref{sec:belyimaps}.
    In this section we define a narrow our focus to a more specific family of Belyi maps
    which we now describe.
    \begin{definition}\label{def:galois}
      A degree $d$ Belyi map $\phi$ with monodromy group $G$
      is said to be \defi{Galois} if $\#G = d$.
    \end{definition}
    \begin{definition}\label{def:2groupbelyi}
      A \defi{$2$-group Belyi map} is a Galois Belyi map
      with monodromy group a $2$-group.
    \end{definition}
    \mm{some exposition}
  }
  \section{Degree $2$ Belyi maps}{\label{sec:degree2}
    \lipsum[1]
  }
  \section{An algorithm to enumerate isomorphism classes of $2$-group Belyi maps}{\label{sec:isoclasses}
    \begin{alg}\label{alg:triples}
    \end{alg}
  }
  \pagebreak
  \section{An algorithm to compute $2$-group Belyi curves and maps}{\label{sec:curvesandmaps}
    The algorithm we describe here is inductive.
    The base case is discussed in Section \ref{sec:degree2}.
    We now set up some notation for the inductive step.
    First we assume
    we have computed a $2$-group Belyi map $\phi:X\to\PP^1_k$ of degree
    $d=2^n$ with monodromy group $G\colonequals\langle\sigma\rangle$
    where $\sigma$ is a permutation triple corresponding to $\phi$
    and $k$ is a number field.
    We will assume $X$ is a projective curve sitting inside $\PP^n_k$
    with coordinates $x=[x_0:\dots:x_n]$ and
    cut out by the equations $\{h_i(x)=0\}$ with $h_i\in k[x_0,\dots,x_n]$.
    We also assume that the Belyi map $\phi$ is given by
    \[
      [x_0:\dots:x_n]\mapsto [x_0:x_n].
    \]
    We will use $Y\subset\AA^n_k$ to denote an affine patch of $X$ with coordinates
    $y=(y_1,\dots,y_n)$ cut out by the equations $\{g_i(y)=0\}_{i=1}^k$ with
    $g_i\in k[y_1,\dots,y_n]$.
    In the affine patch we assume the Belyi map is given by
    \[
      (y_1,\dots,y_n)\mapsto y_1.
    \]
    \mm{notation for residue fields?}
    Moreover, we assume
    that we are given a permutation triple $\wt{\sigma}$ arising as the output
    of Algorithm \ref{alg:triples} applied to the input $\sigma$.
    Algorithm \ref{alg:lift} describes how to lift the degree $d$ Belyi map $\phi$
    to a degree $2d$ Belyi map $\wt{\phi}$ with ramification prescribed by $\wt{\sigma}$
    (see Figure \ref{fig:lift}).
    \begin{figure}[ht]
      \label{fig:lift}
      \[
        \begin{tikzcd}[ampersand replacement=\&]
          \widetilde{X}\arrow{dd}[swap]{\widetilde{\phi}}\arrow{dr}{\psi}\\
          % \widetilde{X}\arrow{dd}[swap]{\widetilde{\phi}}\arrow{dr}{\text{degree 2}}\\
          \&X\arrow{dl}{\phi}\\
          \PP^1
        \end{tikzcd}
      \]
      \caption{
        In Algorithm \ref{alg:lift} we construct $\wt{\phi}$
        from a given $2$-group Belyi map $\phi$.
      }
    \end{figure}
    We are now ready to describe the algorithm.
    \begin{alg}\label{alg:lift}
      Let the notation be as above.
      \newline
      \textbf{Input}: $\phi:X\to\PP^1_k$, $\wt{\sigma}$
      \begin{enumerate}
        \item
          Compare permutation triples $\sigma$ and $\wt{\sigma}$
          to determine the ramification values of $\psi$.
          Denote these points on $X$ by $\{Q_i\}$.
        \item
          Let $D = \sum_{P}n_PP$ be a degree $0$ divisor in
          $\Div(X)$ with every $n_{Q_i}$ odd.
          \mm{class group and base field}
        \item
          Compute $f\in k(X)$ corresponding to a generator
          of the $1$-dimensional Riemann-Roch space $L(D)$.
        \item
          Write $f=a/b$ with $a,b\in k[y_1,\dots,y_n]$ and construct the ideal
          \[
            \wt{I}\colonequals\langle g_1,\dots,g_k, by_{n+1}^2-a\rangle
          \]
          in $k[y_1,\dots,y_n,y_{n+1}]$.
        \item
          Saturate $\wt{I}$ at $\langle b\rangle$ and denote this ideal by $\wt{J}$.
        \item
          Let $\wt{X}$ be the curve corresponding to $\wt{J}$ and
          $\wt{\phi}$ the map $(y_1,\dots,y_{n+1})\mapsto y_1$.
      \end{enumerate}
      \textbf{Output}: $\wt{\phi}:\wt{X}\to\PP^1_k$
    \end{alg}
    \begin{proof}[Proof of correctness]
    \end{proof}
    \begin{example}
    \end{example}
  }
  \section{Running time analysis}{\label{sec:runtime}
  }
  \section{Explicit computations}{\label{sec:computations}
  }
}

%%%%%%%%%%%%%%%%%%%%%%%%%%%%%%%%%%%%%%%%%%%%%%%%%%
%Backend of thesis (references and the like)
%%%%%%%%%%%%%%%%%%%%%%%%%%%%%%%%%%%%%%%%%%%%%%%%%%
\backmatter

%Add your bibliography filename and uncomment these lines
\bibliographystyle{amsplain}
\addcontentsline{toc}{chapter}{References}
\bibliography{references}

%Uncomment if you want an index
%\printindex

\end{document}
