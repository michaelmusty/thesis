% arara: pdflatex
% arara: pdflatex

\documentclass[xcolor=dvipsnames]{beamer}
\usetheme[
  numbering=fraction,
  % numbering=none,
  sectionpage=progressbar,
  subsectionpage=none,
  % block=fill,
  progressbar=foot,
]{metropolis}
\definecolor{DartGreen}{cmyk}{0.95,0,1,0.50}
\setbeamercolor{progress bar}{fg=DartGreen}
\makeatletter
  \setlength{\metropolis@titleseparator@linewidth}{5pt}
  \setlength{\metropolis@progressonsectionpage@linewidth}{5pt}
  \setlength{\metropolis@progressinheadfoot@linewidth}{5pt}
\makeatother

\usepackage{appendixnumberbeamer}

\usepackage{stmaryrd}
\usepackage{soul}
\usepackage{amssymb,amsthm,amsmath,amsxtra,mathrsfs}
% \usepackage[all]{xy}
% \usepackage{xfrac}
% \usepackage{calc}
\usepackage{tikz}
\usetikzlibrary{arrows,calc,automata,shadows,backgrounds,positioning,intersections,fadings,decorations.pathreplacing,shapes,snakes, matrix}
\usepackage{tikz-cd}
\tikzset{commutative diagrams/.cd, arrow style = tikz, diagrams = {>=latex}}
\tikzset{>=latex}

\usepackage{filecontents}
\usepackage{pgfplots, pgfplotstable}
\usepgfplotslibrary{statistics}

\usepackage{marvosym}
\usepackage[marvosym]{tikzsymbols}
\usepackage{colonequals}

\theoremstyle{plain}
\newtheorem*{thm}{Theorem}
\newtheorem*{lem}{Lemma}
\newtheorem*{prop}{Proposition}
\newtheorem*{ques}{Question}
\newtheorem*{conj}{Conjecture}
\newtheorem*{cor}{Corollary}

% \usecolortheme{owl}
\usecolortheme[snowy]{owl}

\newcommand{\OO}{\mathcal O}
\newcommand{\PP}{\mathbb P}
\newcommand{\CC}{\mathbb C}
\newcommand{\RR}{\mathbb R}
\newcommand{\QQ}{\mathbb Q}
\newcommand{\ZZ}{\mathbb Z}
\renewcommand{\AA}{\mathbb A}
\newcommand{\defi}[1]{\textsf{#1}}
\newcommand{\jv}[1]{{\color{red} \sf JV: [#1]}}
\newcommand{\mm}[1]{{\color{blue} \sf MM: [#1]}}
\newcommand{\wt}[1]{\widetilde{#1}}
\newcommand{\QQal}{{\mathbb Q}^{\textup{al}}}
\newcommand{\FFqal}{{\mathbb F}_q^{\textup{al}}}
\newcommand{\QQab}{{\mathbb Q}^{\textup{ab}}}
\newcommand{\QQbar}{{\mathbb Q}^{\textup{al}}}
\newcommand{\Kal}{{{K}^{\textup{al}}}}
\newcommand{\Kab}{{K}^{\textup{ab}}}
\newcommand{\kbar}{\overline{k}}
\newcommand{\Kbar}{\overline{K}}
\newcommand{\LL}{\mathscr L}
\newcommand{\sm}{\setminus}
\newcommand{\FF}{\mathbb{F}}
\renewcommand{\ker}{\operatorname{ker}}

\DeclareMathOperator{\con}{con}
\DeclareMathOperator{\Div}{Div}
\DeclareMathOperator{\Princ}{Princ}
\DeclareMathOperator{\Pic}{Pic}
\DeclareMathOperator{\ddiv}{div}
\DeclareMathOperator{\sat}{sat}
\DeclareMathOperator{\ddeg}{deg}
\DeclareMathOperator{\ddim}{dim}
\DeclareMathOperator{\rred}{red}
% \renewcommand{\rred}{\operatorname{red}}
\DeclareMathOperator{\Lifts}{Lifts}
\DeclareMathOperator{\order}{order}
\DeclareMathOperator{\Aut}{Aut}
\DeclareMathOperator{\PGL}{PGL}
\DeclareMathOperator{\Mon}{Mon}
\DeclareMathOperator{\Gal}{Gal}
\DeclareMathOperator{\supp}{supp}
\DeclareMathOperator{\ord}{ord}
\DeclareMathOperator{\mult}{mult}
\DeclareMathOperator{\stab}{stab}
\DeclareMathOperator{\orb}{orb}
\DeclareMathOperator{\id}{id}
% \DeclareMathOperator{\ker}{ker}
\DeclareMathOperator{\GL}{GL}
\DeclareMathOperator{\charpoly}{charpoly}
% \DeclareMathOperator{\det}{det}
\DeclareMathOperator{\Tr}{tr}
\DeclareMathOperator{\Pl}{Pl}
\DeclareMathOperator{\Cl}{Cl}
\DeclareMathOperator{\Jac}{Jac}
\DeclareMathOperator{\minpoly}{minpoly}


\title{$2$-group Belyi maps}
\author{Michael Musty}
\date{July 9, 2019}
\institute{Dartmouth College}
\begin{document}
  \maketitle
  \begin{frame}{Acknowledgements}
    \begin{itemize}
      \item
        Dave, Tom, Carl, and John
      \item
        Sam, Anna, Jeroen, Edgar, Florian, and Richard
      \item
        Mary, Jim, Matt, and Nicole
    \end{itemize}
  \end{frame}
  \begin{frame}{Outline}
    \tableofcontents
  \end{frame}
  \section{Motivation}{
    \begin{frame}{Galois representations}
      Let $X$ be an irreducible, smooth
      projective algebraic curve of genus $g\geq 1$
      over a number field $K$.
      Let $G_K\colonequals\Gal(\Kal\,|\,K)$ be
      the absolute Galois group of $K$ and
      let $\ell\in\ZZ$ be prime.
      \pause\newline
      Let $J\colonequals\Jac(X)$ be the
      \textbf{Jacobian variety} of $X$.
      $J$ is an abelian variety of dimension $g$.
      \pause\newline
      $G_K$ acts on the $\ell$-torsion points
      $J[\ell](\Kal)\cong(\ZZ/\ell\ZZ)^{2g}$ of $X$.
      \pause\newline
      This action determines a
      \textbf{mod-$\ell$ Galois representation}
      \[
        \rho\colon G_K\to\Aut(J[\ell])\cong\GL_{2g}(\ZZ/\ell\ZZ).
      \]
      \pause
      The geometry of $X$ and the arithmetic of
      $\rho$ are inimately related.
      \pause
      For example,
      if $X$ has good reduction at a prime
      $\mathfrak{p}$ above
      $p\neq\ell$,
      then $\mathfrak{p}$
      will be unramified in the
      \textbf{$\ell$-torsion field}
      $K(J[\ell])$.
    \end{frame}
    \begin{frame}{Belyi's theorem}
      A \textbf{Belyi map}
      is a morphism
      $\phi\colon X\to\PP^1$
      of smooth projective algebraic curves
      over $\CC$
      that is unramified outside of
      $\{0,1,\infty\}$.
      \pause
      \begin{thm}[Belyi 1979]
        \vspace{1pt}
        An algebraic curve (smooth projective)
        $X$ over $\CC$ can be defined over a number
        field if and only if $X$ admits a
        Belyi map.
      \end{thm}
    \end{frame}
    \begin{frame}{Some definitions}
      Let $\phi\colon X\to\PP^1$ be a Belyi map
      defined over $K$.
      \pause\newline
      The \textbf{genus} of $\phi$ is the genus
      of the curve $X$.
      \pause\newline
      The \textbf{degree} of $\phi$
      is the degree of the field extension
      \[
        K(\PP^1)\hookrightarrow K(X).
      \]
      \pause
      $\phi$ is \textbf{geometrically Galois}
      if $\Kal(X)$ is a Galois extension.
      \pause\newline
      The \textbf{monodromy group of $\phi$},
      $\Mon(\phi)$,
      is the image of the map
      \[
        \pi_1(\PP^1\setminus\{0,1,\infty\},\star)
        \to S_d
      \]
      obtained by path lifting.
      \pause\newline
      When $\phi$ is Galois we
      can identify $\Mon(\phi)$ with
      $\Gal(\Kal(X)\,|\,\Kal(\PP^1))$.
      \pause\newline
      We can now state Beckmann's theorem
      to relate Belyi maps to our original
      motivation.
    \end{frame}
    \begin{frame}{Beckmann's theorem}
      \begin{thm}[Beckmann 1989]
        \vspace{1pt}
        Let $\phi\colon X\to\PP^1$ be a Galois
        Belyi map with monodromy group $G$.
        Let $p$ be a prime not dividing
        $\#G$.
        \pause\newline
        Then there exists a number field $M$
        satisfying the following properties.
        \pause
        \begin{itemize}
          \item
            $p$ is unramified in $M$
          \item
            $\phi$ is defined over $M$
          \item
            $X$ is defined over $M$
          \item
            $X$ has good reduction at all
            primes $\mathfrak{p}$ of $M$
            above $p$
        \end{itemize}
      \end{thm}
    \end{frame}
    \begin{frame}{Why $p=2$?}
      \begin{conj}[Gross 1998]
        \vspace{1pt}
        For every prime $p$,
        there exists a nonsolvable Galois number field
        ramified only at $p$.
      \end{conj}
      \pause
      $p\geq 11$ : existence (Serre), explicit (Edixhoven, Mascot)
      \par
      $p=7$ : existence (Dieulefait)
      \par
      $p=5$ : existence (Demb\'{e}l\'{e}, Greenberg, Voight), explicit (Roberts)
      \par
      $p=3$ : existence (Demb\'{e}l\'{e}, Greenberg, Voight)
      \par
      $p=2$ : existence (Demb\'{e}l\'{e})
      \pause
      \par
      The hope is that an explicit nonsolvable field ramified only at $2$
      can be obtained as $K(\Jac(X)[2])$ where
      $X$ is the domain of a Galois Belyi map
      with monodromy group a $2$-group.
      \pause\newline
      We call these Belyi maps
      \textbf{$2$-group Belyi maps}.
    \end{frame}
    \begin{frame}{Main results}
      Motivated by the applications of
      $2$-group Belyi maps to arithmetic geometry,
      we now state the main results.
      \pause
      \begin{itemize}
        \item
          implementation of an algorithm to
          enumerate isomorphism classes
          of $2$-group Belyi maps
          \pause
        \item
          implementation of an algorithm to
          compute equations for
          $2$-group Belyi maps
          over finite fields
          \pause
        \item
          implementation of a \emph{method}
          to compute equations for
          $2$-group Belyi maps
          over number fields
          \pause
        \item
          computational and theoretical evidence
          supporting a conjecture that
          every $2$-group Belyi map
          is defined over an
          abelian extension of the rationals
      \end{itemize}
    \end{frame}
    \begin{frame}[fragile]{$2$-group Belyi maps as iterated quadratic extensions}
      \begin{center}
        \begin{tikzpicture}[scale=1]
          \node at (0,0) (X0) {$X_0 = \PP^1$};
          \node [above left=of X0] (X1) {$X_1$};
          \node [above =of X1] (X2) {$X_2$};
          \node [above =of X2] (dots) {$\vdots$};
          \node [above =of dots] (Xr) {$X_r$};
          %\draw[<->] (X1) to node [above,rotate=-35] {$\sim$} node {} (X0);
          \draw[->] (X1) to node [below] {$2$} node {} (X0);
          \draw[->] (X2) to node [left] {$2$} node {} (X1);
          \draw[->, bend left] (X2) to node [below, left] {} node {} (X0);
          \draw[->] (dots) to node [left] {$2$} node {} (X2);
          \draw[->] (Xr) to node [left] {$2$} node {} (dots);
          \draw[->, bend left] (Xr) to node [below, left] {} node {} (X0);
          \node at (6,0) (KX0) {$K(X_0) = K(\PP^1)$};
          \node [above left=of KX0] (KX1) {$K(X_1)$};
          \node [above =of KX1] (KX2) {$K(X_2)$};
          \node [above =of KX2] (dots) {$\vdots$};
          \node [above =of dots] (KXr) {$K(X_r)$};
          %\draw[<->] (X1) to node [above,rotate=-35] {$\sim$} node {} (X0);
          \draw[-] (KX1) to node [below] {$2$} node {} (KX0);
          \draw[-] (KX2) to node [left] {$2$} node {} (KX1);
          \draw[-, bend left] (KX2) to node [below, left] {} node {} (KX0);
          \draw[-] (dots) to node [left] {$2$} node {} (KX2);
          \draw[-] (KXr) to node [left] {$2$} node {} (dots);
          \draw[-, bend left] (KXr) to node [below, left] {} node {} (KX0);
        \end{tikzpicture}
      \end{center}
    \end{frame}
  }
  \section{Background}{
    \begin{frame}[fragile]{Isomorphism of Belyi maps}
      Let $\phi\colon X\to\PP^1$
      and
      $\phi'\colon X\,'\to\PP^1$
      be Belyi maps of degree $d$.
      \pause\newline
      $\phi$ and $\phi'$ are
      \textbf{isomorphic}
      (respectively \textbf{lax isomorphic})
      if the diagrams
      \[
        \begin{tikzcd}
          X\arrow{rr}{\sim}\arrow{dr}[swap]{\phi}
          &
          &X\,'\arrow{dl}{\phi'}\\
          &\PP^1
        \end{tikzcd},
        \text{ respectively }
        \begin{tikzcd}
          X\arrow{r}{\sim}\arrow{d}[swap]{\phi}&X\,'\arrow{d}{\phi'}\\
          \PP^1\arrow{r}{\sim}[swap]{\beta}&\PP^1
        \end{tikzcd}
      \]
      commute
      where $\beta(\{0,1,\infty\}) = \{0,1,\infty\}$.
    \end{frame}
    \begin{frame}{Permutation Triples}
      A \textbf{transitive permutation triple of degree $d$} is a triple
      \[
        \sigma = (\sigma_0, \sigma_1, \sigma_\infty)\in S_d^3
      \]
      such that
      \begin{itemize}
        \item
          $\sigma_\infty\sigma_1\sigma_0=1$
        \item
          $\sigma$ generates a transitive subgroup of $S_d$
      \end{itemize}
      \pause
      The set of degree $d$ Belyi maps up to isomorphism is in bijection with the
      set of degree $d$ transitive permutation triples up to
      \textbf{simultaneous conjugation} and
      the group $\langle\sigma\rangle$ is the monodromy group of $\phi$.
    \end{frame}
    \begin{frame}{Passports}
      A \textbf{passport} $\mathcal{P}$ consists of the data
      $(g,G,\lambda)$ where $g\geq 0$ is an integer,
      $G\leq S_d$ is a transitive subgroup,
      and $\lambda = (\lambda_0,\lambda_1,\lambda_\infty)$
      is a triple of partitions of $d$.
      \pause
      \par
      The \textbf{passport of a Belyi map} $\phi:X\to\PP^1$
      is $(g(X), \Mon(\phi), (\lambda_0,\lambda_1,\lambda_\infty))$
      with $g(X)$ the genus of $X$,
      $\Mon(\phi)$ the monodromy group of $\phi$,
      % and the partitions specified by ramification.
      and the partitions from ramification.
      \pause
      \par
      The \textbf{passport of a permutation triple} $\sigma$ is
      $(g(\sigma), \langle\sigma\rangle, \lambda(\sigma))$
      where
      $$
      g(\sigma) = 1-d+(e(\sigma_0)-e(\sigma_1)-e(\sigma_\infty))/2
      $$
      with
      $$
      e(\tau) = d-\#\text{cycles of }\tau,
      $$
      and $\lambda(\sigma)$ is specified by cycle structures.
      \pause
      \par
      We now discuss the importance of organizing triples by passport.
      % \pause
      % \par
      % The \textbf{size of a passport} $\mathcal{P}$ is the number of
      % simultaneous conjugacy classes of transitive permutation triples
      % with passport $\mathcal{P}$.
    \end{frame}
    \begin{frame}{Fields of moduli, fields of definition, and passports}
      Let $\phi\colon X\to\PP^1$ be a Belyi map.
      \pause\par
      A number field $K$ is a
      \textbf{field of definition} of $X$
      if $X$ can be defined by polynomial
      equations over $K$.
      Similarly for $\phi$.
      \pause\par
      The \textbf{field of moduli}
      $M(X)$
      of $X$ is
      the fixed field of the field
      automorphisms
      $\{\tau\in\Aut(\CC) : \tau(X)\simeq X\}$.
      Similarly for $\phi$, $M(\phi)$.
      \pause\par
      The \textbf{size} of a passport $\mathcal{P}$
      is the number of simultaneous conjugacy classes
      of permutation triples $\sigma$ with passport
      $\mathcal{P}$.
      \pause
      \begin{theorem}
        Let $\phi\colon X\to\PP^1$ be a
        Belyi map with passport $\mathcal{P}$.
        Then the degree of $M(\phi)$ is
        bounded by the size of $\mathcal{P}$.
      \end{theorem}
      \pause
      A Belyi map need not be defined
      over its field of moduli
      \rWalley
      \pause\par
      The situation improves, however,
      in the Galois setting...
    \end{frame}
    \begin{frame}{The Galois setting}
      Let $\phi\colon X\to\PP^1$ be a
      \emph{Galois} Belyi map of degree $d$
      with monodromy group $G$
      corresponding to the permutation triple
      $\sigma$.
      \pause\par
      Then
      \begin{itemize}
        \item
          $\phi$ and $X$ are defined over $M(\phi)$,
        \item
          $\#G = d$,
        \item
          all cycles of $\sigma_s$ have the same
          length for $s\in\{0,1,\infty\}$,
        \item
          and if we let $a,b,c$ be the orders
          of $\sigma_0,\sigma_1,\sigma_\infty$
          respectively,
          we have
          \[
            g(X) = 1+\frac{\#G}{2}
            \left(
              1-\frac{1}{a}
              -\frac{1}{b}
              -\frac{1}{c}
            \right).
          \]
      \end{itemize}
    \end{frame}
    \begin{frame}{Function fields}
      Let $\phi\colon X\to\PP^1$ be
      a degree $d$ Belyi map defined over $K$.
      \pause\par
      This corresponds to a degree $d$
      extension
      $K(\PP^1)\hookrightarrow K(X)$.
      \pause\par
      $K(\PP^1)\cong K(x)$
      the
      \textbf{rational function field}
      of $K$ in one variable.
      \pause\par
      For $K$ perfect,
      $K(X)$ has a primitive element
      and can be written as
      \[
        K(x)(\alpha) = \frac{K(x)[y]}{(\minpoly_{\alpha,K(x)}(y))}
      \]
      \pause\par
      Ramification in this setting corresponds
      to the factorization of ideals
      $(x),(x-1),(1/x)$
      in maximal orders of $K(X)$.
      \pause\par
      The monodromy group in this setting
      corresponds to field automorphisms
      of the Galois closure of $K(X)$
      fixing $K(x)$.
    \end{frame}
  }
  \section{Computing permutation triples}{
    \begin{frame}{Setup}
      We first define some terminology for
      permutation triples corresponding
      to $2$-group Belyi maps.
      \pause\par
      A \textbf{$2$-group permutation triple}
      of degree $d\in\ZZ_{\geq 1}$ is a triple
      of permutations
      $\sigma\colonequals(\sigma_0,\sigma_1,\sigma_\infty)
      \in S_d^3$ satisfying
      \begin{itemize}
        \item
          $\sigma_\infty\sigma_1\sigma_0=\id$;
        \item
          $G\colonequals\langle\sigma_0,\sigma_1\rangle$
          is a transitive subgroup of $S_d$; and
        \item
          $G$ is a $2$-group of order $d$
          embedded in $S_d$
          via its left regular representation.
      \end{itemize}
      \pause
      $G$ is called the \textbf{monodromy group of $\sigma$}.
      \pause\par
      We say two degree $d$
      $2$-group permutation triples $\sigma,\sigma'$
      are \textbf{simultaneously conjugate} if there exists
      $\tau\in S_d$ such that
      \[
        \sigma^\tau\colonequals
        (
          \tau^{-1}\sigma_0\tau,
          \tau^{-1}\sigma_1\tau,
          \tau^{-1}\sigma_\infty\tau,
        )
        =\sigma'
      \]
    \end{frame}
    \begin{frame}[fragile]{Lifting permutation triples}
      Let $\sigma$ be a $2$-group permutation triple.
      \pause\par
      A \textbf{lift} of $\sigma$ is
      a $2$-group permutation triple
      $\wt{\sigma}\in S_{2d}^3$
      such that
      $\langle\wt{\sigma}\rangle$
      is isomorphic to some extension
      $\wt{G}$
      of $\ZZ/2\ZZ$ by $G$
      as in the exact sequence below.
      \[
        \begin{tikzcd}
          1\arrow{r}&\ZZ/2\ZZ\arrow{r}{\iota}
                    &\wt{G}\arrow{r}{\pi}
                    &\langle\sigma\rangle\arrow{r}
                    &1
        \end{tikzcd}
      \]
      \pause
      For a $2$-group permutation triple
      $\sigma$, we denote the set of lifts
      of $\sigma$ by $\Lifts(\sigma)$
      and $\Lifts(\sigma)/\!\!\sim$
      denotes the set of lifts up
      to simultaneous conjugation.
    \end{frame}
    \begin{frame}[fragile]{Algorithm to compute $\Lifts(\sigma)/\!\!\sim$}
      \textbf{Input}:
      $\sigma$ a $2$-group permutation triple of degree $d$
      \newline
      \textbf{Output}:
      $\Lifts(\sigma)/\!\!\sim$
      \pause\par
      \begin{enumerate}
        \item
          Let $G = \langle\sigma\rangle$
          and compute representatives
          of $H^2(G,A)$
          where $A\colonequals\ZZ/2\ZZ$
          with the trivial $G$-module structure
        \item\pause
          For each $f\in H^2(G,A)$
          compute the corresponding
          extension
          \[
            \begin{tikzcd}
              1\arrow{r}&A\arrow{r}{\iota_f}
                        &\wt{G}_f\arrow{r}{\pi_f}
                        &G\arrow{r}&1
            \end{tikzcd}
          \]
        \item\pause
          For each extension
          $\wt{G}_f$
          compute the set
          $\Lifts(\sigma,f)$ defined by
          \[
            % \Lifts(\sigma, f)
            % \colonequals
            \Big\{
              \wt{\sigma}
              % =(\wt{\sigma}_0,
              % \wt{\sigma}_1,
              % \wt{\sigma}_\infty)
              :
              \wt{\sigma}_s\in\pi_f^{-1}
              (\sigma_s)\text{ for }
              s\in\{0,1,\infty\},\;
              \wt{\sigma}_\infty
              \wt{\sigma}_1
              \wt{\sigma}_0=1,\;
              \langle\wt{\sigma}\rangle
              =\wt{G}_f
            \Big\}
          \]
        \item\pause
          \[
            \Lifts(\sigma)
            \colonequals
            \bigcup_{f\in H^2(G,A)}
            \Lifts(\sigma, f)
          \]
        \item\pause
          Quotient $\Lifts(\sigma)$
          by simultaneous conjugation
          % by the equivalence relation
          % $\sim$ identifying triples
          % in $\Lifts(\sigma)$ that
          % are simultaneously conjugate,
          % to obtain representatives of
          % $\Lifts(\sigma)/\!\!\sim$.
      \end{enumerate}
    \end{frame}
    \begin{frame}[fragile]{Example computing $\Lifts(\sigma)/\!\!\sim$ : setup}
      % for $\sigma =\big((1\,2),\id,(1\,2)\big)$
      Let $\sigma = \big((1\,2),\id,(1\,2)\big)$.
      Then $G = \langle\sigma\rangle = \ZZ/2\ZZ$.
      \pause\par
      $\wt{G}_1\cong\ZZ/2\ZZ\times\ZZ/2\ZZ$
      and
      $\wt{G}_2\cong\ZZ/4\ZZ$ with
      \begin{align*}
        \begin{tikzcd}[ampersand replacement=\&, row sep = small]
          1\arrow{r}\&\ZZ/2\ZZ\arrow{r}{\iota_1}\&\wt{G}_1\arrow{r}{\pi_1}\&G\arrow{r}\&1
        \end{tikzcd}
        \\
        \begin{tikzcd}[ampersand replacement=\&, row sep = small]
          1\arrow{r}\&\ZZ/2\ZZ\arrow{r}{\iota_2}\&\wt{G}_2\arrow{r}{\pi_2}\&G\arrow{r}\&1
        \end{tikzcd}
      \end{align*}
      \pause
      Each map $\pi_1,\pi_2$ pulls back to
      $4$ triples that multiply to $\id$:
      \pause
      % $\Lifts(\sigma,\wt{G}_1) =
      $T_1=
      \Big\{((1\,2)(3\,4), \id, (1\,2)(3\,4)),
      ((1\,2)(3\,4), (1\,3)(2\,4), (1\,4)(2\,3)),$
      $((1\,4)(2\,3), \id, (1\,4)(2\,3)),
      ((1\,4)(2\,3), (1\,3)(2\,4), (1\,2)(3\,4))
      \Big\}$
      \pause
      % $\Lifts(\sigma,\wt{G}_2) =
      $T_2=
      \Big\{
      ((1\,4\,3\,2), \id, (1\,2\,3\,4)),
      ((1\,2\,3\,4), (1\,3)(2\,4), (1\,2\,3\,4)),$
      $((1\,2\,3\,4), \id, (1\,4\,3\,2)),
      ((1\,4\,3\,2), (1\,3)(2\,4), (1\,4\,3\,2))
      \Big\}$
    \end{frame}
    \begin{frame}{Example computing $\Lifts(\sigma)/\!\!\sim$ : action on blocks}
      % for $\sigma =\big((1\,2),\id,(1\,2)\big)$
      Choose $\alpha=(1\,3)(2\,4)$ to be
      the generator of $\iota_1(\ZZ/2\ZZ)$
      in $\wt{G}_1$.
      \pause\par
      Each triple in
      $T_1$ must act on the
      \emph{blocks}
      $\left\{\boxed{1\,3},\boxed{2\,4}\right\}$
      corresponding to the permutations in
      $\sigma$.
      \pause\par
      Let
      $(\wt{\sigma}_0,\wt{\sigma}_1,\wt{\sigma}_\infty)=
      ((1\,2)(3\,4), (1\,3)(2\,4), (1\,4)(2\,4))$.
      \pause\par
      Note that
      $\wt{\sigma}_0\Big(\boxed{1\,3}\Big) = \boxed{2\,4}$
      and
      $\wt{\sigma}_0\Big(\boxed{2\,4}\Big) = \boxed{1\,3}$.
      \pause\par
      The induced permutation
      of $\wt{\sigma}_0$ on blocks is
      $\left(\boxed{1\,3},\boxed{2\,4}\right)$
      which is the same as the
      permutation $\sigma_0 = (1\,2)$
      \pause\par
      Choosing
      \[
        \alpha\colonequals
        (1\;d+1)(2\;d+2)\dots
        (d-1\;2d-1)(d\;2d)
      \]
      allows us to label blocks by reducing modulo $d$.
    \end{frame}
    \begin{frame}{Example computing $\Lifts(\sigma)/\!\!\sim$ : conclude}
      % for $\sigma =\big((1\,2),\id,(1\,2)\big)$
      We currently have triples that multiply to $\id$
      and have the correct action on blocks,
      but we only want triples that
      generate the correct group.
      \pause\par
      $\Lifts(\sigma,\wt{G}_1\cong\ZZ/2\ZZ\times\ZZ/2\ZZ)=$
      $\Big\{ ((1\,2)(3\,4), (1\,3)(2\,4), (1\,4)(2\,3)),
      ((1\,4)(2\,3), (1\,3)(2\,4), (1\,2)(3\,4)) \Big\}$
      \pause\par
      $\Lifts(\sigma,\wt{G}_2\cong\ZZ/4\ZZ)=T_2=$
      $\Big\{ ((1\,4\,3\,2), \id, (1\,2\,3\,4)),
      ((1\,2\,3\,4), (1\,3)(2\,4), (1\,2\,3\,4)),$
      $((1\,2\,3\,4), \id, (1\,4\,3\,2)),
      ((1\,4\,3\,2), (1\,3)(2\,4), (1\,4\,3\,2)) \Big\}$
      \pause\par
      Lastly, we quotient by simultaneous conjugation to obtain
      $\Lifts(\sigma)/\!\!\sim \;=\Big\{
      ((1\,2)(3\,4),(1\,3)(2\,4),(1\,4)(2\,3)),$
      $((1\,4\,3\,2),\id,(1\,2\,3\,4)),
      ((1\,2\,3\,4),(1\,3)(2\,4),(1\,2\,3\,4)) \Big\}$
    \end{frame}
    \begin{frame}{Bipartite graphs of permutation triples}
      Now that we can lift permutation triples,
      we now describe some notation for the bipartite
      graphs that organize these triples.
      \pause\par
      For $i\in\ZZ_{\geq 1}$ we define the bipartite
      graph denoted $\mathscr{G}_{2^i}$ with the following
      node sets.
      \begin{itemize}
        \item
          $\mathscr{G}_{2^i}^\text{above}$ :
          the set of isomorphism classes of $2$-group Belyi maps
          of degree $2^i$
          indexed by $2$-group permutation triples
          $\wt{\sigma}$ up to simultaneous conjugation
          in $S_{2^i}$
        \item
          $\mathscr{G}_{2^i}^\text{below}$ :
          the set of isomorphism classes of $2$-group Belyi maps
          of degree $2^{i-1}$
          indexed by $2$-group permutation triples
          $\sigma$ up to simultaneous conjugation
          in $S_{2^{i-1}}$
      \end{itemize}
      \pause
      For every pair of nodes
      $(\wt{\sigma},\sigma)\in
      \mathscr{G}_{2^i}^\text{above}\times
      \mathscr{G}_{2^i}^\text{below}$
      there is an edge between
      $\sigma$ and $\wt{\sigma}$
      if and only if $\wt{\sigma}$ is
      simultaneously conjugate to a lift of
      $\sigma$.
    \end{frame}
    \begin{frame}{Algorithm to compute $\mathscr{G}_{2^i}$}
      \textbf{Input}:
      The bipartite graph $\mathscr{G}_{2^{i-1}}$
      \newline
      \textbf{Output}:
      The bipartite graph $\mathscr{G}_{2^{i}}$
      \pause\par
      \begin{enumerate}
        \item
          \[
            \Lifts(\mathscr{G}_{2^{i-1}})
            \colonequals
            \bigcup_{\sigma\in\mathscr{G}_{2^{i-1}}^\text{above}}\Lifts(\sigma)/\!\!\sim
          \]
        \item\pause
          Quotient
          $\Lifts(\mathscr{G}_{2^{i-1}})$
          by simultaneous conjugation in $S_{2^i}$
          to obtain
          $\Lifts(\mathscr{G}_{2^{i-1}})/\!\!\sim$
        \item\pause
          Define $\mathscr{G}_{2^i}^\text{below}\colonequals
          \mathscr{G}_{2^{i-1}}^\text{above}$
          and define
          $\mathscr{G}_{2^i}^\text{above}$
          by representatives of
          $\Lifts(\mathscr{G}_{2^{i-1}})/\!\!\sim$
        \item\pause
          For every pair
          $(\wt{\sigma},\sigma)\in
          \mathscr{G}_{2^i}^\text{above}
          \times
          \mathscr{G}_{2^i}^\text{below}$
          place an edge between
          $\wt{\sigma}$ and $\sigma$
          if and only if there is a triple in
          the equivalence class
          $[\wt{\sigma}]\in
          \Lifts(\mathscr{G}_{2^{i-1}})/\!\!\sim$
          that is a lift of $\sigma$
      \end{enumerate}
    \end{frame}
    \begin{frame}{Results : number of triples and passports}
      \begin{theorem}[M.]
        \vspace{1pt}
        The following tables list
        the number of isomorphism classes of
        $2$-group Belyi maps,
        the number of passports, and number of
        lax passports respectively
        up to degree $256$.
        \begin{equation*}
          \begin{tabular}{|c||c|c|c|c|c|c|c|c|c|}
            \hline
            $d$ & $1$ & $2$ & $4$ & $8$ & $16$ & $32$ & $64$ & $128$ & $256$\\
            \hline
            \#\text{ triples } & $1$ & $3$ & $7$ & $19$ & $55$ & $151$ & $503$ & $1799$ & $7175$\\
            \hline
          \end{tabular}
        \end{equation*}
        \begin{equation*}
          \begin{tabular}{|c||c|c|c|c|c|c|c|c|c|}
            \hline
            $d$ & $1$ & $2$ & $4$ & $8$ & $16$ & $32$ & $64$ & $128$ & $256$\\
            \hline
            \# passports & $1$ & $3$ & $7$ & $16$ & $41$ & $96$ & $267$ & $834$ & $2893$\\
            \hline
          \end{tabular}
        \end{equation*}
        \begin{equation*}
          \begin{tabular}{|c||c|c|c|c|c|c|c|c|c|}
            \hline
            $d$ & $1$ & $2$ & $4$ & $8$ & $16$ & $32$ & $64$ & $128$ & $256$\\
            \hline
            \# lax passports & $1$ & $1$ & $3$ & $6$ & $14$ & $31$ & $85$ & $257$ & $882$\\
            \hline
          \end{tabular}
        \end{equation*}
      \end{theorem}
    \end{frame}
    \begin{frame}[fragile]{Results : distribution of genera}
      \begin{figure}[ht]
        \centering
        \begin{tikzpicture}[scale=1]
          \begin{axis}[
            % title=Distribution of $g$,
            xlabel=$g$,
            ylabel=\# triples,
            ybar,
            ymin=0,
            xmin=-0.5,
            height=0.95\textheight,
            width=0.95\textwidth,
            % enlarge x limits=0.1,
            % enlarge y limits=0.05,
            % xtick=data,
            % ytick={100, 500, 1000},
            % x tick label style={
            %   rotate=45,
            %   anchor=east
            % }
          ]
            \addplot +[
              draw=black,
              hist={
                bins=131,
                data min=-0.5,
                data max=130.5
              }
            ]
            % table [x index=0, y index=0] {data.csv};
            table [y index = 0] {genus_up_to_d256.csv};
          \end{axis}
        \end{tikzpicture}
        % \label{fig:tikzplot}
        % \caption{Distribution of genera
        % up to degree $256$}
      \end{figure}
    \end{frame}
    \begin{frame}[fragile]{Results : groups by nilpotency class}
      \begin{figure}[ht]
        \centering
        % \begin{tikzpicture}[scale=1]
        %   \begin{axis}[
        %     ybar stacked,
        %     % scale only axis=true,
        %     height=0.9\textheight,
        %     width=0.45\textwidth,
        %     enlarge x limits=0.1,
        %     enlarge y limits=0.05,
        %     legend style={
        %       % at={(0.5,-0.30)},
        %       % at={(0.1,0.9)},
        %       at={(0,1)},
        %       anchor=north west,
        %       % legend columns=-1
        %     },
        %     ylabel={\# triples},
        %     xlabel={degree},
        %     symbolic x coords={
        %       $1$,$2$,$4$,
        %       $8$,$16$,$32$,
        %       $64$,$128$,$256$
        %     },
        %     xtick=data,
        %     % ytick={100, 500, 1000},
        %     x tick label style={
        %       rotate=45,
        %       anchor=east
        %     }
        %   ]
        %   \addplot+[ybar] plot coordinates {
        %     ($1$,1)
        %     ($2$,3)
        %     ($4$,7)
        %     ($8$,15)
        %     ($16$,31)
        %     ($32$,63)
        %     ($64$,127)
        %     ($128$,255)
        %     ($256$,511)
        %   };
        %   \addplot+[ybar] plot coordinates {
        %     ($1$,0)
        %     ($2$,0)
        %     ($4$,0)
        %     ($8$,4)
        %     ($16$,24)
        %     ($32$,88)
        %     ($64$,376)
        %     ($128$,1544)
        %     ($256$,6664)
        %   };
        %   \legend{abelian, nonabelian}
        %   \end{axis}
        % \end{tikzpicture}
        \begin{tikzpicture}[scale=1]
          \begin{axis}[
            ybar stacked,
            % scale only axis=true,
            height=0.95\textheight,
            width=0.95\textwidth,
            enlarge x limits=0.1,
            enlarge y limits=0.05,
            legend style={
              % at={(0.5,-0.30)},
              % at={(0.1,0.9)},
              at={(0,1)},
              anchor=north west,
              % legend columns=-1
            },
            ylabel={\# triples},
            xlabel={degree},
            symbolic x coords={
              $1$,$2$,$4$,
              $8$,$16$,$32$,
              $64$,$128$,$256$
            },
            xtick=data,
            % ytick={100, 500, 1000},
            x tick label style={
              rotate=45,
              anchor=east
            }
          ]
          % nc1
          \addplot+[ybar] plot coordinates {
            ($1$,1)
            ($2$,3)
            ($4$,7)
            ($8$,15)
            ($16$,31)
            ($32$,63)
            ($64$,127)
            ($128$,255)
            ($256$,511)
          };
          % nc2
          \addplot+[ybar] plot coordinates {
            ($1$,0)
            ($2$,0)
            ($4$,0)
            ($8$,4)
            ($16$,12)
            ($32$,28)
            ($64$,76)
            ($128$,172)
            ($256$,364)
          };
          % nc3
          \addplot+[ybar] plot coordinates {
            ($1$,0)
            ($2$,0)
            ($4$,0)
            ($8$,0)
            ($16$,12)
            ($32$,48)
            ($64$,168)
            ($128$,472)
            ($256$,1288)
          };
          % nc4
          \addplot+[ybar] plot coordinates {
            ($1$,0)
            ($2$,0)
            ($4$,0)
            ($8$,0)
            ($16$,0)
            ($32$,12)
            ($64$,120)
            ($128$,624)
            ($256$,2288)
          };
          % nc5
          \addplot+[ybar] plot coordinates {
            ($1$,0)
            ($2$,0)
            ($4$,0)
            ($8$,0)
            ($16$,0)
            ($32$,0)
            ($64$,12)
            ($128$,264)
            ($256$,2160)
          };
          % nc6
          \addplot+[ybar] plot coordinates {
            ($1$,0)
            ($2$,0)
            ($4$,0)
            ($8$,0)
            ($16$,0)
            ($32$,0)
            ($64$,0)
            ($128$,12)
            ($256$,552)
          };
          % nc7
          \addplot+[ybar] plot coordinates {
            ($1$,0)
            ($2$,0)
            ($4$,0)
            ($8$,0)
            ($16$,0)
            ($32$,0)
            ($64$,0)
            ($128$,0)
            ($256$,12)
          };
          \legend{
            class $1$,
            class $2$,
            class $3$,
            class $4$,
            class $5$,
            class $6$,
            class $7$,
          }
          \end{axis}
        \end{tikzpicture}
        % \label{fig:groupsbydegree}
        % \caption{
        %   \# permutation triples by degree
        %   with abelian and nonabelian monodromy
        %   groups (left)
        %   and
        %   \# permutation triples by degree
        %   with monodromy groups of
        %   various nilpotency classes (right).
        % }
      \end{figure}
    \end{frame}
    \begin{frame}[fragile]{Results : passport sizes}
      \begin{figure}[ht]
        \centering
        \begin{tikzpicture}[scale=1]
          \begin{axis}[
            ybar stacked,
            % scale only axis=true,
            height=0.95\textheight,
            width=0.95\textwidth,
            enlarge x limits=0.1,
            enlarge y limits=0.05,
            legend style={
              % at={(0.5,-0.30)},
              % at={(0.1,0.9)},
              at={(0,1)},
              anchor=north west,
              % legend columns=-1
            },
            ylabel={\# passports},
            xlabel={degree},
            symbolic x coords={
              $1$,$2$,$4$,
              $8$,$16$,$32$,
              $64$,$128$,$256$
            },
            xtick=data,
            % ytick={100, 500, 1000},
            x tick label style={
              rotate=45,
              anchor=east
            }
          ]
          % size1
          \addplot+[ybar] plot coordinates {
            ($1$,1)
            ($2$,3)
            ($4$,7)
            ($8$,13)
            ($16$,34)
            ($32$,70)
            ($64$,149)
            ($128$,402)
            ($256$,1065)
          };
          % size2
          \addplot+[ybar] plot coordinates {
            ($1$,0)
            ($2$,0)
            ($4$,0)
            ($8$,3)
            ($16$,3)
            ($32$,18)
            ($64$,84)
            ($128$,271)
            ($256$,1123)
          };
          % size3
          \addplot+[ybar] plot coordinates {
            ($1$,0)
            ($2$,0)
            ($4$,0)
            ($8$,0)
            ($16$,1)
            ($32$,1)
            ($64$,4)
            ($128$,11)
            ($256$,52)
          };
          % size4
          \addplot+[ybar] plot coordinates {
            ($1$,0)
            ($2$,0)
            ($4$,0)
            ($8$,0)
            ($16$,3)
            ($32$,3)
            ($64$,21)
            ($128$,117)
            ($256$,462)
          };
          % size6
          \addplot+[ybar] plot coordinates {
            ($1$,0)
            ($2$,0)
            ($4$,0)
            ($8$,0)
            ($16$,0)
            ($32$,1)
            ($64$,3)
            ($128$,11)
            ($256$,62)
          };
          % size8
          \addplot+[ybar] plot coordinates {
            ($1$,0)
            ($2$,0)
            ($4$,0)
            ($8$,0)
            ($16$,0)
            ($32$,3)
            ($64$,3)
            ($128$,12)
            ($256$,84)
          };
          % size12
          \addplot+[ybar] plot coordinates {
            ($1$,0)
            ($2$,0)
            ($4$,0)
            ($8$,0)
            ($16$,0)
            ($32$,0)
            ($64$,0)
            ($128$,4)
            ($256$,26)
          };
          % size16
          \addplot+[ybar] plot coordinates {
            ($1$,0)
            ($2$,0)
            ($4$,0)
            ($8$,0)
            ($16$,0)
            ($32$,0)
            ($64$,3)
            ($128$,3)
            ($256$,12)
          };
          % size24
          \addplot+[ybar] plot coordinates {
            ($1$,0)
            ($2$,0)
            ($4$,0)
            ($8$,0)
            ($16$,0)
            ($32$,0)
            ($64$,0)
            ($128$,0)
            ($256$,1)
          };
          % size32
          \addplot+[ybar] plot coordinates {
            ($1$,0)
            ($2$,0)
            ($4$,0)
            ($8$,0)
            ($16$,0)
            ($32$,0)
            ($64$,0)
            ($128$,3)
            ($256$,3)
          };
          % size64
          \addplot+[ybar] plot coordinates {
            ($1$,0)
            ($2$,0)
            ($4$,0)
            ($8$,0)
            ($16$,0)
            ($32$,0)
            ($64$,0)
            ($128$,0)
            ($256$,3)
          };
          \legend{
            size $1$,
            size $2$,
            size $3$,
            size $4$,
            size $6$,
            size $8$,
            size $12$,
            size $16$,
            size $24$,
            size $32$,
            size $64$,
          }
          \end{axis}
        \end{tikzpicture}
        % \label{fig:passportsizes}
        % \caption{
        %   \# passports of various sizes by degree
        % }
      \end{figure}
    \end{frame}
  }
  \section{Computing equations}{
    \begin{frame}{A motivating example}
      % number fields
    \end{frame}
    \begin{frame}{Algorithm in characteristic $p\geq 3$}
    \end{frame}
    \begin{frame}{Implementation in characteristic zero}
    \end{frame}
    \begin{frame}{Results}
    \end{frame}
  }
  \section{A refined conjecture}{
    \begin{frame}{Passports}
      Recall that a passport $\mathcal{P}$
      consists of the data
      $(g,G,\lambda)$
      where $g\in\ZZ_{\geq 0}$,
      $G$ is a transitive subgroup of $S_d$
      and $\lambda = (\lambda_0,\lambda_1,\lambda_\infty)$
      is a triple of partitions of $d$
      corresponding to conjugacy classes $(C_0,C_1,C_\infty)$
      of $S_d$.
      \pause\par
      The size of
      $\mathcal{P}$ is the cardinality of the
      set
      $\Sigma_\mathcal{P}$
      defined by
      \[
        \Big\{
          (\sigma_0,\sigma_1,\sigma_\infty)\in C_0\times C_1\times C_\infty :
          \sigma_\infty\sigma_1\sigma_0=1
          \text{ and }
          \langle\sigma_0,\sigma_1\rangle=G
        \Big\}/\!\!\sim
      \]
      where $\sim$ denotes simultaneous conjugation in $S_d$.
      \pause\par
      As a result of the action of
      $G_\QQ$ on $\mathcal{P}$,
      the size of $\mathcal{P}$ bounds the degree of the field
      of moduli of any Belyi map with passport
      $\mathcal{P}$.
      \pause\par
      To instead analyze
      $\Gal(\QQal\,|\,\QQab)$
      we \emph{refine} the notion of a passport.
    \end{frame}
    \begin{frame}{Refined passports}
      A \textbf{refined passport} $\mathscr{P}$
      consists of the data
      $(g,G,c)$
      where $g\in\ZZ_{\geq 0}$,
      $G$ is a transitive subgroup of $S_d$
      and $c = (c_0,c_1,c_\infty)$
      is a triple of conjugacy classes
      of $G$.
      \pause\par
      The size of
      $\mathscr{P}$ is the cardinality of the
      set
      $\Sigma_\mathscr{P}$
      defined by
      \[
        \Big\{
          (\sigma_0,\sigma_1,\sigma_\infty)\in c_0\times c_1\times c_\infty :
          \sigma_\infty\sigma_1\sigma_0=1
          \text{ and }
          \langle\sigma_0,\sigma_1\rangle=G
        \Big\}/\!\!\sim
      \]
      where
      $(\sigma_0,\sigma_1,\sigma_\infty)\sim
      (\sigma_0',\sigma_1',\sigma_\infty')$
      if there exists $\alpha\in\Aut(G)$
      with $\alpha(\sigma_s) = \sigma_s'$ for
      every $s\in\{0,1,\infty\}$.
      \pause\par
      As was the case with passport,
      every permutation triple $\sigma$
      determines a refined passport
      $\mathscr{P}(\sigma)$.
    \end{frame}
    \begin{frame}{A refined conjecture}
      \begin{thm}[M.]
        \vspace{1pt}
        The size of $\mathscr{P}(\sigma)$ is
        equal to $1$ for every
        $2$-group permutation triple
        $\sigma$ with degree $\leq 256$.
      \end{thm}
      \pause\par
      \begin{conj}[ARC]
        \vspace{1pt}
        The size of $\mathscr{P}(\sigma)$ is
        equal to $1$ for every
        $2$-group permutation triple.
      \end{conj}
      \pause\par
      \begin{thm}[M.]
        \vspace{1pt}
        ARC is true for $2$-group permutation triples $\sigma$
        with $\langle\sigma\rangle$ dihedral.
      \end{thm}
    \end{frame}
  }
  \section{Examples}{
    \begin{frame}{Notation}
      \texttt{4T1-4,1,4-g0}
    \end{frame}
    \begin{frame}{An}
      \texttt{4T1-4,1,4-g0}
    \end{frame}
  }
  \appendix
  \begin{frame}{Backup slides}
  \end{frame}
\end{document}
