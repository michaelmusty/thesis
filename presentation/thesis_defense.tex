% arara: xelatex
% arara: xelatex

\documentclass[xcolor=dvipsnames]{beamer}
\usetheme[
  numbering=fraction,
  numbering=none,
  sectionpage=progressbar,
  subsectionpage=none,
  % block=fill,
  progressbar=foot,
]{metropolis}
\definecolor{DartGreen}{cmyk}{0.95,0,1,0.50}
\setbeamercolor{progress bar}{fg=DartGreen}
\makeatletter
  \setlength{\metropolis@titleseparator@linewidth}{5pt}
  \setlength{\metropolis@progressonsectionpage@linewidth}{5pt}
  \setlength{\metropolis@progressinheadfoot@linewidth}{5pt}
\makeatother

\usepackage{appendixnumberbeamer}

\usepackage{stmaryrd}
\usepackage{soul}
\usepackage{amssymb,amsthm,amsmath,amsxtra}
% \usepackage[all]{xy}
% \usepackage{xfrac}
% \usepackage{calc}
\usepackage{tikz}
\usetikzlibrary{arrows,calc,automata,shadows,backgrounds,positioning,intersections,fadings,decorations.pathreplacing,shapes,snakes, matrix}
\usepackage{tikz-cd}
\tikzset{commutative diagrams/.cd, arrow style = tikz, diagrams = {>=latex}}
\tikzset{>=latex}
\usepackage{marvosym}
\usepackage[marvosym]{tikzsymbols}
\usepackage{colonequals}

\theoremstyle{plain}
\newtheorem*{thm}{Theorem}
\newtheorem*{lem}{Lemma}
\newtheorem*{prop}{Proposition}
\newtheorem*{ques}{Question}
\newtheorem*{conj}{Conjecture}
\newtheorem*{cor}{Corollary}

\usecolortheme{owl}

\newcommand{\OO}{\mathcal O}
\newcommand{\PP}{\mathbb P}
\newcommand{\CC}{\mathbb C}
\newcommand{\RR}{\mathbb R}
\newcommand{\QQ}{\mathbb Q}
\newcommand{\ZZ}{\mathbb Z}
\renewcommand{\AA}{\mathbb A}
\newcommand{\defi}[1]{\textsf{#1}}
\newcommand{\jv}[1]{{\color{red} \sf JV: [#1]}}
\newcommand{\mm}[1]{{\color{blue} \sf MM: [#1]}}
\newcommand{\wt}[1]{\widetilde{#1}}
\newcommand{\QQal}{{\mathbb Q}^{\textup{al}}}
\newcommand{\FFqal}{{\mathbb F}_q^{\textup{al}}}
\newcommand{\QQab}{{\mathbb Q}^{\textup{ab}}}
\newcommand{\QQbar}{{\mathbb Q}^{\textup{al}}}
\newcommand{\Kal}{{{K}^{\textup{al}}}}
\newcommand{\Kab}{{K}^{\textup{ab}}}
\newcommand{\kbar}{\overline{k}}
\newcommand{\Kbar}{\overline{K}}
\newcommand{\LL}{\mathscr L}
\newcommand{\sm}{\setminus}
\newcommand{\FF}{\mathbb{F}}
\renewcommand{\ker}{\operatorname{ker}}

\DeclareMathOperator{\con}{con}
\DeclareMathOperator{\Div}{Div}
\DeclareMathOperator{\Princ}{Princ}
\DeclareMathOperator{\Pic}{Pic}
\DeclareMathOperator{\ddiv}{div}
\DeclareMathOperator{\sat}{sat}
\DeclareMathOperator{\ddeg}{deg}
\DeclareMathOperator{\ddim}{dim}
\DeclareMathOperator{\rred}{red}
% \renewcommand{\rred}{\operatorname{red}}
\DeclareMathOperator{\Lifts}{Lifts}
\DeclareMathOperator{\order}{order}
\DeclareMathOperator{\Aut}{Aut}
\DeclareMathOperator{\PGL}{PGL}
\DeclareMathOperator{\Mon}{Mon}
\DeclareMathOperator{\Gal}{Gal}
\DeclareMathOperator{\supp}{supp}
\DeclareMathOperator{\ord}{ord}
\DeclareMathOperator{\mult}{mult}
\DeclareMathOperator{\stab}{stab}
\DeclareMathOperator{\orb}{orb}
\DeclareMathOperator{\id}{id}
% \DeclareMathOperator{\ker}{ker}
\DeclareMathOperator{\GL}{GL}
\DeclareMathOperator{\charpoly}{charpoly}
% \DeclareMathOperator{\det}{det}
\DeclareMathOperator{\Tr}{tr}
\DeclareMathOperator{\Pl}{Pl}
\DeclareMathOperator{\Cl}{Cl}
\DeclareMathOperator{\Jac}{Jac}


\title{$2$-group Belyi maps}
\author{Michael Musty}
\date{July 9, 2019}
\institute{Dartmouth College}
\begin{document}
  \maketitle
  \begin{frame}{Acknowledgements}
    \begin{itemize}
      \item
        Committee: Dave, Tom, Carl, and John
      \item
        Family: Mary, Jim, Matt, and Nicole
      \item
        Others: Sam, Jeroen, Edgar, Florian, and Richard
    \end{itemize}
  \end{frame}
  \begin{frame}{Outline}
    \tableofcontents
  \end{frame}
  \section{Motivation}{
    \begin{frame}{Galois representations}
      Let $X$ be an irreducible, smooth
      projective algebraic curve of genus $g\geq 1$
      over a number field $K$.
      Let $G_K\colonequals\Gal(\Kal\,|\,K)$ be
      the absolute Galois group of $K$ and
      let $\ell\in\ZZ$ be prime.
      \pause\newline
      Let $J\colonequals\Jac(X)$ be the
      \textbf{Jacobian variety} of $X$.
      $J$ is an abelian variety of dimension $g$.
      \pause\newline
      $G_K$ acts on the $\ell$-torsion points
      $J[\ell](\Kal)\cong(\ZZ/\ell\ZZ)^{2g}$ of $X$.
      \pause\newline
      This action determines a
      \textbf{mod-$\ell$ Galois representation}
      \[
        \rho\colon G_K\to\Aut(J[\ell])\cong\GL_{2g}(\ZZ/\ell\ZZ).
      \]
      \pause
      The geometry of $X$ and the arithmetic of
      $\rho$ are inimately related.
      \pause
      For example,
      if $X$ has good reduction at a prime
      $\mathfrak{p}$ above
      $p\neq\ell$,
      then $\mathfrak{p}$
      will be unramified in the
      \textbf{$\ell$-torsion field}
      $K(J[\ell])$.
    \end{frame}
    \begin{frame}{Belyi's theorem}
      A \textbf{Belyi map}
      is a morphism
      $\phi\colon X\to\PP^1$
      of smooth projective algebraic curves
      over $\CC$
      that is unramified outside of
      $\{0,1,\infty\}$.
      \pause
      \begin{thm}[Belyi 1979]
        \vspace{1pt}
        An algebraic curve (smooth projective)
        over $\CC$ can be defined over a number
        field if and only if $X$ admits a
        Belyi map.
      \end{thm}
    \end{frame}
    \begin{frame}{Some definitions}
      Let $\phi\colon X\to\PP^1$ be a Belyi map
      defined over $K$.
      \pause\newline
      The \textbf{genus} of $\phi$ is the genus
      of the curve $X$.
      \pause\newline
      The \textbf{degree} of $\phi$
      is the degree of the field extension
      \[
        K(\PP^1)\hookrightarrow K(X).
      \]
      \pause
      $\phi$ is \textbf{geometrically Galois}
      if $\Kal(X)$ is a Galois extension.
      \pause\newline
      The \textbf{monodromy group of $\phi$},
      $\Mon(\phi)$,
      is the image of the map
      \[
        \pi_1(\PP^1\setminus\{0,1,\infty\},\star)
        \to S_d
      \]
      obtained by path lifting.
      \pause\newline
      When $\phi$ is Galois we
      can identify $\Mon(\phi)$ with
      $\Gal(\Kal(X)\,|\,\Kal(\PP^1))$.
      \pause\newline
      We can now state Beckmann's theorem.
    \end{frame}
    \begin{frame}{Beckmann's theorem}
      \begin{thm}[Beckmann 1989]
        \vspace{1pt}
        Let $\phi\colon X\to\PP^1$ be a Galois
        Belyi map with monodromy group $G$.
        Let $p$ be a prime not dividing
        $\#G$.
        \pause\newline
        Then there exists a number field $M$
        satisfying the following properties.
        \pause
        \begin{itemize}
          \item
            $p$ is unramified in $M$
          \item
            $\phi$ is defined over $M$
          \item
            $X$ is defined over $M$
          \item
            $X$ has good reduction at all
            primes $\mathfrak{p}$ of $M$
            above $p$
        \end{itemize}
      \end{thm}
    \end{frame}
    \begin{frame}{Why $p=2$?}
      \begin{conj}[Gross 1998]
        \vspace{1pt}
        For every prime $p$,
        there exists a nonsolvable Galois number field
        ramified only at $p$.
      \end{conj}
      \pause
      $p\geq 11$ : existence (Serre), explicit (Edixhoven, Mascot)
      \par
      $p=7$ : existence (Dieulefait)
      \par
      $p=5$ : existence (Demb\'{e}l\'{e}, Greenberg, Voight), explicit (Roberts)
      \par
      $p=3$ : existence (Demb\'{e}l\'{e}, Greenberg, Voight)
      \par
      $p=2$ : existence (Demb\'{e}l\'{e})
      \pause
      \par
      The hope is that an explicit nonsolvable field ramified only at $2$
      can be obtained as $K(\Jac(X)[2])$ where
      $X$ is the domain of a Galois Belyi map
      with monodromy group a $2$-group.
      \pause\newline
      We call these Belyi maps
      \textbf{$2$-group Belyi maps}.
    \end{frame}
    \begin{frame}{Main results}
      Motivated by the applications of
      $2$-group Belyi maps to arithmetic geometry,
      we now state the main results.
      \pause
      \begin{itemize}
        \item
          implementation of an algorithm to
          enumerate isomorphism classes
          of $2$-group Belyi maps
          \pause
        \item
          implementation of an algorithm to
          compute equations for
          $2$-group Belyi maps
          over finite fields
          \pause
        \item
          implementation of a \emph{method}
          to compute equations for
          $2$-group Belyi maps
          over number fields
          \pause
        \item
          computational and theoretical evidence
          supporting a conjecture that
          every $2$-group Belyi map
          is defined over an
          abelian extension of the rationals
      \end{itemize}
    \end{frame}
  }
  \section{Background}{
    \begin{frame}[fragile]{Isomorphism of Belyi maps}
      Let $\phi\colon X\to\PP^1$
      and
      $\phi'\colon X\,'\to\PP^1$
      be Belyi maps of degree $d$.
      \pause\newline
      $\phi$ and $\phi'$ are
      \textbf{isomorphic}
      (respectively \textbf{lax isomorphic})
      if the diagrams
      \[
        \begin{tikzcd}
          X\arrow{rr}{\sim}\arrow{dr}[swap]{\phi}
          &
          &X\,'\arrow{dl}{\phi'}\\
          &\PP^1
        \end{tikzcd},
        \text{ respectively }
        \begin{tikzcd}
          X\arrow{r}{\sim}\arrow{d}[swap]{\phi}&X\,'\arrow{d}{\phi'}\\
          \PP^1\arrow{r}{\sim}[swap]{\beta}&\PP^1
        \end{tikzcd}
      \]
      commute
      where $\beta(\{0,1,\infty\}) = \{0,1,\infty\}$.
    \end{frame}
    \begin{frame}{Permutation Triples}
      A \textbf{transitive permutation triple of degree $d$} is a triple
      \[
        \sigma = (\sigma_0, \sigma_1, \sigma_\infty)\in S_d^3
      \]
      such that
      \begin{itemize}
        \item
          $\sigma_\infty\sigma_1\sigma_0=1$
        \item
          $\sigma$ generates a transitive subgroup of $S_d$
      \end{itemize}
      \pause
      The set of degree $d$ Belyi maps up to isomorphism is in bijection with the
      set of degree $d$ transitive permutation triples up to
      \textbf{simultaneous conjugation} and
      the group $\langle\sigma\rangle$ is the monodromy group of $\phi$.
    \end{frame}
    \begin{frame}{Passports}
      A \textbf{passport} $\mathcal{P}$ consists of the data
      $(g,G,\lambda)$ where $g\geq 0$ is an integer,
      $G\leq S_d$ is a transitive subgroup,
      and $\lambda = (\lambda_0,\lambda_1,\lambda_\infty)$
      is a triple of partitions of $d$.
      \pause
      \par
      The \textbf{passport of a Belyi map} $\phi:X\to\PP^1$
      is $(g(X), \Mon(\phi), (\lambda_0,\lambda_1,\lambda_\infty))$
      with $g(X)$ the genus of $X$,
      $\Mon(\phi)$ the monodromy group of $\phi$,
      and the partitions specified by ramification.
      \pause
      \par
      The \textbf{passport of a permutation triple} $\sigma$ is
      $(g(\sigma), \langle\sigma\rangle, \lambda(\sigma))$
      where
      $$
      g(\sigma) = 1-d+(e(\sigma_0)-e(\sigma_1)-e(\sigma_\infty))/2
      $$
      with
      $$
      e(\tau) = d-\#\text{cycles of }\tau,
      $$
      and $\lambda(\sigma)$ is specified by cycle structures.
      % \pause
      % \par
      % The \textbf{size of a passport} $\mathcal{P}$ is the number of
      % simultaneous conjugacy classes of transitive permutation triples
      % with passport $\mathcal{P}$.
    \end{frame}
    \begin{frame}{Function fields}
      Let $\phi\colon X\to\PP^1$ be
      a Belyi map
    \end{frame}
    \begin{frame}{Fields of moduli and fields of definition}
    \end{frame}
    \begin{frame}{The Galois setting}
    \end{frame}
  }
  \section{Computing permutation triples}{
    \begin{frame}{General idea}
    \end{frame}
    \begin{frame}{Outline of algorithm}
    \end{frame}
    \begin{frame}{Results}
    \end{frame}
  }
  \section{Computing equations}{
    \begin{frame}{A motivating example}
      % number fields
    \end{frame}
    \begin{frame}{Algorithm in characteristic $p\geq 3$}
    \end{frame}
    \begin{frame}{Implementation in characteristic zero}
    \end{frame}
    \begin{frame}{Results}
    \end{frame}
  }
  \section{A refined conjecture}{
    \begin{frame}{Refined passports}
    \end{frame}
    \begin{frame}{A refined conjecture}
    \end{frame}
  }
  \section{Examples}{
    \begin{frame}{Notation}
      \texttt{4T1-4,1,4-g0}
    \end{frame}
    \begin{frame}{An}
      \texttt{4T1-4,1,4-g0}
    \end{frame}
  }
  \appendix
  \begin{frame}{Backup slides}
  \end{frame}
\end{document}
