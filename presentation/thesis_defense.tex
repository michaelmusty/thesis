% arara: pdflatex
% arara: pdflatex
\documentclass[xcolor=dvipsnames]{beamer}
\usetheme[
  numbering=fraction,
  numbering=none,
  sectionpage=progressbar,
  subsectionpage=none,
  % block=fill,
  progressbar=foot,
]{metropolis}
\definecolor{DartGreen}{cmyk}{0.95,0,1,0.50}
\setbeamercolor{progress bar}{fg=DartGreen}
\makeatletter
  \setlength{\metropolis@titleseparator@linewidth}{5pt}
  \setlength{\metropolis@progressonsectionpage@linewidth}{5pt}
  \setlength{\metropolis@progressinheadfoot@linewidth}{5pt}
\makeatother
\usepackage{appendixnumberbeamer}
\usepackage{tikz}
\usetikzlibrary{arrows,calc,automata,shadows,backgrounds,positioning,intersections,fadings,decorations.pathreplacing,shapes,snakes, matrix}
\usepackage{tikz-cd}
\tikzset{commutative diagrams/.cd, arrow style = tikz, diagrams = {>=latex}}
\tikzset{>=latex}
\usepackage{marvosym}
\usepackage[marvosym]{tikzsymbols}
\theoremstyle{plain}
\newtheorem*{thm}{Theorem}
\newtheorem*{lem}{Lemma}
\newtheorem*{prop}{Proposition}
\newtheorem*{ques}{Question}
\newtheorem*{conj}{Conjecture}
\newtheorem*{cor}{Corollary}
\usecolortheme[snowy]{owl}
% \usecolortheme{owl}
\DeclareMathOperator{\disc}{disc}
\DeclareMathOperator{\Gal}{Gal}
\DeclareMathOperator{\id}{id}
\DeclareMathOperator{\Mon}{Mon}
\DeclareMathOperator{\rad}{rad}
\DeclareMathOperator{\Aut}{Aut}
\newcommand{\CC}{\mathbb C}
\newcommand{\wt}[1]{\widetilde{#1}}
\title{$2$-group Belyi maps}
\author{Michael Musty}
\date{July 9, 2019}
\institute{Dartmouth College}
\begin{document}
  \maketitle
  \begin{frame}{Outline}
    \tableofcontents
  \end{frame}
  \section{Motivation}{
    \begin{frame}{Belyi's theorem}
      asdf
    \end{frame}
    \begin{frame}{Gross's conjecture}
      asdf
    \end{frame}
    \begin{frame}{Beckmann's theorem}
      asdf
    \end{frame}
  }
  \section{Background}{
    \begin{frame}{Belyi maps}
      some stuff about Belyi maps in general
    \end{frame}
    \begin{frame}{Function fields}
      some stuff about function fields
    \end{frame}
  }
  \section{Computing permutation triples}{
    \begin{frame}{2-groups}
      some stuff about $2$-groups
    \end{frame}
  }
  \section{Computing equations}{
    \subsection{Main idea}{
      \begin{frame}{Main idea}
        some stuff about how to compute equations
      \end{frame}
    }
    \subsection{Characteristic $p\geq 3$}{
      \begin{frame}{Characteristic $p\geq 3$}
        some stuff about positive char
      \end{frame}
    }
    \subsection{Characteristic zero}{
      \begin{frame}{Characteristic $0$}
        some stuff about char $0$
      \end{frame}
    }
  }
  \section{A refined conjecture}{
    \begin{frame}{Conjecture}
      some stuff about the conjecture
    \end{frame}
  }
  \section{Examples}{
    \begin{frame}{Notation}
      \texttt{4T1-4,1,4-g0}
    \end{frame}
    \begin{frame}{An}
      \texttt{4T1-4,1,4-g0}
    \end{frame}
  }
  \appendix
  \begin{frame}{Backup slides}
  \end{frame}
\end{document}
