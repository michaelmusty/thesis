% arara: pdflatex
% arara: bibtex
% arara: pdflatex
% arara: bibtex
% arara: pdflatex

\documentclass{dcthesis}

%This version of the thesis template has been updated for the February 2017 thesis guidelines provided by the School of Graduate and Advanced Studies by David Freund and Daryl DeFord.

%  Some good options are draft and singlespacing, for drafts, and *final*,
%for the final cut, and noheadings, for a printout without headings.  
%You can also use the copyright option to add a copyright.  Note:
%final will enforce a bunch of different options, like oneside, 12pt
%and doublespacing, as well as make the margins correct.   Draft, has
%larger margins and is appropriate for two sided printing.   


%%%%%%%%%%%%%%%%%%%%%%%%%%%
%%%%    IMPORTANT     %%%%%
%%%%%%%%%%%%%%%%%%%%%%%%%%%
%
%  Because dvips doesn't know/care about the page size of the dvi file
%  its working on, and because its so important that the margins of
%  your thesis are correct you have to make sure that you are using
%  the command dvi -t letter *thesisnamehere*
%
%  If you are using a terminal this is straightforward, but if you are
%  using a tex editing program it make take a little bit of searching
%  before you figure out how to make this work. 
%  
%  Alternatively, you might consider compiling using pdflatex, which
%  compiles straight to a PDF and doesn't have this problem.  

%%%%%%%%%%%%%%%%%%%%%%%%%%%%%%%%%%%%%%%%%%%%%%%%%%
% Some Defaults
%%%%%%%%%%%%%%%%%%%%%%%%%%%%%%%%%%%%%%%%%%%%%%%%%%
\committee[F. Jon Kull, Ph.D.]{}{}{}{}
\school{Dartmouth College}{Hanover, New Hampshire}
\degree{Doctor of Philosophy}
\field{Mathematics}

%%%%%%%%%%%%%%%%%%%%%%%%%%%%%%%%%%%%%%%%%%%%%%%%%%
% Additional Packages (add as desired)
%%%%%%%%%%%%%%%%%%%%%%%%%%%%%%%%%%%%%%%%%%%%%%%%%%
\usepackage{amsmath,amssymb,amsthm,amsxtra}
\usepackage[chapter]{algorithm}
% \usepackage{showkeys}
\usepackage{chngcntr}
\usepackage{mathrsfs} 
\usepackage{lipsum}
\usepackage{xcolor}
\usepackage{hyperref}
\hypersetup{colorlinks=true,urlcolor=blue,citecolor=blue,linkcolor=blue}
\usepackage{tikz}
\usepackage{colonequals}
% \usepackage{caption}
\usepackage{enumerate}
\usepackage{booktabs}
\usepackage{longtable}
\usetikzlibrary{arrows,calc,automata,shadows,backgrounds,positioning,intersections,fadings,decorations.pathreplacing,shapes,matrix}
\usepackage{tikz-cd}
\tikzset{commutative diagrams/.cd, arrow style = tikz, diagrams = {>=latex}}
\tikzset{>=latex}
\usepackage{filecontents}
\usepackage{pgfplots, pgfplotstable}
\usepgfplotslibrary{statistics}

%%%%%% begin TCOLORBOX
\usepackage{tcolorbox, listings}
% \lstdefinestyle{mystyle}{
%   basicstyle=\ttfamily
% %  basicstyle=\ttfamily,
% %  numbers=left,
% %  numberstyle=\tiny,
% %  numbersep=5pt
% }
\tcbuselibrary{listings, skins, breakable}
\newtcblisting{magma}{%
  title={\textsf{Magma}},
  %every listing line={\textcolor{blue}{\texttt{> }}},
  arc=0mm,
  top=0mm,
  bottom=0mm,
  left=3mm,
  right=0mm,
  width=\textwidth,
  boxrule=0.5pt,
  colback=gray!20,
  %spartan,
  listing only,
  % listing options={style=mystyle},
  breakable
}
\newtcblisting{code}{%
  title={\textsf{Shell}},
  %every listing line={\textcolor{blue}{\texttt{> }}},
  arc=0mm,
  top=0mm,
  bottom=0mm,
  left=3mm,
  right=0mm,
  width=\textwidth,
  boxrule=0.5pt,
  colback=gray!20,
  %spartan,
  listing only,
  % listing options={style=mystyle},
  breakable
}
%%%%%% end TCOLORBOX

%\usepackage{index} %Uncomment if you would like to have an index. Compiling with an index takes more work than compiling without one. You will have to look up how to use the index package.

%%%%%%%%%%%%%%%%%%%%%%%%%%%%%%%%%%%%%%%%%%%%%%%%%%
%Formatting
%%%%%%%%%%%%%%%%%%%%%%%%%%%%%%%%%%%%%%%%%%%%%%%%%%
%  Change or add to as desired. 
%  These two commands make the first enumerations look like (a)
%  And the second level like (i).  
% \renewcommand{\labelenumi}{(\alph{enumi})}
% \renewcommand{\labelenumii}{(\roman{enumii})}
%  These commands make the section headings Boldface and not all
%  caps. It also removes the chapter numbers  
\renewcommand{\chaptermark}[1]{\markboth{{\sc #1}}{{\sc #1}}}
\renewcommand{\sectionmark}[1]{\markright{{\sc \thesection\ #1}}}
%  These commands have lowercase headings with chapter numbers. 
%\renewcommand{\chaptermark}[1]{markboth{#1}{}}
%\renewcommand{\sectionmark}[1]{\markright{\thesection\ #1}} 
\newcommand{\OO}{\mathcal O}
\newcommand{\PP}{\mathbb P}
\newcommand{\CC}{\mathbb C}
\newcommand{\RR}{\mathbb R}
\newcommand{\QQ}{\mathbb Q}
\newcommand{\ZZ}{\mathbb Z}
\renewcommand{\AA}{\mathbb A}
\newcommand{\defi}[1]{\textsf{#1}}
\newcommand{\jv}[1]{{\color{red} \sf JV: [#1]}}
\newcommand{\mm}[1]{{\color{blue} \sf MM: [#1]}}
\newcommand{\wt}[1]{\widetilde{#1}}
\newcommand{\QQal}{{\mathbb Q}^{\textup{al}}}
\newcommand{\FFqal}{{\mathbb F}_q^{\textup{al}}}
\newcommand{\QQab}{{\mathbb Q}^{\textup{ab}}}
\newcommand{\QQbar}{{\mathbb Q}^{\textup{al}}}
\newcommand{\Kal}{{{K}^{\textup{al}}}}
\newcommand{\Kab}{{K}^{\textup{ab}}}
\newcommand{\kbar}{\overline{k}}
\newcommand{\Kbar}{\overline{K}}
\newcommand{\LL}{\mathscr L}
\newcommand{\sm}{\setminus}
\newcommand{\FF}{\mathbb{F}}
\renewcommand{\ker}{\operatorname{ker}}

\DeclareMathOperator{\con}{con}
\DeclareMathOperator{\Div}{Div}
\DeclareMathOperator{\Princ}{Princ}
\DeclareMathOperator{\Pic}{Pic}
\DeclareMathOperator{\ddiv}{div}
\DeclareMathOperator{\sat}{sat}
\DeclareMathOperator{\ddeg}{deg}
\DeclareMathOperator{\ddim}{dim}
\DeclareMathOperator{\rred}{red}
% \renewcommand{\rred}{\operatorname{red}}
\DeclareMathOperator{\Lifts}{Lifts}
\DeclareMathOperator{\order}{order}
\DeclareMathOperator{\Aut}{Aut}
\DeclareMathOperator{\PGL}{PGL}
\DeclareMathOperator{\Mon}{Mon}
\DeclareMathOperator{\Gal}{Gal}
\DeclareMathOperator{\Inn}{Inn}
\DeclareMathOperator{\Out}{Out}
\DeclareMathOperator{\supp}{supp}
\DeclareMathOperator{\ord}{ord}
\DeclareMathOperator{\mult}{mult}
\DeclareMathOperator{\stab}{stab}
\DeclareMathOperator{\orb}{orb}
\DeclareMathOperator{\id}{id}
% \DeclareMathOperator{\ker}{ker}
\DeclareMathOperator{\GL}{GL}
\DeclareMathOperator{\charpoly}{charpoly}
% \DeclareMathOperator{\det}{det}
\DeclareMathOperator{\Tr}{tr}
\DeclareMathOperator{\Pl}{Pl}
\DeclareMathOperator{\Cl}{Cl}
\DeclareMathOperator{\Jac}{Jac}

%%%%%%%%%%%%%%%%%%%%%%%%%%%%%%%%%%%%%%%%%%%
%numberwithin section for
%tables equations and figures
%%%%%%%%%%%%%%%%%%%%%%%%%%%%%%%%%%%%%%%%%%%
\numberwithin{equation}{section}
\renewcommand{\thefigure}{\arabic{chapter}.\arabic{section}.\arabic{equation}}
% \numberwithin{equation}{section}
% \numberwithin{figure}{section}
% \numberwithin{table}{section}
% \counterwithin{equation}{section}
% \counterwithin{figure}{section}
% \counterwithin{table}{section}
% \renewcommand{\thefigure}{\arabic{figure}}
%%%%%%%%%
\makeatletter
\let\c@equation\c@figure
\makeatother
%%%%%%%%%%%%%%%%%%%%%%%%%%%%%%%%%%%%%%%%%%%%%%%%%%
% Theorem Declarations
%%%%%%%%%%%%%%%%%%%%%%%%%%%%%%%%%%%%%%%%%%%%%%%%%%
% A basic set of theorem declarations.  Add or remove as desired. 
% \newtheorem{prop}{Proposition}[chapter]
\newtheorem{theorem}[equation]{Theorem}
\newtheorem{prop}[equation]{Proposition}
\newtheorem{conj}[equation]{Conjecture}
\newtheorem{lemma}[equation]{Lemma}
\newtheorem{corr}[equation]{Corollary}
\theoremstyle{definition}
\newtheorem{definition}[equation]{Definition}
\newtheorem{alg}[equation]{Algorithm}
\newtheorem{notation}[equation]{Notation}
\theoremstyle{remark}
\newtheorem{remark}[equation]{Remark}
\newtheorem{example}[equation]{Example}
\newtheorem*{claim}{Claim}
\newtheorem{question}[equation]{Question}

%%%%%%%%%%%%%%%%%%%%%%%%%%%%%%%%%%%%%%%%%%%%%%%%%%
%Indices
%%%%%%%%%%%%%%%%%%%%%%%%%%%%%%%%%%%%%%%%%%%%%%%%%%
%This is for an index.  More work is needed for a notation index
%\makeindex

%%%%%%%%%%%%%%%%%%%%%%%%%%%%%%%%%%%%%%%%%%%%%%%%%%
% Macros
%%%%%%%%%%%%%%%%%%%%%%%%%%%%%%%%%%%%%%%%%%%%%%%%%%
% Add your math macros here

%%%%%%%%%%%%%%%%%%%%%%%%%%%%%%%%%%%%%%%%%%%%%%%%%%
% Title Page Information
%%%%%%%%%%%%%%%%%%%%%%%%%%%%%%%%%%%%%%%%%%%%%%%%%%
%  Your personal info goes here!
\title{2-Group Belyi Maps}
\author{Michael James Musty}
% \date{\today}
\date{July 9, 2019}
\field{Mathematics}
\degree{Doctor of Philosophy}
%As an optional argument to \committee you can specify the dean of graduate studies.
\committee{John Voight}{Thomas Shemanske}{Carl Pomerance}{David P. Roberts}

\begin{document}

%%%%%%%%%%%%%%%%%%%%%%%%%%%%%%%%%%%%%%%%%%%%%%%%%%
%Front end of thesis
%%%%%%%%%%%%%%%%%%%%%%%%%%%%%%%%%%%%%%%%%%%%%%%%%%
\frontmatter

\maketitle

%%%%%%%%%%%%%%%%%%%%%%%%%%%%%%%%%%%%%%%%%%%%%%%%%%
%Abstract
%%%%%%%%%%%%%%%%%%%%%%%%%%%%%%%%%%%%%%%%%%%%%%%%%%
%NOTE:  Must be less than 350 words.  
\chapter*{Abstract}
%This is so the abstract appears in the ToC
\addcontentsline{toc}{section}{Abstract}
This thesis concerns the explicit computation
of Galois Belyi maps
$\phi\colon X\to\PP^1$
with monodromy group
a $2$-group,
which we call $2$-group Belyi maps.
The computation has two parts.
The first is a combinatorial computation
to enumerate the isomorphism classes
of $2$-group Belyi maps.
% This computation provides evidence to
% a conjecture that every $2$-group
% Belyi map is defined over an abelian
% extension of $\QQ$.
% A partial proof to the
% conjecture for certain $2$-groups is given.
The second part is an explicit algorithm
to compute equations for the algebraic curve
$X$ and the Belyi map $\phi$.
\par
The motivation behind computing these maps comes from Beckmann's theorem,
which relates the primes of bad reduction of
$X$ to the primes dividing the order of the monodromy group of $\phi$.
Beckmann's theorem also implies that the field of
moduli of a $2$-group Belyi map is unramified away from $2$.
Are these moduli fields always solvable?
Is the field generated by the $2$-power torsion
subgroup of the Jacobian of $X$ solvable over $\QQ$?
This work aims to provide the computational framework
to begin answering these questions.
% All computations are carried out using
% \textsf{Magma}
% \cite{magma}
% and the source code
% written by the author
% is available at
% \cite{twogroupdessins}.

\chapter*{Acknowledgments}
\addcontentsline{toc}{section}{Acknowledgments}
I first want to thank my advisor John
for his time, support, and patience.
I doubt there is anyone more passionate about
math than John,
and I am just lucky enough to be part of his
tireless devotion to the cause.
\par
Second,
I would like to thank my committee
(Tom, Carl, and Dave)
for their efforts towards
improving this work.
I would also like to thank Sam Schiavone,
Edgar Costa, and Jeroen Sijsling
for helpful comments.
\par
Lastly,
I would like to thank my immediate
family (Mary, Jim, and Matt)
and my partner Nicole
for their constant support and love.
% Preface and Acknowledgments go here!
% \mm{what is a preface?}
% \mm{
%   Acknowledgments
% }
% Jim, Mary, Matt
% Nicole
% committee
% Richard Foote, Jeroen Sijsling, Florian Hess
% Sam Schiavone
% JV


%%%%%%%%%%%%%%%%%%%%%%%%%%%%%%%%%%%%%%%%%%%%%%%%%%
%Table of Contents
%%%%%%%%%%%%%%%%%%%%%%%%%%%%%%%%%%%%%%%%%%%%%%%%%%
\tableofcontents

%Add a list of tables
% \listoftables

%Add a list of figures
% \listoffigures

%%%%%%%%%%%%%%%%%%%%%%%%%%%%%%%%%%%%%%%%%%%%%%%%%%
%Main Portion of Thesis
%%%%%%%%%%%%%%%%%%%%%%%%%%%%%%%%%%%%%%%%%%%%%%%%%%
\mainmatter

%%%%%%%%%%%%%%%%%%%%%%%%%
%%  THESIS  %%
%%%%%%%%%%%%%%%%%%%%%%%%%

\chapter{Introduction}{\label{chapter:intro}
  \section{Motivation}{\label{sec:motivation}{
    A broad goal of arithmetic geometry is
    to use tools from algebraic geometry
    to study questions that arise
    in number theory.
    % One example of this connection is
    % the theorem of
    % Faltings,
    % which bounds
    % the number of rational points
    % (an arithmetic quantity)
    % according to
    % the genus $g \geq 2$
    % of a nonsingular algebraic curve
    % (a geometric quantity).
    An example of this connection
    is a theorem due to Belyi,
    which states that a nice algebraic curve
    $X$
    over the complex numbers
    % (equivalently a compact Riemann surface)
    can be defined by equations with
    coefficients in a number field
    if and only if
    $X$ admits a \defi{Belyi map},
    a finite cover $\phi\colon X\to\PP^1$
    unramified outside
    $\{0,1,\infty\}$.
    The way in which Belyi maps
    capture precisely when a transcendental
    object is also an algebraic object
    is just one of the remarkable properties
    of these covers.
    The goal of this thesis is to exploit
    these properties in the
    particular setting which we
    now describe.
    \par
    We begin with a motivating example.
    Let $E$ be an elliptic curve over $\QQ$,
    let $\ell\in\ZZ$ be prime,
    and let $G_\QQ\colonequals\Gal(\QQal\,|\,\QQ)$
    be the absolute Galois group of $\QQ$.
    There is an action of $G_\QQ$ on the
    $\ell$-torsion points
    $E[\ell](\QQal)\cong\ZZ/\ell\ZZ\times
    \ZZ/\ell\ZZ$
    of $E$,
    which determines a $2$-dimensional
    mod-$\ell$ Galois representation
    \begin{equation}
      \label{eqn:modlgalois}
      \rho\colon
      G_\QQ\to
      \Aut(E[\ell])\cong
      \GL_2(\ZZ/\ell\ZZ).
    \end{equation}
    The kernel of this
    representation fixes the field
    $\QQ(E[\ell])$,
    the
    $\ell$-torsion field of $E$
    obtained by adjoining the coordinates
    of all $\ell$-torsion points of $E$.
    The geometry of $E$ and the arithmetic
    of $\rho$ are intimately related.
    For example,
    if $E$ has good reduction at a prime $p\neq\ell$,
    then $p$ will be unramified in
    the field $\QQ(E[\ell])$
    by the criterion of
    N\'{e}ron--Ogg-Shafarevich.
    \par
    This relationship between curves and Galois
    representations extends to higher genus
    curves.
    Let $X$ be an
    irreducible, smooth projective
    curve
    of genus $g\geq 1$
    over a number field $K$.
    The Jacobian variety of $X$,
    $J\colonequals \Jac(X)$,
    is an abelian variety over $K$
    of dimension $g$.
    Again the $\ell$-torsion points
    $J[\ell]$ of $J$
    define a mod-$\ell$ Galois representation
    and a number field $K(J[\ell])$.
    As was the case for elliptic curves,
    if $X$ has good reduction at a prime
    $\mathfrak{p}$ in $K$,
    then $\mathfrak{p}$ is unramified in
    the $\ell$-torsion field
    $K(J[\ell])$.
    \par
    The application of Belyi maps to this
    situation comes from Beckmann's theorem.
    % \ref{thm:beckmann}.
    To state Beckmann's theorem requires
    a bit more terminology.
    Associated to every Belyi map
    $\phi\colon X\to\PP^1$
    over $\CC$
    is the \defi{monodromy group}
    of the covering
    obtained by lifting paths around
    the ramification points on $\PP^1$.
    We say that a Belyi map is
    \defi{Galois} if the covering is Galois
    (equivalently if the degree
    of the cover equals the size
    of the monodromy group).
    We can now state Beckmann's theorem.
    \begin{theorem}[Beckmann]
      \label{thm:beckmann}
      Let $\phi\colon X\to\PP^1$
      be a Galois Belyi map with
      monodromy group $G$
      and suppose $p$ does not divide $\# G$.
      Then there exists a number field $M$
      with the following properties:
      \begin{itemize}
        \item
          $p$ is unramified in $M$;
        \item
          the Belyi map $\phi$ is defined over $M$; and
        \item
          $\phi$ and $X$ have good reduction at all primes $\mathfrak{p}$
          of $M$ above $p$.
      \end{itemize}
    \end{theorem}
    \begin{proof}
      See \cite{beckmann} and \cite[Proposition 2.9]{beckmannref}.
    \end{proof}
    If we insist that $G$ is a $2$-group
    in Beckmann's theorem,
    then we can hope to find
    fields $M$
    and $M(\Jac(X)[2])$ unramified away from $2$.
    The main interest in finding such a field
    comes from a conjecture
    (now theorem) of Gross.
    \begin{conj}[Gross]
      \label{conj:gross}
      For every prime $p$,
      there exists a nonsolvable Galois
      number field
      ramified only at $p$.
    \end{conj}
    For $p\geq 11$, the existence of such
    fields is attributed to Serre in
    \cite{serre32,serre33}
    and explicit examples are given
    in \cite{mascot}.
    For $p=3,5$, the existence of such fields
    is proved in \cite{DGV}
    and an explicit example for $p=5$
    is given in \cite{roberts5}.
    Existence of such a field in the $p=7$ case
    is attributed to
    \cite{die} using techniques from
    \cite{DGV} and some corrections by
    David P. Roberts.
    For $p=2$, the existence of such a field
    is proven in \cite{lassina}.
    % Although Gross's conjecture is now resolved,
    % the only explicit example of such
    % a field for $p<11$ is given
    % in \cite{roberts5} for $p=5$.
    A long-term application of the work
    in this thesis is to
    find an explicit nonsolvable number field
    ramified only at $2$.
    % The existence of such a field was proved
    % in \cite{lassina} using techniques
    % unrelated to Belyi maps.
    \par
    But first, to use Beckmann's theorem
    to construct interesting number fields,
    we must have explicit
    Galois Belyi maps with monodromy group
    a $2$-group.
    The explicit construction of these
    objects is the focus of this thesis,
    and it is these objects we refer to as
    \defi{$2$-group Belyi maps}.
    We now summarize the results of this thesis
    concerning
    $2$-group Belyi maps.
  }
  \section{Main results}{\label{sec:mainresults}
    Motivated by the discussion in
    Section \ref{sec:motivation},
    this work aims to address the task
    of explicitly computing
    $2$-group Belyi maps up to isomorphism.
    This computation consists of two
    main parts:
    \begin{itemize}
      \item
        enumerating isomorphism classes,
      \item
        computing explicit equations
        for each isomorphism class.
    \end{itemize}
    \par
    Let $d\in\ZZ_{\geq 1}$.
    Isomorphism classes of degree $d$
    Belyi maps
    are in bijection with
    the set of
    transitive permutation triples
    up to
    simultaneous conjugation
    in the symmetric group $S_d$.
    A \defi{transitive permutation triple}
    is a triple of permutations
    $\sigma\colonequals
    (\sigma_0,\sigma_1,\sigma_\infty)
    \in S_d^3$
    that multiply to the identity
    and generate a transitive subgroup
    of $S_d$.
    We say
    that two permutation triples
    $\sigma,\sigma'$ are
    \defi{simultaneously conjugate}
    if there exists $\tau\in S_d$ such that
    \begin{equation}\label{eqn:simconj_results}
      \sigma^\tau \colonequals
      (\tau^{-1}\sigma_0\tau, \tau^{-1}\sigma_1\tau, \tau^{-1}\sigma_\infty\tau)
      = \left(\sigma'_0,\sigma'_1,\sigma'_\infty\right)
      = \sigma'.
    \end{equation}
    The first main result of this thesis
    is an explicit algorithm
    (see Algorithm
    \ref{alg:alltriplesiteration}
    in Section \ref{sec:triplesalgorithm})
    to compute permutation triples
    corresponding to all $2$-group Belyi
    maps up to a given degree.
    We use this algorithm (implemented in
    \textsf{Magma} \cite{magma})
    to enumerate
    all such permutation triples
    up to conjugation
    for $2$-power degree
    up to $256$.
    The algorithms in Section
    \ref{sec:triplesalgorithm}
    are used to prove results such as
    the following theorem.
    \begin{theorem}\label{thm:isoclasses_results}
      The following table lists
      the number of isomorphism classes of
      permutation triples corresponding to
      $2$-group Belyi maps
      of degree $d$ up to $256$.
      \begin{equation}
        \label{eqn:}
        \begin{tabular}{|c||c|c|c|c|c|c|c|c|c|}
          \hline
          $d$ & $1$ & $2$ & $4$ & $8$ & $16$ & $32$ & $64$ & $128$ & $256$\\
          \hline
          \#\text{ permutation triples } & $1$ & $3$ & $7$ & $19$ & $55$ & $151$ & $503$ & $1799$ & $7175$\\
          \hline
        \end{tabular}
      \end{equation}
    \end{theorem}
    Other results of this type are
    detailed in
    Section \ref{sec:grouptheorycomputations}.
    Having explicit permutation triples
    allows us to apply techniques from
    \cite{belyidb}
    to obtain information about the possible
    number fields that $2$-group Belyi
    maps can be defined over.
    In particular,
    in Section \ref{sec:computerefined},
    we prove the
    following theorem.
    \begin{theorem}
      \label{thm:256_results}
      Every $2$-group Belyi map
      of degree $d\leq 256$ is defined over
      a quadratic extension of an abelian number field
      ramified only at $2$.
    \end{theorem}
    Theorem \ref{thm:256_results}
    raises the following questions.
    \begin{question}
      \label{ques:solvablemotivation}
      Are all $2$-group Belyi maps defined over solvable
      number fields?
      Is the field generated
      by the $2$-power torsion subgroup of the
      Jacobian of a $2$-group Belyi curve
      solvable over $\QQ$?
    \end{question}
    A first step in answering
    Question
    \ref{ques:solvablemotivation}
    is to compute explicit equations for
    $2$-group Belyi maps.
    % The beginning of
    % Section \ref{sec:computerefined}
    % explains the above theorem and then
    % pivots to the following conjecture.
    % \begin{conj}
    %   \label{conj:conjecture_results}
    %   Every $2$-group Belyi map
    %   is defined over an abelian number field.
    % \end{conj}
    % In addition to verifying this conjecture
    % for $d\leq 256$,
    % we were able to prove this conjecture
    % for $2$-group Belyi maps
    % with monodromy group abelian
    % (Lemma \ref{lem:conjectureabelian})
    % or dihedral
    % (Theorem \ref{thm:conjecturedihedral}).
    %
    % \par
    The rest of the results of this thesis
    pertain to computing equations
    of $2$-group Belyi maps.
    The first of these is an algorithm
    to compute $2$-group Belyi maps
    over a finite field $\FF_q$
    where $q=p^k$ and $p\neq 2$
    (see Section \ref{sec:functionfieldalgorithm}).
    This algorithm has been implemented in
    \textsf{Magma}
    and used to construct a database
    of all $2$-group Belyi maps
    (up to isomorphism)
    over
    $\FF_3^\textup{al}$
    up to degree $32$.
    \par
    The other main result is an implementation
    (similar to the algorithm over 
    $\FF_q$)
    in characteristic zero
    (see Section
    \ref{sec:characteristiczero}).
    Although the characteristic zero
    implementation does not succeed in all cases,
    it often works well in practice.
    In particular,
    the \textsf{Magma} implementation succeeded
    in computing hundreds of $2$-group Belyi maps
    over $\QQal$ up to degree $256$.
    \par
    The rest of Chapter \ref{chapter:equations}
    is devoted to describing the computations
    along with interesting examples
    encountered along the way.
  }
  % \section{Explicit computations}{
  %   \label{sec:mainresults}
  %   All source code and examples
  %   can be found at the following link:
  %   \begin{center}
  %     \url{https://github.com/michaelmusty/2GroupDessins}
  %   \end{center}
  % }
  \section{Navigation}{\label{sec:navigation}
    Having motivated and stated the main
    results, we now provide
    some explanation of how this thesis is
    organized
    and where to find details pertaining
    to the main results.
    \par
    Chapter \ref{chapter:backgroundbelyimaps}
    details some of the
    necessary background material
    related to Belyi maps,
    permutation triples,
    and function fields.
    \par
    Chapter \ref{chapter:grouptheory}
    describes an algorithm
    to enumerate the
    isomorphism classes
    of $2$-group Belyi maps
    using permutation triples
    (see
    Algorithm \ref{alg:triples}
    and
    Algorithm \ref{alg:alltriplesiteration}).
    These algorithms have been used to
    enumerate all isomorphism classes of
    $2$-group Belyi maps with degree
    up to $256$.
    The results of these computations are
    detailed in
    Section \ref{sec:grouptheorycomputations}.
    \par
    Chapter \ref{chapter:fieldsofdefinition}
    explains how
    the results of
    computations in
    Chapter \ref{chapter:grouptheory}
    can be used to obtain information
    on the possible fields of definition
    of $2$-group Belyi maps.
    % to provide evidence for a conjecture
    % that all
    % $2$-group Belyi maps
    % are defined over a cyclotomic field.
    \par
    Chapter \ref{chapter:equations}
    discusses an algorithm to
    compute explicit equations
    for $2$-group Belyi maps
    over finite fields with
    characteristic not $2$
    (see
    Algorithm \ref{alg:getcandidates},
    Algorithm \ref{alg:liftbelyimap},
    and
    Algorithm \ref{alg:computepassport}).
    Algorithm \ref{alg:charzero}
    describes the modifications to
    in characteristic zero.
    \par
    In Chapter \ref{chapter:future}
    we discuss the future direction of this work.
    \par
    The source code for the implementation used
    in this thesis can be found at
    \cite{twogroupdessins}.
    % the following link.
    % \begin{center}
    %   \url{https://github.com/michaelmusty/2GroupDessins}
    % \end{center}
  }
  % all main results motivated and stated
  % RETppr exposition about inverse Galois
  % \section{Belyi maps from a historical perspective}{
  %   In \cite{belyi}, G.V. Belyi proved that a Riemann surface
  %   $X$
  %   can be defined over a number field
  %   (when viewed as an algebraic curve over $\CC$)
  %   if and only if there exists a non-constant
  %   meromorphic function $\phi:X\to\PP^1_\CC$ unramified outside the set $\{0,1,\infty\}$.
  %   This result came to be known as Belyi's Theorem
  %   and the maps $\phi$ came to be known as Belyi maps (or Belyi functions).
  %   Although Belyi's Theorem has an elementary proof,
  %   it was a starting point for a great deal of modern research in the area.
  %   This work was largely spurred on by Grothendieck's
  %   \emph{Esquisse d'un programme}
  %   \cite{esquisse}
  %   where he was impressed enough to write
  %   \begin{quote}
  %     \emph{jamais sans doute un r\'{e}sultat profond et d\'{e}routant ne fut
  %     d\'{e}montr\'{e} en si peu de lignes!}
  %     \newline
  %     never, without a doubt, was such a deep and disconcerting result proved in so few lines!
  %   \end{quote}
  %   An intriguing aspect of the theory of Belyi maps that arose from Grothendieck's
  %   work in the 1980s is the reformulation of these objects
  %   in a purely topological way.
  %   The preimage $\phi^{-1}([0,1])$ is a graph embedded on $X$,
  %   and Grothendieck developed axioms for embedded graphs in such a way
  %   that they coincided exactly with the category of Belyi maps.
  %   He called these graphs
  %   \emph{dessins d'enfants} or children's drawings.
  %   \par
  %   Even as a standalone theorem, Belyi's Theorem
  %   is a remarkable result in the mysterious way that it allows us to distinguish between
  %   algebraic and transcendental objects.
  %   However, the main interest in Belyi maps arises from Galois theory.
  %   The absolute Galois group of $\QQ$ acts on the set of Belyi maps
  %   via the defining equations.
  %   The induced action on the set of dessins
  %   \subsection[Inverse Galois theory]{Inverse Galois theory, Hurwitz families, and fields with few ramified primes}{
  %     \subsubsection{Inverse Galois theory}{
  %     }
  %     \subsubsection{Hurwitz families}{
  %     }
  %     \subsubsection{Number fields with few ramified primes}{
  %     }
  %   }
  %   \subsection[Dessins d'enfants]{Grothendieck's theory of dessins d'enfants}{
  %   }
  % }
  % abc in JonesWolfart
}
\chapter{Background on Belyi maps}{\label{chapter:backgroundbelyimaps}
  % \section{Complex manifolds, Riemann surfaces, and branched covers}{\label{sec:riemannsurfaces}
    % In this section we outline
    % basic results needed to define a ($2$-group) Belyi map
    % as a holomorphic map of Riemann surfaces.
    % For a more detailed discussion
    % see \cite{miranda, farkas}.
    % \begin{definition}
    %   \label{def:chart}
    %   A \defi{chart} on a topological space $X$
    %   is a homeomorphism
    %   $\phi\colon U\to V$
    %   where $U$ is an open subset of $X$
    %   and $V$ an open subset of $\CC$.
    %   We say the chart is \defi{centered}
    %   at $p\in U$
    %   if $\phi(p) = 0$.
    %   We say that $z = \phi(x)$ for $x\in U$
    %   is a \defi{local coordinate on $X$}.
    % \end{definition}
    % \begin{definition}
    %   \label{def:compatible}
    %   Let $\phi_1\colon U_1\to V_1$
    %   and $\phi_2\colon U_2\to V_2$
    %   be charts.
    %   $\phi_1$ and $\phi_2$ are
    %   \defi{compatible}
    %   if they are disjoint or
    %   the \defi{transition map}
    %   \[
    %     \phi_2\circ\phi_1^{-1}\colon
    %     \phi_1(U_1\cap U_2)\to
    %     \phi_2(U_1\cap U_2)
    %   \]
    %   is holomorphic.
    % \end{definition}
    % \begin{definition}
    %   \label{def:atlas}
    %   A \defi{complex atlas}
    %   on $X$ is a collection of compatible
    %   charts that cover $X$.
    % \end{definition}
    % \begin{definition}
    %   \label{def:equivatlas}
    %   Two atlases $\mathscr{A}_1$ and $\mathscr{A}_2$
    %   are \defi{equivalent}
    %   if every pair of charts
    %   $\phi_1, \phi_2$
    %   with
    %   $\phi_1\in\mathscr{A}_1$
    %   and
    %   $\phi_2\in\mathscr{A}_2$
    %   are compatible.
    % \end{definition}
    % \begin{definition}
    %   \label{def:complexstructure}
    %   A \defi{complex structure}
    %   on a topological space $X$
    %   is an equivalence class of atlases.
    %   % maximal atlas
    % \end{definition}
    % \begin{definition}
    %   \label{def:riemannsurface}
    %   A \defi{Riemann surface}
    %   is a second countable, connected,
    %   Hausdorff topological space $X$
    %   equipped with a complex structure.
    % \end{definition}
    % \begin{example}
    %   \label{exm:PP1}
    %   Let $\PP^1_\CC$ (or simply $\PP^1$)
    %   denote the set of $1$-dimensional subspaces of $\CC^2$
    %   which we can write as
    %   \[
    %     \{[z:w] : z,w\in\CC\text{ and }zw\ne 0\}
    %   \]
    %   where $[z:w]$ denotes the $\CC$-span of $(z,w)\in\CC^2$.
    %   Let $U_0 = \{[z:w]\in\PP^1 : z\neq 0\}$,
    %   $U_1 = \{[z:w]\in\PP^1 : w\neq 0\}$,
    %   define $\phi_0\colon U_0\to\CC$
    %   by $[z:w]\mapsto \frac{w}{z}$,
    %   and
    %   define $\phi_1\colon U_1\to\CC$
    %   by $[z:w]\mapsto \frac{z}{w}$.
    %   On $V\colonequals \phi_i(U_0\cap U_1) = \CC^\times$
    %   we have the holomorphic transition function
    %   $\phi_1\circ\phi_0^{-1}\colon V\to\CC$
    %   defined by $z\mapsto \frac{1}{z}$.
    %   The atlas consisting of these two charts $\phi_0,\phi_1$
    %   define a complex structure on $\PP^1$
    %   giving it the structure of a Riemann surface.
    % \end{example}
    % \begin{example}
    %   \label{exm:planecurves}
    %   \mm{plane curves and local complete intersections in $\PP^n$}
    % \end{example}
    % \begin{definition}
    %   \label{def:singularities}
    %   A function $f\colon X\to\CC$ is
    %   \defi{holomorphic
    %     (respectively has a removable singularity,
    %     has a pole, has an essential singularity)
    %   }
    %   at $p\in X$ if there exists a chart
    %   $\phi\colon U\to V$ such that
    %   $f\circ\phi^{-1}$ is
    %   holomorphic (respectively has a removable singularity, has a pole,
    %   has an essential singularity)
    %   at $\phi(p)$.
    %   $f$ is \defi{holomorphic on an open set $W\subseteq X$} if
    %   $f$ is holomorphic at all $p\in W$.
    %   $f$ is \defi{meromorphic} at $p\in X$
    %   if $f$ is holomorphic, has a removable singularity,
    %   or has a pole at $p$.
    %   $f$ is \defi{meromorphic on an open set $W\subseteq X$}
    %   if
    %   $f$ is meromorphic at all $p\in W$.
    % \end{definition}
    % % O_X(W) = {f holo on W}
    % \begin{definition}
    %   \label{def:laurentseries}
    %   Let $W$ be an open subset of $X$ and denote the set of
    %   meromorphic functions on $W$ by
    %   \[
    %     \mathcal{M}_X(W)
    %     =
    %     \{f\colon W\to\CC : f\text{ is meromorphic on $W$ }\}.
    %   \]
    %   Let $p\in W$ and
    %   let $f\in\mathcal{M}_X(W)$.
    %   Then there exists a chart $\phi$ on $W$
    %   with local coordinate $z$
    %   and
    %   $\phi(p) = z_0$
    %   such that $f\circ\phi^{-1}$ is meromorphic at $z_0$.
    %   Thus,
    %   we can write a \defi{Laurent series expansion
    %   for $f\circ\phi^{-1}$}
    %   in a neighborhood of $z_0$
    %   in the local coordinate $z$
    %   as
    %   \[
    %     (f\circ\phi^{-1})(z) = \sum_{n}c_n(z-z_0)^n.
    %   \]
    % \end{definition}
    % \begin{definition}
    %   \label{def:ord}
    %   The minimum $n$ such that $c_n\neq 0$
    %   in Definition \ref{def:laurentseries}
    %   is the \defi{order of $f$ at $p$}
    %   and denoted
    %   $\ord_p(f)$.
    % \end{definition}
    % \begin{definition}
    %   \label{ref:holomorphicmapofRS}
    %   $F\colon X\to Y$ is
    %   \defi{holomorphic}
    %   at $p\in X$
    %   if there exists
    %   charts
    %   $\phi_1\colon U_1\to\CC$
    %   $\phi_2\colon U_2\to\CC$
    %   with $p\in U_1$ and $F(p)\in U_2$
    %   such that
    %   $\phi_2\circ F\circ\phi_1^{-1}$
    %   is holomorphic at $\phi_1(p)$.
    %   Similarly,
    %   $F$ is \defi{holomorphic on an open set $W\subseteq X$}
    %   if it is holomorphic at every $p\in W$.
    % \end{definition}
    % \begin{definition}
    %   \label{def:RSautoiso}
    %   An \defi{isomorphism} of Riemann surfaces is
    %   a bijective holomorphic map $F\colon X\to Y$
    %   where $F^{-1}$ is holomorphic.
    %   An isomorphism from $X$ to $X$ is
    %   an \defi{automorphism}.
    % \end{definition}
    % \begin{example}
    %   \label{exm:PP1isoCCoo}
    %   $\PP^1$ defined in Example
    %   \ref{exm:PP1}
    %   is isomorphic to $\CC\cup \{\infty\}$
    %   the compactification of the complex plane
    %   via stereographic projection.
    % \end{example}
    % \begin{theorem}
    %   \label{thm:compactonto}
    %   Let $X$ be a compact Riemann surface
    %   and $F\colon X\to Y$
    %   a nonconstant holomorphic map.
    %   Then $Y$ is compact an $F$ is onto.
    % \end{theorem}
    % \begin{prop}
    %   \label{prop:discretefibers}
    %   Let $F\colon X\to Y$ be a nonconstant
    %   holomorphic map of Riemann surfaces.
    %   Then for every $y\in Y$,
    %   the fiber $F^{-1}(y)$ is a discrete subset of $X$.
    % \end{prop}
    % % meromorphic functions on X correspond to holomorphic maps to PP1 not identically oo
    % \begin{theorem}
    %   \label{thm:localnormalform}
    %   Let $F\colon X\to Y$ be a nonconstant holomorphic map.
    %   Let $p\in X$.
    %   Then there exists a positive integer $m$ such that
    %   for all charts
    %   $\phi_2$ centered at $F(p)$
    %   there exists a chart $\phi_1$ centered at $p$
    %   (let $z$ be the local coordinate)
    %   with $(\phi_2\circ F\circ\phi_1^{-1})(z) = z^m$.
    % \end{theorem}
    % \begin{definition}
    %   \label{def:multiplicity}
    %   Let $F\colon X\to Y$ be a holomorphic map
    %   of Riemann surfaces.
    %   The \defi{multiplicity}
    %   of $F$ at $p\in X$ is denoted
    %   $\mult_p(F)$ and is defined to be the unique
    %   integer $m$ from Theorem \ref{thm:localnormalform}
    %   such that there exist local coordinates about $p$
    %   and $F(p)$ so that
    %   $F$ can be written as $z\mapsto z^m$.
    % \end{definition}
    % \begin{definition}
    %   \label{def:ramificationRS}
    %   Let $F\colon X\to Y$ be a nonconstant holomorphic
    %   map of Riemann surfaces.
    %   $p\in X$ is a \defi{ramification point}
    %   of $F$
    %   if $\mult_p(F)\geq 2$.
    %   $y\in X$ is a \defi{branch point} of $F$
    %   if $F^{-1}(y)$ contains a ramification point.
    % \end{definition}
    % \begin{example}
    %   \label{exm:planecurve}
    %   \mm{plane curves, p.46, hyperelliptic curves, \ldots}
    % \end{example}
    % \begin{definition}
    %   \label{def:degreemapofRS}
    %   The \defi{degree}
    %   of a nonconstant holomorphic map
    %   $F\colon X\to Y$
    %   is
    %   \[
    %     \deg(F) \colonequals
    %     \sum_{p\in F^{-1}(y)}\mult_p(F)
    %   \]
    %   for any $y\in Y$.
    % \end{definition}
    % % sum_p ord_p(f) = 0
    % \begin{theorem}[Riemann-Hurwitz]
    %   \label{thm:riemannhurwitzforriemannsurfaces}
    %   Let $F\colon X\to Y$
    %   be a nonconstant holomorphic map of
    %   compact Riemann surfaces.
    %   Let $g(X), g(Y)$ be the topological genus
    %   of $X, Y$ respectively.
    %   Then
    %   \begin{equation}
    %     \label{eqn:riemannhurwitzforriemannsurfaces}
    %     2g(X)-2=
    %     \deg(F)(2g(Y)-2)+
    %     \sum_{p\in X}(\mult_p(F)-1).
    %   \end{equation}
    % \end{theorem}
    % % groups acting on Riemann surfaces
    % \begin{definition}
    %   \label{def:coveringspace}
    %   A \defi{covering space}
    %   of a real or complex manifold $V$
    %   is a continuous map
    %   $F\colon U\to V$ such that
    %   the following conditions hold:
    %   \begin{itemize}
    %     \item
    %       $F$ is surjective
    %     \item
    %       For all $v\in V$
    %       there exists a neighborhood
    %       $W$ of $v\in V$ such that
    %       $F^{-1}(W)$ consists of a disjoint
    %       union of open sets of $U$
    %       $\{U_\alpha\}_{\alpha\in I}$
    %       with $F|_{U_\alpha}\colon U_\alpha\to W$
    %       a homeomorphism.
    %   \end{itemize}
    %   The cardinality of $I$ is the \defi{degree}
    %   of the cover.
    % \end{definition}
    % \begin{definition}
    %   \label{def:coveringspaceiso}
    %   Two covering spaces $U_1\to V$
    %   and $U_2\to V$ are
    %   \defi{isomorphic}
    %   if there exists a homeomorphism
    %   $U_1\to U_2$
    %   making the diagram
    %   \begin{equation}
    %     \label{eqn:coveringspaceiso}
    %     \begin{tikzcd}
    %       U_1\arrow{dr}\arrow{rr}{}&&U_2\arrow{dl}\\
    %                      &V
    %     \end{tikzcd}
    %   \end{equation}
    %   commute.
    % \end{definition}
    % \begin{prop}
    %   \label{prop:universalcoveringspace}
    %   Given a real or complex manifold $V$,
    %   there exists a covering space
    %   $F_0\colon U_0\to V$
    %   such that $U_0$ is simply connected.
    %   $F_0$ is unique up to isomorphism
    %   and is universal in the following sense:
    %   If $F\colon U\to V$ is another cover of $V$,
    %   then there exists
    %   $G\colon U_0\to V$
    %   such that $F_0 = F\circ G$.
    % \end{prop}
    % Pick a basepoint $q\in V$
    % and let $\pi_1(V,q)$ denote
    % the \defi{fundamental group} of $V$
    % with loops based at $q$.
    % Then $\pi_1(V,q)$ acts on the cover
    % $F_0\colon U_0\to V$
    % via path lifting.
    % % iso classes of connected covers F:U->V correspond to conj classes of subgroups of pi_1(V,q)
    % We now restrict to the case of finite degree covers.
    % Let $F\colon U\to V$ be a degree $d$ cover
    % and consider the fiber of $q$,
    % $F^{-1}(q) = \{x_1,\dots,x_n\}$.
    % To a loop $\gamma$ on $V$ based at $q$,
    % we can lift $\gamma$ to $d$ paths
    % $\wt{\gamma}_1,\dots,\wt{\gamma}_d$
    % in $U$ where $\wt{\gamma}_i$
    % starts at $x_i$
    % and ends at $x_j$ for some $j$.
    % For each $i\in \{1,\ldots,d\}$ denote the terminal
    % point of $\wt{\gamma}_i$ by $x_{\sigma(i)}\in F^{-1}(q)$.
    % $\sigma$ defines
    % a \defi{monodromy representation}
    % \begin{equation}
    %   \label{eqn:monodromyrep}
    %   \rho\colon
    %   \pi_1(V,q)\to S_d.
    % \end{equation}
    % \begin{lemma}
    %   \label{lem:transitive}
    %   Let $\rho\colon\pi_1(V,q)\to S_d$
    %   be the monodromy representation
    %   of a
    %   finite degree cover $F\colon U\to V$
    %   with $U$ connected.
    %   Then the image of $\rho$
    %   is a transitive subgroup of $S_d$.
    % \end{lemma}
    % \begin{definition}
    %   \label{def:monodromyrepofholomorphicmap}
    %   Let $F\colon X\to Y$ be a nonconstant holomorphic
    %   map of Riemann surfaces.
    %   Let
    %   \begin{align*}
    %     V &\colonequals Y\sm \{\text{branch points of $F$}\}\\
    %     U &\colonequals X\sm \{\text{preimages of branch points of $F$}\}.
    %   \end{align*}
    %   Then $F|_U\colon U\to V$ is a covering space
    %   and induces a monodromy representation
    %   which we refer to as the
    %   \defi{monodromy representation of $F$}.
    % \end{definition}
    % \begin{definition}
    %   \label{def:branchedcoverofriemannsurface}
    %   A \defi{branched cover of Riemann surfaces} is a nonconstant holomorphic map
    %   of Riemann surfaces
    %   $\phi\colon X\to Y$ where $X$ is a compact connected Riemann surface.
    % \end{definition}
    % Let $Y$ be a compact connected Riemann surface,
    % let $B\subseteq Y$ be a finite set,
    % let $V\colonequals Y\sm B$,
    % and let $F\colon U\to V$ be a finite degree cover.
    % Then there is a unique complex structure on $U$
    % making $F$ holomorphic.
    % Let $b\in B$
    % and consider a neighborhood $W$ of $b$
    % small enough so that $W\sm \{b\}$ is homeomorphic to a punctured disk.
    % Then $F^{-1}(W\sm \{b\})$ is a finite disjoint union
    % of punctured disks $\{\wt{U}_j\}_j$.
    % Moreover,
    % by Theorem \ref{thm:localnormalform}
    % there are integers $m_j$ for each $j$ such that
    % \[
    %   F|_{\wt{U}_j}\colon\wt{U}_j\to W\sm \{b\}
    % \]
    % can be written as $z\mapsto z^{m_j}$ in local coordinates.
    % Extending this holomorphic map to the unpunctured disks
    % for every $b\in B$ yields the following correspondence.
    % \begin{prop}
    %   \label{prop:branchedcoverofriemannsurfaces}
    %   Let $Y$ be a compact Riemann surface,
    %   $B\subseteq Y$ a finite set,
    %   and $q\in Y\sm B$ a basepoint.
    %   Then there is a bijection of sets
    %   \begin{center}
    %     \begin{tikzpicture}[scale=1]
    %       \node at (-4,0) (maps)
    %       {
    %         $
    %           \left\{
    %             \begin{minipage}{30ex}
    %               isomorphism classes of
    %               degree $d$
    %               holomorphic maps
    %               $F\colon X\to Y$
    %               with branch points
    %               contained in $B$
    %             \end{minipage}
    %           \right\}
    %         $
    %       };
    %       \node at (4,0) (perms)
    %       {
    %         $
    %           \left\{
    %             \begin{minipage}{30ex}
    %               group homomorphisms
    %               $\rho\colon\pi_1(Y\sm B, q)\to S_d$
    %               with image a transitive subgroup
    %               up to conjugation in $S_d$
    %             \end{minipage}
    %           \right\}
    %         $
    %       };
    %       % \draw[<->] (maps) to node [above,rotate=-35] {$\sim$} node {} (perms);
    %       \draw[<->] (maps) to node [above] {$\sim$} node {} (perms);
    %     \end{tikzpicture}
    %   \end{center}
    % \end{prop}
    % If we let $Y = \PP^1$ in Proposition \ref{prop:branchedcoverofriemannsurfaces}
    % and $B = \{b_1,\ldots,b_n\}\subseteq\PP^1$ we obtain
    % the following correspondence:
    % \begin{center}
    %   \begin{tikzpicture}[scale=1]
    %   \label{branchedcoverscorrespondtopermutations}
    %     \node at (-4,0) (maps)
    %     {
    %       $
    %         \left\{
    %           \begin{minipage}{30ex}
    %             isomorphism classes of
    %             degree $d$
    %             holomorphic maps
    %             $F\colon X\to \PP^1$
    %             with branch points
    %             contained in $B$
    %           \end{minipage}
    %         \right\}
    %       $
    %     };
    %     \node at (4,0) (perms)
    %     {
    %       $
    %         \left\{
    %           \begin{minipage}{37ex}
    %             $n$-tuples of permutations
    %             $(\sigma_1,\ldots,\sigma_n)\in S_d^n$
    %             generating a transitive subgroup
    %             with $\sigma_1\cdots\sigma_n=1$
    %             up to simultaneous conjugation in $S_d$
    %           \end{minipage}
    %         \right\}
    %       $
    %     };
    %     % \draw[<->] (maps) to node [above,rotate=-35] {$\sim$} node {} (perms);
    %     \draw[<->] (maps) to node [above] {$\sim$} node {} (perms);
    %   \end{tikzpicture}
    % \end{center}
    % Moreover,
    % if $\sigma_i$ has cycle structure $(m_1,\dots,m_k)$,
    % then there are $k$ preimages
    % $u_1,\dots,u_k$ of $b_i$
    % in the cover $F\colon X\to\PP^1$
    % with $\mult_{u_j}(F) = m_j$ for all $j$.
    % \begin{definition}
    %   \label{def:belyimapriemannsurface}
    %   A \defi{Belyi map} is a nonconstant holomorphic map of
    %   compact connected Riemann surfaces
    %   $F\colon X\to\PP^1$ with no more than $3$ branch points.
    % \end{definition}
    % \begin{definition}\label{def:galoiscover}
    % \end{definition}
  % }
  % \section{Algebraic curves}{\label{sec:algebraiccurves}
    %Let $K$ be a field isomorphic to the complex numbers,
    %the real numbers, a number field, or a finite field.
    %Let $\Kal$ denote an algebraic closure of $K$.
    %For a detailed treatment see
    %\cite[Chapters 1-2]{silverman}.
    %%%%%%%%%%%%%AFFINE VARIETIES
    %\begin{definition}
    %  \label{def:affinespace}
    %  \defi{Affine $n$-space} over $K$ is the
    %  set of $n$-tuples of elements in $\Kal$
    %  and denoted $\AA^n(\Kal)$ or $\AA^n$.
    %  % set of K-rational points of AA^n
    %\end{definition}
    %We will denote \defi{points} in $\AA^n$ by $P$.
    %% Galois action on AA^n
    %Let $\Kal[x_1,\dots,x_n]$ be the $n$-variable
    %polynomial ring over $\Kal$.
    %To each ideal $I\trianglelefteq\Kal[x_1,\dots,x_n]$
    %we associate the following subset of $\AA^n$.
    %\begin{equation}
    %  \label{eqn:algebraicset}
    %  V(I)\colonequals
    %  \{P\in\AA^n : f(P) = 0 \text{ for all } f\in I\}
    %\end{equation}
    %\begin{definition}
    %  \label{def:algebraicset}
    %  Subsets of $\AA^n$
    %  of the form $V(I)$ for some ideal $I$
    %  as in Equation \ref{eqn:algebraicset}
    %  are called
    %  \defi{affine algebraic sets}.
    %\end{definition}
    %Let $V$ be an affine algebraic set.
    %To such a set we can associate
    %the following ideal of $\Kal[x_1,\dots,x_n]$.
    %\begin{equation}
    %  \label{eqn:idealofalgebraicset}
    %  I(V)\colonequals
    %  \{f\in\Kal[x_1,\dots,x_n]: f(P)=0 \text{ for all }f\in I\}.
    %\end{equation}
    %\begin{definition}
    %  \label{def:definedoverK}
    %  Let $\AA^n(K)$ denote the set of
    %  \defi{$K$-rational points} of $\AA^n$ defined by
    %  \[
    %    \AA^n(K)\colonequals
    %    \{(x_1,\dots,x_n)\in\AA^n : x_i\in K\}.
    %  \]
    %  Let $V$ be an algebraic set.
    %  We say $V$ is
    %  \defi{defined over $K$}
    %  if $I(V)$ can be generated by
    %  polynomials in $K[x_1,\dots,x_n]$.
    %  For such $V$ we can define the
    %  \defi{$K$-rational points of $V$} by
    %  \[
    %    V(K)\colonequals
    %    V\cap\AA^n(K).
    %  \]
    %\end{definition}
    %Let $G_K = \Gal(\Kal/K)$.
    %Another way to characterize $V(K)$ is
    %the points fixed under the action of $G_K$:
    %\begin{equation}
    %  \label{eqn:galoisactiononV}
    %  V(K) =
    %  \{P\in V : P^\sigma = P\text{ for all }\sigma\in G_K\}
    %\end{equation}
    %\begin{definition}
    %  \label{def:affinevariety}
    %  An affine algebraic set $V$ is called
    %  an \defi{affine variety}
    %  if $I(V)\trianglelefteq\Kal[x_1,\dots,x_n]$ is a prime ideal.
    %\end{definition}
    %If an affine algebraic set $V$ is defined over $K$,
    %then $V$ is an affine variety if
    %$I(V)\trianglelefteq K[x_1,\dots,x_n]$
    %is a prime ideal.
    %\begin{definition}
    %  \label{def:coordinateringfunctionfield}
    %  Let $V$ be an affine variety defined over $K$.
    %  We define the
    %  \defi{affine coordinate ring}
    %  by
    %  \[
    %    K[V]
    %    \colonequals
    %    \frac{
    %      K[x_1,\dots,x_n]
    %    }{
    %    I(V)
    %    }
    %  \]
    %  and the
    %  \defi{function field of $V$}
    %  by the field of fractions of $K[V]$
    %  denoted $K(V)$.
    %  We can similarly define this construction
    %  for $\Kal[V]$ and $\Kal(V)$.
    %\end{definition}
    %\begin{definition}
    %  \label{def:dimension}
    %  The \defi{dimension}
    %  of an affine variety $V$
    %  is the transcendence degree of the field extension
    %  $\Kal(V)$ over $\Kal$.
    %\end{definition}
    %% \begin{definition}
    %%   \label{def:nonsingular}
    %%   Let $V$ be an affine variety of dimension $d$,
    %%   let $P\in V$,
    %%   and $I(V)$ generated by
    %%   $\{f_1,\dots,f_m\}$.
    %%   We say $V$ is \defi{nonsingular at $P$}
    %%   if the $m\times n$ matrix of partial
    %%   derivatives
    %%   whose $(i,j)$-entry is
    %%   \[
    %%     \frac{\partial f_i}{\partial x_j}
    %%   \]
    %%   has rank $n-d$.
    %% \end{definition}
    %\begin{definition}
    %  \label{def:nonsingular}
    %  Let $V$ be a variety of dimension $d$
    %  and $P\in V$.
    %  Consider the maximal ideal
    %  \[
    %    M_P\colonequals
    %    \{f\in\Kal[x_1,\dots,x_n]:f(P)=0\}.
    %  \]
    %  The quotient $M_P/M_P^2$ is a finite dimensional
    %  vector space over $\Kal$.
    %  We say $P$ is \defi{nonsingular}
    %  if the dimension of $M_P/M_P^2$ as
    %  a vector space over $\Kal$
    %  is equal to $d$.
    %\end{definition}
    %\begin{definition}
    %  \label{def:localringatP}
    %  Let $V$ be an affine variety
    %  and $P\in V$.
    %  The \defi{ring of regular functions on $V$ at $P$}
    %  is defined to be the localization
    %  of $\Kal[V]$ at the maximal ideal $M_P$
    %  (denoted $\Kal[V]_P$).
    %  More explicitly we have
    %  \[
    %    \Kal[V]_P
    %    \colonequals
    %    \Kal[V]_{M_P}
    %    =
    %    \{f/g\in\Kal[V]:g(P)\neq 0\}
    %  \]
    %  so that
    %  the elements of
    %  $\Kal[V]_P$
    %  are well-defined
    %  as functions on $V$.
    %\end{definition}
    %%%%%%%%%%%%%PROJECTIVE VARIETIES
    %\begin{definition}
    %  \label{def:projectivespace}
    %  \defi{Projective $n$-space}
    %  over $K$ is denoted by $\PP^n(\Kal)$
    %  or $\PP^n$
    %  and is defined to be
    %  \[
    %    \PP^n\colonequals
    %    \{(x_0,\dots,x_{n})\in\AA^{n+1}:\text{ not all $x_i=0$ }\}/\sim
    %  \]
    %  where
    %  $(x_0,\dots,x_n)\sim(x_0',\dots,x_n')$
    %  if there exists $\lambda\in(\Kal)^\times$ with
    %  $x_i=\lambda y_i$ for all $i\in \{0,\dots,n\}$.
    %  The equivalence class of $(x_0,\dots,x_n)\in\AA^{n+1}$
    %  with respect to $\sim$ is denoted by
    %  $[x_0,\dots,x_n]$ or $[x_0:\cdots:x_n]$.
    %  We call these $x_i$
    %  \defi{homogeneous coordinates}
    %  of the point in $\PP^n$.
    %  As in the affine case,
    %  we define the
    %  \defi{$K$-rational points} of $\PP^n$
    %  to be
    %  \[
    %    \PP^n(K)\colonequals
    %    \{[x_0,\dots,x_n]\in\PP^n:x_i\in K \text{ for all } i\}.
    %  \]
    %\end{definition}
    %\begin{definition}
    %  \label{def:minimalfieldofdefinitionPPn}
    %  Let $P\in\PP^n$ with homogeneous coordinates
    %  $[x_0,\dots,x_n]$.
    %  The \defi{minimal field of definition of $P$ over $K$}
    %  is
    %  \[
    %    K(P)\colonequals
    %    K(x_0/x_i,\dots,x_n/x_i)
    %  \]
    %  for any $i\in \{0,\dots,n\}$.
    %\end{definition}
    %$\PP^n(K)$ is the set of $P\in\PP^n$
    %fixed by the action of $G_K$.
    %On the other hand,
    %$K(P)$ is the fixed field of the
    %subgroup
    %$\{\sigma\in G_K: P = P^\sigma\}$.
    %\begin{definition}
    %  \label{def:homogeneousideal}
    %  An ideal $I\trianglelefteq\Kal[x_0,\dots,x_n]$
    %  is \defi{homogeneous}
    %  if it can be generated by homogeneous polynomials.
    %  To a homogeneous ideal $I$ we can associate a subset of
    %  $\PP^n$ as follows.
    %  \[
    %    V(I)\colonequals
    %    \{P\in\PP^n : f(P)=0\text{ for all homogeneous $f\in\Kal[x_0,\dots,x_n]$ }\}
    %  \]
    %  A \defi{projective algebraic set}
    %  is a subset of $\PP^n$ which is $V(I)$
    %  for some homogeneous ideal $I$.
    %\end{definition}
    %To any projective algebraic set $V$,
    %we can associate a homogeneous ideal $I(V)$ defined by
    %\begin{equation}
    %  \label{eqn:projectivealgebraicsetideal}
    %  I(V)
    %  \colonequals
    %  \{f\in\Kal[x_0,\dots,x_n]:f \text{ is homogeneous and }f(P) = 0\text{ for all $P\in V$ }\}
    %\end{equation}
    %% \begin{theorem}\label{thm:curvesandfunctionfields}
    %%   \mm{correspondence curves and function fields}
    %% \end{theorem}
    %% \begin{definition}
    %%   Let $K\subseteq\CC$ be a field.
    %%   A \defi{branched cover of algebraic curves over $K$}
    %%   is a finite map of curves
    %%   $\phi\colon X\to\PP^1$ defined over $K$.
    %% \end{definition}
    %% \mm{enough to define good curves and function fields}
  % }
  % \section{Riemann's existence theorem}{\label{sec:riemannsexistence}
    % Riemann surfaces are defined in Section \ref{sec:riemannsurfaces}.
    % Algebraic curves are defined in Section \ref{sec:algebraiccurves}.
    % Here in Section \ref{sec:riemannsexistence}
    % we establish the connection between these objects over the complex numbers.
    % \par
    % Let $X$ be an algebraic curve over $\CC$.
    % Let $\CC(t)$ denote the function field of $\PP^1$.
    % By Theorem \ref{thm:curvesandfunctionfields},
    % $X$ corresponds to a finite extension $L\colonequals\CC(X)$
    % over $\CC(t)$.
    % Let $\alpha$ be a primitive element of $L/\CC(t)$.
    % Then there exists a polynomial
    % \begin{equation}
    %   \label{eqn:minpoly}
    %   f(x,t) = a_0(t)+a_1(t)x+a_2(t)x^2+\cdots+a_n(t)x^n\in\CC(t)[x]
    % \end{equation}
    % where $f(\alpha,t) = 0$ and (after possibly clearing denominators)
    % $a_i(t) \in\CC[t]$.
    % The polynomial $f$ in Equation \ref{eqn:minpoly}
    % defines a Riemann surface $X'$
    % as a branched cover of $\PP^1$
    % with branch points
    % \[
    %   S \colonequals\left\{t_0\in\CC : f(x,t_0) \text{ has repeated roots }\right\}.
    % \]
    % Here $x$ can be viewed as a meromorphic function on $X'$
    % and we can identify the field of meromorphic functions on $X'$
    % with $L$.
    % This explains how we obtain a Riemann surface from an algebraic curve.
    % \par
    % Suppose instead we start with a compact Riemann surface $X$.
    % Can we reverse the above process to construct an algebraic curve?
    % The crucial part of this process is proving that
    % there exists a meromorphic function on $X$ that realizes $X$
    % as a branched cover of $\PP^1$
    % (see Theorem \ref{thm:riemannsexistence} below).
    % Given the existence of such a function,
    % the field of meromorphic functions on $X$ is then realized
    % as a finite extension of the meromorphic functions on $\PP^1$.
    % Finally,
    % by Theorem \ref{thm:curvesandfunctionfields},
    % this corresponds to an algebraic curve.
    % The existence of such a function
    % is given by
    % Theorem \ref{thm:riemannsexistence}
    % (Riemann's existence theorem).
    % \begin{theorem}\label{thm:riemannsexistence}
    %   Let $X$ be a compact Riemann surface.
    %   Then there exists a meromorphic function on $X$
    %   that separates points.
    %   That is, for any set of distinct points
    %   $\{x_1,\dots,x_n\}\subset X$
    %   and any set of distinct points
    %   $\{t_1,\dots,t_n\}\subset\PP^1$
    %   there exists a meromorphic function $f$
    %   on $X$ such that $f(x_i) = t_i$ for all $i$.
    % \end{theorem}
    % \mm{todo: more details...other formulations}
  % }
  % \section{Belyi's theorem}{\label{sec:belyistheorem}
    % In Sections
    % \ref{sec:riemannsufraces}, \ref{sec:algebraiccurves},
    % and \ref{sec:riemannsexistence}
    % we established the equivalence between
    % compact Riemann surfaces and algebraic curves over $\CC$.
    % This was done, in part,
    % using branched covers.
    % It turns out that branched covers are the key
    % to descending from the transcendental world to
    % the number-theoretic world in the following sense.
    % \begin{theorem}[Belyi's theorem \cite{belyi}]\label{thm:belyistheorem}
    %   An algebraic curve $X$ over $\CC$ can be defined over a number field
    %   if and only if there exists a branched cover
    %   $\phi\colon X\to\PP^1$ unramified outside
    %   $\{0,1,\infty\}$.
    % \end{theorem}
    % These remarkable covers are the main focus of this work.
  % }
  \section{Belyi maps and Galois Belyi maps}{\label{sec:belyimaps}
    We now set up the framework to discuss
    the main mathematical objects of interest in this work.
    \begin{definition}\label{def:belyimap}
      A
      \defi{Belyi map}
      is a morphism
      $\phi\colon X \to \PP^1$
      of
      smooth projective
      algebraic curves over $\CC$
      that is
      unramified outside
      $\{0,1,\infty\}$.
      We define the \defi{genus} of $\phi$
      to be the genus of $X$.
    \end{definition}
    \begin{definition}\label{def:belyiiso}
      Two Belyi maps
      $\phi\colon X\to\PP^1$ and
      $\phi'\colon X'\to\PP^1$
      are \defi{isomorphic}
      if there exists an isomorphism
      of curves from
      $X$ to $X'$
      such that the diagram
      \begin{equation}
        \label{eqn:belyiiso}
        \begin{tikzcd}
          X\arrow{rr}{\sim}\arrow{dr}[swap]{\phi}
          &
          &X'\arrow{dl}{\phi'}\\
          &\PP^1
        \end{tikzcd}
      \end{equation}
      commutes.
      If instead we only insist that
      the isomorphism
      makes a diagram
      \begin{equation}
        \label{eqn:}
        \begin{tikzcd}
          X\arrow{r}{\sim}\arrow{d}[swap]{\phi}&X'\arrow{d}{\phi'}\\
          \PP^1\arrow{r}{\sim}[swap]{\beta}&\PP^1
        \end{tikzcd}
      \end{equation}
      commute,
      with the bottom map
      $\beta$ satisfying
      $\beta(\{0,1,\infty\}) =
      \{0,1,\infty\}$,
      then we say that $\phi$ and $\phi'$
      are \defi{lax isomorphic}.
      Note that a lax isomorphism
      between $\phi$ and $\phi'$ is an
      isomorphism if and only if
      $\beta$ fixes the three points
      $\{0,1,\infty\}$.
    \end{definition}
    \begin{definition}\label{def:ramificationtype}
      The triple of partitions
      $(\lambda_0,\lambda_1,\lambda_\infty)$
      encoding the ramification
      above
      $0$, $1$, and $\infty$
      is called
      the \defi{ramification type} of $\phi$.
    \end{definition}
    Let $\phi\colon X\to\PP^1$ be a
    Belyi map of degree $d$.
    Once we label the sheets of the cover
    and pick a basepoint
    $\star\not\in\{0,1,\infty\}$,
    we obtain a homomorphism
    \begin{equation}\label{eqn:monodromy}
      h\colon\pi_1(\PP^1\setminus\{0,1,\infty\},
      \star)\to S_d
    \end{equation}
    by lifting paths around the
    branch points of $\phi$.
    \begin{definition}\label{def:monodromy}
      The image of $h$ in \eqref{eqn:monodromy}
      is the \defi{monodromy group} of $\phi$,
      denoted $\Mon(\phi)$.
    \end{definition}
    \begin{definition}
      \label{def:definedover}
      A Belyi map
      $\phi\colon X\to\PP^1$
      is \defi{defined over}
      a number field $K\subseteq\CC$
      if the defining equations
      of $\phi$ and $X$
      can be described by
      polynomial expressions
      with coefficients in $K$.
      We say that $K$ is a
      \defi{field of definition}
      for $\phi$.
    \end{definition}
    \begin{theorem}
      [Belyi's theorem \cite{belyi}]
      \label{thm:belyistheorem}
      An algebraic curve $X$ over $\CC$
      can be defined over a number field
      if and only if $X$
      admits a Belyi map.
    \end{theorem}
    Belyi's theorem implies that
    every Belyi map can be described by
    a morphism $\phi\colon X\to\PP^1$
    of algebraic curves
    defined over a number field $K\subseteq\CC$
    (instead of over $\CC$).
    Since maps of curves
    correspond to function field
    extensions,
    we can consider a Belyi map
    $\phi\colon X\to\PP^1$
    (defined over $K$)
    as equivalently given by
    an extension of function fields
    $K(X)\supseteq K(\PP^1)$.
    Note that
    $K(\PP^1)$ is isomorphic the field of
    rational functions
    (referred to as the
    \defi{rational function field} of $K$)
    in one variable,
    say $K(x)$,
    and $K(X)$
    can be written as $K(x)(\alpha)$
    for some primitive element $\alpha$.
    \par
    The degree of a Belyi map
    in this setting
    is the degree of the corresponding
    function field extension $K(X)$
    over the rational function field.
    Ramification in this setting corresponds
    to the factorization of ideals
    $(x)$, $(x-1)$, and $(1/x)$
    in maximal orders of $K(X)$.
    The monodromy group in this setting
    corresponds to field automorphisms
    of the Galois closure of $K(X)$
    fixing $K(x)$.
    \par
    Let $\Kal$
    denote an algebraic closure of $K$ in $\CC$.
    \begin{definition}\label{def:galoisbelyi}
      A Belyi map $\phi\colon X\to\PP^1$
      defined over $K$
      is \defi{(geometrically) Galois}
      if the corresponding
      function field extension
      $\Kal(X)$ is a Galois field extension
      over the rational function field
      $\Kal(x)=\Kal(\PP^1)$.
    \end{definition}
    When $\phi$ is Galois,
    the ramification type of
    $\phi$ can be
    more simply encoded by
    a triple of integers
    $(a,b,c)\in\ZZ_{\geq 1}^3$.
    When $\phi$ is a Galois Belyi map,
    we can identify $\Mon(\phi)$
    in Definition \ref{def:monodromy}
    as the Galois group
    $\Gal(\Kal(X)\,|\,\Kal(\PP^1))$.
    For this reason,
    we may also write $\Gal(\phi)$
    to denote $\Mon(\phi)$ when $\phi$
    is Galois.
    \par
    We can now define the main object of
    interest in this thesis.
    \begin{definition}
      \label{def:2gbm}
      A \defi{$2$-group Belyi map}
      is a Galois Belyi map
      of degree $d$
      with monodromy group a
      $2$-group of order $d$.
    \end{definition}
    % Necessarily,
    For a Galois Belyi map
    $\Mon(\phi)=\Gal(\phi)\subseteq S_d$
    is the regular representation.
    % \begin{prop}\label{prop:galoiscover}
      %   Let $\phi\colon X\to\PP^1$ be a Galois Belyi map
      %   and let $\CC(X)$ be the function field of $X$.
      %   Then the field extension
      %   $\CC(X)/\CC(\PP^1)$ is Galois.
    % \end{prop}
    % \begin{proof}
    % \end{proof}
    % \begin{definition}\label{def:Gbelyi}
    %   A \defi{$G$-Galois Belyi map}
    %   is a Galois Belyi map
    %   $\phi\colon X\to\PP^1$
    %   with monodromy group $G$
    %   equipped with an isomorphism
    %   \[
    %     i\colon G\stackrel{\sim}{\to}\Mon(\phi)\leq\Aut(X).
    %   \]
    %   An \defi{isomorphism of $G$-Galois Belyi maps}
    %   $(\phi\colon X\to\PP^1, i\colon G\to\Mon(\phi))$
    %   and
    %   $(\phi'\colon X'\to\PP^1, i'\colon G\to\Mon(\phi)$
    %   is an isomorphism $h\colon X\stackrel{\sim}{\to} X'$ such that
    %   for all $g\in G$
    %   the diagram in
    %   Figure \ref{fig:Gbelyiiso} commutes.
    %   \begin{figure}[ht]
    %     \begin{center}
    %       \begin{tikzcd}
    %         X\arrow{rr}{h}\arrow{d}[swap]{i(g)}&&X'\arrow{d}{i'(g)}\\
    %         X\arrow{rr}{h}\arrow{dr}[swap]{\phi}&&X'\arrow{dl}{\phi'}\\
    %                                                &\PP^1
    %       \end{tikzcd}
    %     \end{center}
    %     \caption{$G$-Galois Belyi map isomorphism}
    %     \label{fig:Gbelyiiso}
    %   \end{figure}
    %   \begin{prop}\label{prop:Gauts}
    %     \mm{\cite[Prop. 3.6 ish]{triangles}}
    %   \end{prop}
    % \end{definition}
  }
  \section{Permutation triples and passports}{\label{sec:passports}
    % A \defi{(nice) curve} over $K$ is
    % a smooth, projective, geometrically connected (irreducible) scheme of finite
    % type over $K$ that is pure of dimension $1$.
    % After extension to $\CC$, a curve
    % may be thought of as a compact, connected Riemann surface.  A \defi{Belyi map}
    % over $K$ is a finite morphism $\phi\colon X \to \PP^1$ over $K$ that is
    % unramified outside $\{0,1,\infty\}$; we will sometimes write $(X,\phi)$ when we
    % want to pay special attention to the source curve $X$.  Two Belyi maps
    % $\phi,\phi'$ are \defi{isomorphic} if there is an isomorphism $\iota\colon X
    % \xrightarrow{\sim} X'$ of curves such that $\phi'\iota=\phi$.
    % Let $\phi\colon X\to\PP^1$ be a Belyi map over $\overline{\QQ}$ of degree
    % $d \in \ZZ_{\geq 1}$.
    % The \defi{monodromy group} of $\phi$ is the Galois group
    % $\Mon(\phi) \colonequals \Gal(\CC(X)\,|\,\CC(\PP^1)) \leq S_d$ of the
    % corresponding extension of function fields (understood as the action of the
    % automorphism group of the normal closure); the group $\Mon(\phi)$ may also be
    % obtained by lifting paths around $0,1,\infty$ to $X$.
    \begin{definition}
      \label{def:permutationtriple}
      A \defi{permutation triple} of degree
      $d \in \ZZ_{\geq 1}$ is a tuple
      $\sigma =
      (\sigma_0,\sigma_1,\sigma_\infty)\in S_d^3$
      such that
      $\sigma_\infty \sigma_1 \sigma_0 = 1$.
      A permutation triple is
      \defi{transitive} if the subgroup
      $\langle \sigma \rangle \leq S_d$
      generated by $\sigma$ is transitive.
      We say
      that two permutation triples
      $\sigma,\sigma'$ are
      \defi{simultaneously conjugate}
      if there exists $\tau\in S_d$ such that
      \begin{equation}\label{eqn:simconj}
        \sigma^\tau \colonequals
        (\tau^{-1}\sigma_0\tau, \tau^{-1}\sigma_1\tau, \tau^{-1}\sigma_\infty\tau)
        = \left(\sigma'_0,\sigma'_1,\sigma'_\infty\right)
        = \sigma'.
      \end{equation}
      An \defi{automorphism} of
      a permutation triple $\sigma$
      is an element of $S_d$ that
      simultaneously conjugates
      $\sigma$ to itself, i.e.,
      $\Aut(\sigma)=
      C_{S_d}(\langle \sigma \rangle)$,
      the centralizer inside $S_d$.
    \end{definition}
    \begin{lemma}
      \label{lem:simulisom}
      The set of transitive permutation triples
      of degree $d$ up to simultaneous
      conjugation is in bijection with
      the set of Belyi maps of degree $d$ up to
      isomorphism.
    \end{lemma}
    \begin{proof}
      The correspondence is via monodromy
      \cite[Lemma 1.1]{KMSV}; in particular,
      the monodromy group of a Belyi map
      is (conjugate in $S_d$ to) the group
      generated by~$\sigma$.
    \end{proof}
    The group $G_\QQ\colonequals\Gal(\QQal\,|\,\QQ)$
    acts on Belyi maps by acting on the
    coefficients of a set of defining equations;
    under the bijection of Lemma
    \ref{lem:simulisom},
    it thereby acts on the set of
    transitive permutation
    triples, but this action is rather mysterious.
    We can cut this action down to size
    by identifying some basic invariants, as
    follows.
    \begin{definition}
      \label{def:passport}
      A \defi{passport} consists of the data
      $\mathcal{P}=(g,G,\lambda)$
      where $g \geq 0$ is an integer,
      $G \leq S_d$ is a transitive subgroup,
      and
      $\lambda=
      (\lambda_0,\lambda_1,\lambda_\infty)$
      is a tuple of partitions
      $\lambda_s$ of $d$ for $s=0,1,\infty$.
      These partitions will be also be
      thought of as a tuple of conjugacy classes
      $C=(C_0,C_1,C_\infty)$ by cycle
      type,
      so we will also write passports as $(g,G,C)$.
      Two passports $(g,G,C)$ and $(g',G',C')$
      are \defi{equal}
      if $g=g'$, $C=C'$, and
      $G$ is conjugate to $G'$.
    \end{definition}
    \begin{definition}
      \label{def:passportofbelyimap}
      The \defi{passport} of a
      Belyi map $\phi\colon X \to \PP^1$ is
      \begin{equation}
        \label{eqn:passportofbelyimap}
        \mathcal{P}(\phi)=
        (g(X),\Mon(\phi),
        (\lambda_0,\lambda_1,\lambda_\infty))
      \end{equation}
      where $g(X)$ is the genus of $X$ and
      $\lambda_s$ is the partition of $d$ obtained by the ramification degrees above
      $s=0,1,\infty$, respectively.
    \end{definition}
    \begin{definition}
      \label{def:passportofpermutationtriple}
      The \defi{passport} of a transitive
      permutation triple $\sigma$ is
      \begin{equation}
        \label{eqn:passportofpermutationtriple}
        \mathcal{P}(\sigma)=
        (g(\sigma),\langle \sigma \rangle,
        \lambda(\sigma))
      \end{equation}
      where (following Riemann--Hurwitz)
      \begin{equation}
        \label{eqn:riemannhurwitzfortriples}
        g(\sigma) \colonequals
        1-d+(e(\sigma_0)+e(\sigma_1)+
        e(\sigma_\infty))/2
      \end{equation}
      and $e$ is the index of a
      permutation
      ($d$ minus the number of orbits), and
      $\lambda(\sigma)$ is the cycle
      type of $\sigma_s$ for $s=0,1,\infty$.
    \end{definition}
    \begin{definition}
      \label{def:passportsize}
      The
      \defi{size} of a passport $\mathcal{P}$
      is the number of simultaneous conjugacy
      classes as in \eqref{eqn:simconj} of
      (necessarily transitive) permutation
      triples $\sigma$ with passport $\mathcal{P}$.
    \end{definition}
    The action of $G_\QQ$ on
    Belyi maps preserves passports.
    Therefore, after computing equations
    for all Belyi maps with a given
    passport, we can try to
    identify the Galois orbits of this action.
    \begin{definition}
      \label{def:reduciblepassport}
      We say a passport is
      \defi{irreducible} if it has one
      $G_\QQ$-orbit and
      \defi{reducible} otherwise.
    \end{definition}
    We finish this section
    with an observation about ramification
    and the Riemann-Hurwitz formula
    in the case where we have a
    Galois Belyi map.
    \begin{lemma}\label{lem:regular}
      Let $\sigma$ be a degree $d$
      permutation triple corresponding to
      $\phi\colon X\to\PP^1$,
      a Galois Belyi map with monodromy group $G$,
      and let
      $a,b,c$ be the orders of
      $\sigma_0,\sigma_1,\sigma_\infty$
      respectively.
      Then
      $\sigma_0$
      consists of
      $d/a$
      many $a$-cycles,
      $\sigma_1$ consists of
      $d/b$ many $b$-cycles,
      and $\sigma_\infty$
      consists of $d/c$
      many $c$-cycles.
      % $m_s$ be the order of
      % $\sigma_s$ for $s\in\{0,1,\infty\}$.
      % Then $\sigma_s$ consists of
      % $d/m_s$ many $m_s$-cycles.
      In particular,
      for a $2$-group Belyi map,
      $a,b,c,$ and $\#G$ are powers of $2$.
    \end{lemma}
    \begin{proof}
      This follows from the condition that
      the field extension $K(X)$ is Galois
      over the rational function field $K(x)$.
      The Galois action is transitive on
      primes above any prime of $K(x)$
      and in particular implies that the
      ramified primes all have the same ramification
      index if they lie above the same prime
      of $K(x)$.
    \end{proof}
    Lemma \ref{lem:regular}
    allows for a simplified version of the
    Riemann-Hurwitz formula for Galois Belyi maps.
    \begin{theorem}
      [Riemann-Hurwitz]
      \label{thm:riemannhurwitzgalois}
      Let $\sigma$ be a degree $d$
      permutation triple corresponding to
      $\phi\colon X\to\PP^1$, a
      Galois Belyi map with monodromy group $G$.
      Let $a,b,c$ be the orders of
      $\sigma_0,\sigma_1,\sigma_\infty$
      respectively.
      Then
      \begin{equation}
        \label{eqn:riemannhurwitzgalois}
        g(X) = 1+\frac{\#G}{2}
        \left(1-\frac{1}{a}-\frac{1}{b}-
        \frac{1}{c}\right).
      \end{equation}
    \end{theorem}
  }
  \section{Triangle groups}{\label{sec:trianglegroups}
    \begin{definition}
      \label{def:geometrytype}
      Let $(a,b,c)\in\ZZ_{\geq 1}^3$.
      If $1\in\{a,b,c\}$, then we say the triple is \defi{degenerate}.
      Otherwise, we call the triple
      \defi{spherical},
      \defi{Euclidean},
      or \defi{hyperbolic}
      according to whether the value of
      \begin{equation}
        \label{eqn:eulerchar}
        \chi(a,b,c) = \frac{1}{a}+\frac{1}{b}
        +\frac{1}{c}-1
      \end{equation}
      is positive, zero, or negative.
      We call this the \defi{geometry type}
      of the triple.
      We associate the \defi{geometry}
      \begin{equation}
        \label{eqn:geometrytype}
        H=
        \begin{cases}
          \PP^1,&\text{if }\chi(a,b,c)>0\\
          \CC,&\text{if }\chi(a,b,c)=0\\
          \mathfrak{H},&\text{if }\chi(a,b,c)<0
        \end{cases}
      \end{equation}
      where $\mathfrak{H}$ denotes the complex upper half-plane.
    \end{definition}
    \begin{definition}
      \label{def:trianglegroup}
      For each triple $(a,b,c)$ in Definition \ref{def:geometrytype}
      we define the \defi{triangle group}
      \begin{equation}
        \label{eqn:trianglegroup}
        \Delta(a,b,c)
        =
        \langle
        \delta_a, \delta_b, \delta_c \,|\,
        \delta_a^a=\delta_b^b=\delta_c^c=\delta_c\delta_b\delta_a=1
        \rangle
      \end{equation}
      The \defi{geometry type}
      of a triangle group $\Delta(a,b,c)$
      is the geometry type of the triple $(a,b,c)$.
    \end{definition}
    \begin{definition}\label{def:geometrytypeofbelyimap}
      The \defi{geometry type} of a Galois Belyi map
      with ramification type $(a,b,c)$
      is the geometry type of $(a,b,c)$.
    \end{definition}
    \begin{definition}\label{def:geometrytypeofpermutationtriple}
      Let $\sigma=(\sigma_0,\sigma_1,\sigma_\infty)$ be a transitive permutation triple.
      Let $a,b,c$ be the orders of
      $\sigma_0,\sigma_1,\sigma_\infty$ respectively.
      The \defi{geometry type} of $\sigma$
      is the geometry type of $(a,b,c)$.
    \end{definition}
    The connection between Belyi maps
    and triangle groups
    of various geometry types is explained by Lemma
    \ref{lem:belyimapsandtrianglegroups}.
    \begin{lemma}
      \label{lem:belyimapsandtrianglegroups}
      The set of isomorphism classes of degree-$d$
      Belyi maps with ramification type $(a,b,c)$
      is in bijection with the set of
      index $d$ subgroups
      $\Gamma\leq\Delta(a,b,c)$
      up to conjugation.
    \end{lemma}
    For a detailed explanation of this relationship
    see the first part of Section $1$
    in \cite{KMSV}.
  }
  % \section{Background results on Belyi maps}{\label{sec:backgroundresults}
    % \begin{theorem}\label{thm:bigbijection}
    %   \mm{big bijection}
    % \end{theorem}
    % \begin{prop}\label{prop:galoisaction}
    %   \mm{Galois action on Belyi maps}
    % \end{prop}
    % \begin{prop}\label{prop:galoiscorrespondence}
    %   Galois correspondence of Belyi maps
    % \end{prop}
    % \begin{proof}
    % \end{proof}
    % \mm{\cite[1.6, 1.7]{SV}}
  % }
  \section{Fields of moduli and fields of definition}{\label{sec:fieldsofmodulifieldsofdefinition}
    We now discuss the background
    material necessary to describe
    the results in
    Chapter \ref{chapter:fieldsofdefinition}.
    This section aims
    to define a canonical number field
    associated to a Belyi map
    which is well-defined on isomorphism classes,
    bound the degree of this number field,
    and discuss when a Belyi map can be defined
    over this field.
    To start let $\Aut(\CC)$ denote the
    field automorphisms of $\CC$.
    \begin{definition}
      \label{def:fieldofmoduli}
      Let $X$ be an algebraic curve over $\CC$.
      The \defi{field of moduli} of $X$,
      denoted $M(X)$,
      is the fixed field of $\CC$ under
      the subgroup of field automorphisms
      \begin{equation}
        \label{eqn:fixedfield}
        \{\tau\in\Aut(\CC) : \tau(X)\simeq X\}
      \end{equation}
      where $\tau\in\Aut(\CC)$ acts on
      a set of defining equations of $X$.
    \end{definition}
    \begin{definition}
      \label{def:fieldofmodulibelyimap}
      Let $\phi\colon X\to\PP^1$ be a Belyi map.
      The \defi{field of moduli} of $\phi$,
      denoted $M(\phi)$,
      is the fixed field of $\CC$ under
      the subgroup of
      field automorphisms
      \begin{equation}
        \label{eqn:fixedfieldphi}
        \{\tau\in\Aut(\CC) : \tau(\phi)\simeq\phi\}
      \end{equation}
      where $\tau\in\Aut(\CC)$ acts on
      a set of defining equations of $\phi$
      and isomorphism is determined by
      Definition \ref{def:belyiiso}.
    \end{definition}
    Recall from Definition \ref{def:definedover}
    that a Belyi map
    $\phi\colon X\to\PP^1$
    is defined over a number field $K$
    if $\phi$ and $X$ can be defined with
    equations over $K$.
    We say that $K$
    is a \defi{field of definition}
    for $\phi$.
    \begin{remark}
      One should think of the moduli
      field as the intersection of all fields of definition.
      For more on fields of moduli and fields of definition
      see
      \cite{huggins}.
    \end{remark}
    % \begin{definition}
    %   \label{def:fieldofmoduliGbelyimap}
    %   Let $\phi\colon X\to\PP^1$ be a
    %   Galois Belyi map.
    %   The \defi{field of moduli} of
    %   $\phi$ is the fixed field of the
    %   field automorphisms
    %   \[
    %     \{\tau\in\Aut(\CC) : \phi^\tau\cong \phi\}
    %   \]
    %   where $\tau\in\Aut(\CC)$ acts on the defining equations of $\phi$
    %   and isomorphism is determined by
    %   Definition \ref{def:Gbelyi}.
    %   Denote this field as $M(\phi)$.
    % \end{definition}
    \begin{theorem}\label{thm:fieldofmoduli}
      Let $\phi:X\to\PP^1$ be a Belyi map
      with passport $\mathcal{P}(\phi)$.
      Then the degree
      of $M(\phi)$
      % of the field of
      % moduli of $\phi$
      is bounded by the size of
      $\mathcal{P}(\phi)$.
    \end{theorem}
    \begin{proof}
      Let $\tau\in G_\QQ$ and consider the
      conjugated map
      $\tau(\phi)\colon \tau(X)\to\PP^1$.
      By \cite[Appendix]{jones_streit_manfred}
      $\tau(\phi)$ is a Belyi map with
      $\mathcal{P}(\phi) =
      \mathcal{P}(\tau(\phi))$.
      Thus $G_\QQ$ acts on the set
      of (isomorphism classes of) Belyi maps with
      a given passport.
      The degree of $M(\phi)$
      is bounded by the index of the
      stabilizer of $\phi$ in $G_\QQ$
      under this action,
      and this index is bounded by
      the size of $\mathcal{P}(\phi)$.
    \end{proof}
    For a general Belyi map it may not be possible
    to define the Belyi map over its field
    of moduli.
    However,
    in the setting we are concerned with
    this is always possible.
    \begin{theorem}
      \label{thm:galoisbelyimapoverfieldofmoduli}
      A Galois Belyi map can always be defined over
      its field of moduli.
    \end{theorem}
    \begin{proof}
      See 
      \cite[Proposition 2.5]{coombes_harbater}
      and
      \cite[Theorem 2.2]{kock}.
    \end{proof}
  }
}
\chapter{Group theory}{\label{chapter:grouptheory}
  We begin this chapter with some background
  on $2$-groups and group extensions which we
  use to explain the algorithms
  in Section \ref{sec:triplesalgorithm}
  on computing explicit permutation
  triples corresponding to $2$-group Belyi maps.
  We conclude the chapter with 
  Section \ref{sec:grouptheorycomputations}
  where we summarize the computation
  of all permutation triples corresponding
  to $2$-group Belyi maps up to degree $256$.
  We also do some coarse data analysis of these
  results.
  \section{$2$-groups}{\label{sec:twogroups}
    % def, center, 
    % exact sequence, group coho, cosets
    % examples, families:
    % gen dihedral, gen quats, etc
    % \mm{references \cite{DF}\ldots}
    In this section we set up some notation
    and
    summarize some background material
    on $2$-groups all of which can be found
    in \cite[\S 6.1]{DF}.
    \par
    Let $G$ be a finite group.
    Denote the \defi{centralizer} and
    \defi{normalizer} of a subset $S\subseteq G$
    by $C_G(S)$ and $N_G(S)$ respectively.
    Let $G$ act on a set $X$.
    For $x\in X$
    denote the \defi{stabilizer of $x$} by
    $\stab_x(G)$
    and the \defi{orbit of $x$} by
    $\orb_x(G)$.
    \begin{definition}
      \label{def:pgroup}
      Let $p\in\ZZ$ be prime.
      A finite group $G$ is a
      \defi{$p$-group}
      if the cardinality of $G$
      is a power of $p$.
    \end{definition}
    \begin{lemma}
      \label{lem:pgrouphasacenter}
      The center of a nontrivial $p$-group is nontrivial.
    \end{lemma}
    % \begin{proof}
      % Let $G$ be a $p$-group acting on itself by conjugation.
      % Note that for $g\in G$ we have
      % $C_G(g)=\stab_g(G)=N_G(\{g\})$,
      % and
      % $Z(G)=\cap_g C_G(g)$.
      % Let
      % $C_g\colonequals\orb_g(G)$
      % denote the conjugacy class of $g\in G$.
      % Then $\#C_g = [G:C_G(g)]$ for every $g$.
      % Partitioning $G$ into conjugacy classes we obtain
      % \begin{equation}\label{eqn:classequation}
      %   \#G =
      %   \#Z(G)+\sum_{i=1}^r[G:C_G(g_i)]
      % \end{equation}
      % where $\{g_1,\dots,g_r\}$ is a set of representatives of distinct
      % conjugacy classes not contained in $Z(G)$.
      % Since $g_i\not\in Z(G)$, $p$ divides $[G:C_G(g_i)]$ for every $i$.
      % Then \eqref{eqn:classequation} implies $p$ divides
      % $\#Z(G)$.
    % \end{proof}
    \begin{lemma}
      \label{lem:conjugacyinsubgroups}
      Let $H$ be a normal subgroup of a $p$-group $G$.
      Let $C$ be a conjugacy class of $G$.
      Then either $C\subseteq H$ or $C\cap H = \emptyset$.
    \end{lemma}
    % \begin{proof}
      % Suppose $a\in C\cap H$.
      % Let $x\in C$.
      % Then there exists $g\in G$ so that
      % $x = gag^{-1}$.
      % But $a\in H$ and $H$ is normal,
      % so $x=gag^{-1}\in H$.
      % Thus $C\subseteq H$.
    % \end{proof}
    \begin{lemma}
      \label{lem:normalimpliescentralintersect}
      Let $G$ be a $p$-group.
      Let $H$ be a nontrivial normal subgroup of $G$.
      Then $H$ intersects the center $Z(G)$ nontrivially.
    \end{lemma}
    % \begin{proof}
      % Let $\{g_1,\dots,g_r\}$ be a set of representatives of the
      % $r$ distinct conjugacy classes (denoted $C_i$)
      % of $G$ with $\#C_i\ge 2$.
      % We will use \eqref{eqn:classequation}
      % for the subgroup $H$, so
      % by Lemma \ref{lem:conjugacyinsubgroups}
      % we may assume all $g_i\in H$.
      % The conjugacy classes of size $1$ are contained in the center $Z(G)$
      % and as in \eqref{eqn:classequation} we
      % can write
      % \begin{equation}
      %   \label{eqn:classequationsubgroup}
      %   \#H = \#(H\cap Z(G))+
      %   \sum_{i=1}^r[G:C_G(g_i)].
      % \end{equation}
      % As in the proof of
      % Lemma \ref{lem:pgrouphasacenter}
      % we see that $p$ divides $\#(H\cap Z(G))$.
    % \end{proof}
    \begin{corr}
      \label{cor:normalcentral}
      Let $H$ be a normal subgroup of order $p$ of a $p$-group $G$.
      Then $H$ is central.
    \end{corr}
    % \begin{proof}
      % By Lemma \ref{lem:normalimpliescentralintersect},
      % $H\cap Z(G)$ is a nontrivial subgroup of $G$
      % or order at least $p$.
      % Since $\#H=p$ this tells us
      % $H=H\cap Z(G)$.
      % In particular,
      % $H$ is contained in $Z(G)$.
    % \end{proof}
    \begin{lemma}
      \label{lem:normalsubgroupsofallorders}
      Let $H$ be a normal subgroup of a $p$-group $G$.
      % Let $\#G=p^\alpha$.
      Then for every divisor $p^\beta$ of $\# H$,
      $H$ contains a subgroup $H_\beta$,
      normal in $G$, of order $p^\beta$.
      % Then $H$ contains a subgroup $H_\beta$ of order $p^\beta$
      % for every divisor $p^\beta$ of $\#H$
      % with the property that
      % $H_\beta$ is normal in $G$ for every $\beta$.
    \end{lemma}
    \begin{lemma}
      \label{lem:maximalsubgroup}
      Every maximal subgroup $H$ of a $p$-group $G$
      has $[G:H]=p$ and $H\trianglelefteq G$.
    \end{lemma}
    \begin{definition}
      \label{def:uppercentralseries}
      Let $G$ be a finite group.
      We define a sequence of subgroups of $G$ iteratively as follows.
      Let $Z_0(G) = \{1\}$
      and let $Z_1(G) = Z(G)$.
      For $i\geq 2$ consider
      the map
      \[
        \pi\colon G\to G/Z_i(G),
      \]
      and define $Z_{i+1}(G)$ to be the preimage of the center of $G/Z_i(G)$ under $\pi$
      as follows.
      \[
        Z_{i+1}(G)\colonequals\pi^{-1}
        \left(Z\left(\frac{G}{Z_i(G)}\right)\right)
      \]
      Continuing this process produces a sequence of
      characteristic subgroups of $G$
      \[
        Z_0(G)\trianglelefteq Z_1(G)\trianglelefteq\cdots\trianglelefteq Z_{i}(G)\trianglelefteq\cdots
      \]
      called the \defi{upper central series} of $G$.
    \end{definition}
    \begin{definition}
      \label{def:lowercentralseries}
      For $x,y\in G$ a finite group,
      define the \defi{commutator of $x$ and $y$}
      by
      $[x,y]\colonequals x^{-1}y^{-1}xy$.
      For subgroups $H,K$ of $G$ define
      $[H,K]\colonequals\langle[h,k]:h\in H\text{ and }k\in K\rangle$.
      We define the \defi{lower central series} of $G$ iteratively as follows.
      Let $G_0=G$,
      let $G_1 = [G,G]$,
      and for $i\geq 1$ define
      $G_{i+1}=[G,G_i]$.
    \end{definition}
    \begin{definition}
      \label{def:nilpotentgroup}
      A finite group $G$ is \defi{nilpotent}
      if the upper central series
      \[
        Z_0(G)\trianglelefteq Z_1(G)\trianglelefteq\cdots\trianglelefteq Z_{i}(G)\trianglelefteq\cdots
      \]
      has $Z_c(G) = G$ for some nonnegative integer $c$.
      The integer $c$ is called the \defi{nilpotency class} of
      the nilpotent group $G$.
    \end{definition}
    \begin{lemma}
      \label{lem:upperlowernilpotent}
      % Thm 8 chapter 6.1 DF
      A finite group $G$ is nilpotent if and only if
      $G^c=\{1\}$ for some nonnegative integer $c$.
      Moreover,
      the smallest $c$ such that $G^c=\{1\}$
      is the nilpotency class of $G$ and
      \[
        Z_i(G)\leq G^{c-i-1}\leq Z_{i+1}(G)
      \]
      for all $i\in \{0,1,\dots,c-1\}$.
    \end{lemma}
    \begin{lemma}
      \label{lem:pgroupsarenilpotent}
      Every $p$-group is nilpotent.
    \end{lemma}
    % \begin{lemma}
    %   % Prop 2 chapter 6.1 DF
    %   A $p$-group of order $p^\alpha$ is nilpotent with nilpotency class
    %   at most $\alpha-1$.
    % \end{lemma}
    % \begin{definition}
    %   \label{def:frattinisubgroup}
    %   For a group $G$,
    %   define $\Phi(G)$
    %   to be the intersection of all maximal subgroups of $G$.
    %   $\Phi(G)$ is called the
    %   \defi{Frattini subgroup of $G$}.
    % \end{definition}
    % frattini quotient and generators
  }
  % \section{Examples of $2$-groups}{\label{sec:twogroupexamples}
    % % examples, families:
    % % gen dihedral, gen quats, etc
    % % In this section we discuss some
    % % special families of $2$-groups.
    % % maybe these show up later in the
    % % "proof of the conjecture"
    % In this section we define
    % some families of
    % nonabelian $2$-groups
    % that we refer to later in
    % the partial proof of
    % Conjecture \ref{conj:refinedpassportsizeone}.
    % These examples are all $2$-groups
    % with an index $2$ cyclic subgroup.
    % According to
    % Berkovich and Janko
    % (\cite[Theorem 1.2]{berkovich1}),
    % these groups all have order $2$ center,
    % order $4$ abelianization,
    % and highest possible nilpotency class.
    % % \begin{example}
    %   % \label{exm:cyclic}
    %   % For $n\geq 1$ define
    %   % $C_{2^n}$ to be the \defi{cyclic group of order $2^n$}
    %   % with presentation
    %   % \[
    %   %   \left\langle
    %   %     a\mid
    %   %     a^{2^n}=1
    %   %   \right\rangle.
    %   % \]
    %   % % \begin{itemize}
    %   % %   \item
    %   % % \end{itemize}
    % % \end{example}
    % % \begin{example}
    %   % \label{exm:abeliannoncyclic}
    %   % For $n\geq 2$ define
    %   % $A_{2^n}$ to be the
    %   % abelian group
    %   % with presentation
    %   % \[
    %   %   \left\langle
    %   %     a, b\mid
    %   %     a^{2^{n-1}}=b^2=1,
    %   %     ab=ba
    %   %   \right\rangle.
    %   % \]
    %   % \begin{itemize}
    %   %   \item
    %   %     $A_{2^n}$ is the unique noncyclic abelian group
    %   %     with a cyclic subgroup of index $2$.
    %   % \end{itemize}
    % % \end{example}
    % % \begin{example}
    %   % \label{exm:randomtwogroups}
    %   % For $n\geq 4$ define
    %   % the following $2$-groups by their presentations.
    %   % \begin{itemize}
    %   %   \item
    %   %     \[
    %   %       \left\langle
    %   %         a,b\mid
    %   %         a^{2^{n-1}}=b^2=1,
    %   %         ba=a^{1+2^{n-2}}b
    %   %       \right\rangle.
    %   %     \]
    %   %   \item
    %   %     \[
    %   %       \left\langle
    %   %         a,b\mid
    %   %         a^{2^{n-1}}=b^2=1,
    %   %         ba=a^{-1+2^{n-2}}b
    %   %       \right\rangle.
    %   %     \]
    %   % \end{itemize}
    % % \end{example}
    % % some more examples in pgroups.pdf
    % \begin{example}[Dihedral]
    %   \label{exm:dihedral}
    %   For $n\geq 2$ define
    %   \begin{equation}
    %     \label{eqn:dihedral}
    %     D_{2^{n+1}}\colonequals
    %     \left\langle
    %       a,b\mid
    %       a^{2^{n}}=
    %       b^2=1,
    %       \,
    %       bab=a^{-1}
    %     \right\rangle.
    %   \end{equation}
    %   We summarize some properties of
    %   $D_{2^{n+1}}$:
    %   \begin{itemize}
    %     \item
    %       $D_{2^{n+1}}$ has $2^{n+1}$ elements which can be written as
    %       \begin{equation}
    %         \label{eqn:elementsofdihedral}
    %         \{
    %           1,a,a^2,\dots,a^{2^n-1},
    %           b,ab,a^2b,\dots,a^{2^n-1}b
    %         \}.
    %       \end{equation}
    %     \item
    %       The center
    %       $Z(D_{2^{n+1}}) = \{1,a^{2^{n-1}}\}$
    %       is characteristic.
    %     \item
    %       $D_{2^{n+1}}$ is a split extension
    %       of cyclic groups.
    %       There exists an exact sequence
    %       \begin{equation}
    %         \label{eqn:dihedralses}
    %         \begin{tikzcd}
    %           1\arrow{r}
    %           &Z(D_{2^{n+1}})\cong\ZZ/2\ZZ\arrow{r}
    %           &D_{2^{n+1}}\arrow{r}
    %           &\ZZ/2^{n}\ZZ\arrow{r}&1.
    %         \end{tikzcd}
    %       \end{equation}
    %     \item
    %       All elements in
    %       $D_{2^{n+1}}\setminus\langle a\rangle$
    %       are involutions.
    %     \item
    %       The conjugacy classes of
    %       $D_{2^{n+1}}$ are as follows.
    %       There are $2$ conjugacy classes
    %       of size $1$,
    %       namely $\{1\}$ and $\{a^{2^{n-1}}\}$.
    %       There are $2^{n-1}-1$
    %       conjugacy classes of
    %       size $2$,
    %       % $\{a,a^{-1}\}$,
    %       % $\{a^2,a^{-2}\}$,
    %       % \dots,
    %       % $\{a^{2^{n-1}-2},a^{2-2^{n-1}}\}$,
    %       % $\{a^{2^{n-1}-1},a^{1-2^{n-1}}\}$,
    %       \begin{equation}
    %         \label{eqn:conjclassesdihedral}
    %         \left\{
    %           \{a^i,a^{-i}\}
    %         \right\}
    %         _{i=1}^{2^{n-1}-1}.
    %       \end{equation}
    %       Finally, there are
    %       $2$ conjugacy classes of
    %       size $2^{n-1}$,
    %       \begin{equation}
    %         \label{eqn:conjclassesdihedralbiggest}
    %         \{a^{2i}b : 0\leq i\leq 2^{n-1}-1\}
    %         \text{ and }
    %         \{a^{2i+1}b : 0\leq i\leq 2^{n-1}-1\}.
    %       \end{equation}
    %   \end{itemize}
    % \end{example}
    % \begin{example}[Generalized Quaternion]
    %   \label{exm:generalizedquaternion}
    %   For $n\geq 2$ define
    %   \begin{equation}
    %     \label{eqn:quaternionpresentation}
    %     Q_{2^{n+1}}
    %     \colonequals
    %     \left\langle
    %       % a,b\mid
    %       % a^{2^{n-1}}=1,
    %       % b^2=a^{2^{n-2}},
    %       % ba=a^{-1}b
    %       a,b\mid
    %       a^{2^n}=1,\,
    %       b^2=a^{2^{n-1}},\,
    %       b^{-1}ab=a^{-1}
    %     \right\rangle.
    %   \end{equation}
    %   % We summarize some properties of
    %   % $Q_{2^{n+1}}$:
    %   % \begin{itemize}
    %   %   \item
    %   %     $Q_{2^{n+1}}$ has $2^{n+1}$ elements
    %   %     which can be written as
    %   %     \mm{todo}
    %   %   \item
    %   %     $Q_{2^{n+1}}$ is a nonsplit extension
    %   %     of cyclic groups.
    %   %   \item
    %   %     All elements in
    %   %     $Q_{2^{n+1}}\setminus\langle a\rangle$
    %   %     have order $4$.
    %   %   \item
    %   %     $Q_{2^{n+1}}$ has a unique involution
    %   %     \mm{todo}
    %   %   \item
    %   %     $Q_{2^{n+1}}/Z(Q_{2^{n+1}})$ is dihedral
    %   %     for $n\geq 3$.
    %   %   \item
    %   %     The conjugacy classes of
    %   %     $Q_{2^{n+1}}$ are as follows.
    %   %     \mm{todo}
    %   % \end{itemize}
    % \end{example}
    % \begin{example}[Semi-dihedral]
    %   \label{exm:semidihedral}
    %   For $n\geq 3$ define
    %   \begin{equation}
    %     \label{eqn:semidehedralpresentation}
    %     SD_{2^{n+1}}
    %     \colonequals
    %     \left\langle
    %       a,b\mid
    %       a^{2^{n}}=b^2=1,\;
    %       bab=a^{-1+2^{n-1}}
    %     \right\rangle.
    %   \end{equation}
    %   % We summarize some properties of
    %   % $SD_{2^{n+1}}$:
    %   % \begin{itemize}
    %   %   \item
    %   %     $SD_{2^{n+1}}$ has $2^{n+1}$ elements
    %   %     which can be written as
    %   %     \mm{todo}
    %   %   \item
    %   %     $SD_{2^{n+1}}$ is a split extension
    %   %     of cyclic groups.
    %   %   \item
    %   %     \mm{involutions?}
    %   %   \item
    %   %     $Q_{2^{n+1}}/Z(Q_{2^{n+1}})$ is dihedral.
    %   %   \item
    %   %     Maximal subgroups in $SD_{2^{n+1}}$ are characteristic:
    %   %     \begin{align}
    %   %       \label{eqn:maximalsubgroupssemidihedral}
    %   %       \begin{split}
    %   %         \langle a^2,b\rangle=\Omega_1(SD_{2^{n+1}})&\cong D_{2^n}\\
    %   %         \langle a^2,ab\rangle&\cong Q_{2^n}
    %   %       \end{split}
    %   %     \end{align}
    %   %   \item
    %   %     The conjugacy classes of
    %   %     $SD_{2^{n+1}}$ are as follows.
    %   %     \mm{todo}
    %   % \end{itemize}
    % \end{example}
    % \begin{lemma}
    %   \label{lem:dihedralquotient}
    %   Let $G$ be one of the groups
    %   $D_{2^{n+1}}$, $Q_{2^{n+1}}$, $SD_{2^{n+1}}$
    %   discussed in the previous examples
    %   with center $Z(G)$.
    %   Then $\#Z(G) = 2$
    %   and $G/Z(G)$ is a dihedral group.
    % \end{lemma}
    % \begin{proof}
    %   % % Conrad has elementary proof for dihedral
    %   % \mm{todo}
    %   This is proved in
    %   Berkovich and Janko
    %   \cite[Theorem 1.2]{berkovich1}
    %   where they attribute a stronger result
    %   to Burnside.
    % \end{proof}
    % % \mm{
    % %   In fact, these groups are the $p$-groups
    % %   of maximal nilpotency class and carry
    % %   many properties\ldots
    % % }
  % }
  \section{Computing group extensions}{\label{sec:modcoho}
    % trivial G-Module ->
    % Cohomology module -> Holt's algorithm
    In Section \ref{sec:triplesalgorithm},
    we will be interested in constructing
    $2$-groups
    as (central) extensions of other $2$-groups.
    The computations we rely on are implemented in
    \textsf{Magma}
    and described in
    % \emph{Cohomology and group extensions in
    % \textsf{Magma}}
    \cite{magmabook}.
    We now describe the broad strokes of
    this implementation
    emphasizing the particular
    setting we are interested in.
    The background material concerning group
    extensions in this section is summarized
    from
    \cite[\S 17.4]{DF}.
    \begin{definition}
      \label{def:groupextension}
      Let $G$ be a finite group
      and $A$ a finite abelian group.
      An \defi{extension of $A$ by $G$} is
      a group $\wt{G}$ such that the sequence
      \begin{equation}
        \label{eqn:extension}
        \begin{tikzcd}
          1\arrow{r}&A\arrow{r}{\iota}&\wt{G}\arrow{r}{\pi}&G\arrow{r}&1
        \end{tikzcd}
      \end{equation}
      is exact.
      An extension \eqref{eqn:extension}
      is \defi{central}
      if $\iota(A)$ is
      contained in the center of $\wt{G}$.
    \end{definition}
    Note that for a group extension
    \eqref{eqn:extension}
    there is an action of $G$ on
    $\iota(A)$ by conjugation.
    This action is obtained by choosing
    a lift in $\wt{G}$ and conjugating.
    Conjugating $\iota(A)$ by
    this lift is well-defined
    since $A$ is abelian.
    From now on we identify $A$ with
    its image $\iota(A)$ in $\wt{G}$
    to ease notation.
    % This motivates the following definition.
    To keep track of the action of $G$
    on $A$ we make the
    following definition.
    \begin{definition}
      \label{def:Gmodule}
      Let $G$ be a finite group.
      A \defi{$G$-module}
      is a finite abelian group $A$
      and a group homomorphism
      $\phi\colon G\to\Aut(A)$.
    \end{definition}
    \begin{prop}
      \label{prop:centralextensionstrivialGaction}
      The extension in \eqref{eqn:extension}
      is central
      if and only if
      $A$
      (identified with its image $\iota(A)$
      in $\wt{G}$)
      has trivial $G$-module
      structure.
    \end{prop}
    \begin{proof}
      Let $a\in A$, let $g\in G$,
      and let $\wt{g}\in\pi^{-1}(g)$.
      Then $g$ acts on $a$ by
      \begin{equation}
        \label{eqn:gactiontrivial1}
        g\cdot a=\wt{g}a\wt{g}^{-1},
      \end{equation}
      so the action is trivial
      if and only if
      $a=\wt{g}a\wt{g}^{-1}$
      for all $\wt{g}\in\wt{G}$
      if and only if $a\in Z(\wt{G})$.
    \end{proof}
    % \begin{proof}
      % Let $a\in\iota(A)$, let $g\in G$,
      % and let $\wt{g}\in\pi^{-1}(g)$.
      % Then $g$ acts on $a$ by
      % \begin{equation}
      %   \label{eqn:gaction}
      %   ga=\iota^{-1}\left(\wt{g}\iota(a){\wt{g}}^{-1}\right).
      % \end{equation}
      % For the trivial action this is just
      % \begin{equation}
      %   \label{eqn:gactiontrivial1}
      %   a=\iota^{-1}\left(\wt{g}\iota(a)\wt{g}^{-1}\right)
      % \end{equation}
      % or equivalently
      % \begin{equation}
      %   \label{eqn:gactiontrivial2}
      %   \iota(a)=\wt{g}\iota(a)\wt{g}^{-1}.
      % \end{equation}
      % Since every element of $\wt{G}$ can be written as some
      % $\wt{g}$ (the lift of some $g$ under the surjective map $\pi$),
      % this is equivalent to saying $\iota(a)$
      % is central in $\wt{G}$.
    % \end{proof}
    \begin{definition}
      \label{def:equivalentgroupextension}
      Two extensions of $A$ by $G$ are
      \defi{equivalent}
      if there exists an isomorphism of groups
      $\phi$ making the diagram
      \begin{equation}
        \label{eqn:twoextensions}
        \begin{tikzcd}
          1\arrow{r}&A\arrow{d}{\id}\arrow{r}{}&\wt{G}_1\arrow{r}{}\arrow{d}{\phi}&G\arrow{d}{\id}\arrow{r}&1\\
          1\arrow{r}&A\arrow{r}{}&\wt{G}_2\arrow{r}{}&G\arrow{r}&1
        \end{tikzcd}
      \end{equation}
      commute.
    \end{definition}
    \begin{remark}
      \label{rmk:otherequivalences}
      The notion of equivalence from
      Definition \ref{def:equivalentgroupextension}
      requires an isomorphism $\phi$
      inducing the identity map on $A$ and $G$.
      This definition comes from the
      $G$-module structure of $A$
      in the sense that equivalent extensions
      induce (by conjugation) the same
      $G$-module structure on $A$.
      A weaker notion of equivalence
      (where we only require $\phi$
      to be any isomorphism from $A$ to $A$)
      is useful to characterize
      the groups $\wt{G}$ up to group isomorphism,
      but will not be used in our situation.
    \end{remark}
    % It is possible for isomorphic groups
    % $\wt{G}_1$
    % and
    % $\wt{G}_2$
    % to appear as inequivalent extensions
    % of $A$ by $G$.
    % We can weaken this notion of equivalent extensions
    % with the following definition.
    % % in the end we are just trying to find isomorphism classes
    % % of groups sitting in extensions of A by G
    % \begin{definition}
    %   \label{def:isoextensions}
    %   Let
    %   $\wt{G}_1$
    %   be an extension of
    %   $A_1$ by $G$,
    %   let
    %   $\wt{G}_2$
    %   an extension of
    %   $A_2$ by $G$,
    %   and let
    %   $A_1$ isomorphic to $A_2$.
    %   We say the extensions are
    %   \defi{isomorphic}
    %   if there exists an isomorphism of groups
    %   $\phi\colon\wt{G}_1\to\wt{G}_2$
    %   mapping $A_1$ to $A_2$.
    % \end{definition}
    % Given a finite group $G$
    % and an abelian group $A$,
    % we would like to find all isomorphism classes of groups
    % $\wt{G}$ sitting in the exact sequence
    % in Equation \ref{eqn:extension}.
    % Computing any of the following sets
    % helps us to achieve this goal.
    % \begin{enumerate}
    %   \item
    %     \label{itm:equal}
    %     $\{
    %       \text{
    %         isomorphism classes of groups
    %         $\wt{G}$ sitting in the exact sequence
    %         in Equation \ref{eqn:extension}
    %       }
    %     \}$
    %   \item
    %     \label{itm:iso}
    %     $\{
    %       \text{
    %         isomorphism classes of extensions of $A$ by $G$
    %       }
    %     \}$
    %   \item
    %     \label{itm:equiv}
    %     $\{
    %       \text{
    %         equivalence classes of extensions of $A$ by $G$
    %       }
    %     \}$
    % \end{enumerate}
    % There are surjective maps
    % from the set in \ref{itm:equiv}
    % to the set in \ref{itm:iso}
    % and from the set in \ref{itm:iso}
    % to the set in \ref{itm:equal},
    % but ultimately
    % we will not make too much fuss about these
    % various distinctions.
    We now look at a motivating example.
    \begin{example}
      \label{exm:semidirectproduct}
      Let $A$ be a $G$-module
      with $\phi\colon G\to\Aut(A)$
      defining the action of $G$ on $A$.
      Then we can construct
      the \defi{(external) semidirect product}
      $A\rtimes G$ which is the set
      $A\times G$ equipped with multiplication
      defined by
      \[
        (a_1,g_1)(a_2,g_2)\colonequals
        (a_1+\phi(g_1)(a_2), g_1g_2).
      \]
      Then $A\rtimes G$
      is an extension of $A$ by $G$
      \[
        \begin{tikzcd}
          1\arrow{r}&A\arrow{r}{\iota}&A\rtimes G\arrow{r}{\pi}&G\arrow{r}&1
        \end{tikzcd}
      \]
      where the conjugation action of $\pi^{-1}(G)$
      on $A$
      (identified with $\iota(A)$)
      coincides with the original
      $G$-module action of $A$.
    \end{example}
    We now explain the bijection between
    equivalence classes of extensions
    (of $A$ by $G$)
    and elements of the group
    $H^2(G,A)$.
    The latter
    can be efficiently computed in \textsf{Magma}
    \cite{magmabook},
    and is a crucial part of the algorithms
    in
    Section \ref{sec:triplesalgorithm}.
    % examples at the end..
    \begin{definition}
      \label{def:section}
      A function $s\colon G\to\wt{G}$
      such that $\pi\circ s = \id_G$
      is called a
      \defi{section}
      of $\pi$.
      A section is
      \defi{normalized}
      if it maps $\id_G$
      to
      $\id_{\wt{G}}$.
      An extension
      is \defi{split}
      if there exists a section $s$ such that
      $s$ is a homomorphism.
    \end{definition}
    \begin{prop}
      \label{prop:splitextensions}
      The extension in
      \eqref{eqn:extension}
      is split
      if and only if
      it is equivalent to
      \[
        \begin{tikzcd}
          1\arrow{r}&A\arrow{r}{\iota'}
                    &A\rtimes G\arrow{r}{\pi'}
                    &G\arrow{r}&1
        \end{tikzcd}
      \]
      where $A\rtimes G$ is the semidirect product
      of $G$ and $A$ relative to the given action
      described in
      Example \ref{exm:semidirectproduct}.
    \end{prop}
    \begin{proof}
      Suppose $\phi\colon\wt{G}\to A\rtimes G$
      is an isomorphism inducing the identity maps
      on $A$ and $G$.
      Let $s'\colon G\to A\rtimes G$ be the section
      $g\mapsto (\id_A,g)$.
      Then the section $s\colonequals\phi^{-1}s'$
      is a group homomorphism
      $s\colon G\to\wt{G}$
      showing the extension is split.
      % For a proof see \cite[Proposition 2.1]{brown}
      % or \cite[\S 17.4]{DF}
      Conversely,
      assume there exists
      a section
      $s\colon G\to\wt{G}$
      which is a group homomorphism.
      Then the map $\phi\colon A\rtimes G\to\wt{G}$ defined by
      \[
        (a,g)\mapsto\iota(a)s(g)
      \]
      is a bijection.
      We now show that this map is a group isomorphism
      by analyzing the multiplication of two
      elements in the image of $\phi$.
      Let $\iota(a)s(g)$ and $\iota(a')s(g')$
      in the image of $\phi$.
      Then
      from the $G$-module structure of $A$ we have
      \begin{equation}
        \label{eqn:gmodulestructureofA}
        s(g)\iota(a')=\iota(ga)s(g).
      \end{equation}
      \eqref{eqn:gmodulestructureofA}
      then implies
      \[
        \iota(a)s(g)\iota(a')s(g')=\iota(a)\iota(ga')s(g)s(g')=\iota(a+ga')s(gg')
      \]
      which is precisely the semidirect product multiplication
      rule on $A\times G$.
    \end{proof}
    Proposition \ref{prop:splitextensions}
    completely describes split extensions.
    For nonsplit extensions,
    we must analyze sections that are
    not homomorphisms.
    To measure the failure of $s$
    to be a homomorphism,
    we make the following definition.
    % \begin{definition}
    %   \label{def:factorset}
    %   Consider an extension as in Equation \ref{eqn:extension}
    %   and a section $s$.
    %   Let $f\colon G\times G\to A$ be defined by the equation
    %   \begin{equation}
    %     \label{eqn:factorset}
    %     s(g_1)s(g_2) = \iota(f(g_1,g_2))s(g_1g_2).
    %   \end{equation}
    %   In other words,
    %   $\pi(s(g_1g_2)) = \pi(s(g_1)s(g_2)) = g_1g_2$,
    %   so we know that
    %   $s(g_1g_2)$ and $s(g_1)s(g_2)$ differ by an element of $\iota(A)$.
    %   We define $f(g_1,g_2)$ to be the element $a\in A$
    %   such that
    %   Equation \ref{eqn:factorset}
    %   is satisfied.
    %   The function $f$ is called the \defi{factor set}
    %   for the extension and the section $s$.
    %   A factor set is \defi{normalized}
    %   if $s$ is normalized.
    %   A normalized factor set $f$ satisfies
    %   \[
    %     f(g,1)=f(1,g)=0
    %   \]
    %   for all $g\in G$.
    % \end{definition}
    \begin{definition}
      \label{def:factorset}
      Let $s$ be a section of an extension
      \eqref{eqn:extension}.
      Let $f\colon G\times G\to A$
      be defined by the equation
      \begin{equation}
        \label{eqn:factorset}
        s(g)s(h) = \iota(f(g,h))s(gh).
      \end{equation}
      In other words,
      $\pi(s(gh)) = \pi(s(g)s(h)) = gh$,
      so we know that
      $s(gh)$ and $s(g)s(h)$ differ by an element of $\iota(A)$.
      We define $f(g,h)$ to be the element $a\in A$
      such that
      \eqref{eqn:factorset}
      is satisfied.
      The function $f$ is called the \defi{factor set}
      for the extension and the section $s$.
      A factor set is \defi{normalized}
      if $s$ is normalized.
      A normalized factor set $f$ satisfies
      \[
        f(g,1)=f(1,g)=0
      \]
      for all $g\in G$.
    \end{definition}
    In Lemma \ref{lem:factorsetisacocycle}
    we will see that a factor set
    for an extension with a section
    is a special case of a
    \emph{$2$-cocycle}
    which we now define.
    \begin{definition}
      \label{def:twococycle}
      A \defi{$2$-cocycle}
      is a map
      $f\colon G\times G\to A$
      satisfying
      \begin{equation}
        \label{eqn:twococycle}
        f(g,h)+f(gh,k)=gf(h,k)+f(g,hk)
      \end{equation}
      for all $g,h,k\in G$.
      A $2$-cocycle $f$ is \defi{normalized}
      if
      \[
        f(g,1)=f(1,g)=0
      \]
      for all $g\in G$.
    \end{definition}
    \begin{definition}
      \label{def:twocoboundary}
      A \defi{$2$-coboundary}
      is a map
      $f\colon G\times G\to A$
      such that there exists
      $f_1\colon G\to A$ satisfying
      \begin{equation}
        \label{eqn:twocoboundary}
        f(g,h) = gf_1(h)-f_1(gh)+f_1(g)
      \end{equation}
      for all $g,h\in G$.
    \end{definition}
    \begin{definition}
      \label{def:h2}
      Let $Z^2(G,A)$ denote the set
      of $2$-cocycles and $B^2(G,A)$
      denote the set of all $2$-coboundaries.
      The \defi{second cohomology group} $H^2(G,A)$
      is defined by the
      quotient $Z^2(G,A)/B^2(G,A)$.
    \end{definition}
    \begin{lemma}
      \label{lem:factorsetisacocycle}
      The factor set $f$ of an extension
      as in \eqref{eqn:extension}
      and a section $s$ is a $2$-cocycle.
    \end{lemma}
    % \begin{proof}
      % % DF p.825
      % Since $s\colon G\to\wt{G}$ is a section,
      % we can write elements of $\wt{G}$ in the form
      % $\iota(a)s(g)$ for $a\in A, g\in G$.
      % Now we can write the multiplication of arbitrary elements in $\wt{G}$ as
      % $\iota(a_1)s(g)\iota(a_2)s(h)$.
      % From the action of $G$ on $A$ we have
      % \begin{equation}
      %   \label{eqn:factorsetisacocycle1}
      %   \iota(a_1)s(g)\iota(a_2)s(h) = \iota(a_1)\iota(ga_2)s(g)s(h)
      %   % = \iota(a_1+s(g)a_2s(g)^{-1})s(g)s(h)
      % \end{equation}
      % which, by \eqref{eqn:factorset},
      % is equal to
      % \begin{equation}
      %   \label{eqn:factorsetisacocycle2}
      %   \iota(a_1)\iota(ga_2)\iota(f(g,h))s(gh)
      %   =\iota(a_1+ga_2+f(g,h))s(gh)
      %   % \iota(a_1+s(g)a_2s(g)^{-1})\iota(f(g,h))s(gh)=
      %   % \iota((a_1+s(g)a_2s(g)^{-1})f(g,h))s(gh)
      % \end{equation}
      % so that
      % \begin{equation}
      %   \label{eqn:factorsetisacocycle3}
      %   \iota(a_1)s(g)\iota(a_2)s(h)
      %   =\iota(a_1+ga_2+f(g,h))s(gh).
      % \end{equation}
      % Now let $g,h,k\in G$ and,
      % using \eqref{eqn:factorsetisacocycle3},
      % we have
      % \begin{align}
      %   \label{eqn:associative1}
      %   \begin{split}
      %     [s(g)s(h)]s(k) &= [\iota(f(g,h))s(gh)]s(k)\\
      %                    &= \iota(f(g,h)+f(gh,k))s(ghk)
      %   \end{split}
      % \end{align}
      % and
      % \begin{align}
      %   \label{eqn:associative2}
      %   \begin{split}
      %     s(g)[s(h)s(k)] &= s(g)[\iota(f(h,k))s(hk)]\\
      %                    &= \iota(gf(h,k)+f(g,hk))s(ghk).
      %   \end{split}
      % \end{align}
      % Since the right hand sides of
      % \eqref{eqn:associative1}
      % and
      % \eqref{eqn:associative2}
      % are equal
      % by associativity in $\wt{G}$,
      % we get
      % \begin{equation}
      %   \label{eqn:justiota}
      %   \iota(f(g,h)+f(gh,k))s(ghk)
      %   =
      %   \iota(gf(h,k)+f(g,hk))s(ghk).
      % \end{equation}
      % After canceling $s(ghk)$ from both sides
      % and using the injectivity of $\iota$
      % \eqref{eqn:justiota}
      % shows that $f$ satisfies the condition in
      % Definition
      % \ref{def:twococycle}.
    % \end{proof}
    \begin{lemma}
      \label{lem:choiceofsectioncoboundary}
      % \mm{choice of section changes a factor set by a coboundary
      % and therefore an extension determines a class in $H^2$}
      Consider an extension as in
      \eqref{eqn:extension}.
      Let $s$ and $s'$ be sections of this extension
      with corresponding factor sets $f$ and $f'$
      respectively.
      Then $f'-f$ is a $2$-coboundary.
    \end{lemma}
    % \begin{proof}
      % For $g\in G$ we have
      % $s(g)$ and $s'(g)$ define the same (right) coset of
      % $\wt{G}/\iota(A)$.
      % We can therefore write
      % \begin{equation}
      %   \label{eqn:definef1}
      %   s'(g) = \iota(a)s(g)
      % \end{equation}
      % for some $a\in A$.
      % This defines a map
      % $f_1\colon G\to A$
      % by mapping $g\in G$ to $a\in A$
      % satisfying
      % \eqref{eqn:definef1}.
      % Thus,
      % \begin{equation}
      %   \label{eqn:definef1again}
      %   s'(g)=\iota(f_1(g))s(g)
      % \end{equation}
      % for every $g\in G$.
      % Now on one hand we have
      % \begin{equation}
      %   \label{eqn:onehand}
      %   s'(g)s'(h)=
      %   \iota(f'(g,h))s'(gh)=
      %   \iota(f'(g,h))\iota(f_1(gh))s(gh)
      % \end{equation}
      % for all $g,h\in G$.
      % On the other hand we have
      % \begin{align}
      %   \label{eqn:otherhand}
      %   \begin{split}
      %     s'(g)s'(h)
      %     &=\iota(f_1(g))s(g)\iota(f_1(h))s(h)\\
      %     &=\iota(f_1(g))\iota(gf_1(h))s(g)s(h)\\
      %     &=\iota(f_1(g))\iota(gf_1(h))\iota(f(g,h))s(gh).
      %   \end{split}
      % \end{align}
      % Combining \eqref{eqn:onehand}
      % and
      % \eqref{eqn:otherhand}
      % we get
      % \begin{equation}
      %   \label{eqn:result1}
      %   \iota(f'(g,h))\iota(f_1(gh))s(gh)
      %   =\iota(f_1(g))\iota(gf_1(h))\iota(f(g,h))s(gh)
      % \end{equation}
      % which implies
      % \begin{equation}
      %   \label{eqn:result2}
      %   \iota(f'(g,h)+f_1(gh))
      %   =\iota(f_1(g)+gf_1(h)+f(g,h))
      % \end{equation}
      % which implies (by injectivity of $\iota$)
      % that
      % \begin{equation}
      %   \label{eqn:result3}
      %   f'(g,h)+f_1(gh)
      %   =f_1(g)+gf_1(h)+f(g,h).
      % \end{equation}
      % Rewriting \eqref{eqn:result3} as
      % \begin{equation}
      %   \label{eqn:result4}
      %   f'(g,h)-f(g,h)=gf_1(h)-f_1(gh)+f_1(g)
      % \end{equation}
      % shows that $f'-f$ satisfies
      % the conditions in
      % Definition \ref{def:twocoboundary}
      % and is a $2$-coboundary.
      % % DF p.826
    % \end{proof}
    Lemma \ref{lem:factorsetisacocycle}
    and Lemma
    \ref{lem:choiceofsectioncoboundary}
    are explained on page 825 and 826 of
    \cite{DF}.
    \begin{lemma}
      \label{lem:equivalentextensionssamecohomologyclass}
      % \mm{equivalent extensions determine the same element of $H^2$}
      An equivalence class of extensions of $A$ by $G$
      determine a unique element of
      $H^2(G,A)$.
    \end{lemma}
    \begin{proof}
      Let $f$ be the factor set for
      any section of the extension.
      Lemma \ref{lem:factorsetisacocycle} shows
      that $f\in Z^2(G,A)$.
      Lemma \ref{lem:choiceofsectioncoboundary}
      shows that any other choice of $f$
      corresponding to another choice of section
      differs from $f$ by an element of
      $B^2(G,A)$.
      Thus,
      any single extension of $A$ by $G$
      determines a unique cohomology class
      in $H^2(G,A)$.
      It remains to show that equivalent extensions
      determine the same element of $H^2(G,A)$.
      Consider the
      equivalent extensions
      \begin{equation}
        \label{eqn:twoextensions1}
        \begin{tikzcd}
          1\arrow{r}&A\arrow{d}{\id}\arrow{r}{}&\wt{G}_1\arrow{r}{\pi_1}\arrow{d}{\phi}&G\arrow{d}{\id}\arrow{r}{}&1\\
          1\arrow{r}&A\arrow{r}{}&\wt{G}_2\arrow{r}{\pi_2}&G\arrow{r}{}&1.
        \end{tikzcd}
      \end{equation}
      and let
      $s_1\colon G\to\wt{G}$
      be a section of $\pi_1$.
      From \eqref{eqn:twoextensions1} we have that
      $s_2\colonequals \phi\circ s_1$ is a
      section of $\pi_2$.
      Let $f_1$ and $f_2$ be the factor sets
      corresponding to $s_1$
      and $s_2$ respectively
      defined by
      \begin{align}
        \label{eqn:twofactorsets}
        \begin{split}
          s_1(g)s_1(h) &= f_1(g,h)s_1(gh)\\
          s_2(g)s_2(h) &= f_2(g,h)s_2(gh)\\
        \end{split}
      \end{align}
      for all $g,h\in G$.
      Chasing through the diagram in
      \eqref{eqn:twoextensions1}
      we have
      \begin{equation}
        \label{eqn:samefactorsets}
        \begin{split}
          s_2(g)s_2(h)
          &=\phi(s_1(g))\phi(s_1(h))\\
          &=\phi(s_1(g)s_1(h))\\
          &=\phi(f_1(g,h)s_1(gh))\\
          &=\phi(f_1(g,h))\phi(s_1(gh))\\
          &=f_1(g,h)s_2(gh)
        \end{split}
      \end{equation}
      where the last equality in
      \eqref{eqn:samefactorsets}
      follows from chasing the diagram through the
      identity map $\id\colon A\to A$.
      This shows if two extensions are equivalent,
      then we can define sections for both extensions
      such that the corresponding factor sets
      are the same $2$-cocycle.
      In particular,
      equivalent extensions define the same
      element of $H^2(G,A)$,
      which completes the proof.
      % DF p.826
    \end{proof}
    Lemma
    \ref{lem:equivalentextensionssamecohomologyclass}
    proves that any factor set for an
    extension of $A$ by $G$
    defines a unique class in
    $H^2(G,A)$.
    We now discuss the reverse process
    of constructing an extension of $A$ by $G$
    from a $2$-cocycle.
    % \begin{lemma}
    %   \label{lem:normalizedcocyclesufficient}
    %   Every class in $H^2(G,A)$
    %   contains a normalized $2$-cocycle.
    % \end{lemma}
    % \begin{proof}
    % \end{proof}
    \begin{lemma}
      \label{lem:constructextensionfromcocycle}
      % \mm{how to construct an extension from a normalized $2$-cocycle}
      Let $f\in H^2(G,A)$ for some finite group $G$
      and $G$-module $A$.
      Then there is an extension
      \begin{equation}
        \label{eqn:extensionfromcocycle}
        \begin{tikzcd}
          1\arrow{r}&A\arrow{r}{\iota}&\wt{G}\arrow{r}{\pi}
                    &G\arrow{r}&1
        \end{tikzcd}
      \end{equation}
      whose factor set is equivalent to $f$ in $H^2(G,A)$.
    \end{lemma}
    \begin{proof}
      Let $\wt{G}$ be defined by the set $A\times G$ equipped
      with the operation
      \begin{equation}
        \label{eqn:Efmultiplication}
        (a_1,g_1)(a_2,g_2)=
        (a_1+g_1a_2+f(g_1,g_2), g_1g_2).
      \end{equation}
      % We are first required to prove that
      $A\times G$ with this operation is a group
      with identity element
      $(-f(1,1), 1)$
      and the inverse of $(a,g)\in A\times G$
      given by
      \begin{equation}
        \label{eqn:aginverses}
        (a,g)^{-1}
        =
        (-g^{-1}a-f(g^{-1},g)-f(1,1), g^{-1}).
      \end{equation}
      % We will do this in three steps.
      % \begin{enumerate}
      %   \item
      %     \label{itm:identityelement}
      %     % We claim the identity element is
      %     The identity element is
      %     $(-f(1,1), 1)$.
      %     % Indeed if we let $(a,g)\in\wt{G}$,
      %     % then
      %     % \begin{align}
      %     %   \label{eqn:identityinwtG}
      %     %   \begin{split}
      %     %     (-f(1,1),1)(a,g)
      %     %     &=(-f(1,1)+1a+f(1,g),1g)\\
      %     %     &=(f(1,g)-f(1,1)+a,g)\\
      %     %     (a,g)(-f(1,1),1)
      %     %     &=(a+g(-f(1,1))+f(g,1), g1)\\
      %     %     &=(a+f(g,1)-gf(1,1), g)
      %     %   \end{split}
      %     % \end{align}
      %     % so it suffices to show
      %     % \begin{align}
      %     %   \label{eqn:sufficientoshow}
      %     %   \begin{split}
      %     %     f(1,g)-f(1,1)=0=f(g,1)-gf(1,1).
      %     %   \end{split}
      %     % \end{align}
      %     % \eqref{eqn:sufficientoshow}
      %     % follows from the equations
      %     % \begin{align}
      %     %   \label{eqn:g11}
      %     %   \begin{split}
      %     %     f(1,1)+f(1,g)&=2f(1,g)\\
      %     %     2f(g,1)&=gf(1,1)+f(g,1)
      %     %   \end{split}
      %     % \end{align}
      %     % which are obtained
      %     % by substituting
      %     % $g=1,h=1,k=g$
      %     % and
      %     % $g=g, h=1, k=1$
      %     % respectively into
      %     % \eqref{eqn:twococycle}.
      %     % % Then Equation \ref{eqn:g11} implies
      %     % % \begin{align}
      %     % %   \label{eqn:twotimeszero}
      %     % %   \begin{split}
      %     % %     f(1,g)-f(1,1)
      %     % %     &=2f(1,g)-2f(1,1)\\
      %     % %     &=2[f(1,g)-f(1,1)]\\
      %     % %     f(g,1)-gf(1,1)
      %     % %     &=2f(g,1)-2gf(1,1)\\
      %     % %     &=2[f(g,1)-gf(1,1)].
      %     % %   \end{split}
      %     % % \end{align}
      %     % % Thus Equation \ref{eqn:sufficientoshow}
      %     % % is satisfied.
      %   \item
      %     \label{itm:inverses}
      %     Let $(a,g)\in A\times G$.
      %     % We claim that
      %     Then
      %     \begin{equation}
      %       \label{eqn:aginverses}
      %       (a,g)^{-1}
      %       =
      %       (-g^{-1}a-f(g^{-1},g)-f(1,1), g^{-1}).
      %     \end{equation}
      %     % We are required to show
      %     % \begin{align}
      %     %   \label{eqn:sufficientforinverses}
      %     %   \begin{split}
      %     %     (a,g)(-g^{-1}a-f(g^{-1},g)-f(1,1), g^{-1})&=(-f(1,1),1)\\
      %     %     (-g^{-1}a-f(g^{-1},g)-f(1,1), g^{-1})(a,g)&=(-f(1,1),1).
      %     %   \end{split}
      %     % \end{align}
      %     % We have \mm{TODO: verify inverse}
      %     % \begin{align}
      %     %   \label{eqn:justgoforit}
      %     %   \begin{split}
      %     %     (a,g)(-g^{-1}a-f(g^{-1},g)-f(1,1), g^{-1})
      %     %     &=(,gg^{-1})\\
      %     %     &=(-f(1,1),1)\\
      %     %     (-g^{-1}a-f(g^{-1},g)-f(1,1), g^{-1})(a,g)
      %     %     &=(,g^{-1}g)\\
      %     %     &=(-f(1,1),1)
      %     %   \end{split}
      %     % \end{align}
      %   \item
      %     \mm{TODO: verify associativity}
      %     \label{itm:associativity}
      % \end{enumerate}
      We now construct the rest of the extension.
      Let $A^*$ be defined by
      \begin{equation}
        \label{eqn:Astar}
        A^*\colonequals
        \{(a-f(1,1), 1):a\in A\}.
      \end{equation}
      % We first show that $A^*$ is a subgroup of $\wt{G}$.
      $A^*$ is a normal subgroup of
      $\wt{G}$ with the inverses given by
      \begin{align}
        \label{eqn:subgroupinversesAstar}
        \begin{split}
          (a-f(1,1),1)^{-1}
          % &=(-(1(a-f(1,1)))-f(1,1)-f(1,1), 1)\\
          % &=(-(a-f(1,1))-f(1,1)-f(1,1), 1)\\
          &=(-a-f(1,1), 1)
        \end{split}
      \end{align}
      % Let
      % $a_1^*\colonequals(a_1-f(1,1,),1)$
      % and
      % $a_2^*\colonequals(a_2-f(1,1),1)$
      % be elements of $A^*$.
      % Then
      % \begin{align}
      %   \label{eqn:subgroupAstar}
      %   \begin{split}
      %     % (a_1-f(1,1),1)(a_2-f(1,1),1)
      %     a_1^*a_2^*
      %     &=(a_1-f(1,1)+1(a_2-f(1,1))+f(1,1), 1)\\
      %     &=(a_1+a_2-f(1,1),1)
      %   \end{split}
      % \end{align}
      % shows that $A^*$ is closed under the group operation.
      % Let $(a-f(1,1),1)\in A^*$.
      % Then
      % \begin{align}
      %   \label{eqn:subgroupinversesAstar}
      %   \begin{split}
      %     (a-f(1,1),1)^{-1}
      %     &=(-(1(a-f(1,1)))-f(1,1)-f(1,1), 1)\\
      %     &=(-(a-f(1,1))-f(1,1)-f(1,1), 1)\\
      %     &=(-a-f(1,1), 1)
      %   \end{split}
      % \end{align}
      % shows that $A^*$ is closed under inverses.
      % Thus $A^*$ is a subgroup of $\wt{G}$.
      % To see that $A^*$ is a normal subgroup,
      % let $a^*\colonequals (a-f(1,1),1)\in A^*$
      % and $(a',g)\in\wt{G}$.
      % Then
      % \mm{TODO: verify $A^*$ is normal}
      % \begin{align}
      %   \label{eqn:Astarnormal}
      %   \begin{split}
      %     (a',g)a^*(a',g)^{-1}
      %     &=(a',g)(a-f(1,1),1)(a',g)^{-1}\\
      %     &=(a',g)(a-f(1,1),1)(-g^{-1}a'-f(g^{-1},g)-f(1,1),g^{-1})\\
      %     &=(a'+g(a-f(1,1))+f(g,1),g)(-g^{-1}a'-f(g^{-1},g)-f(1,1),g^{-1})\\
      %     % Nicole says comic sans would work here
      %     &=(,gg^{-1})\\
      %     &=
      %   \end{split}
      % \end{align}
      The isomorphism $\iota\colon A\to A^*$
      is defined by
      \begin{equation}
        \label{eqn:iotaAstar}
        a\mapsto
        (a-f(1,1),1).
      \end{equation}
      % To show that $\iota$ is a homomorphism
      % Let $a_1,a_2\in A$.
      % Then
      % \begin{align}
      %   \label{eqn:iotahomo}
      %   \begin{split}
      %     \iota(a_1+a_2)
      %     &=(a_1+a_2-f(1,1),1)\\
      %     &=(a_1-f(1,1)+1(a_2-f(1,1))+f(1,1), 1)\\
      %     &=\iota(a_1)\iota(a_2).
      %   \end{split}
      % \end{align}
      % Now let $a\in\ker\iota$ so that
      % \begin{equation}
      %   \label{eqn:keriota}
      %   (-f(1,1),1) =\iota(a)
      %               =(a-f(1,1),1)
      % \end{equation}
      % implies that $a=0$
      % and $\iota$ is injective.
      % To see that $\iota$ maps onto $A^*$,
      % let $(a-f(1,1),1)\in A^*$.
      % Then $\iota(a)=(a-f(1,1),1)$.
      % Thus $\iota\colon A\to A^*$
      % is an isomorphism.
      Define $\pi\colon\wt{G}\to G$ by
      the projection
      $(a,g)\mapsto g$.
      Now $A^*$, the image of $\iota$,
      is contained in $\ker\pi$
      since the second coordinate is $1\in G$
      for every element of $A^*$.
      Thus
      \eqref{eqn:extensionfromcocycle}
      is an extension of $A$ by $G$.
      Lastly,
      let $s\colon G\to\wt{G}$
      be a section of $\pi$
      and let $f_s$ be the factor set
      of the extension in
      \eqref{eqn:extensionfromcocycle}.
      One can show that $f_s$ and $f$
      are equal in
      $H^2(G,A)$.
      \par
      For more of the details of this proof
      see page 827 of \cite{DF}
      and page 92 of \cite{brown}.
      % DF p.827
      % Brown p.92
    \end{proof}
    \begin{remark}
      \label{rmk:semidirectproducttrivialcocycle}
      The construction in
      Lemma \ref{lem:constructextensionfromcocycle}
      generalizes the
      semidirect product construction
      in Example \ref{exm:semidirectproduct}.
    \end{remark}
    \begin{theorem}
      \label{thm:dfthm36}
      % \mm{
      %   bijection between extensions of $A$ by $G$
      %   and $H^2(G,A)$,
      %   everything normalized
      % }
      There is a bijection between equivalence
      classes of extensions of $A$ by $G$
      as in
      \eqref{eqn:extension}
      and
      elements of $H^2(G,A)$.
    \end{theorem}
    \begin{proof}
      The least technical proof is
      by reducing to the normalized setting
      and is described in detail
      on page 826 and page 827
      in \cite{DF}.
      Below we give a summary of the proof.
      \par
      Every $2$-cocycle $f$ has a
      normalized $2$-cocycle in its
      cohomology class,
      so without loss of generality we
      can assume $f$ is normalized.
      One then shows that
      extension constructed from $f$
      using Lemma
      \ref{lem:constructextensionfromcocycle}
      has normalized factor set equal to $f$.
      The last step is showing that
      this procedure does not depend on the
      choice of normalized $2$-cocycle
      by showing that as long
      as the normalized $2$-cocycles
      are in the same cohomology class
      then the corresponding extensions
      will be equivalent.
    \end{proof}
    %NORMALIZED REMARK
    % \begin{remark}
    %   \label{rmk:normalcocyclesandfactorsets}
    %   The bijection in Theorem \ref{thm:dfthm36}
    %   is simplified by taking
    %   $2$-cocycles representing
    %   elements
    %   of $H^2(G,A)$ to be normalized.
    % \end{remark}
    Having established Theorem \ref{thm:dfthm36},
    we are interested in computing representatives
    of $H^2(G,A)$.
    To do this
    we use the implementation
    in \textsf{Magma} described in
    \cite[Cohomology and group extensions]{magmabook}.
    Describing this implementation in
    detail is beyond the scope of this work.
    Instead, we provide
    Example \ref{exm:magmaH2example}
    at the end of this section
    detailing how we use these
    implementations in practice.
    % discuss Holt's algorithm
    % to enumerate H^2(G,A) elements
    % then move on to our specific setting
    In our computation of permutation triples
    corresponding to $2$-group Belyi maps
    in the next section,
    we will first be concerned with computing
    extensions of $A$ by $G$
    where $G$ is a finite $2$-group
    and $A\simeq\ZZ/2\ZZ$.
    The first consideration in producing these
    extensions is the possible $G$-module
    structures on $A$.
    Fortunately,
    the only $G$-module structure on $A$
    is the trivial action
    corresponding to the only
    homomorphism
    \begin{equation}
      \label{eqn:trivialGmodule}
      G\to\Aut(\ZZ/2\ZZ).
    \end{equation}
    According to Theorem
    \ref{thm:dfthm36},
    the equivalent extensions of $A$ by $G$
    correspond to elements of $H^2(G,A)$
    which can be computed efficiently
    in \textsf{Magma}
    and explicitly converted to
    group extensions
    as in Example \ref{exm:magmaH2example}.
    \begin{remark}
      % \mm{decide what level of generality we want for the next section.}
      \label{rmk:abelianconsiderations}
      Modifications are
      required to compute
      extensions when
      $A$ is cyclic of prime order $p$.
      All possible homomorphisms
      $
        G\to\Aut(\ZZ/p\ZZ)\cong
        (\ZZ/p\ZZ)^\times
      $
      must be computed,
      and for each $G$-module $A$,
      the corresponding group
      $H^2(G,A)$ must also be computed.
      When $A$ has more than one cyclic factor,
      the situation becomes more complicated.
      For example,
      the possible $G$-module structures on
      $A\cong\underbrace{\ZZ/p\ZZ\times\dots\times\ZZ/p\ZZ}_{d\text{ times }}$
      correspond to
      irreducible $\FF_p[G]$-modules of dimension
      $d$.
      Although \textsf{Magma} is capable of
      computing these modules,
      we do not require this level
      of generality for the computations
      in the next section.
      % We avoid this added complexity in the next
      % section where we only consider
      % cases where $A$ is cyclic.
    \end{remark}
    We conclude this section with an example
    of how we compute group extensions
    in \textsf{Magma}.
    \begin{example}
      \label{exm:magmaH2example}
      % \mm{example of how to use \textsf{Magma} implementation in our specific setting}
      A file with the source code
      for this example can be found in
      the repository
      \cite{twogroupdessins}
      % \begin{center}
      %   \url{https://github.com/michaelmusty/2GroupDessins}
      % \end{center}
      and can be run from a shell in the
      repository as follows.
      \begin{code}
magma thesis_examples/group_extensions.m
      \end{code}
      Let $\sigma$ be the permutation triple
      representing the size $1$ passport
      $(41,G,(16,2,8))$
      % 256S100-16,2,8-g41-3759776077.m
      where $G$ is the permutation group
      generated by $\sigma$ of
      order $256$ and small group database
      label $(256,100)$.
      Suppressing the permutations in
      $\sigma$, the source code is as follows.
      \begin{magma}
...
G := sub<Sym(256)|sigma>;
assert IsTransitive(G);
assert #G eq 256;
A := TrivialModule(G, GF(2));
CM := CohomologyModule(G, A);
H2 := CohomologyGroup(CM, 2);
extensions := [* *];
for h in H2 do
  E_fp, pi_fp, iota_fp := Extension(CM, h);
  iso, E, K := CosetAction(E_fp, sub<E_fp|Id(E_fp)>);
  iotaE := iota_fp*iso;
  piE := (iso^-1)*pi_fp;
  assert Image(iotaE) eq Kernel(piE);
  assert Image(iotaE).1 in Center(E);
  Append(~extensions, [* E, iotaE, piE , h *]);
end for;
      \end{magma}
      We first construct $A$ as a $G$-module
      and construct $H^2(G,A)$
      using the \emph{Cohomology module}
      functionality in \textsf{Magma}.
      In this example $\#H^2(G,A)=32$.
      For each cohomology class,
      we compute the corresponding extension
      (as a finitely presented group)
      along with mappings defining the extension.
      Lastly, we act on the identity coset
      to obtain the extension as a permutation
      group along with the appropriate mappings.
    \end{example}
  }
  \section{An iterative algorithm to produce generating triples}{\label{sec:triplesalgorithm}
    The aim of this section is to use
    the group cohomology algorithms,
    discussed in Section \ref{sec:modcoho},
    to iteratively
    compute \emph{$p$-group permutation triples}
    which we define below.
    \begin{definition}
      \label{def:pgroupbelyitriple}
      Let $p$ be prime.
      Let $d\in\ZZ_{\geq 1}$.
      A \defi{$p$-group permutation triple}
      of degree $d$
      is a triple of permutations
      $\sigma\colonequals
      (\sigma_0,\sigma_1,\sigma_\infty)\in S_d^3$
      satisfying
      \begin{itemize}
        \item
          $\sigma_\infty\sigma_1\sigma_0=1$;
        \item
          $G\colonequals\langle\sigma\rangle$ is a transitive
          subgroup of $S_d$; and
        \item
          $G$ is a $p$-group
          of order $d$
          embedded in $S_d$
          via its left regular representation.
      \end{itemize}
      The group $G$ is called the
      \defi{monodromy group of $\sigma$}.
      We say that two $p$-group permutation triples
      $\sigma,\sigma'$ are \defi{simultaneously
      conjugate} if there exists $\tau\in S_d$ such that
      \begin{equation}\label{eqn:simconjbelyitriple}
        \sigma^\tau \colonequals
        (\tau^{-1}\sigma_0\tau, \tau^{-1}\sigma_1\tau, \tau^{-1}\sigma_\infty\tau)
        = \left(\sigma'_0,\sigma'_1,\sigma'_\infty\right)
        = \sigma'.
      \end{equation}
    \end{definition}
    \begin{remark}
      \label{rmk:leftregularrepn}
      In the process of computing extensions of
      monodromy groups of $p$-group Belyi maps
      we must pass back and forth
      between permutation groups
      and
      abstract groups given by a presentation.
      Insisting that $G$ embeds into $S_d$
      via its regular representation
      eliminates the ambiguity
      in embedding a finitely presented group
      into $S_d$.
      This explains the last property
      in Definition \ref{def:pgroupbelyitriple}.
    \end{remark}
    % \begin{lemma}
    %   \label{lem:pgroupleftregularrepn}
    %   Let $\sigma$ be a $p$-group permutation
    %   triple
    %   of degree $d$
    %   with monodromy group $G$.
    %   Then there is a unique
    %   injective
    %   homomorphism
    %   \begin{equation}
    %     \label{eqn:regularrepnunique}
    %     G\to S_d.
    %   \end{equation}
    % \end{lemma}
    \begin{example}
      \label{exm:degree1belyitriple}
      When $d=1$ we define the triple
      $(\id,\id,\id)\in S_1^3$ to be a $p$-group
      permutation triple for every $p$.
      This is the unique $p$-group permutation
      triple of degree $1$.
    \end{example}
    \begin{example}
      \label{exm:degreepbelyitriple}
      Let $d=p$ and let $\sigma_s$ be
      any $p$-cycle in $S_p$.
      Then we can write $3$ distinct
      $p$-group permutation triples
      of degree $p$:
      \begin{align}
        \label{eqn:dequalsp}
        \begin{split}
          \Big(\sigma_s,\sigma_s^{-1},\id\Big),
          \Big(\sigma_s,\id,\sigma_s^{-1}\Big),
          \Big(\id,\sigma_s,\sigma_s^{-1}\Big).
        \end{split}
      \end{align}
      These are the only $p$-group permutation
      triples
      of degree $p$ up to
      simultaneous conjugation.
    \end{example}
    We will describe the algorithms in this
    section in this slightly more general setting
    even though the $p=2$ case is our primary
    concern.
    \begin{notation}
      \label{not:oursetting}
      Let $\sigma$ be a $p$-group permutation
      triple
      with monodromy group $G$
      and
      let $A\cong\ZZ/p\ZZ$
      cyclic of prime order.
      Let $\wt{G}$ be an extension of $A$
      by $G$ sitting in the exact sequence
      \begin{equation}
        \label{eqn:motivateliftdefinition}
        \begin{tikzcd}
          1\arrow{r}&A\arrow{r}{\iota}
                    &\wt{G}\arrow{r}{\pi}
                    &G\arrow{r}
                    &1.
        \end{tikzcd}
      \end{equation}
      By Corollary \ref{cor:normalcentral}
      the image of $\iota$
      is a central subgroup of $\wt{G}$.
      % \mm{any other observations that
      % should go here?}
      The algorithm discussed in this section
      is iterative,
      and the base case for this
      iteration is described in
      Example \ref{exm:degree1belyitriple}.
    \end{notation}
    \begin{definition}
      \label{def:lift}
      Let $\sigma$ be a $p$-group permutation
      triple of degree $d$
      with monodromy group $G$.
      We say that another
      $p$-group permutation
      triple $\wt{\sigma}$
      is a \defi{$p$-lift}
      (or simply a \defi{lift})
      of $\sigma$
      if $\wt{\sigma}$ is
      a $p$-group permutation triple of degree $pd$
      with monodromy group $\wt{G}$
      sitting in the exact sequence
      in
      \eqref{eqn:motivateliftdefinition}
      with $A\cong\ZZ/p\ZZ$.
      % equipped with a
      % $G$-module structure.
    \end{definition}
    \begin{notation}
      \label{not:bipartitegraphs}
      In Algorithm \ref{alg:triples},
      the objective will be to lift a
      $p$-group permutation triple
      $\sigma$
      of degree $d$
      to $p$-group permutation
      triples $\wt{\sigma}$
      of degree $pd$.
      We will denote the set
      of lifts of $\sigma$
      by $\Lifts(\sigma)$
      and write $\Lifts(\sigma)/\!\!\sim$
      to denote the equivalence classes
      of lifts up to simultaneous conjugation
      in $S_{pd}$.
      \par
      Once we can compute
      $\Lifts(\sigma)$,
      the next objective is
      to enumerate all $p$-group permutation
      triples up to a given degree
      along with the bipartite graph
      structure determined by lifting triples.
      More precisely,
      let
      $\mathscr{G}_{p^i}$ denote the bipartite
      graph with the following node sets.
      \begin{itemize}
        \item
          $\mathscr{G}_{p^i}^\text{above}$ :
          the set of isomorphism classes
          of $p$-group permutation triples
          of degree $p^i$ indexed by
          permutation triples $\wt{\sigma}$
          up to simultaneous conjugation
          in $S_{p^i}$
        \item
          $\mathscr{G}_{p^i}^\text{below}$ :
          the set of isomorphism classes
          of $p$-group permutation triples
          of degree $p^{i-1}$ indexed
          by permutation triples $\sigma$
          up to simultaneous conjugation
          in $S_{p^{i-1}}$
      \end{itemize}
      The edge set of
      $\mathscr{G}_{p^i}$ is defined as follows.
      For every pair of nodes
      $(\wt{\sigma},\sigma)\in
      \mathscr{G}_{p^i}^\text{above}
      \times
      \mathscr{G}_{p^i}^\text{below}$
      there is an edge between
      $\wt{\sigma}$
      and $\sigma$
      if and only if
      $\wt{\sigma}$ is simultaneously conjugate
      to a lift of $\sigma$.
    \end{notation}
    Now that we have set up some notation
    and definitions,
    we now describe the algorithms.
    % \begin{notation}\label{not:triples}
      % First we suppose that we are given
      % $\sigma$ a permutation triple corresponding
      % to a $2$-group Belyi map $\phi:X\to\PP^1$.
      % \begin{definition}
      %   \label{def:lift}
      %   We say that a permutation triple $\wt{\sigma}$
      %   is a \defi{degree $2$ lift} (or simply a \defi{lift})
      %   of a permutation triple $\sigma$ if
      %   there exists a short exact sequence of groups
      %   as in Figure \ref{fig:lifttriple}
      %   with $\iota(\ZZ/2\ZZ)$ contained in the center of
      %   $\langle\wt{\sigma}\rangle$.
      % \end{definition}
      % \begin{figure}[ht]\label{fig:lifttriple}
      %   \[
      %     \begin{tikzcd}[ampersand replacement=\&, row sep = small]
      %       1\arrow{r}\&\ZZ/2\ZZ\arrow{r}{\iota}\&\langle\widetilde{\sigma}\rangle\arrow{r}{\pi}\&\langle\sigma\rangle\arrow{r}\&1
      %     \end{tikzcd}
      %   \]
      %   \caption{$\wt{\sigma}$ a lift of $\sigma$}
      % \end{figure}
      % In Algorithm \ref{alg:triples} below
      % we describe how to determine all lifts $\wt{\sigma}$ (up to isomorphism)
      % of a given permutation triple $\sigma$.
      % \begin{lemma}
      %   \label{lem:central}
      %   Let $\sigma$ be a permutation triple corresponding to a $2$-group Belyi map
      %   $\phi:X\to\PP^1$ and $\wt{\sigma}$ a lift of $\sigma$
      %   corresponding to a $2$-group Belyi map $\wt{\phi}:\wt{X}\to\PP^1$.
      %   Then there exists a permutation triple $\wt{\sigma}'$
      %   that is simultaneously conjugate to $\wt{\sigma}$ with
      %   $\iota(\langle\wt{\sigma}'\rangle)$ contained in the center of
      %   $\langle\sigma\rangle$.
      % \end{lemma}
      % \begin{proof}
      % \end{proof}
      % \begin{remark}
      %   \label{rmk:central}
      %   In light of Lemma \ref{lem:central},
      %   we can restrict our attention to central extensions of
      %   $\langle\sigma\rangle$
      %   in Definition \ref{def:lift}.
      % \end{remark}
    % \end{notation}
    \begin{alg}\label{alg:triples}
      % Let the notation be as described above in \ref{not:triples}.
      Let $p$ be prime and
      let $d\in\ZZ_{\geq 1}$.
      \newline
      \textbf{Input}:
      \begin{itemize}
        \item 
          $\sigma=(\sigma_0,\sigma_1,\sigma_\infty)\in S_d^3$ a $p$-group permutation triple
          with monodromy group $G$
        \item
          $A$ a $G$-module
      \end{itemize}
      \textbf{Output}:
      \newline
      All degree $p$ lifts $\wt{\sigma}$
      of $\sigma$ up to
      simultaneous conjugation in $S_{pd}$
      where the induced $G$-module structure on
      $A$ from
      the extension in
      \eqref{eqn:motivateliftdefinition}
      matches the $G$-module structure of $A$
      given as input.
      \begin{enumerate}
        \item
          \label{alg:triplescomputeH2reps}
          Let $G = \langle\sigma\rangle$
          and compute representatives
          of $H^2(G,A)$.
        \item
          \label{alg:triplescomputeextensions}
          For each $f\in H^2(G,A)$
          compute the corresponding
          extension
          \begin{equation}
            \label{eqn:extensiondependson2cocycle}
            \begin{tikzcd}
              1\arrow{r}&A\arrow{r}{\iota_f}
                        &\wt{G}_f\arrow{r}{\pi_f}
                        &G\arrow{r}&1
            \end{tikzcd}
          \end{equation}
        \item
          \label{alg:triplescomputelifts}
          For each extension
          $\wt{G}_f$ in
          \eqref{eqn:extensiondependson2cocycle}
          % For $s\in\{0,1,\infty\}$,
          % $\sigma_s$ has $2$ preimages under the map $\pi$
          % which yields $8$ triples $\wt{\sigma} = (\wt{\sigma}_0,\wt{\sigma}_1,\wt{\sigma}_\infty)$
          % with the property that $\pi(\wt{\sigma}_s) = \sigma_s$ for each $s$.
          compute the set
          % $\Lifts(\sigma, f)$
          % defined by
          \begin{equation}
            \label{eqn:lifts}
            \Lifts(\sigma, f)
            \colonequals
            \Big\{
              \wt{\sigma}
              % =(\wt{\sigma}_0,
              % \wt{\sigma}_1,
              % \wt{\sigma}_\infty)
              :
              \wt{\sigma}_s\in\pi_f^{-1}
              (\sigma_s)\text{ for }
              s\in\{0,1,\infty\},\;
              \wt{\sigma}_\infty
              \wt{\sigma}_1
              \wt{\sigma}_0=1,\;
              \langle\wt{\sigma}\rangle
              =\wt{G}_f
            \Big\}
          \end{equation}
        % \item
          % \label{alg:triplessortbyorders}
          % For each
          % $\wt{\sigma}\in\Lifts(\sigma)$
          % compute
          % \begin{equation}
          %   \label{eqn:ordertosortby}
          %   \order(\wt{\sigma})
          %   \colonequals
          %   (\order(\wt{\sigma}_0),
          %   \order(\wt{\sigma}_1),
          %   \order(\wt{\sigma}_\infty))
          %   \in\ZZ^3
          % \end{equation}
          % and sort
          % $\Lifts(\sigma)$
          % according to
          % $\order(\wt{\sigma})$.
          % Let
          % \begin{equation}
          %   \label{eqn:liftsorders}
          %   \Lifts(\sigma,(a,b,c))\colonequals
          %   \{
          %     \wt{\sigma}\in\Lifts(\sigma) :
          %     \order(\wt{\sigma}) = (a,b,c)
          %   \}.
          % \end{equation}
        % \item
          % For each set of triples $\Lifts(\sigma,(a,b,c))$
          % remove simultaneously conjugate triples so that
          % $\Lifts(\sigma,(a,b,c))$ has exactly one representative
          % from each simultaneous conjugacy class.
          % \mm{TODO: reword}
        % \item
          % Return the union of the sets $\Lifts(\sigma,(a,b,c))$
          % ranging over all extensions as in Figure
          % \ref{fig:centralext}
          % and for each extension ranging over all orders
          % $(a,b,c)$.
        \item
          \label{alg:triplesunionoflifts}
          Let
          \begin{equation}
            \label{eqn:unionoflifts}
            \Lifts(\sigma)
            \colonequals
            \bigcup_{f\in H^2(G,A)}
            \Lifts(\sigma, f)
          \end{equation}
        \item
          \label{alg:triplesuptoisomorphism}
          Quotient $\Lifts(\sigma)$
          by the equivalence relation
          $\sim$ identifying triples
          in $\Lifts(\sigma)$ that
          are simultaneously conjugate,
          as in \eqref{eqn:simconjbelyitriple},
          to obtain representatives of
          $\Lifts(\sigma)/\!\!\sim$.
          % Compute representatives of
          % $\Lifts(\sigma)/\!\!\sim$
          % where the equivalence relation
          % $\sim$ is simultaneous conjugation
          % of triples defined in Equation
      \end{enumerate}
    \end{alg}
    \begin{proof}[Proof of correctness]
      The computation of $H^2(G,A)$ is described
      in \cite{magmabook} and implemented in
      \cite{magma}.
      Theorem \ref{thm:dfthm36}
      in Section \ref{sec:modcoho} implies
      the following.
      \begin{itemize}
        \item
          The elements of $H^2(G,A)$
          are in bijection
          with extensions
          $\wt{G}_f$
          as in
          \eqref{eqn:extensiondependson2cocycle}.
        \item
          Any lift of $\sigma$
          inducing the $G$-module
          structure of $A$ on $\ZZ/p\ZZ$
          must have monodromy group sitting
          in an exact sequence obtained in
          Step \ref{alg:triplescomputeextensions}.
      \end{itemize}
      In Step \ref{alg:triplescomputelifts}
      all possible lifts of $\sigma$
      for a single extension $\wt{G}_f$
      are computed.
      This is done by computing
      all $(\#A)^3$
      triples mapping to
      $\sigma$ under $\pi_f$
      and checking which satisfy
      the conditions to be a lift of
      $\sigma$.
      After collecting all the lifts together
      in Step
      \ref{alg:triplesunionoflifts}
      it is possible there are
      simultaneously conjugate
      $p$-group permutation triples
      in $\Lifts(\sigma)$.
      In Step
      \ref{alg:triplesuptoisomorphism}
      we quotient by
      simultaneous conjugation
      to obtain the desired
      set of lifts as output.
        % The algorithms in Step 1 are addressed in Section \ref{subsec:holtsalgorithm}.
        % Let $\phi:X\to\PP^1$ be the $2$-group Belyi map corresponding to $\sigma$.
        % By Proposition \ref{prop:galoiscorrespondence},
        % the groups obtained from Step 1 are precisely the
        % groups that can occur as monodromy groups of degree $2$ covers of $X$.
        % \mm{
        %   lemma in section about extensions
        %   (or in background about Belyi maps)
        %   to prove
        %   that two isomorphic extensions cannot produce
        %   nonisomorphic Belyi maps
        %   and that two nonisomorphic extensions cannot produce
        %   isomorphic Belyi maps
        % }
        % In Step 2 we restrict our attention to a single extension of $G$
        % as in Figure \ref{fig:centralext}.
        % When we pullback a triple $\sigma$ under the map $\pi$,
        % there are $2^3=8$ preimages $\wt{\sigma}$.
        % Of these $8$ preimages, exactly $4$ have the property that
        % $\wt{\sigma}_\infty\wt{\sigma}_1\wt{\sigma}_0=1$.
        % Of these $4$ triples, we only take those that generate $\wt{G}$
        % and this makes up the set $\Lifts(\sigma)$.
        % In Step 2(b),
        % we are sorting $\Lifts(\sigma)$ by passport.
        % Since $2$-group Belyi maps are Galois,
        % the cycle structure of each $\wt{\sigma}_s\in\wt{\sigma}$
        % is determined by the order of $\wt{\sigma}_s$
        % so that sorting by order is the same as sorting by cycle structure.
        % At this point,
        % we have constructed the sets $\Lifts(\sigma,(a,b,c))$.
        % In light of Remark \ref{rmk:twouniquepassports},
        % there are only $2$ possibilities:
        % \begin{itemize}
        %   \item
        %     There is only one such set $\Lifts(\sigma, (a,b,c))$
        %     consisting of at most $4$ triples.
        %   \item
        %     There are $2$ sets $\Lifts(\sigma, (a,b,c))$ and $\Lifts(\sigma, (a',b',c'))$
        %     each consisting of at most $2$ triples.
        % \end{itemize}
        % Step 2(c) is to eliminate simultaneous conjugation in each
        % set $\Lifts(\sigma, (a,b,c))$.
        % After Step 2(c) is complete,
        % the sets $\Lifts(\sigma, (a,b,c))$ contain exactly one permutation triple
        % for each isomorphism class of $2$-group Belyi map with passport determined by
        % $(a,b,c)$ and monodromy group $\wt{G}$ such that the diagram in
        % Figure \ref{fig:liftbelyimap} commutes.
        % \begin{figure}[ht]
        %   \[
        %     \begin{tikzcd}[ampersand replacement=\&]
        %       \widetilde{X}\arrow{dd}[swap]{\widetilde{\phi}}\arrow{dr}{2}\\
        %       \&X\arrow{dl}{\phi}\\
        %       \PP^1
        %     \end{tikzcd}
        %   \]
        %   \caption{
        %     The permutation triples $\wt{\sigma}$ constructed in
        %     Algorithm \ref{alg:triples}
        %     correspond to Belyi maps $\wt{\phi}:\wt{X}\to\PP^1$
        %     in the above diagram.
        %   }
        %   \label{fig:liftbelyimap}
        % \end{figure}
        % In Step 3 we collect together all sets $\Lifts(\sigma, (a,b,c))$
        % as we range over all possible extensions in Step 1,
        % and by the discussion for Step 2 yields the desired output.
    \end{proof}
    % \begin{remark}\label{rmk:twouniquepassports}
      % In fact,
      % even though we do not need this for the algorithm,
      % there are at most $2$ different passports that can occur
      % in $\Lifts(\sigma)$.
      % $2$ different passports occur when one of $\sigma_s\in\sigma$
      % is the identity.
      % If $\sigma$ does not contain an identity element,
      % then all triples in $\Lifts(\sigma)$ have the same passport.
    % \end{remark}
    Algorithm \ref{alg:triples}
    reduces the problem of finding all lifts
    of a given $p$-group permutation triple
    $\sigma$ to determining all possible
    $\langle\sigma\rangle$-module
    structures on $\ZZ/p\ZZ$.
    Although computations of this sort
    are implemented
    in \cite{magma},
    it is especially easy to do
    when $p=2$.
    \begin{lemma}
      \label{lem:trivialGmoduleonly}
      Let $G$ be a finite group.
      The only $G$-module structure on
      $\ZZ/2\ZZ$ is trivial.
    \end{lemma}
    \begin{proof}
      A $G$-module structure on $\ZZ/2\ZZ$
      is a homomorphism
      from $G$ to $\Aut(\ZZ/2\ZZ)$.
      But $\Aut(\ZZ/2\ZZ)\cong(\ZZ/2\ZZ)^\times$
      which is the trivial group,
      so there is only one such
      homomorphism.
    \end{proof}
    For the rest of this section we suppose
    that $p=2$.
    In this special case,
    Algorithm
    \ref{alg:triples}
    does not require
    a $G$-module as input
    since
    (by Lemma
    \ref{lem:trivialGmoduleonly})
    the trivial $G$-module structure
    on $\ZZ/2\ZZ$ can be assumed.
    % \mm{
    %   can 2 lifts of the
    %   \emph{same} triple downstairs
    %   but \emph{different} extensions
    %   yield isomorphic triples upstairs?
    %   I don't think this can happen,
    %   but I guess it doesn't matter
    %   \begin{itemize}
    %     \item
    %       make lifts for all
    %       $\sigma$ downstairs
    %     \item
    %       then compare every pair of lifts?
    %       overnight up to d256
    %   \end{itemize}
    % }
    \begin{remark}
      \label{rmk:distinguishedinvolution}
      Suppose $p=2$ using Notation
      \ref{not:oursetting}.
      Then $\iota(A)$ is an
      order $2$ normal subgroup of $\wt{G}$.
      Let $\alpha$ denote the generator of
      $\iota(A)$.
      From the perspective of branched covers,
      $\alpha$ is identifying $2d$ sheets
      in a degree $2d$ cover down to $d$ sheets
      in a degree $d$ cover.
      To relate the degree $2d$ cover
      corresponding to $\wt{G}$
      with the
      degree $d$ cover
      corresponding to $G$
      it is convenient to choose $\alpha$
      to be the following product of $d$
      transpositions.
      \begin{equation}
        \label{eqn:alpha}
        \alpha\colonequals
        (1\;d+1)(2\;d+2)\dots
        (d-1\;2d-1)(d\;2d)
      \end{equation}
      The benefit of following this convention
      can be seen in Example
      \ref{exm:lift}
      where we
      illustrate Algorithm \ref{alg:triples}.
    \end{remark}
    \begin{example}\label{exm:lift}
      In this example we carry out Algorithm
      \ref{alg:triples} for
      the degree $2$ permutation triple
      $\sigma = ((1\,2),\id,(1\,2))$.
      Here $G = \langle\sigma\rangle\cong\ZZ/2\ZZ$.
      In
      Algorithm \ref{alg:triples}
      Step \ref{alg:triplescomputeextensions},
      we obtain two group extensions
      $\wt{G}_1\cong\ZZ/2\ZZ\times\ZZ/2\ZZ$
      and
      $\wt{G}_2\cong\ZZ/4\ZZ$
      sitting in the following
      exact sequences.
      \begin{align}
        \label{eqn:liftexampleextensions}
        \begin{split}
          \begin{tikzcd}[ampersand replacement=\&, row sep = small]
            1\arrow{r}\&\ZZ/2\ZZ\arrow{r}{\iota_1}\&\wt{G}_1\arrow{r}{\pi_1}\&G\arrow{r}\&1
          \end{tikzcd}
          \\
          \begin{tikzcd}[ampersand replacement=\&, row sep = small]
            1\arrow{r}\&\ZZ/2\ZZ\arrow{r}{\iota_2}\&\wt{G}_2\arrow{r}{\pi_2}\&G\arrow{r}\&1
          \end{tikzcd}
        \end{split}
      \end{align}
      We will consider the
      two extensions separately.
      \begin{itemize}
        \item
          For $\wt{G}_1$,
          we can look at preimages of
          $\sigma_s$ under the map $\pi_1$
          to obtain $4$ triples that
          multiply to the identity:
          \begin{align}
            \label{eqn:examplelifts1}
            \begin{split}
            % \Lifts(\sigma) =
            &\Big\{
              ((1\,2)(3\,4), \id, (1\,2)(3\,4)),
              ((1\,2)(3\,4), (1\,3)(2\,4), (1\,4)(2\,3)),\\
            &((1\,4)(2\,3), \id, (1\,4)(2\,3)),
              ((1\,4)(2\,3), (1\,3)(2\,4), (1\,2)(3\,4))
            \Big\}
            \end{split}
          \end{align}
          Before we continue with the algorithm,
          let us take a moment to
          analyze these triples
          more closely.
          The generator $\alpha$
          of $\iota(\ZZ/2\ZZ)$
          in $\wt{G}_1$ is $(1\,3)(2\,4)$.
          Each triple in
          \eqref{eqn:examplelifts1}
          must act on the blocks
          $\left\{\boxed{1\,3},\boxed{2\,4}\right\}$
          so that the induced permutations of these
          blocks is the same as the
          corresponding permutation in $\sigma$.
          For
          \begin{equation}
            \label{eqn:examplelifts1triple}
            (\wt{\sigma}_0,\wt{\sigma}_1
            ,\wt{\sigma}_\infty)=
            ((1\,2)(3\,4),
            (1\,3)(2\,4), (1\,4)(2\,4))
          \end{equation}
          we have
          $\wt{\sigma}_0\Big(\boxed{1\,3}\Big)
          = \boxed{2\,4}$
          and
          $\wt{\sigma}_0\Big(\boxed{2\,4}\Big)
          = \boxed{1\,3}$
          so that the induced permutation
          of blocks is
          \begin{equation}
            \label{eqn:inducedpermutationofblocks}
              \left(\boxed{1\,3},\boxed{2\,4}\right)
          \end{equation}
          which is the same as the
          permutation $\sigma_0 = (1\,2)$
          (as long as we identity
          $\boxed{1\,3}$ with $1$
          and $\boxed{2\,4}$ with $2$).
          Insisting $\alpha$
          has the form in Remark
          \ref{rmk:distinguishedinvolution}
          allows us to label blocks
          by reducing modulo $d$
          as in
          \eqref{eqn:inducedpermutationofblocks}.
          The last requirement for a triple
          $\wt{\sigma}$
          in Equation \ref{eqn:examplelifts1}
          to be in $\Lifts(\sigma,\wt{G}_1)$
          is that $\wt{\sigma}$ generates
          $\wt{G}_1$.
          We obtain
          $\Lifts(\sigma,\wt{G}_1)$
          to be
          \begin{align}
            \label{eqn:liftssigma1}
            \Big\{
              ((1\,2)(3\,4), (1\,3)(2\,4), (1\,4)(2\,3)),
              ((1\,4)(2\,3), (1\,3)(2\,4), (1\,2)(3\,4))
            \Big\}
          \end{align}
        \item
          For $\wt{G}_2$, we obtain
          $\Lifts(\sigma,\wt{G}_2)$ to be
          \begin{align}
            \label{eqn:examplelifts2}
            \begin{split}
            &\Big\{
              ((1\,4\,3\,2), \id, (1\,2\,3\,4)),
              ((1\,2\,3\,4), (1\,3)(2\,4), (1\,2\,3\,4)),\\
            &((1\,2\,3\,4), \id, (1\,4\,3\,2)),
              ((1\,4\,3\,2), (1\,3)(2\,4), (1\,4\,3\,2))
            \Big\}
            \end{split}
          \end{align}
      \end{itemize}
      At the end of
      % Algorithm \ref{alg:triples}
      Step \ref{alg:triplesunionoflifts}
      we have that
      $\Lifts(\sigma)$
      contains the $2$ triples in
      \eqref{eqn:liftssigma1}
      and the $4$ triples in
      \eqref{eqn:examplelifts2}.
      Lastly, in Step
      \ref{alg:triplesuptoisomorphism}
      we quotient by simultaneous conjugation
      to obtain the $3$ triples
      \begin{align}
        \label{eqn:exampleoutput}
        \begin{split}
          \Lifts(\sigma)/\!\!\sim
          \;=\Big\{
            &((1\,2)(3\,4),(1\,3)(2\,4),(1\,4)(2\,3)),\\
            &((1\,4\,3\,2),\id,(1\,2\,3\,4)),\\
            &((1\,2\,3\,4),(1\,3)(2\,4),(1\,2\,3\,4))
          \Big\}
        \end{split}
      \end{align}
      as output.
    \end{example}
    Now that we have an algorithm to
    find all lifts of a single
    permutation triple,
    we now describe how to use this
    to compute all isomorpism classes
    of $2$-group permutation triples
    up to a given degree.
    In the algorithms to follow,
    we are concerned with constructing
    the bipartite graphs
    $\mathscr{G}_{2^i}$
    defined in
    Notation \ref{not:bipartitegraphs}.
    \begin{alg}
      \label{alg:alltriplesbasecase}
      Let $p=2$ and the notation be as in
      \ref{not:oursetting} and
      \ref{not:bipartitegraphs}.
      Then we can construct
      $\mathscr{G}_2$ as follows.
      \begin{itemize}
        \item
          \label{alg:alltriplesbasecasebelow}
          The set of nodes
          $\mathscr{G}_2^\text{below}$
          consists of a single triple
          $(\id,\id,\id)\in S_1^3$
        \item
          \label{alg:alltriplesbasecaseabove}
          The set of nodes
          $\mathscr{G}_2^\text{above}$
          consists of $3$ triples
          described in
          Example
          \ref{exm:degreepbelyitriple}.
        \item
          \label{alg:alltriplesbasecaseedges}
          The edge set of $\mathscr{G}_2$
          consists of $3$ edges
          (i.e. it is the complete
          bipartite graph for the sets
          $\mathscr{G}_2^\text{below}$
          and
          $\mathscr{G}_2^\text{above}$
          )
      \end{itemize}
    \end{alg}
    \begin{proof}[Proof of correctness]
      By definition,
      the $3$ degree $2$ permutation triples
      from Example \ref{exm:degreepbelyitriple}
      are the only $2$-group permutation triples
      of degree $2$.
      These are all lifts of the unique
      $2$-group permutation triple
      (in Example \ref{exm:degree1belyitriple})
      via the extension
      \begin{equation}
        \label{eqn:trivialliftdegree2}
        \begin{tikzcd}
          1\arrow{r}&\ZZ/2\ZZ\arrow{r}{\id}
                    &\ZZ/2\ZZ\arrow{r}{\pi}
                    &\{\id\}\arrow{r}&1
        \end{tikzcd}
      \end{equation}
    \end{proof}
    Having constructed $\mathscr{G}_2$,
    we now describe the iterative process to
    compute $\mathscr{G}_{2^i}$
    from $\mathscr{G}_{2^{i-1}}$.
    \begin{alg}
      \label{alg:alltriplesiteration}
      Let $p=2$ and the notation be as in
      \ref{not:oursetting} and
      \ref{not:bipartitegraphs}.
      This algorithm describes the process
      of computing $\mathscr{G}_{2^i}$
      given $\mathscr{G}_{2^{i-1}}$.
      \newline
      \textbf{Input}:
      The bipartite graph
      $\mathscr{G}_{2^{i-1}}$
      \newline
      \textbf{Output}:
      The bipartite graph
      $\mathscr{G}_{2^{i}}$
      \begin{enumerate}
        \item
          \label{alg:alltriplesiterationcomputelifts}
          For every
          $\sigma\in\mathscr{G}_{2^{i-1}}^\text{above}$
          apply Algorithm \ref{alg:triples}
          to obtain the set
          $\Lifts(\sigma)/\!\!\sim$ for each
          $\sigma$.
          Combine these lifts into a single set
          \begin{equation}
            \label{eqn:unionofliftsabove}
            % \Lifts(\mathscr{G}_{2^{i-1}}^\text{above})
            \Lifts(\mathscr{G}_{2^{i-1}})
            \colonequals\bigcup_{\sigma\in\mathscr{G}_{2^{i-1}}^\text{above}}\Lifts(\sigma)
          \end{equation}
        \item
          \label{alg:alltriplesiterationmodout}
          Compute
          $\Lifts(\mathscr{G}_{2^{i-1}})/\!\!\sim$
          which we define to be the equivalence
          classes of
          $\Lifts(\mathscr{G}_{2^{i-1}})$
          where two triples
          $\wt{\sigma}$ and $\wt{\sigma}'$
          in
          $\Lifts(\mathscr{G}_{2^{i-1}})$
          are equivalent
          if and only if
          they are simultaneously
          conjugate in $S_{2^{i}}$.
          Denote the equivalence class
          of $\wt{\sigma}\in
          \Lifts(\mathscr{G}_{2^{i-1}})$
          by $[\wt{\sigma}]\in
          \Lifts(\mathscr{G}_{2^{i-1}})/\!\!\sim$.
        \item
          \label{alg:alltriplesiterationnodes}
          Define
          $\mathscr{G}_{2^i}^\text{below}
          \colonequals\mathscr{G}_{2^{i-1}}^\text{above}$.
          Define
          $\mathscr{G}_{2^i}^\text{above}$
          by choosing a single representative
          for each equivalence class of
          $\Lifts(\mathscr{G}_{2^{i-1}})/\!\!\sim$.
          This defines
          the nodes of
          $\mathscr{G}_{2^i}$.
        \item
          \label{alg:alltriplesiterationedges}
          For every pair
          $(\wt{\sigma},\sigma)\in
          \mathscr{G}_{2^i}^\text{above}
          \times
          \mathscr{G}_{2^i}^\text{below}$
          place an edge between
          $\wt{\sigma}$ and $\sigma$
          if and only if
          there is a triple in the equivalence
          class $[\wt{\sigma}]\in
          \Lifts(\mathscr{G}_{2^{i-1}})/\!\!\sim$
          that is a lift of $\sigma$.
        \item
          \label{alg:alltriplesiterationreturn}
          Return
          $\mathscr{G}_{2^i}$
          as output.
      \end{enumerate}
    \end{alg}
    \begin{proof}[Proof of correctness]
      % the "heavy lifting" is done by alg:triples
      Since $2$-groups are nilpotent,
      every
      $2$-group permutation triple
      of degree $2^i$
      is the lift of at least one
      $2$-group permutation triple
      of degree $2^{i-1}$.
      Let
      $\wt{\sigma}\in\mathscr{G}_{2^i}^\text{above}$
      be an arbitrary representative
      of an isomorphism class of
      $2$-group permutation triples of degree $2^i$
      contained in $\Lifts(\sigma)$
      for some degree $2^{i-1}$ triple $\sigma$.
      Let $\sigma'$ denote the representative
      in $\mathscr{G}_{2^{i-1}}^\text{above}$
      that is simultaneously conjugate to
      $\sigma$.
      Algorithm \ref{alg:triples}
      ensures that there is a
      $2$-group permutation triple $\wt{\sigma}'$
      of degree $2^i$ in
      $\Lifts(\sigma')$
      that is simultaneously conjugate
      to $\wt{\sigma}$.
      Thus,
      $\Lifts(\mathscr{G}_{2^{i-1}})$
      computed in Step
      \ref{alg:alltriplesiterationcomputelifts}
      contains at least one 
      triple for every
      isomorphism class of $2$-group
      permutation triples of degree $2^i$.
      It is,
      however,
      possible for
      $\Lifts(\mathscr{G}_{2^{i-1}})$
      to contain simultaneously conjugate
      triples arising as lifts of
      different triples in
      $\mathscr{G}_{2^{i-1}}^\text{above}$.
      Step
      \ref{alg:alltriplesiterationmodout}
      quotients
      $\Lifts(\mathscr{G}_{2^{i-1}})$
      by simultaneous conjugation
      and
      Steps
      \ref{alg:alltriplesiterationnodes}
      and
      \ref{alg:alltriplesiterationedges}
      define the desired graph
      $\mathscr{G}_{2^i}$ in such a way
      that the edge structure of the lifts
      is preserved.
    \end{proof}
    % \begin{alg}\label{alg:alltriples}
      % Let the notation be as in \ref{not:oursetting}
      % and let $p=2$.
      % \newline
      % \textbf{Input}: $d = 2^m$ for
      % some positive integer $m$
      % \newline
      % \textbf{Output}: a sequence of
      % bipartite graphs
      % $\mathscr{G}_1, \mathscr{G}_2,
      % \mathscr{G}_4,\dots,\mathscr{G}_{2^m}$
      % where the two sets of nodes
      % of $\mathscr{G}_{2^i}$
      % are
      % \begin{itemize}
      %   \item
      %     $\mathscr{G}_{2^i}^\text{above}$ :
      %     the set of isomorphism classes of $2$-group Belyi maps
      %     of degree $2^i$ indexed by permutation triples $\wt{\sigma}$
      %   \item
      %     $\mathscr{G}_{2^i}^\text{below}$ :
      %     the set of isomorphism classes of $2$-group Belyi maps
      %     of degree $2^{i-1}$ indexed by permutation triples $\sigma$
      % \end{itemize}
      % and there is an edge between
      % $\wt{\sigma}$ and $\sigma$
      % if and only if $\wt{\sigma}$ is a lift
      % of $\sigma$.
      % % Moreover,
      % % for each $i$,
      % % we have
      % % $\mathscr{G}_{2^{i-1}}^\text{above} = \mathscr{G}_{2^i}^\text{below}$.
      % This algorithm is iterative.
      % For each $i=1,\dots,m$,
      % we use $\mathscr{G}_{2^i}^\text{below}$ to
      % compute $\mathscr{G}_{2^i}^\text{above}$
      % and then we define
      % \[
      %   \mathscr{G}_{2^{i+1}}^\text{below}\colonequals\mathscr{G}_{2^i}^\text{above}
      % \]
      % and continue the process.
      % \begin{enumerate}
      %   \item
      %     To begin the iteration we
      %     let $\mathscr{G}_2^\text{below}$ = $\{\sigma\}$
      %     where $\sigma = ((1),(1),(1))\in S_1^3$
      %     corresponds to the degree $1$ Belyi map.
      %   \item
      %     Now suppose we have computed $\mathscr{G}_{2^i}^\text{below}$.
      %     We compute $\mathscr{G}_{2^i}^\text{above}$ as follows:
      %     \begin{enumerate}
      %       \item
      %         Apply Algorithm \ref{alg:triples} to every
      %         $\sigma\in\mathscr{G}_{2^i}^\text{below}$
      %         to obtain
      %         $\#\mathscr{G}_{2^i}^\text{below}$
      %         sets
      %         $\Lifts(\sigma)$.
      %         As a word of caution,
      %         the notation $\Lifts(\sigma)$ has
      %         a different meaning here than in
      %         Algorithm \ref{alg:triples}.
      %         Here $\Lifts(\sigma)$ is the set of
      %         lifts of $\sigma$ up to simultaneous
      %         conjugation.
      %         Let
      %         \[
      %           \mathscr{G}_{2^i}^\text{above}
      %           \colonequals
      %           \bigcup_{\sigma\in\mathscr{G}_{2^i}^\text{below}}
      %           \Lifts(\sigma)
      %         \]
      %         and place an edge 
      %         of $\mathscr{G}_{2^i}$
      %         between
      %         $\wt{\sigma}\in\mathscr{G}_{2^i}^\text{above}$
      %         and
      %         $\sigma\in\mathscr{G}_{2^i}^\text{below}$
      %         if and only if $\wt{\sigma}\in\Lifts(\sigma)$.
      %       \item
      %         Consider all pairs $(\wt{\sigma},\wt{\sigma}')\in\mathscr{G}_{2^i}^\text{above}$
      %         and for each pair
      %         test if $\wt{\sigma}$ is simultaneously conjugate to
      %         $\wt{\sigma}'$ in $S_{2^i}$.
      %         If the pair is simultaneously conjugate,
      %         then combine the nodes $\wt{\sigma}$ and $\wt{\sigma}'$
      %         into a single node (take either triple)
      %         and combine the edge sets of $\wt{\sigma}$
      %         and $\wt{\sigma}'$ to be the edge set of the new node.
      %       \item
      %         Return the resulting bipartite graph as
      %         $\mathscr{G}_{2^i}$.
      %       \item
      %         If $i<m$, then let
      %         $\mathscr{G}_{2^{i+1}}^\text{below}\colonequals\mathscr{G}_{2^i}^\text{above}$
      %         and repeat Step 2 with $i+1$.
      %         If $i=m$,
      %         then return the sequence of bipartite graphs
      %         $\mathscr{G}_2,\mathscr{G}_4,\dots,\mathscr{G}_{2^m}$.
      %     \end{enumerate}
      % \end{enumerate}
    % \end{alg}
    % \begin{proof}[Proof of correctness]
      %% first use 2-groups and Galois correspondence to see we get all iso classes
      %We first address the claim that
      %every $2$-group Belyi map $\phi:X\to\PP^1$ of degree $2^i$
      %is represented by a permutation triple in
      %$\mathcal{G}_{2^i}^\text{above}$.
      %Let $G$ be the monodromy group of $\phi$.
      %Since $\#G=2^i$,
      %by Lemma \ref{lem:2groupfiltration},
      %there exists a normal tower of groups
      %\begin{equation}\label{eqn:2groupfiltration}
      %  G_0\trianglelefteq G_1\trianglelefteq\dots
      %  \trianglelefteq G_i
      %\end{equation}
      %where $G_0 = \{1\}$, $G_i = G$,
      %and each consecutive quotient is isomorphic
      %to $\ZZ/2\ZZ$.
      %By the Galois correspondence,
      %Proposition \ref{prop:galoiscorrespondence},
      %this normal tower of groups corresponds to
      %the diagram in Figure \ref{fig:2groupfiltration}.
      %\begin{figure}[ht]
      %  \begin{center}
      %    \begin{tikzpicture}[scale=0.3]
      %      \node at (0,0) (X0) {$X_0 = \PP^1$};
      %      \node [above left=of X0] (X1) {$X_1$};
      %      \node [above =of X1] (X2) {$X_2$};
      %      \node [above =of X2] (dots) {$\vdots$};
      %      \node [above =of dots] (Xr) {$X_i=X$};
      %      %\draw[<->] (X1) to node [above,rotate=-35] {$\sim$} node {} (X0);
      %      \draw[->] (X1) to node [below] {$2$} node [right] {$\phi_1$} (X0);
      %      \draw[->] (X2) to node [left] {$2$} node {} (X1);
      %      \draw[->, bend left] (X2) to node [below, right] {$\phi_2$} node {} (X0);
      %      \draw[->] (dots) to node [left] {$2$} node {} (X2);
      %      \draw[->] (Xr) to node [left] {$2$} node {} (dots);
      %      \draw[->, bend left] (Xr) to node [below, right] {$\phi_i=\phi$} node {} (X0);
      %      %\node at (8,0) (KX0) {$K(X_1) = K(\PP^1)$};
      %      %\node [above left=of KX0] (KX1) {$K(X_2)$};
      %      %\node [above =of KX1] (KX2) {$K(X_3)$};
      %      %\node [above =of KX2] (dots) {$\vdots$};
      %      %\node [above =of dots] (KXr) {$K(X_r)$};
      %      %%\draw[<->] (X1) to node [above,rotate=-35] {$\sim$} node {} (X0);
      %      %\draw[-] (KX1) to node [below] {$2$} node {} (KX0);
      %      %\draw[-] (KX2) to node [left] {$2$} node {} (KX1);
      %      %\draw[-, bend left] (KX2) to node [below, left] {} node {} (KX0);
      %      %\draw[-] (dots) to node [left] {$2$} node {} (KX2);
      %      %\draw[-] (KXr) to node [left] {$2$} node {} (dots);
      %      %\draw[-, bend left] (KXr) to node [below, left] {} node {} (KX0);
      %    \end{tikzpicture}
      %  \end{center}
      %  \caption{
      %    A $2$-group Belyi map $\phi$
      %    as a sequence of degree $2$ covers.
      %    For $j\in\{1,\dots,i\}$,
      %    $\phi$ factors through
      %    a degree $2^j$ Belyi map denoted $\phi_j$.
      %  }
      %  \label{fig:2groupfiltration}
      %\end{figure}
      %Let $\sigma_j$ be the permutation triple corresponding to
      %$\phi_j$ in Figure \ref{fig:2groupfiltration}.
      %Applying Algorithm \ref{alg:triples} to $\sigma_j$
      %we obtain $\sigma_{j+1}$ as a lift of $\sigma_j$
      %so that the permutation triple corresponding to $\phi$
      %appears in $\mathscr{G}_{2^i}^\text{above}$.
      %This shows that every $2$-group Belyi map
      %of degree $2^i$ is represented by at least one node in
      %$\mathscr{G}_{2^i}$.
      %We now claim that every $2$-group Belyi map of degree $2^i$
      %is represented by exactly one node in $\mathscr{G}_{2^i}$.
      %% then show why only one in each iso class
      %Since we are applying Algorithm \ref{alg:triples}
      %to every permutation triple in $\mathscr{G}_{2^i}^\text{below}$,
      %it is possible that in Step 2(a),
      %the set of permutation triples in $\mathscr{G}_{2^i}^\text{above}$
      %has simultaneously conjugate triples
      %which arise when a degree $2^i$ Belyi map
      %is a degree $2$ cover of more than one
      %nonisomorphic Belyi map of degree $2^{i-1}$.
      %In Step 2(b),
      %we combine permutation triples in $\mathscr{G}_{2^i}^\text{above}$
      %that are simultaneously conjugate by taking a single
      %permutation triple to represent this isomorphism class
      %of $2$-group Belyi map.
      %Note that in Step 2(b)
      %we never remove any edges in the graph
      %$\mathscr{G}_{2^i}$.
      %It follows from Step 2(b)
      %that $\mathscr{G}_{2^i}^\text{above}$ has at most
      %one node for each
      %$2$-group Belyi map
      %isomorphism class of degree $2^i$.
    % \end{proof}
    % \mm{
    %   In your comments you
    %   mention an possibly doing
    %   an example...maybe write out the bipartite
    %   graphs up to degree 8?
    % }
    Algorithm
    \ref{alg:alltriplesbasecase}
    combined with
    Algorithm
    \ref{alg:alltriplesiteration}
    allows us to compute
    \begin{equation}
      \label{eqn:sequenceofgraphs}
      \mathscr{G}_2,
      \mathscr{G}_4,
      \dots,
      \mathscr{G}_{2^i},
      \dots,
      \mathscr{G}_{2^m}
    \end{equation}
    up to any degree $d=2^m$.
    A \textsf{Magma} implementation of
    Algorithms
    \ref{alg:triples},
    \ref{alg:alltriplesbasecase},
    and
    \ref{alg:alltriplesiteration}
    can be found at
    \cite{twogroupdessins}.
    % \begin{center}
    %   \url{https://github.com/michaelmusty/2GroupDessins}
    % \end{center}
    In the next section we discuss
    the results of these computations.
  }
  \section{Results of computations}{\label{sec:grouptheorycomputations}
    % how long did it take?
    % some coarse data analysis
    % what (a,b,c) did we get
    % how many permutations, passports, (a,b,c) for each degree
    % graph distribution of genera
    % hyperbolic vs. nonhyperbolic
    % anything else you can say using just the group theory
    % Paulhus code. . .
    In this section we discuss the
    \textsf{Magma} implementation of
    Algorithms
    \ref{alg:triples},
    \ref{alg:alltriplesbasecase},
    and
    \ref{alg:alltriplesiteration}
    available at
    \cite{twogroupdessins}
    % \url{https://github.com/michaelmusty/2GroupDessins}
    where the techniques of this chapter
    are used to tabulate a database
    of $2$-group permutation triples
    up to degree $256$.
    This computation took roughly
    $50$ CPU hours on a standard desktop.
    The majority
    of this time is spent
    checking conjugacy of degree $256$
    permutation triples.
    This database consists of
    roughly $340$MB worth of text files.
    We devote the rest of this section
    to summarizing the results of
    these computations.
    \begin{theorem}\label{thm:isoclasses}
      The following table lists
      the number of isomorphism classes of
      $2$-group permutation triples
      of degree $d$ up to $256$.
      \begin{equation}
        \label{eqn:isoclasses}
        \begin{tabular}{|c||c|c|c|c|c|c|c|c|c|}
          \hline
          $d$ & $1$ & $2$ & $4$ & $8$ & $16$ & $32$ & $64$ & $128$ & $256$\\
          \hline
          \#\text{ permutation triples } & $1$ & $3$ & $7$ & $19$ & $55$ & $151$ & $503$ & $1799$ & $7175$\\
          \hline
        \end{tabular}
      \end{equation}
    \end{theorem}
    \begin{theorem}\label{thm:passports}
      The following table lists
      the number of passports of
      $2$-group permutation triples
      of degree $d$ up to $256$.
      \begin{equation}
        \label{eqn:}
        \begin{tabular}{|c||c|c|c|c|c|c|c|c|c|}
          \hline
          $d$ & $1$ & $2$ & $4$ & $8$ & $16$ & $32$ & $64$ & $128$ & $256$\\
          \hline
          \# passports & $1$ & $3$ & $7$ & $16$ & $41$ & $96$ & $267$ & $834$ & $2893$\\
          \hline
        \end{tabular}
      \end{equation}
    \end{theorem}
    \begin{theorem}\label{thm:laxpassports}
      The following table lists
      the number of lax passports of
      $2$-group permutation triples
      of degree $d$ up to $256$.
      \begin{equation}
        \label{eqn:}
        \begin{tabular}{|c||c|c|c|c|c|c|c|c|c|}
          \hline
          $d$ & $1$ & $2$ & $4$ & $8$ & $16$ & $32$ & $64$ & $128$ & $256$\\
          \hline
          \# lax passports & $1$ & $1$ & $3$ & $6$ & $14$ & $31$ & $85$ & $257$ & $882$\\
          \hline
        \end{tabular}
      \end{equation}
    \end{theorem}
    % \begin{longtable}{| p{.20\textwidth} | p{.80\textwidth} |} 
    \begin{theorem}\label{thm:abc}
      The following table lists
      the number of $2$-group
      permutation triples
      up to degree $256$ with
      $\{\order(\sigma_s) : s\in\{0,1,\infty\}\}$
      equal to $\{a,b,c\}$ as sets.
      \begin{longtable}{|l|l|} 
        \hline
        $(a,b,c)$ & \# permutation triples \\
        \hline
        \hline
        $(1,1,1)$ & $1$ \\ \hline
        $(1,2,2)$ & $3$ \\ \hline
        $(1,4,4)$ & $3$ \\ \hline
        $(1,8,8)$ & $3$ \\ \hline
        $(1,16,16)$ & $3$ \\ \hline
        $(1,32,32)$ & $3$ \\ \hline
        $(1,64,64)$ & $3$ \\ \hline
        $(1,128,128)$ & $3$ \\ \hline
        $(1,256,256)$ & $3$ \\ \hline
        $(2,2,2)$ & $1$ \\ \hline
        $(2,2,4)$ & $24$ \\ \hline
        $(2,2,8)$ & $132$ \\ \hline
        $(2,2,16)$ & $144$ \\ \hline
        $(2,2,32)$ & $60$ \\ \hline
        $(2,2,64)$ & $24$ \\ \hline
        $(2,2,128)$ & $12$ \\ \hline
        $(2,4,4)$ & $24$ \\ \hline
        $(2,4,8)$ & $78$ \\ \hline
        $(2,4,16)$ & $78$ \\ \hline
        $(2,4,32)$ & $30$ \\ \hline
        $(2,4,64)$ & $18$ \\ \hline
        $(2,4,128)$ & $6$ \\ \hline
        $(2,8,8)$ & $132$ \\ \hline
        $(2,8,16)$ & $156$ \\ \hline
        $(2,8,32)$ & $60$ \\ \hline
        $(2,8,64)$ & $12$ \\ \hline
        $(2,16,16)$ & $144$ \\ \hline
        $(2,16,32)$ & $36$ \\ \hline
        $(2,32,32)$ & $60$ \\ \hline
        $(2,64,64)$ & $24$ \\ \hline
        $(2,128,128)$ & $12$ \\ \hline
        $(2,256,256)$ & $3$ \\ \hline
        $(4,4,4)$ & $65$ \\ \hline
        $(4,4,8)$ & $1581$ \\ \hline
        $(4,4,16)$ & $969$ \\ \hline
        $(4,4,32)$ & $225$ \\ \hline
        $(4,4,64)$ & $69$ \\ \hline
        $(4,4,128)$ & $15$ \\ \hline
        $(4,8,8)$ & $1581$ \\ \hline
        $(4,8,16)$ & $960$ \\ \hline
        $(4,8,32)$ & $168$ \\ \hline
        $(4,8,64)$ & $24$ \\ \hline
        $(4,16,16)$ & $969$ \\ \hline
        $(4,16,32)$ & $84$ \\ \hline
        $(4,32,32)$ & $225$ \\ \hline
        $(4,64,64)$ & $69$ \\ \hline
        $(4,128,128)$ & $15$ \\ \hline
        $(4,256,256)$ & $6$ \\ \hline
        $(8,8,8)$ & $726$ \\ \hline
        $(8,8,16)$ & $1542$ \\ \hline
        $(8,8,32)$ & $378$ \\ \hline
        $(8,8,64)$ & $78$ \\ \hline
        $(8,16,16)$ & $1542$ \\ \hline
        $(8,16,32)$ & $72$ \\ \hline
        $(8,32,32)$ & $378$ \\ \hline
        $(8,64,64)$ & $78$ \\ \hline
        $(8,128,128)$ & $24$ \\ \hline
        $(8,256,256)$ & $12$ \\ \hline
        $(16,16,16)$ & $136$ \\ \hline
        $(16,16,32)$ & $552$ \\ \hline
        $(16,32,32)$ & $552$ \\ \hline
        $(16,64,64)$ & $144$ \\ \hline
        $(16,128,128)$ & $48$ \\ \hline
        $(16,256,256)$ & $24$ \\ \hline
        $(32,64,64)$ & $288$ \\ \hline
        $(32,128,128)$ & $96$ \\ \hline
        $(32,256,256)$ & $48$ \\ \hline
        $(64,128,128)$ & $192$ \\ \hline
        $(64,256,256)$ & $96$ \\ \hline
        $(128,256,256)$ & $192$ \\ \hline
      \end{longtable}
    \end{theorem}
    \begin{remark}
      \label{rmk:extrasymmetry}
      The above table in Theorem \ref{thm:abc}
      has a pattern.
      The table entries corresponding to
      $(8,8,16)$ and $(8,16,16)$ are equal
      and the table entries corresponding to
      $(16,16,32)$ and $(16,32,32)$ are equal.
      This appears to be more than just a coincidence,
      but at this point we do not have an
      explanation of this phenomenon.
    \end{remark}
    % \begin{proof}
      % Apply Algorithm \ref{alg:alltriples}.
      % \mm{maybe go up to 512 or 1024 and include source code link to implementation}
    % \end{proof}
    % \begin{alg}\label{alg:allpassports}
      % We use Algorithm \ref{alg:alltriples}
      % to count the number of Passports of $2$-group Belyi maps
      % of a given degree.
      % \mm{todo}
    % \end{alg}
    % \begin{theorem}\label{thm:passports}
      % The following table lists the number of passports of
      % $2$-group Belyi maps of degree $d$ for $d$ up to $256$.
      % \[
      %   \begin{tabular}{|c||c|c|c|c|c|c|c|c|}
      %     \hline
      %     $d$ & $2$ & $4$ & $8$ & $16$ & $32$ & $64$ & $128$ & $256$\\
      %     \hline
      %     \# passports & $3$ & $7$ & $16$ & $41$ & $96$ & $267$ & $834$ & $2893$\\
      %     \hline
      %   \end{tabular}
      % \]
    % \end{theorem}
    % \begin{proof}
    %   Apply Algorithm \ref{alg:allpassports}.
    % \end{proof}
    \begin{figure}[ht]
      \centering
      \begin{tikzpicture}[scale=1]
        \begin{axis}[
          % title=Distribution of $g$,
          xlabel=$g$,
          ylabel=\# triples,
          ybar,
          ymin=0,
          xmin=-0.5,
          % height=0.9\textheight,
          width=\textwidth,
          % enlarge x limits=0.1,
          % enlarge y limits=0.05,
          % xtick=data,
          % ytick={100, 500, 1000},
          % x tick label style={
          %   rotate=45,
          %   anchor=east
          % }
        ]
          \addplot +[
            draw=black,
            hist={
              bins=131,
              data min=-0.5,
              data max=130.5
            }
          ]
          % table [x index=0, y index=0] {data.csv};
          table [y index = 0] {genus_up_to_d256.csv};
        \end{axis}
      \end{tikzpicture}
      \label{fig:tikzplot}
      \caption{Distribution of genera
      up to degree $256$.}
    \end{figure}
    \begin{figure}[ht]
      \centering
      % Belyi Triples
      % \begin{tikzpicture}[scale=1]
        % \begin{axis}[
        %     ybar stacked,
        %     % scale only axis=true,
        %     height=0.9\textheight,
        %     width=0.45\textwidth,
        %     % enlargelimits=0.15,
        %     legend style={
        %       % at={(0.5,-0.30)},
        %       % at={(0.1,0.9)},
        %       at={(0,1)},
        %       anchor=north west,
        %       % legend columns=-1
        %     },
        %     ylabel={\# triples},
        %     xlabel={degree},
        %     symbolic x coords={
        %       $1$,$2$,$4$,
        %       $8$,$16$,$32$,
        %       $64$,$128$,$256$
        %     },
        %     xtick=data,
        %     x tick label style={
        %       rotate=45,
        %       anchor=east
        %     }
        %   ]
        %   \addplot+[ybar] plot coordinates {
        %     ($1$,1)
        %     ($2$,3)
        %     ($4$,7)
        %     ($8$,9)
        %     ($16$,9)
        %     ($32$,9)
        %     ($64$,9)
        %     ($128$,9)
        %     ($256$,9)
        %   };
        %   \addplot+[ybar] plot coordinates {
        %     ($1$,0)
        %     ($2$,0)
        %     ($4$,0)
        %     ($8$,10)
        %     ($16$,46)
        %     ($32$,142)
        %     ($64$,494)
        %     ($128$,1790)
        %     ($256$,7166)
        %   };
        % \legend{nonhyperbolic, hyperbolic}
        % \end{axis}
      % \end{tikzpicture}
      % Passports
      \begin{tikzpicture}[scale=1]
        \begin{axis}[
            ybar stacked,
            % scale only axis=true,
            height=0.9\textheight,
            width=0.45\textwidth,
            enlarge x limits=0.1,
            enlarge y limits=0.05,
            legend style={
              % at={(0.5,-0.30)},
              % at={(0.1,0.9)},
              at={(0,1)},
              anchor=north west,
              % legend columns=-1
            },
            ylabel={\# passports},
            xlabel={degree},
            symbolic x coords={
              $1$,$2$,$4$,
              $8$,$16$,$32$,
              $64$,$128$,$256$
            },
            xtick=data,
            % ytick={100, 500, 1000},
            x tick label style={
              rotate=45,
              anchor=east
            }
          ]
          \addplot+[ybar] plot coordinates {
            ($1$,1)
            ($2$,3)
            ($4$,7)
            ($8$,9)
            ($16$,9)
            ($32$,9)
            ($64$,9)
            ($128$,9)
            ($256$,9)
          };
          \addplot+[ybar] plot coordinates {
            ($1$,0)
            ($2$,0)
            ($4$,0)
            ($8$,7)
            ($16$,32)
            ($32$,87)
            ($64$,258)
            ($128$,825)
            ($256$,2884)
          };
        \legend{nonhyperbolic, hyperbolic}
        \end{axis}
      \end{tikzpicture}
      % LaxPassports
      \begin{tikzpicture}[scale=1]
        \begin{axis}[
            ybar stacked,
            % scale only axis=true,
            height=0.9\textheight,
            width=0.45\textwidth,
            enlarge x limits=0.1,
            enlarge y limits=0.05,
            legend style={
              % at={(0.5,-0.30)},
              % at={(0.1,0.9)},
              at={(0,1)},
              anchor=north west,
              % legend columns=-1
            },
            ylabel={\# lax passports},
            xlabel={degree},
            symbolic x coords={
              $1$,$2$,$4$,
              $8$,$16$,$32$,
              $64$,$128$,$256$
            },
            xtick=data,
            % ytick={100, 500, 1000},
            x tick label style={
              rotate=45,
              anchor=east
            }
          ]
          \addplot+[ybar] plot coordinates {
            ($1$,1)
            ($2$,1)
            ($4$,3)
            ($8$,3)
            ($16$,3)
            ($32$,3)
            ($64$,3)
            ($128$,3)
            ($256$,3)
          };
          \addplot+[ybar] plot coordinates {
            ($1$,0)
            ($2$,0)
            ($4$,0)
            ($8$,3)
            ($16$,11)
            ($32$,28)
            ($64$,82)
            ($128$,254)
            ($256$,879)
          };
        \legend{nonhyperbolic, hyperbolic}
        \end{axis}
      \end{tikzpicture}
      \label{fig:geometrybydegree}
      \caption{
        % \# nonhyperbolic and hyperbolic Belyi
        % triples by degree (left),
        \# nonhyperbolic and hyperbolic
        passports by degree (left),
        and
        \# nonhyperbolic and hyperbolic
        lax passports by degree (right).
      }
    \end{figure}
    \begin{figure}[ht]
      \centering
      \begin{tikzpicture}[scale=1]
        \begin{axis}[
          ybar stacked,
          % scale only axis=true,
          height=0.9\textheight,
          width=0.45\textwidth,
          enlarge x limits=0.1,
          enlarge y limits=0.05,
          legend style={
            % at={(0.5,-0.30)},
            % at={(0.1,0.9)},
            at={(0,1)},
            anchor=north west,
            % legend columns=-1
          },
          ylabel={\# triples},
          xlabel={degree},
          symbolic x coords={
            $1$,$2$,$4$,
            $8$,$16$,$32$,
            $64$,$128$,$256$
          },
          xtick=data,
          % ytick={100, 500, 1000},
          x tick label style={
            rotate=45,
            anchor=east
          }
        ]
        \addplot+[ybar] plot coordinates {
          ($1$,1)
          ($2$,3)
          ($4$,7)
          ($8$,15)
          ($16$,31)
          ($32$,63)
          ($64$,127)
          ($128$,255)
          ($256$,511)
        };
        \addplot+[ybar] plot coordinates {
          ($1$,0)
          ($2$,0)
          ($4$,0)
          ($8$,4)
          ($16$,24)
          ($32$,88)
          ($64$,376)
          ($128$,1544)
          ($256$,6664)
        };
        \legend{abelian, non-abelian}
        \end{axis}
      \end{tikzpicture}
      \begin{tikzpicture}[scale=1]
        \begin{axis}[
          ybar stacked,
          % scale only axis=true,
          height=0.9\textheight,
          width=0.45\textwidth,
          enlarge x limits=0.1,
          enlarge y limits=0.05,
          legend style={
            % at={(0.5,-0.30)},
            % at={(0.1,0.9)},
            at={(0,1)},
            anchor=north west,
            % legend columns=-1
          },
          % ylabel={\# triples},
          xlabel={degree},
          symbolic x coords={
            $1$,$2$,$4$,
            $8$,$16$,$32$,
            $64$,$128$,$256$
          },
          xtick=data,
          % ytick={100, 500, 1000},
          x tick label style={
            rotate=45,
            anchor=east
          }
        ]
        % nc1
        \addplot+[ybar] plot coordinates {
          ($1$,1)
          ($2$,3)
          ($4$,7)
          ($8$,15)
          ($16$,31)
          ($32$,63)
          ($64$,127)
          ($128$,255)
          ($256$,511)
        };
        % nc2
        \addplot+[ybar] plot coordinates {
          ($1$,0)
          ($2$,0)
          ($4$,0)
          ($8$,4)
          ($16$,12)
          ($32$,28)
          ($64$,76)
          ($128$,172)
          ($256$,364)
        };
        % nc3
        \addplot+[ybar] plot coordinates {
          ($1$,0)
          ($2$,0)
          ($4$,0)
          ($8$,0)
          ($16$,12)
          ($32$,48)
          ($64$,168)
          ($128$,472)
          ($256$,1288)
        };
        % nc4
        \addplot+[ybar] plot coordinates {
          ($1$,0)
          ($2$,0)
          ($4$,0)
          ($8$,0)
          ($16$,0)
          ($32$,12)
          ($64$,120)
          ($128$,624)
          ($256$,2288)
        };
        % nc5
        \addplot+[ybar] plot coordinates {
          ($1$,0)
          ($2$,0)
          ($4$,0)
          ($8$,0)
          ($16$,0)
          ($32$,0)
          ($64$,12)
          ($128$,264)
          ($256$,2160)
        };
        % nc6
        \addplot+[ybar] plot coordinates {
          ($1$,0)
          ($2$,0)
          ($4$,0)
          ($8$,0)
          ($16$,0)
          ($32$,0)
          ($64$,0)
          ($128$,12)
          ($256$,552)
        };
        % nc7
        \addplot+[ybar] plot coordinates {
          ($1$,0)
          ($2$,0)
          ($4$,0)
          ($8$,0)
          ($16$,0)
          ($32$,0)
          ($64$,0)
          ($128$,0)
          ($256$,12)
        };
        \legend{
          class $1$,
          class $2$,
          class $3$,
          class $4$,
          class $5$,
          class $6$,
          class $7$,
        }
        \end{axis}
      \end{tikzpicture}
      \label{fig:groupsbydegree}
      \caption{
        \# permutation triples by degree
        with abelian and non-abelian monodromy
        groups (left)
        and
        \# permutation triples by degree
        with monodromy groups of
        various nilpotency classes (right).
      }
    \end{figure}
    \begin{figure}[ht]
      \centering
      \begin{tikzpicture}[scale=1]
        \begin{axis}[
          ybar stacked,
          % scale only axis=true,
          height=0.95\textheight,
          width=0.95\textwidth,
          enlarge x limits=0.1,
          enlarge y limits=0.05,
          legend style={
            % at={(0.5,-0.30)},
            % at={(0.1,0.9)},
            at={(0,1)},
            anchor=north west,
            % legend columns=-1
          },
          ylabel={\# passports},
          xlabel={degree},
          symbolic x coords={
            $1$,$2$,$4$,
            $8$,$16$,$32$,
            $64$,$128$,$256$
          },
          xtick=data,
          % ytick={100, 500, 1000},
          x tick label style={
            rotate=45,
            anchor=east
          }
        ]
        % size1
        \addplot+[ybar] plot coordinates {
          ($1$,1)
          ($2$,3)
          ($4$,7)
          ($8$,13)
          ($16$,34)
          ($32$,70)
          ($64$,149)
          ($128$,402)
          ($256$,1065)
        };
        % size2
        \addplot+[ybar] plot coordinates {
          ($1$,0)
          ($2$,0)
          ($4$,0)
          ($8$,3)
          ($16$,3)
          ($32$,18)
          ($64$,84)
          ($128$,271)
          ($256$,1123)
        };
        % size3
        \addplot+[ybar] plot coordinates {
          ($1$,0)
          ($2$,0)
          ($4$,0)
          ($8$,0)
          ($16$,1)
          ($32$,1)
          ($64$,4)
          ($128$,11)
          ($256$,52)
        };
        % size4
        \addplot+[ybar] plot coordinates {
          ($1$,0)
          ($2$,0)
          ($4$,0)
          ($8$,0)
          ($16$,3)
          ($32$,3)
          ($64$,21)
          ($128$,117)
          ($256$,462)
        };
        % size6
        \addplot+[ybar] plot coordinates {
          ($1$,0)
          ($2$,0)
          ($4$,0)
          ($8$,0)
          ($16$,0)
          ($32$,1)
          ($64$,3)
          ($128$,11)
          ($256$,62)
        };
        % size8
        \addplot+[ybar] plot coordinates {
          ($1$,0)
          ($2$,0)
          ($4$,0)
          ($8$,0)
          ($16$,0)
          ($32$,3)
          ($64$,3)
          ($128$,12)
          ($256$,84)
        };
        % size12
        \addplot+[ybar] plot coordinates {
          ($1$,0)
          ($2$,0)
          ($4$,0)
          ($8$,0)
          ($16$,0)
          ($32$,0)
          ($64$,0)
          ($128$,4)
          ($256$,26)
        };
        % size16
        \addplot+[ybar] plot coordinates {
          ($1$,0)
          ($2$,0)
          ($4$,0)
          ($8$,0)
          ($16$,0)
          ($32$,0)
          ($64$,3)
          ($128$,3)
          ($256$,12)
        };
        % size24
        \addplot+[ybar] plot coordinates {
          ($1$,0)
          ($2$,0)
          ($4$,0)
          ($8$,0)
          ($16$,0)
          ($32$,0)
          ($64$,0)
          ($128$,0)
          ($256$,1)
        };
        % size32
        \addplot+[ybar] plot coordinates {
          ($1$,0)
          ($2$,0)
          ($4$,0)
          ($8$,0)
          ($16$,0)
          ($32$,0)
          ($64$,0)
          ($128$,3)
          ($256$,3)
        };
        % size64
        \addplot+[ybar] plot coordinates {
          ($1$,0)
          ($2$,0)
          ($4$,0)
          ($8$,0)
          ($16$,0)
          ($32$,0)
          ($64$,0)
          ($128$,0)
          ($256$,3)
        };
        \legend{
          size $1$,
          size $2$,
          size $3$,
          size $4$,
          size $6$,
          size $8$,
          size $12$,
          size $16$,
          size $24$,
          size $32$,
          size $64$,
        }
        \end{axis}
      \end{tikzpicture}
      \label{fig:passportsizes}
      \caption{
        \# passports of various sizes by degree.
      }
    \end{figure}
  }
}
\chapter{Fields of definition of $2$-group Belyi maps}{\label{chapter:fieldsofdefinition}
  % \mm{this chapter could possibly get combined into the group theory chapter}
  % \mm{need some sort of lead up for this chapter to connect group theory to field of moduli field of definition}
  % \mm{K\"{o}ck: $X$ and $\phi$ defined over field of moduli in Galois case}
  Recall from Definition
  \ref{def:definedover},
  that a field of definition of a Belyi map
  $\phi\colon X\to\PP^1$
  is a number field
  $K\subseteq\CC$
  such that $X$ and $\phi$
  are defined using algebraic equations
  having coefficients in $K$.
  Theorem
  \ref{thm:fieldofmoduli}
  proves
  that the moduli field of a Belyi map
  has degree bounded by the size of its
  passport.
  This is obtained from the action
  of $\Gal(\QQal\,|\,\QQ)$
  on the Belyi maps with a given passport.
  There is an analogous action of
  $\Gal(\QQal\,|\,\QQab)$
  on \emph{refined passports}
  which we define and compute in this chapter.
  % Using data from Chapter
  % \ref{chapter:grouptheory},
  % we formulate a conjecture about
  % the possible fields of definition of
  % $2$-group Belyi maps.
  %
  % \section{Fields of moduli}{\label{sec:fieldsofmoduli}}{
    % Recall the action of $G_\QQ$ on the set of Belyi maps
    % described in Proposition \ref{prop:galoisaction}.
    % For a fixed Belyi map,
    % we can simplify matters as described in the following definition.
    % \begin{definition}\label{def:fieldofmoduli}
    %   The \defi{field of moduli} of a Belyi map
    %   $\phi:X\to\PP^1$ is the fixed field
    %   \[
    %     \{\tau\in G_\QQ : \phi^\tau\cong\phi\}.
    %   \]
    % \end{definition}
    % Definition \ref{def:fieldofmoduli} allows us to study
    % a more manageable finite extension.
    % Moreover, passports (recall Definition \ref{def:passport})
    % allow us to bound the degree of the field of moduli.
    % \begin{theorem}\label{thm:fieldofmoduli}
    %   Let $\phi:X\to\PP^1$ be a Belyi map
    %   with passport $\mathcal{P}$.
    %   Then the degree of the field of moduli of $\phi$
    %   is bounded by the size of $\mathcal{P}$.
    % \end{theorem}
    % \begin{proof}
    %   % proof in JVSijsling
    % \end{proof}
  % }
  \section{Refined passports}{\label{sec:refined}}{
    Let $\sigma$ be a $2$-group permutation triple.
    Recall, from Definition \ref{def:passport},
    that the passport of $\sigma$
    consists of the data
    $(g(\sigma), \langle\sigma\rangle,
    \lambda(\sigma))$
    where $g(\sigma)$ is the genus,
    $\langle\sigma\rangle$ is the monodromy
    group
    ($2$-group in its regular
    representation)
    as a subgroup of $S_d$,
    and $\lambda(\sigma)$ is the
    triple of partitions
    specifying the three ordered
    $S_d$
    conjugacy classes
    $C_0,C_1,C_\infty$
    of
    $\sigma_0,\sigma_1,\sigma_\infty$
    respectively.
    Let $\mathcal{P}$ be the passport
    of $\sigma$.
    The size of
    $\mathcal{P}$
    is the cardinality of the set
    \begin{equation}
      \label{eqn:coarse}
      \Sigma_\mathcal{P}
      =
      \{
        (\sigma_0,\sigma_1,\sigma_\infty)\in
        C_0\times C_1\times C_\infty:
        \sigma_\infty\sigma_1\sigma_0=1
        \text{ and }
        \langle\sigma_0,\sigma_1,\sigma_\infty\rangle=G
      \}/\sim
    \end{equation}
    where
    $(\sigma_0,\sigma_1,\sigma_\infty)\sim
    (\sigma_0',\sigma_1',\sigma_\infty')$
    if the triples are simultaneously conjugate
    by an element of $S_d$.
    By Theorem \ref{thm:fieldofmoduli},
    the cardinality of $\Sigma_\mathcal{P}$
    bounds the field of moduli of
    the Belyi map corresponding to $\sigma$.
    \par
    Let $G$ be a transitive subgroup of $S_d$
    and let $C$ be a conjugacy class of $S_d$.
    Then $C$ can be partitioned into
    conjugacy classes of $G$.
    To analyze conjugacy in $G$ we
    make the following definition.
    \begin{definition}\label{def:refined}
      A \defi{refined passport} $\mathscr{P}$
      consists of the data
      $(g,G,c)$
      where $g\geq 0$ is an integer,
      $G\leq S_d$ is a transitive subgroup,
      and $c = (c_0,c_1,c_\infty)$
      is a triple of conjugacy classes of $G$.
      For a refined passport
      $\mathscr{P}$ consider the set
      \begin{equation}
        \label{eqn:refined}
        \Sigma_\mathscr{P} =
        \{(\sigma_0,\sigma_1,\sigma_\infty)\in c_0\times c_1\times c_\infty : \sigma_\infty\sigma_1\sigma_0=1\text{ and } \langle\sigma_0,\sigma_1,\sigma_\infty\rangle=G\}/\sim
      \end{equation}
      where $(\sigma_0,\sigma_1,\sigma_\infty)\sim(\sigma_0', \sigma_1', \sigma_\infty')$
      if and only if there exists $\alpha\in\Aut(G)$ with
      $\alpha(\sigma_s) = \sigma_s'$ for $s\in\{0,1,\infty\}$.
      \par
      Let $\sigma$ be a
      permutation triple
      and let $c_s$ denote the conjugacy
      class of $\langle\sigma\rangle$
      containing $\sigma_s$ for
      $s\in \{0,1,\infty\}$.
      We define the
      \defi{refined passport of $\sigma$}
      to be
      \begin{equation}
        \label{eqn:refinedpassportofsigma}
        \mathscr{P}(\sigma)
        =(g(\sigma), \langle\sigma\rangle, (c_0,c_1,c_\infty)).
      \end{equation}
    \end{definition}
    % \begin{theorem}
    %   \label{thm:refinedsizebounds}
    %   \mm{
    %     The group $\Gal(\QQal/\QQab)$ acts on the refined passport
    %   }
    % \end{theorem}
  }
  \section{Computing refined passports}{
    \label{sec:computerefined}
    Let $\sigma$ be a $2$-group permutation triple.
    Let $\mathcal{P}$ and $\mathscr{P}$ denote the
    passport and refined passport of $\sigma$
    respectively.
    Let
    $\Sigma_\mathcal{P}$ and $\Sigma_\mathscr{P}$
    denote the sets in
    \eqref{eqn:coarse}
    and
    \eqref{eqn:refined}
    respectively.
    % Let $\wt{\sigma}$ be a lift of $\sigma$
    % with passport and refined passport
    % $\wt{\mathcal{P}}$ and $\wt{\mathscr{P}}$.
    \par
    Chapter \ref{chapter:grouptheory}
    provides us with an explicit list
    of all $2$-group permutation triples
    (up to simultaneous conjugation in $S_d$)
    for fixed degree.
    Using techniques from
    Musty, Schiavone, Sijsling, and Voight
    in \cite{belyidb},
    we now describe the computation
    of $\Sigma_{\mathscr{P}(\sigma)}$ for
    every $2$-group permutation triple
    $\sigma$ of degree $d$ for $d\leq 256$.
    \par
    The main tool for efficiently computing refined
    passports comes from
    the \emph{passport lemma},
    \cite[Lemma 2.2.1]{belyidb},
    which we now state.
    \begin{lemma}[Passport lemma]
      \label{lem:passportlemma}
      Let $S$ be a group,
      let $G\leq S$ be a subgroup,
      let $N\colonequals N_S(G)$
      be the normalizer of $G$ in $S$,
      and let $C_0,C_1$ be conjugacy classes
      in $N$ represented by
      $\tau_0,\tau_1\in G$.
      Let $C_N(g)$ denote the centralizer of
      $g$ in $N$.
      Let
      \begin{equation}
        \label{eqn:passportlemmaU}
        U\colonequals
        \{
          (\sigma_0,\sigma_1)\in C_0\times C_1 :
          \langle\sigma_0,\sigma_1\rangle
          \subseteq G
        \}
        /\!\sim
      \end{equation}
      where $\sim$ indicates simultaneous
      conjugation by elements in $S$.
      Then the map
      \begin{align}
        \label{eqn:passportlemmau}
        \begin{split}
          u\colon C_N(\tau_0)\!\setminus\!N/C_N(\tau_1)&\to U\\
          C_N(\tau_0)\nu C_N(\tau_1)&\mapsto
          [(\tau_0,\nu\tau_1\nu^{-1})]
        \end{split}
      \end{align}
      is surjective,
      and for all
      $[(\sigma_0,\sigma_1)]\in U$
      such that
      $\langle\sigma_0,\sigma_1\rangle=G$,
      there is a unique preimage under $u$.
    \end{lemma}
    With the passport lemma in hand,
    we can now describe an efficient algorithm
    to compute refined passports.
    \begin{alg}
      \label{alg:computerefinedpassport}
      \,
      \newline\textbf{Input}:
      $\sigma$ a $2$-group permutation triple
      \newline
      \textbf{Output}:
      Refined passport representatives for $\mathscr{P}(\sigma)$
      \begin{enumerate}
        \item
          Compute the set $U$ from the passport lemma
          with
          $\tau_0 = \sigma_0$,
          $\tau_1 = \sigma_1$,
          and $S = G$.
        \item
          \label{alg:refinedpassportlemma}
          Let
          $U'
          \colonequals
          \{
          (g_0,g_1)\in U :
          \langle g_0,g_1\rangle = G
          \}
          $.
        \item
          Extend all pairs
          $(g_0,g_1)$
          in $U'$ from
          Step
          \ref{alg:refinedpassportlemma}
          to triples
          $(g_0,g_1,g_\infty)$ satisfying
          $g_\infty g_1 g_0 = 1$.
          Let $T$ denote the set of triples
          obtained in this way from the pairs of $U'$.
        \item
          Let $C_0,C_1,C_\infty$
          denote the conjugacy classes of
          $\sigma_0,\sigma_1,\sigma_\infty$
          respectively and let
          $T'\colonequals
          \{(g_0,g_1,g_\infty)\in T : g_\infty\in C_\infty\}$.
        \item
          Quotient $T'$ by outer automorphisms
          of $G$.
          That is,
          for every pair of triples
          $(g_0,g_1,g_\infty)$
          and
          $(g_0',g_1',g_\infty')$
          in $T$ with
          $(g_0',g_1',g_\infty') =
          (\alpha(g_0),\alpha(g_1),\alpha(g_\infty))$
          for some $\alpha\in\Out(G)$
          only keep one such triple in $T$.
          Return this quotient of $T'$ as output.
      \end{enumerate}
    \end{alg}
    \begin{proof}[Proof of correctness]
      By the passport lemma,
      we have that $T'$ is equal to the set
      \begin{equation}
        \label{eqn:innerauts}
        \{
          (g_0,g_1,g_\infty)\in C_0\times C_1\times C_\infty :
          g_\infty g_1 g_0 = 1,\text{ and }
          \langle g_0,g_1,g_\infty\rangle = G
        \}/\!\sim
      \end{equation}
      where $\sim$ denotes simultaneous conjugation in
      $N_G(G)=G$.
      The refined passport of $\sigma$
      can now be obtained from $T'$
      by eliminating redundant triples that can
      be identified by an outer automorphism of $G$.
      This is done in the last step of the algorithm
      and completes the proof.
    \end{proof}
    Applying
    Algorithm \ref{alg:computerefinedpassport}
    to the $2$-group permutation triples
    computed from Section \ref{sec:triplesalgorithm}
    yields the following result about the
    sizes of refined passports.
    \begin{theorem}
      \label{thm:refinedpassportsizes}
      Every
      $2$-group permutation triple
      of degree $d$ with $d\leq 64$
      has refined passport size $1$.
      There are $48$ $2$-group permutation triples
      (up to simultaneous conjugation)
      of degree $128$ with refined passport size $2$
      and the rest have refined passport size $1$.
      There are $288$ $2$-group permutation triples
      (up to simultaneous conjugation)
      of degree $256$ with refined passport size $2$
      and the rest have refined passport size $1$.
    \end{theorem}
    Theorem \ref{thm:refinedpassportsizes}
    can now be used to prove the following theorem.
    % Theorem \ref{thm:refinedpassportsizes},
    % together with refined field of definition equal to
    % field of moduli for Galois Belyi maps
    % and a strong version of Beckmann's theorem
    % imply the following.
    \begin{theorem}
      \label{thm:rerefined}
      Every $2$-group Belyi map
      of degree $d$ with $d\leq 256$
      is defined over a quadratic
      extension of an abelian extension of $\QQ$
      ramified only at $2$.
    \end{theorem}
    \begin{proof}
      Let $\phi\colon X\to\PP^1$ be a
      $2$-group Belyi map of degree $d\leq 256$.
      Let $M$ denote the field of moduli of $\phi$.
      Theorem \ref{thm:beckmann} (Beckmann's theorem)
      implies that
      for every odd prime $p$
      there exists a number field
      $M_p$ that is a field of definition for $\phi$
      that is unramified at $p$.
      Since $M\subseteq M_p$ for all $p$
      we have that $M$ is unramified away from $2$.
      Moreover, by Theorem \ref{thm:galoisbelyimapoverfieldofmoduli},
      $\phi$ is defined over $M$.
      So $\phi$ is defined over $M$
      which is
      unramified away from $2$.
      \par
      Let $\mathscr{P}$ denote the refined passport of $\phi$.
      Since $\Gal(\QQal\,|\,\QQab)$ acts on $\mathscr{P}$,
      see \cite[Proposition 7.6]{SV},
      the size of $\mathscr{P}$ bounds the
      degree of $M$ over $\QQab$.
      The calculation in Theorem
      \ref{thm:refinedpassportsizes}
      shows that the size of $\mathscr{P}$ is at
      most $2$ and completes the proof.
    \end{proof}
    For a visual representation of Theorem
    \ref{thm:refinedpassportsizes} see
    the figure below.
    % Figure \ref{fig:refinedpassportsizes}.
    Note how small the refined passport sizes
    % in Figure
    % \ref{fig:refinedpassportsizes}
    are in comparison
    to the passport sizes
    from Section \ref{sec:grouptheorycomputations}.
    % computed in
    % Section \ref{sec:grouptheorycomputations}.
    % Figure
    % \ref{fig:passportsizes}.
    \begin{figure}[ht]
      \centering
      \begin{tikzpicture}[scale=1]
        \begin{axis}[
          ybar stacked,
          % scale only axis=true,
          height=0.95\textheight,
          width=0.95\textwidth,
          enlarge x limits=0.1,
          enlarge y limits=0.05,
          legend style={
            % at={(0.5,-0.30)},
            % at={(0.1,0.9)},
            at={(0,1)},
            anchor=north west,
            % legend columns=-1
          },
          ylabel={\# triples},
          xlabel={degree},
          symbolic x coords={
            $1$,$2$,$4$,
            $8$,$16$,$32$,
            $64$,$128$,$256$
          },
          xtick=data,
          % ytick={100, 500, 1000},
          x tick label style={
            rotate=45,
            anchor=east
          }
        ]
        % size1
        \addplot+[ybar] plot coordinates {
          ($1$,1)
          ($2$,3)
          ($4$,7)
          ($8$,19)
          ($16$,55)
          ($32$,151)
          ($64$,503)
          ($128$,1751)
          ($256$,6887)
        };
        % size2
        \addplot+[ybar] plot coordinates {
          ($1$,0)
          ($2$,0)
          ($4$,0)
          ($8$,0)
          ($16$,0)
          ($32$,0)
          ($64$,0)
          ($128$,48)
          ($256$,288)
        };
        \legend{
          size $1$,
          size $2$
        }
        \end{axis}
      \end{tikzpicture}
      \label{fig:refinedpassportsizes}
      \caption{
        \# permutation triples with refined passports of various sizes by degree.
      }
    \end{figure}
    % Using techniques from
    % Musty, Schiavone, Sijsling, and Voight
    % in \cite{belyidb},
    % we computed $\Sigma_{\mathscr{P}(\sigma)}$ for
    % every $2$-group permutation triple
    % $\sigma$
    % of degree $d$
    % for $d\leq 256$.
    % \begin{equation}
    %   \label{eqn:theobservation}
    %   \text{
    %     \emph{
    %       We observed that
    %       $\#\Sigma_{\mathscr{P}(\sigma)}=1$
    %       for every such permutation
    %       triple $\sigma$!
    %     }
    %   }
    % \end{equation}
    % This observation
    % % combined with Theorem
    % % \ref{thm:refinedsizebounds},
    % provides motivation to
    % take a closer look at
    % the behavior
    % of refined passports with respect
    % to the iterative structure of
    % $2$-group permutation triples.
    % % \begin{lemma}
    % %   \label{lem:lineemuplemma}
    % %   Let $d\in\ZZ_{\geq 1}$,
    % %   let $\lambda=(\lambda_0,\lambda_1,\lambda_\infty)$ be a partition of $d$,
    % %   and let $G$ be a transitive subgroup of $S_d$.
    % %   Let $\mathcal{C}$ denote the
    % %   set of conjugacy classes of $G$.
    % %   For $C\in\mathcal{C}$
    % %   let $c(C)$ denote the
    % %   partition of $d$
    % %   encoding the
    % %   cycle structure of $C$
    % %   (i.e. the conjugacy class of $S_d$
    % %   containing $C$).
    % %   Let
    % %   \begin{equation}
    % %     \label{eqn:triplesofclasses}
    % %     \mathscr{C}
    % %     \colonequals
    % %     \{
    % %       (C_0,C_1,C_\infty)\in\mathcal{C}^3:
    % %       (c(C_0),c(C_1), c(C_\infty))=
    % %       (\lambda_0,\lambda_1,\lambda_\infty)
    % %     \}
    % %   \end{equation}
    % % \end{lemma}
    % % \begin{proof}
    % % \end{proof}
    % \begin{lemma}
    %   \label{lem:coarsetorefined}
    %   Let $\sigma$ be a $2$-group
    %   permutation triple
    %   with passport
    %   $\mathcal{P}=(g,G,C)$
    %   where $C=(C_0,C_1,C_\infty)$.
    %   Let $\mathscr{P} = (g,G,c)$
    %   be the refined passport of $\sigma$
    %   where $c=(c_0,c_1,c_\infty)$.
    %   Let $\sigma'$ be a $2$-group
    %   permutation triple
    %   % with passport $\mathcal{P}$
    %   simultaneously conjugate to $\sigma$
    %   with refined passport $\mathscr{P}'=(g,G,c')$
    %   where $c' = (c_0',c_1',c_\infty')$.
    %   Then $\#\Sigma_\mathscr{P} = \#\Sigma_{\mathscr{P}'}$.
    % \end{lemma}
    % \begin{proof}
    %   Let $S_d$ denote the ambient symmetric group.
    %   By hypothesis,
    %   there exists a permutation
    %   $\tau\in S_d$
    %   such that
    %   $\sigma^\tau=\sigma'$.
    %   Now define a map
    %   ${}^\tau\colon\Sigma_\mathscr{P}
    %   \to S_d^3$
    %   by simultaneously conjugating
    %   a representative
    %   triple in $\Sigma_\mathscr{P}$
    %   by $\tau$.
    %   Since conjugation by $\tau$ in $S_d$
    %   is an automorphism of $G$,
    %   simultaneous conjugation by $\tau$
    %   is a bijection from
    %   $\Sigma_\mathscr{P}$
    %   to
    %   $\Sigma_{\mathscr{P}'}$.
    %   % \mm{line em up..}
    %   % Choose representatives
    %   % $\{\psi_i\}_{i=1}^k$
    %   % for $\Sigma_\mathscr{P}$
    %   % and representatives
    %   % $\{\psi_i'\}_{i=1}^{k'}$
    %   % for $\Sigma_{\mathscr{P}'}$.
    %   % Let $\tau\in S_d$ satisfying
    %   % Equation \ref{eqn:simconj}.
    %   % % sigma^tau = (tau^-1sigma0tau,..) = sigma'
    % \end{proof}
    % \begin{definition}
    %   \label{def:characteristiclift}
    %   Let $\wt{\sigma}$
    %   be a lift
    %   of a $2$-group
    %   permutation triple $\sigma$
    %   obtained from the exact sequence
    %   \begin{equation}
    %     \label{eqn:characteristiclift}
    %     \begin{tikzcd}
    %       1\arrow{r}&\ZZ/2\ZZ\arrow{r}{\iota}
    %                 &\wt{G}\arrow{r}{\pi}
    %                 &G\arrow{r}&1
    %     \end{tikzcd}
    %   \end{equation}
    %   with $G=\langle\sigma\rangle$
    %   and $\wt{G}=\langle\sigma\rangle$.
    %   We say that $\wt{\sigma}$
    %   is a
    %   \defi{characteristic lift}
    %   of $\sigma$ if $\iota(\ZZ/2\ZZ)$
    %   is a characteristic subgroup
    %   of $\wt{G}$.
    % \end{definition}
    % \begin{lemma}
    %   \label{lem:identifiedinlift}
    %   Let $\sigma$ and $\sigma'$
    %   be $2$-group permutation triples
    %   with distinct refined passports
    %   $(g,G,c)$ and $(g,G,c')$ respectively.
    %   Then the refined passport of any
    %   characteristic lift
    %   of $\sigma$ is not equal to the refined
    %   passport of any
    %   characteristic lift of $\sigma'$.
    % \end{lemma}
    % \begin{proof}
    %   Assume for contradiction that we have
    %   lifts
    %   $\wt{\sigma}$ and $\wt{\sigma}'$
    %   of $\sigma$ and $\sigma'$
    %   with the same refined passport
    %   $(\wt{g},\wt{G},\wt{C})$.
    %   Let
    %   \begin{equation}
    %     \label{eqn:twoextensionsrefinedlemma}
    %     \begin{tikzcd}
    %       1\arrow{r}&\ZZ/2\ZZ\arrow{r}{\iota}
    %                 &\wt{G}\arrow{r}{\pi}
    %                 &G\arrow{r}&1\\
    %       1\arrow{r}&\ZZ/2\ZZ\arrow{r}{\iota'}
    %                 &\wt{G}\arrow{r}{\pi'}
    %                 &G\arrow{r}&1
    %     \end{tikzcd}
    %   \end{equation}
    %   be extensions of $\ZZ/2\ZZ$ by $G$
    %   such that $\wt{\sigma}\in\pi^{-1}(\sigma)$
    %   and $\wt{\sigma}'\in\pi'^{-1}(\sigma')$.
    %   Since $\wt{\sigma}$ and $\wt{\sigma}'$
    %   have the same refined passport,
    %   there exists
    %   an automorphism
    %   $\wt{\alpha}\in\Aut(\wt{G})$
    %   with
    %   $\wt{\alpha}(\wt{\sigma}_s) =
    %   \wt{\sigma}_s'$
    %   for each $s\in \{0,1,\infty\}$.
    %   \par
    %   Define $\alpha\colon G\to G$ by
    %   projecting $\wt{\alpha}$,
    %   that is,
    %   $\alpha\colonequals
    %   \pi'\circ\wt{\alpha}\circ\pi^{-1}$.
    %   Since $\alpha(\sigma_s)=\sigma_s'$
    %   by definition,
    %   we will have a contradiction if we can
    %   show that $\alpha\in\Aut(G)$.
    %   Let $\tau$
    %   be the generator of the characteristic
    %   subgroup
    %   $\iota(\ZZ/2\ZZ)=\iota'(\ZZ/2\ZZ)$
    %   and note that
    %   $\wt{\alpha}(\tau) = \tau$.
    %   \par
    %   First we show that $\alpha$ is well-defined.
    %   For $\sigma_s\in\sigma$,
    %   $\pi^{-1}(\sigma_s)
    %   \in\{\wt{\sigma}_s,\tau\wt{\sigma}_s\}$,
    %   and the calculation
    %   \begin{equation}
    %     \label{eqn:alphawelldefined}
    %     \pi'(\wt{\alpha}(\tau\wt{\sigma}_s))
    %     =\pi'(\tau\wt{\alpha}(\wt{\sigma}_s))
    %     =\pi'(\tau\wt{\sigma}_s')
    %   \end{equation}
    %   shows that $\alpha$ is well-defined
    %   homomorphism.
    %   \par
    %   To finish the proof we now show that
    %   $\alpha$ is injective.
    %   Let $g\in\ker\alpha$
    %   so that
    %   $\pi'(\wt{\alpha}(\pi^{-1}(g)))=\id_G$.
    %   Then
    %   $\wt{\alpha}(\pi^{-1}(g))\in
    %   \iota'(\ZZ/2\ZZ)$
    %   which implies
    %   $\pi^{-1}(g)\in\iota(\ZZ/2\ZZ)$
    %   so that $g=\id_G$.
    %   % Let $\tau_s\colonequals\pi(\wt{\tau}_s)$.
    %   % Applying $\pi$ to
    %   % \eqref{eqn:eachconjugate} yields
    %   % \begin{equation}
    %   %   \label{eqn:applypi}
    %   %   \sigma_s' = \tau_s\sigma_s\tau_s^{-1}.
    %   % \end{equation}
    %   % Since \eqref{eqn:applypi}
    %   % holds for each $s\in \{0,1,\infty\}$
    %   % this implies that
    %   % $\sigma_s$ and $\sigma_s'$
    %   % are conjugate in $G$
    %   % for each $s$.
    %   % But this implies $\sigma$
    %   % and $\sigma'$ have the same refined passport
    %   % which contradicts the hypothesis
    %   % that these refined passports were distinct.
    %   % Thus the refined passports of
    %   % $\wt{\sigma}$ and $\wt{\sigma}'$
    %   % cannot be equal.
    % \end{proof}
    % \begin{conj}
    %   \label{conj:refinedpassportsizeone}
    %   Every $2$-group permutation triple $\sigma$
    %   has refined passport size $1$.
    % \end{conj}
    % % \mm{computational evidence, true in easy cases}
    % \begin{lemma}
    %   \label{lem:conjectureabelian}
    %   Conjecture
    %   \ref{conj:refinedpassportsizeone}
    %   is true for
    %   $\langle\sigma\rangle$
    %   abelian.
    % \end{lemma}
    % \begin{proof}
    %   In an abelian group the conjugacy classes
    %   consist of sets of size $1$.
    %   Therefore every refined passport
    %   $\mathscr{P}=(g,G,c)$
    %   with $G$ abelian has
    %   \begin{equation}
    %     \label{eqn:leqone}
    %     \Sigma_{\mathscr{P}}\leq 1.
    %   \end{equation}
    %   Since $\sigma$ is a $2$-group
    %   permutation triple with refined passport
    %   $\mathscr{P}(\sigma)$,
    %   we have that
    %   \begin{equation}
    %     \label{eqn:geqone}
    %     \Sigma_{\mathscr{P}(\sigma)}\geq 1.
    %   \end{equation}
    %   \eqref{eqn:leqone} and
    %   \eqref{eqn:geqone}
    %   imply
    %   $\Sigma_{\mathscr{P}(\sigma)}=1$
    %   for $\langle\sigma\rangle$ abelian.
    % \end{proof}
    % % \begin{proof}[Proof for $\langle\sigma\rangle$ abelian]
    %   % The proof is by induction on the degree of $\sigma$.
    %   % There is a unique $2$-group permutation
    %   % triple of degree $1$ which therefore has refined passport size $1$.
    %   % For induction,
    %   % assume that every $2$-group
    %   % permutation triple
    %   % of degree $d$
    %   % with $\langle\sigma\rangle$ abelian
    %   % has refined passport size $1$.
    %   % Let $\wt{\sigma}$ be a $2$-group
    %   % permutation triple of degree
    %   % $2d$ with $\langle\wt{\sigma}\rangle$
    %   % abelian.
    %   % We are required to show that the refined
    %   % passport of $\wt{\sigma}$ has size $1$.
    %   % \par
    %   % Let $\Sigma_\mathscr{P}\colonequals\Sigma_{\mathscr{P}(\wt{\sigma})}$ be the set of refined
    %   % passport representatives defined in
    %   % \eqref{eqn:refined}.
    %   % By Algorithm \ref{alg:triples}
    %   % there exists a $2$-group permutation triple
    %   % $\sigma$ such that the following sequence is exact.
    %   % \begin{equation}
    %   %   \label{eqn:sigmasequence}
    %   %   \begin{tikzcd}
    %   %     1\arrow{r}&\ZZ/2\ZZ\arrow{r}{\iota}
    %   %               &\langle\wt{\sigma}\rangle\arrow{r}{\pi}
    %   %               &\langle\sigma\rangle\arrow{r}
    %   %               &1
    %   %   \end{tikzcd}
    %   % \end{equation}
    %   % In other words,
    %   % $\wt{\sigma}$ is a lift of $\sigma$.
    %   % Since quotients of abelian groups are abelian,
    %   % $\langle\sigma\rangle$ is abelian.
    %   % Thus, by induction,
    %   % the refined passport of $\sigma$
    %   % has size $1$.
    %   % By Lemma \ref{lem:identifiedinlift}
    %   % every element of $\Sigma_\mathscr{P}$
    %   % must also be a lift of $\sigma$.
    %   % Thus, it is sufficient to prove that every
    %   % lift $\wt{\sigma}'$ satisfies one
    %   % of the following.
    %   % \begin{enumerate}
    %   %   \item\label{itm:distinctconjclasses}
    %   %     The $\langle\wt{\sigma}\rangle$
    %   %     conjugacy class of
    %   %     $\wt{\sigma}_s'$ differs from
    %   %     the $\langle\wt{\sigma}\rangle$
    %   %     conjugacy class of
    %   %     $\wt{\sigma}_s$ for
    %   %     some $s\in \{0,1,\infty\}$
    %   %   \item\label{itm:thereexistsanauto}
    %   %     There exists an automorphism
    %   %     $\phi\in\Aut(\langle\wt{\sigma}\rangle)$
    %   %     with
    %   %     $\phi(\wt{\sigma}_s') = \wt{\sigma}_s$
    %   %     for all $s\in \{0,1,\infty\}$
    %   % \end{enumerate}
    %   % Let $\alpha\in\iota(\ZZ/2\ZZ)$ be the generator of the image.
    %   % There are $2^3 = 8$ preimages of $\sigma$
    %   % under the map $\pi$.
    %   % Since $\wt{\sigma}_\infty\wt{\sigma}_1\wt{\sigma}_0=1$
    %   % and $\alpha$ is central,
    %   % there are exactly $4$ preimages
    %   % that multiply to $1$.
    %   % They are as follows.
    %   % \begin{equation}
    %   %   \label{eqn:preimagesto1}
    %   %   \{
    %   %     (\wt{\sigma}_0,\wt{\sigma}_1,\wt{\sigma}_\infty),
    %   %     (\wt{\sigma}_0,\alpha\wt{\sigma}_1,\alpha\wt{\sigma}_\infty),
    %   %     (\alpha\wt{\sigma}_0,\wt{\sigma}_1,\alpha\wt{\sigma}_\infty),
    %   %     (\alpha\wt{\sigma}_0,\alpha\wt{\sigma}_1,\wt{\sigma}_\infty)
    %   %   \}
    %   % \end{equation}
    %   % Since $\langle\wt{\sigma}\rangle$ is abelian,
    %   % the lifts in \eqref{eqn:preimagesto1}
    %   % all define distinct triples of conjugacy
    %   % classes of $\langle\wt{\sigma}\rangle$.
    %   % Thus every lift satisfies \ref{itm:distinctconjclasses} which completes the proof in
    %   % the abelian case.
    % % \end{proof}
    % \begin{theorem}
    %   \label{thm:conjecturedihedral}
    %   Conjecture
    %   \ref{conj:refinedpassportsizeone}
    %   is true for
    %   $\langle\sigma\rangle$
    %   dihedral.
    % \end{theorem}
    % \begin{proof}
    %   The proof is by induction on the
    %   degree of $\sigma$.
    %   There is a unique $2$-group permutation
    %   triple of degree $1$ which
    %   therefore has refined passport size $1$.
    %   For induction,
    %   assume that every $2$-group
    %   permutation triple
    %   of degree $d$
    %   with $\langle\sigma\rangle$ dihedral
    %   has refined passport size $1$.
    %   Let $\wt{\sigma}$ be a $2$-group
    %   permutation triple of degree
    %   $2d$ with $\langle\wt{\sigma}\rangle$
    %   dihedral.
    %   We are required to show that the refined
    %   passport of $\wt{\sigma}$ has size $1$.
    %   \par
    %   Let $\Sigma_\mathscr{P}\colonequals
    %   \Sigma_{\mathscr{P}(\wt{\sigma})}$
    %   be a set of refined
    %   passport representatives defined in
    %   \eqref{eqn:refined}.
    %   By Algorithm \ref{alg:triples}
    %   there exists a $2$-group permutation triple
    %   $\sigma$ such that the
    %   following sequence is exact.
    %   \begin{equation}
    %     \label{eqn:sigmasequence}
    %     \begin{tikzcd}
    %       1\arrow{r}&\ZZ/2\ZZ\arrow{r}{\iota}
    %                 &\langle\wt{\sigma}\rangle\arrow{r}{\pi}
    %                 &\langle\sigma\rangle\arrow{r}
    %                 &1
    %     \end{tikzcd}
    %   \end{equation}
    %   In other words,
    %   $\wt{\sigma}$ is a lift of $\sigma$.
    %   By Lemma
    %   \ref{lem:dihedralquotient},
    %   quotients of dihedral groups are
    %   dihedral,
    %   and therefore
    %   $\langle\sigma\rangle$ is dihedral.
    %   Thus, by induction,
    %   the refined passport of $\sigma$
    %   has size $1$.
    %   \par
    %   First note that
    %   for $\langle\wt{\sigma}\rangle$
    %   dihedral we have
    %   $Z(\langle\wt{\sigma}\rangle)
    %   =\ZZ/2\ZZ$.
    %   Now, since $\iota(\ZZ/2\ZZ)$ is contained
    %   in the center of
    %   $\langle\wt{\sigma}\rangle$,
    %   we have that
    %   $\iota(\ZZ/2\ZZ) = \ZZ/2\ZZ$.
    %   Thus, we can apply
    %   Lemma \ref{lem:identifiedinlift}
    %   to get that
    %   every element of $\Sigma_\mathscr{P}$
    %   must also be a lift of $\sigma$.
    %   Thus, it is sufficient to prove that every
    %   lift $\wt{\sigma}'$ satisfies one
    %   of the following.
    %   \begin{enumerate}
    %     \item\label{itm:distinctconjclasses}
    %       The $\langle\wt{\sigma}\rangle$
    %       conjugacy class of
    %       $\wt{\sigma}_s'$ differs from
    %       the $\langle\wt{\sigma}\rangle$
    %       conjugacy class of
    %       $\wt{\sigma}_s$ for
    %       some $s\in \{0,1,\infty\}$
    %     \item\label{itm:thereexistsanauto}
    %       There exists an automorphism
    %       $\alpha\in\Aut(\langle\wt{\sigma}\rangle)$
    %       with
    %       $\alpha(\wt{\sigma}_s') = \wt{\sigma}_s$
    %       for all $s\in \{0,1,\infty\}$
    %   \end{enumerate}
    %   Let $\tau\in\iota(\ZZ/2\ZZ)$
    %   be the generator of the image.
    %   There are $2^3 = 8$ preimages of $\sigma$
    %   under the map $\pi$.
    %   Since $\wt{\sigma}_\infty
    %   \wt{\sigma}_1\wt{\sigma}_0=1$
    %   and $\tau$ is central,
    %   there are exactly $4$ preimages
    %   that multiply to $1$.
    %   They are as follows.
    %   \begin{equation}
    %     \label{eqn:preimagesto1}
    %     \{
    %       (\wt{\sigma}_0,\wt{\sigma}_1,\wt{\sigma}_\infty),
    %       (\wt{\sigma}_0,\tau\wt{\sigma}_1,\tau\wt{\sigma}_\infty),
    %       (\tau\wt{\sigma}_0,\wt{\sigma}_1,\tau\wt{\sigma}_\infty),
    %       (\tau\wt{\sigma}_0,\alpha\wt{\sigma}_1,\wt{\sigma}_\infty)
    %     \}
    %   \end{equation}
    %   % Since $\langle
    %   % \wt{\sigma}\rangle$ is abelian,
    %   % the lifts in \eqref{eqn:preimagesto1}
    %   % all define distinct triples of conjugacy
    %   % classes of $\langle\wt{\sigma}\rangle$.
    %   % Thus every lift satisfies \ref{itm:distinctconjclasses} which completes the proof in
    %   % the abelian case.
    %   % The proof in the dihedral case
    %   % follows the same outline as the abelian case.
    %   % By Lemma \ref{lem:dihedralquotient},
    %   % the quotient of a dihedral group is
    %   % dihedral, and we use induction
    %   % as in the abelian case.
    %   % For the induction hypothesis we now assume
    %   % that every $2$-group
    %   % permutation triple of degree
    %   % $d$ with $\langle\sigma\rangle$
    %   % abelian or dihedral has
    %   % refined passport size $1$.
    %   % Let $\wt{\sigma}$
    %   % be a $2$-group
    %   % permutation triple of degree $2d$
    %   % with $\langle\wt{\sigma}\rangle$
    %   % dihedral.
    %   % Using the same notation from
    %   % the abelian case in
    %   % \eqref{eqn:sigmasequence}
    %   % we are required to show that the
    %   % $4$ lifts in
    %   % \eqref{eqn:preimagesto1}
    %   % satisfy either
    %   % \ref{itm:distinctconjclasses}
    %   % or
    %   % \ref{itm:thereexistsanauto}
    %   % in the proof of the abelian case.
    %   Using the notation for dihedral groups
    %   from Example \ref{exm:dihedral}
    %   with $2^{n+1}$ replaced with $2d$
    %   we have the following conjugacy classes
    %   of $\langle\wt{\sigma}\rangle$.
    %   \begin{itemize}
    %     \item
    %       $ \{1\} $,
    %       $ \{a^{\frac{d}{2}}\} $
    %     \item
    %       $c_i\colonequals \{a^i,a^{-i}\}$
    %       for $i\in \{1,\dots,\frac{d}{2}-1\}$
    %     \item
    %       $c_\text{even}
    %       \colonequals
    %       \{a^{2i}b : 0\leq i\leq \frac{d}{2}-1\}$
    %     \item
    %       $c_\text{odd}
    %       \colonequals
    %       \{a^{2i+1}b : 0\leq i\leq \frac{d}{2}-1\}$
    %   \end{itemize}
    %   Note that $c_\text{even}$ and $c_\text{odd}$
    %   consist of all the involutions of the group.
    %   We will say that an involution has
    %   \defi{even parity}
    %   if it is in $c_\text{even }$
    %   and \defi{odd parity}
    %   if it is in $c_\text{odd}$.
    %   \par
    %   If $\wt{\sigma}_s$ is central,
    %   then the conjugacy class of
    %   $\tau\wt{\sigma}_s$ cannot be equal to
    %   the conjugacy class of
    %   $\wt{\sigma}_s$.
    %   Thus we can assume
    %   the conjugacy class of
    %   $\wt{\sigma}_s$
    %   is $c_i$ for some $i$,
    %   $c_\text{even}$, or
    %   $c_\text{odd}$ for every $s\in \{0,1,\infty\}$.
    %   If $\wt{\sigma}_s\in c_i$,
    %   then
    %   \begin{equation}
    %     \label{eqn:alphaci}
    %     \tau\wt{\sigma}_s
    %     =a^{d/2}a^{\pm i}=a^{\frac{d}{2}\pm i}.
    %   \end{equation}
    %   Now $a^{\frac{d}{2}\pm i}\in c_i$
    %   if and only if
    %   $i=\pm d/4$.
    %   Thus we can assume the conjugacy class of
    %   $\wt{\sigma}_s$
    %   is $c_{d/4}$,
    %   $c_\text{even}$, or
    %   $c_\text{odd}$
    %   for every $s\in \{0,1,\infty\}$.
    %   \par
    %   Now let us focus on the condition that
    %   $\wt{\sigma}_\infty\wt{\sigma}_1
    %   \wt{\sigma}_0=1$.
    %   First note that
    %   every element of
    %   $c_\text{even}\,\cup\,c_\text{odd}$
    %   is an involution, so we need
    %   at least one of $\wt{\sigma}_s\in c_{d/4}$
    %   to satisfy
    %   $\wt{\sigma}_\infty\wt{\sigma}_1
    %   \wt{\sigma}_0=1$.
    %   Without loss of generality
    %   assume that
    %   $\wt{\sigma}_\infty\in c_{d/4}$.
    %   Then since $\wt{\sigma}_\infty$
    %   has order $4$,
    %   we must have
    %   $\wt{\sigma}_0,\wt{\sigma}_1
    %   \in c_\text{even}\cup c_\text{odd}$
    %   (again to have a chance of satisfying
    %   their product equals $1$).
    %   Let $\wt{\sigma}_1
    %   = a^kb
    %   \in c_\text{even}\cup
    %   c_\text{odd}$ be an involution.
    %   Then
    %   \begin{align}
    %     \label{eqn:sigma1sigmaoo}
    %     \begin{split}
    %       \wt{\sigma}_\infty\wt{\sigma}_1
    %       &= a^{\pm d/4}a^kb
    %       = a^{\pm (d/4)+k}b
    %       % \wt{\sigma}_1\wt{\sigma}_0
    %       % &= a^kba^{\pm d/4}
    %       % = a^{k\mp (d/4)}b.
    %     \end{split}
    %   \end{align}
    %   The parity
    %   of the involution $\wt{\sigma}_1$
    %   is the same as the
    %   parity of the involution
    %   $\wt{\sigma}_\infty\wt{\sigma}_1$
    %   when $d\geq 8$
    %   and the parity is different for $d=4$.
    %   \begin{claim}
    %     \label{claim:sameparitydoesnotgenerate}
    %     Two involutions of the same parity
    %     do not generate $D_{2d}$ for $d=2^n$
    %     and $n\geq 2$.
    %   \end{claim}
    %   \begin{proof}[Proof of Claim]
    %     Let $a^kb$ and $a^{k'}b$ be involutions
    %     of $D_{2d}$ with $k$ and $k'$ the same
    %     parity.
    %     Since $d\geq 4$ is a power of $2$,
    %     $\pm k, \pm k'$ all have the same parity.
    %     Now since
    %     $\langle a^kb,a^{k'}b\rangle$
    %     is generated by involutions
    %     we have
    %     \begin{align}
    %       \begin{split}
    %         \label{eqn:generatedbyinvolutions}
    %         \langle a^kb,a^{k'}b\rangle
    %         &=
    %         \{a^kb,a^{k'}b\}
    %         \cup
    %         \langle a^kba^{k'}b\rangle\\
    %         &=
    %         \{a^kb,a^{k'}b\}
    %         \cup
    %         \langle a^{k-k'}\rangle\\
    %         &=
    %         \{a^kb,a^{k'}b\}
    %         \cup
    %         \langle a^{k'-k}\rangle
    %       \end{split}
    %     \end{align}
    %     which never contains $a$
    %     since $k$ and $k'$ have the same parity.
    %   \end{proof}
    %   The claim finishes the proof
    %   for $d\geq 8$.
    %   To summarize,
    %   the condition that
    %   $\wt{\sigma}_\infty\wt{\sigma}_1
    %   \wt{\sigma}_0=1$
    %   with each $\wt{\sigma}_s\in
    %   c_{d/4}\cup c_\text{even} \cup
    %   c_\text{odd}$
    %   implies that
    %   the triple $\wt{\sigma}$
    %   consists of exactly one element from
    %   $c_{d/4}$
    %   and the other two elements
    %   are involutions with the same parity.
    %   By the claim,
    %   no such triple can generate $D_{2d}$
    %   which completes the proof for
    %   $d\geq 8$.
    %   % It remains to consider $d=4$ when the $2$
    %   % involutions have opposite parity.
    %   \par
    %   It remains to consider the case
    %   $d=4$ (i.e. $D_{2d} = D_8$)
    %   when the $2$ involutions have
    %   opposite parity.
    %   At this point we could simply appeal
    %   to the explicit computation of
    %   this refined passport in observation
    %   \eqref{eqn:theobservation},
    %   but we also provide a proof in
    %   this specific case for completeness.
    %   \par
    %   In this case,
    %   $c_{d/4} = c_1=\{a,a^{-1}\}$,
    %   $c_\text{even}=\{b,a^2b\}$,
    %   $c_\text{odd}=\{ab,a^3b\}$,
    %   and we are considering triples
    %   $\wt{\sigma}\in
    %   c^3$
    %   where $c=c_1\cup c_\text{even}
    %   \cup c_\text{odd}$.
    %   As discussed above,
    %   to satisfy $\wt{\sigma}_\infty
    %   \wt{\sigma}_1\wt{\sigma}_0=1$
    %   we must have exactly one of
    %   $\wt{\sigma}_s\in c_1$.
    %   Without loss of generality assume
    %   $\wt{\sigma}_\infty\in c_1$.
    %   Then from \eqref{eqn:sigma1sigmaoo}
    %   we have that $\wt{\sigma_1}$
    %   and $\wt{\sigma_0}$ are involutions with
    %   opposite parity.
    %   Without loss of generality let
    %   $\wt{\sigma}_1\in c_\text{odd}$
    %   and $\wt{\sigma}_0\in c_\text{even}$.
    %   We then have the following triples
    %   $\wt{\sigma}$ that generate
    %   $D_8$ and satisfy
    %   $\wt{\sigma}_\infty\wt{\sigma}_1
    %   \wt{\sigma}_0=1$.
    %   \begin{align}
    %     \label{eqn:possibleD8triples}
    %     \begin{split}
    %       (a^2b,ab,a),
    %       (b,a^3b,a),
    %       (b,ab,a^{-1}),
    %       (a^2b,a^3b,a^{-1})
    %     \end{split}
    %   \end{align}
    %   To show that the refined passport
    %   of a $\wt{\sigma}$ taken from
    %   \eqref{eqn:possibleD8triples}
    %   has size $1$,
    %   we are required to find elements of
    %   $\Aut(D_8)$
    %   showing all $4$ of these triples are
    %   equivalent.
    %   Let $f_1\in\Aut(D_8)$ be defined by
    %   $a\mapsto a$, and
    %   $b\mapsto a^2b$.
    %   Let $f_2\in\Aut(D_8)$ be defined by
    %   $a\mapsto a^{-1}$, and
    %   $b\mapsto b$.
    %   Then $f_1$ identifies
    %   $(a^2b,ab,a)$
    %   and
    %   $(b,a^3b,a)$,
    %   $f_2$ identifies
    %   $(b,a^3b,a)$
    %   and
    %   $(b,ab,a^{-1})$,
    %   and
    %   $f_1$ identifies
    %   $(b,ab,a^{-1})$
    %   and
    %   $(a^2b,a^3b,a^{-1})$.
    %   This completes the proof for $D_8$
    %   and thus for the entire dihedral case.
    %   % \begin{equation}
    %   %   \label{eqn:dihedral8}
    %   %   D_8=
    %   %   \{1,a,a^2,a^3,b,ab,a^2b,a^3b\}
    %   % \end{equation}
    % \end{proof}
    % % \begin{proof}[Proof for $\langle\sigma\rangle$ dihedral]
    %   % The proof in the dihedral case
    %   % follows the same outline as the abelian case.
    %   % By Lemma \ref{lem:dihedralquotient},
    %   % the quotient of a dihedral group is
    %   % dihedral, and we use induction
    %   % as in the abelian case.
    %   % For the induction hypothesis we now assume
    %   % that every $2$-group
    %   % permutation triple of degree
    %   % $d$ with $\langle\sigma\rangle$
    %   % abelian or dihedral has
    %   % refined passport size $1$.
    %   % Let $\wt{\sigma}$
    %   % be a $2$-group
    %   % permutation triple of degree $2d$
    %   % with $\langle\wt{\sigma}\rangle$
    %   % dihedral.
    %   % Using the same notation from
    %   % the abelian case in
    %   % \eqref{eqn:sigmasequence}
    %   % we are required to show that the
    %   % $4$ lifts in
    %   % \eqref{eqn:preimagesto1}
    %   % satisfy either
    %   % \ref{itm:distinctconjclasses}
    %   % or
    %   % \ref{itm:thereexistsanauto}
    %   % in the proof of the abelian case.
    %   % Using the notation for dihedral groups
    %   % from Example \ref{exm:dihedral}
    %   % with $2^{n+1}$ replaced with $2d$
    %   % we have the following conjugacy classes
    %   % of $\langle\wt{\sigma}\rangle$.
    %   % \begin{itemize}
    %   %   \item
    %   %     $ \{1\} $,
    %   %     $ \{a^{\frac{d}{2}}\} $
    %   %   \item
    %   %     $c_i\colonequals \{a^i,a^{-i}\}$
    %   %     for $i\in \{1,\dots,\frac{d}{2}-1\}$
    %   %   \item
    %   %     $c_\text{even}
    %   %     \colonequals
    %   %     \{a^{2i}b : 0\leq i\leq \frac{d}{2}-1\}$
    %   %   \item
    %   %     $c_\text{odd}
    %   %     \colonequals
    %   %     \{a^{2i+1}b : 0\leq i\leq \frac{d}{2}-1\}$
    %   % \end{itemize}
    %   % Note that $c_\text{even}$ and $c_\text{odd}$
    %   % consist of all the involutions of the group.
    %   % We will say that an involution has
    %   % \defi{even parity}
    %   % if it is in $c_\text{ even }$
    %   % and \defi{odd parity}
    %   % if it is in $c_\text{odd}$.
    %   % \par
    %   % If $\wt{\sigma}_s$ is central,
    %   % then the conjugacy class of
    %   % $\alpha\wt{\sigma}_s$ cannot be equal to
    %   % the conjugacy class of
    %   % $\wt{\sigma}_s$.
    %   % Thus we can assume
    %   % the conjugacy class of
    %   % $\wt{\sigma}_s$
    %   % is $c_i$ for some $i$,
    %   % $c_\text{even}$, or
    %   % $c_\text{odd}$ for every $s\in \{0,1,\infty\}$.
    %   % If $\wt{\sigma}_s\in c_i$,
    %   % then
    %   % \begin{equation}
    %   %   \label{eqn:alphaci}
    %   %   \alpha\wt{\sigma}_s
    %   %   =a^{d/2}a^{\pm i}=a^{\frac{d}{2}\pm i}.
    %   % \end{equation}
    %   % Now $a^{\frac{d}{2}\pm i}\in c_i$
    %   % if and only if
    %   % $i=\pm d/4$.
    %   % Thus we can assume the conjugacy class of
    %   % $\wt{\sigma}_s$
    %   % is $c_{d/4}$,
    %   % $c_\text{even}$, or
    %   % $c_\text{odd}$
    %   % for every $s\in \{0,1,\infty\}$.
    %   % \par
    %   % Now let us focus on the condition that
    %   % $\wt{\sigma}_\infty\wt{\sigma}_1
    %   % \wt{\sigma}_0=1$.
    %   % First note that
    %   % every element of
    %   % $c_\text{even}\cup c_\text{odd}$
    %   % is an involution, so we need
    %   % at least one of $\wt{\sigma}_s\in c_{d/4}$
    %   % to satisfy
    %   % $\wt{\sigma}_\infty\wt{\sigma}_1
    %   % \wt{\sigma}_0=1$.
    %   % Without loss of generality
    %   % assume that
    %   % $\wt{\sigma}_\infty\in c_{d/4}$.
    %   % Then since $\wt{\sigma}_\infty$ has order $4$,
    %   % we must have
    %   % $\wt{\sigma}_0,\wt{\sigma}_1
    %   % \in c_\text{even}\cup c_\text{odd}$
    %   % (again to have a chance of satisfying
    %   % their product equals $1$).
    %   % Let $\wt{\sigma}_1
    %   % = a^kb
    %   % \in c_\text{even}\cup
    %   % c_\text{odd}$ be an involution.
    %   % Then
    %   % \begin{align}
    %   %   \label{eqn:sigma1sigmaoo}
    %   %   \begin{split}
    %   %     \wt{\sigma}_\infty\wt{\sigma}_1
    %   %     &= a^{\pm d/4}a^kb
    %   %     = a^{\pm (d/4)+k}b
    %   %     % \wt{\sigma}_1\wt{\sigma}_0
    %   %     % &= a^kba^{\pm d/4}
    %   %     % = a^{k\mp (d/4)}b.
    %   %   \end{split}
    %   % \end{align}
    %   % The parity
    %   % of the involution $\wt{\sigma}_1$
    %   % is the same as the
    %   % parity of the involution
    %   % $\wt{\sigma}_\infty\wt{\sigma}_1$
    %   % when $d\geq 8$
    %   % and the parity is different for $d=4$.
    %   % \begin{claim}
    %   %   \label{claim:sameparitydoesnotgenerate}
    %   %   Two involutions of the same parity
    %   %   do not generate $D_{2d}$ for $d=2^n$
    %   %   and $n\geq 2$.
    %   % \end{claim}
    %   % \begin{proof}[Proof of Claim]
    %   %   Let $a^kb$ and $a^{k'}b$ be involutions
    %   %   of $D_{2d}$ with $k$ and $k'$ the same
    %   %   parity.
    %   %   Since $d\geq 4$ is a power of $2$,
    %   %   $\pm k, \pm k'$ all have the same parity.
    %   %   Now since
    %   %   $\langle a^kb,a^{k'}b\rangle$
    %   %   is generated by involutions
    %   %   we have
    %   %   \begin{align}
    %   %     \begin{split}
    %   %       \label{eqn:generatedbyinvolutions}
    %   %       \langle a^kb,a^{k'}b\rangle
    %   %       &=
    %   %       \{a^kb,a^{k'}b\}
    %   %       \cup
    %   %       \langle a^kba^{k'}b\rangle\\
    %   %       &=
    %   %       \{a^kb,a^{k'}b\}
    %   %       \cup
    %   %       \langle a^{k-k'}\rangle\\
    %   %       &=
    %   %       \{a^kb,a^{k'}b\}
    %   %       \cup
    %   %       \langle a^{k'-k}\rangle
    %   %     \end{split}
    %   %   \end{align}
    %   %   which never contains $a$
    %   %   since $k$ and $k'$ have the same parity.
    %   % \end{proof}
    %   % The claim finishes the proof
    %   % for $d\geq 8$.
    %   % To summarize,
    %   % the condition that
    %   % $\wt{\sigma}_\infty\wt{\sigma}_1
    %   % \wt{\sigma}_0=1$
    %   % with each $\wt{\sigma}_s\in
    %   % c_{d/4}\cup c_\text{even} \cup
    %   % c_\text{odd}$
    %   % implies that
    %   % the triple $\wt{\sigma}$
    %   % consists of exactly one element from
    %   % $c_{d/4}$
    %   % and the other two elements
    %   % are involutions with the same parity.
    %   % By the claim,
    %   % no such triple can generate $D_{2d}$
    %   % which completes the proof for
    %   % $d\geq 8$.
    %   % It remains to consider $d=4$ when the $2$
    %   % involutions have opposite parity.
    %   % \par
    %   % It remains to consider the case
    %   % $d=4$ (i.e. $D_{2d} = D_8$).
    %   % In this case,
    %   % $c_{d/4} = c_1=\{a,a^{-1}\}$,
    %   % $c_\text{even}=\{b,a^2b\}$,
    %   % $c_\text{odd}=\{ab,a^3b\}$,
    %   % and we are considering triples
    %   % $\wt{\sigma}\in
    %   % c^3$
    %   % where $c=c_1\cup c_\text{even}\cup c_\text{odd}$.
    %   % As discussed above,
    %   % to satisfy $\wt{\sigma}_\infty
    %   % \wt{\sigma}_1\wt{\sigma}_0=1$
    %   % we must have exactly one of
    %   % $\wt{\sigma}_s\in c_1$.
    %   % Without loss of generality assume
    %   % $\wt{\sigma}_\infty\in c_1$.
    %   % Then from \eqref{eqn:sigma1sigmaoo}
    %   % we have that $\wt{\sigma_1}$
    %   % and $\wt{\sigma_0}$ are involutions with
    %   % opposite parity.
    %   % Without loss of generality let
    %   % $\wt{\sigma}_1\in c_\text{odd}$
    %   % and $\wt{\sigma}_0\in c_\text{even}$.
    %   % We then have the following triples
    %   % $\wt{\sigma}$ that generate
    %   % $D_8$ and satisfy
    %   % $\wt{\sigma}_\infty\wt{\sigma}_1
    %   % \wt{\sigma}_0=1$.
    %   % \begin{align}
    %   %   \label{eqn:possibleD8triples}
    %   %   \begin{split}
    %   %     (a^2b,ab,a),
    %   %     (b,a^3b,a),
    %   %     (b,ab,a^{-1}),
    %   %     (a^2b,a^3b,a^{-1})
    %   %   \end{split}
    %   % \end{align}
    %   % To show that the refined passport
    %   % of a $\wt{\sigma}$ taken from
    %   % \eqref{eqn:possibleD8triples}
    %   % has size $1$,
    %   % we are required to find elements of
    %   % $\Aut(D_8)$
    %   % showing all $4$ of these triples are
    %   % equivalent.
    %   % Let $f_1\in\Aut(D_8)$ be defined by
    %   % $a\mapsto a$, and
    %   % $b\mapsto a^2b$.
    %   % Let $f_2\in\Aut(D_8)$ be defined by
    %   % $a\mapsto a^{-1}$, and
    %   % $b\mapsto b$.
    %   % Then $f_1$ identifies
    %   % $(a^2b,ab,a)$
    %   % and
    %   % $(b,a^3b,a)$,
    %   % $f_2$ identifies
    %   % $(b,a^3b,a)$
    %   % and
    %   % $(b,ab,a^{-1})$,
    %   % and
    %   % $f_1$ identifies
    %   % $(b,ab,a^{-1})$
    %   % and
    %   % $(a^2b,a^3b,a^{-1})$.
    %   % This completes the proof for $D_8$
    %   % and thus for the entire dihedral case.
    %   % % \begin{equation}
    %   % %   \label{eqn:dihedral8}
    %   % %   D_8=
    %   % %   \{1,a,a^2,a^3,b,ab,a^2b,a^3b\}
    %   % % \end{equation}
    % % \end{proof}
    % % \begin{proof}[Alternate proof for $\langle\sigma\rangle$ dihedral]
    %   % The proof in the dihedral case
    %   % follows the same outline as the abelian case.
    %   % By Lemma \ref{lem:dihedralquotient},
    %   % the quotient of a dihedral group is
    %   % dihedral, and we use induction
    %   % as in the abelian case.
    %   % For the induction hypothesis we now assume
    %   % that every $2$-group
    %   % permutation triple of degree
    %   % $d$ with $\langle\sigma\rangle$
    %   % abelian or dihedral has
    %   % refined passport size $1$.
    %   % Let $\wt{\sigma}$
    %   % be a $2$-group
    %   % permutation triple of degree $2d$
    %   % with $\langle\wt{\sigma}\rangle$
    %   % dihedral.
    %   % Using the same notation from
    %   % the abelian case in
    %   % \eqref{eqn:sigmasequence}
    %   % we are required to show that the
    %   % $4$ lifts in
    %   % \eqref{eqn:preimagesto1}
    %   % satisfy either
    %   % \ref{itm:distinctconjclasses}
    %   % or
    %   % \ref{itm:thereexistsanauto}
    %   % in the proof of the abelian case.
    %   % Let $\langle\wt{\sigma}\rangle\cong D_{2d}$
    %   % dihedral of order $2d$.
    %   % Using the notation for dihedral groups
    %   % from Example \ref{exm:dihedral}
    %   % with $2^{n+1}$ replaced with $2d$
    %   % we have the following conjugacy classes
    %   % of $\langle\wt{\sigma}\rangle\cong D_{2d}$.
    %   % \begin{itemize}
    %   %   \item
    %   %     $ \{1\} $,
    %   %     $ \{a^{\frac{d}{2}}\} $
    %   %   \item
    %   %     $c_i\colonequals \{a^i,a^{-i}\}$
    %   %     for $i\in \{1,\dots,\frac{d}{2}-1\}$
    %   %   \item
    %   %     $c_\text{even}
    %   %     \colonequals
    %   %     \{a^{2i}b : 0\leq i\leq \frac{d}{2}-1\}$
    %   %   \item
    %   %     $c_\text{odd}
    %   %     \colonequals
    %   %     \{a^{2i+1}b : 0\leq i\leq \frac{d}{2}-1\}$
    %   % \end{itemize}
    %   % Note that $c_\text{even}$ and $c_\text{odd}$
    %   % consist of all the involutions of the group.
    %   % We will say that an involution has
    %   % \defi{even parity}
    %   % if it is in $c_\text{ even }$
    %   % and \defi{odd parity}
    %   % if it is in $c_\text{odd}$.
    %   % \par
    %   % Let $\rho\colon D_{2d}\to\GL_2(\CC)$ be the
    %   % faithful $2$-dimensional representation
    %   % of $D_{2d}$ defined by
    %   % \begin{equation}
    %   %   \label{eqn:dihedralrepresentation}
    %   %   a\mapsto
    %   %   \begin{bmatrix}
    %   %     \cos(2\pi/d)&-\sin(2\pi/d)\\
    %   %     \sin(2\pi/d)&\cos(2\pi/d)
    %   %   \end{bmatrix},\quad
    %   %   b\mapsto
    %   %   \begin{bmatrix}
    %   %     0&1\\
    %   %     1&0
    %   %   \end{bmatrix}
    %   % \end{equation}
    %   % so that for $k\in \{0,\dots,d-1\}$
    %   % we have
    %   % \begin{equation}
    %   %   \label{eqn:dihedralelts}
    %   %   \rho(a^k)=
    %   %   \begin{bmatrix}
    %   %     \cos(2\pi k/d)&-\sin(2\pi k/d)\\
    %   %     \sin(2\pi k/d)&\cos(2\pi k/d)
    %   %   \end{bmatrix},\quad
    %   %   \rho(a^kb)=
    %   %   \begin{bmatrix}
    %   %     -\sin(2\pi k/d)&\cos(2\pi k/d)\\
    %   %     \cos(2\pi k/d)&\sin(2\pi k/d)
    %   %   \end{bmatrix}
    %   % \end{equation}
    %   % is a complete list of elements of
    %   % the image of $\rho$.
    %   % Let $A^k$ and $A^kB$ denote the images
    %   % of $a^k$ and $a^kb$ in the matrix algebra.
    %   % Let $\alpha = \iota(1)\in D_{2d}$.
    %   % Then
    %   % \begin{equation}
    %   %   \label{eqn:rhoofalpha}
    %   %   \rho(\alpha)
    %   %   =\rho(a^{d/2})
    %   %   =A^{d/2}
    %   %   =
    %   %   \begin{bmatrix}
    %   %     \cos(\pi)&-\sin(\pi)\\
    %   %     \sin(\pi)&\cos(\pi)
    %   %   \end{bmatrix}
    %   %   =
    %   %   \begin{bmatrix}
    %   %     -1&0\\
    %   %     0&-1
    %   %   \end{bmatrix}
    %   % \end{equation}
    %   % Suppose that $\wt{\sigma}_s$ and
    %   % $\alpha\wt{\sigma}_s$ are
    %   % conjugate in $D_{2d}$ for
    %   % some $s\in \{0,1,\infty\}$.
    %   % Let $\charpoly(M)$ denote the
    %   % characteristic polynomial of $M$
    %   % for $M\in\GL_n(\CC)$.
    %   % % The constant term of $\charpoly(A)$
    %   % % is $\det(A)$,
    %   % % the second highest degree
    %   % % coefficient is $-\Tr(A)$,
    %   % % and when $\charpoly(A)$ is quadratic
    %   % % these determine the polynomial.
    %   % Then
    %   % \begin{equation}
    %   %   \label{eqn:charpolysequal}
    %   %   \charpoly(\rho(\alpha\wt{\sigma}_s))
    %   %   =\charpoly(\rho(\wt{\sigma}_s)).
    %   % \end{equation}
    %   % Since $\rho(\alpha)=-1$,
    %   % \eqref{eqn:charpolysequal} becomes
    %   % \begin{equation}
    %   %   \label{eqn:charpolysequalalpha}
    %   %   \charpoly(-\rho(\wt{\sigma}_s))=
    %   %   \charpoly(\rho(\wt{\sigma}_s))
    %   % \end{equation}
    %   % which implies that
    %   % $\Tr(\rho(\wt{\sigma}_s)) =
    %   % -\Tr(\rho(\wt{\sigma}_s))$
    %   % so that the trace is zero.
    %   % \mm{
    %   %    the only way I see to prove these types
    %   %    of theorems is to write down an
    %   %    explicit representation,
    %   %    compute the charpoly in general,
    %   %    and then reason by cases
    %   %    as in the original dihedral proof.
    %   % }
    % % \end{proof}
    % % \begin{proof}[Proof for $G$ Generalized Quaternion]
    % % \end{proof}
    % \begin{remark}
    %   \label{rmk:othermaxnilpotency}
    %   It is reasonable to expect that a similar
    %   direct approach in the proof of
    %   Theorem \ref{thm:conjecturedihedral}
    %   could succeed for other families
    %   of $2$-groups that satisfy
    %   Theorem
    %   \ref{lem:dihedralquotient},
    %   namely generalized quaternion groups
    %   and semi-dihedral groups.
    %   These families also have characteristic
    %   centers so that
    %   Lemma
    %   \ref{lem:identifiedinlift}
    %   can be applied.
    %   However,
    %   a different technique is required for a
    %   general proof,
    %   since experimental evidence in
    %   Section
    %   \ref{sec:grouptheorycomputations}
    %   suggests that $2$-groups
    %   of all nilpotency classes
    %   appear as monodromy groups
    %   of $2$-group permutation triples.
    % \end{remark}
    % Conjecture \ref{conj:refinedpassportsizeone},
    % together with refined field of definition
    % equal to field of moduli for
    % Galois Belyi maps
    % and a strong version of
    % Beckmann's theorem would imply the following.
    % \begin{conj}
    %   \label{conj:fieldofdefinitionofbelyimap}
    %   Every $2$-group Belyi map is
    %   defined over an abelian extension of $\QQ$.
    %   % defined over a cyclotomic field
    %   % $\QQ(\zeta_{2^m})$ for some $m$.
    % \end{conj}
  }
  % \section{Representation theory}{\label{sec:reptheory}
    % \mm{
    %   this section was just some general notes
    %   to see if we could possibly prove
    %   something in general
    % }
    % Let $\sigma$ be a $2$-group permutation triple
    % of degree $d$ with monodromy
    % $G=\langle\sigma\rangle$
    % and $\wt{\sigma}$ a lift of $\sigma$
    % of degree $2d$ and monodromy
    % $\wt{G}=\langle\wt{\sigma}\rangle$.
    % Recall that we have the
    % following exact sequence.
    % \begin{equation}
    %   \label{eqn:reptheoryextension}
    %   \begin{tikzcd}
    %     1\arrow{r}&\ZZ/2\ZZ\arrow{r}{\iota}
    %               &\wt{G}\arrow{r}{\pi}
    %               &G\arrow{r}
    %               &1
    %   \end{tikzcd}
    % \end{equation}
    % Let $\alpha$ be a generator of
    % $\iota(\ZZ/2\ZZ)\leq Z(\wt{G})$.
    % There are $4$ triples of permutations
    % that map to $\sigma$ under $\pi$
    % with the additional property that
    % they multiply to $1$.
    % \begin{equation}
    %   \label{eqn:possiblelifts}
    %   \{
    %     (\wt{\sigma}_0,\wt{\sigma}_1,\wt{\sigma}_\infty),
    %     (\wt{\sigma}_0,\alpha\wt{\sigma}_1,\alpha\wt{\sigma}_\infty),
    %     (\alpha\wt{\sigma}_0,\wt{\sigma}_1,\alpha\wt{\sigma}_\infty),
    %     (\alpha\wt{\sigma}_0,\alpha\wt{\sigma}_1,\wt{\sigma}_\infty)
    %   \}
    % \end{equation}
    % Let $\mathscr{P} = \mathscr{P}(\wt{\sigma})$
    % be the refined passport of $\wt{\sigma}$
    % and $\Sigma_\mathscr{P}$ a set of
    % representatives.
    % Let $\wt{\sigma}'$ be an arbitrary lift in
    % \eqref{eqn:possiblelifts}.
    % To prove that $\#\Sigma_\mathscr{P}=1$
    % we are required to show that
    % every lift in
    % \eqref{eqn:possiblelifts}
    % satisfies at least one of the following
    % conditions.
    % \begin{enumerate}
    %   \item\label{itm:distinctconjclassesreptheory}
    %     The $\wt{G}$
    %     conjugacy class of
    %     $\wt{\sigma}_s'$ differs from
    %     % the $\langle\wt{\sigma}\rangle$
    %     the $\wt{G}$
    %     conjugacy class of
    %     $\wt{\sigma}_s$ for
    %     some $s\in \{0,1,\infty\}$
    %   \item\label{itm:thereexistsanautoreptheory}
    %     There exists an automorphism
    %     $\phi\in\Aut(\wt{G})$
    %     with
    %     $\phi(\wt{\sigma}_s') = \wt{\sigma}_s$
    %     for all $s\in \{0,1,\infty\}$
    % \end{enumerate}
    % Suppose now that $\#\Sigma_\mathscr{P}\geq 2$.
    % According to
    % \ref{itm:distinctconjclasses}
    % it is necessary to have
    % \begin{equation}
    %   \label{eqn:alphatau}
    %   \wt{\sigma}_s\alpha
    %   =
    %   \wt{\tau}\wt{\sigma}_s\wt{\tau}^{-1}
    % \end{equation}
    % for some $s\in \{0,1,\infty\}$
    % and some $\wt{\tau}\in\wt{G}$.
    % Let $\tau = \pi(\wt{\tau})$.
    % Applying $\pi$ to
    % \eqref{eqn:alphatau}
    % we get the following.
    % \begin{align}
    %   \label{eqn:taucentralizes}
    %   \begin{split}
    %     \pi(\wt{\sigma}_s\alpha)=
    %     \pi(\wt{\tau}\wt{\sigma}_s\wt{\tau}^{-1})
    %     &\implies
    %     \sigma_s = \tau\sigma_s\tau^{-1}\\
    %     &\implies
    %     \tau\in C_G(\sigma_s)
    %   \end{split}
    % \end{align}
    % \begin{remark}
    %   \label{rmk:size3}
    %   If we assume the more stringent condition
    %   that $\#\Sigma_\mathscr{P}\geq 3$,
    %   then
    %   \eqref{eqn:possiblelifts}
    %   implies that
    %   \eqref{eqn:alphatau}
    %   holds for all $s\in \{0,1,\infty\}$.
    %   Then by
    %   \eqref{eqn:taucentralizes}
    %   we have that a centralizing element
    %   $\tau_s\in C_G(\sigma_s)$ for every $s$.
    %   If all these $\tau_s$ are equal to some
    %   $\tau\in G$, then $\tau$ lives in the
    %   center of $G$
    %   and the element
    %   $\wt{\tau}\in\pi^{-1}(Z(G))$
    %   upstairs
    %   satisfies
    %   \eqref{eqn:alphatau}
    %   for all $s$.
    % \end{remark}
    % Let $\rho\colon \wt{G}\to\GL(V)$ be a
    % representation
    % of $\wt{G}$ over $\CC$.
    % For $M\in\GL(V)$, let
    % $\charpoly(M)$ denote the
    % characteristic polynomial of $M$.
    % Let $A = \rho(\alpha)$.
    % Then $A^2=1$.
    % Since $\alpha\in Z(\wt{G})$,
    % we have
    % $\rho(\alpha \wt{g}) = \rho(\wt{g}\alpha)$
    % so that $A$ commutes with $\rho(\wt{g})$
    % for all $\wt{g}\in \wt{G}$.
    % If $\rho$ is faithful,
    % then $A\neq 1$.
    % In general,
    % $\rho(\wt{g})$ will have finite order
    % and thus can be diagonalized
    % over a cyclotomic field.
    % If \eqref{eqn:alphatau}
    % is satisfied for some $s$,
    % then
    % \begin{equation}
    %   \label{eqn:rhoalphatau}
    %   \rho(\wt{\sigma}_s\alpha)
    %   =
    %   \rho(\wt{\tau}\wt{\sigma}_s\wt{\tau}^{-1})
    % \end{equation}
    % which implies
    % \begin{align}
    %   \label{eqn:charpoly}
    %   \begin{split}
    %     \charpoly(\rho(\wt{\sigma}_s\alpha))
    %     &=\charpoly(\rho(\wt{\tau}\wt{\sigma}_s\wt{\tau}))\\
    %     &=\charpoly(\rho(\wt{\sigma}_s)).
    %   \end{split}
    % \end{align}
    % If we let $S = \rho(\wt{\sigma}_s)$
    % and $A=\rho(\alpha)$ (as above),
    % then \eqref{eqn:charpoly}
    % becomes
    % \begin{equation}
    %   \label{eqn:charpoly2}
    %   \charpoly(SA) = \charpoly(S).
    % \end{equation}
    % Since $A^2=1$ and $A\neq 1$,
    % we have that $A$ is a diagonal matrix
    % with $\pm 1$ along the diagonal with
    % at least one occurrence of $-1$.
    % Over $\CC$,
    % we can apply Shur's Lemma
    % to get that $A$ is a scalar matrix
    % and therefore $A=-1$.
    % The eigenspaces of $A$ are $\wt{G}$-stable.
    % % two nxn matrices A,B over char(K)=0
    % % have same charpolys iff
    % % trace(A^k) = trace(A^k) for all k \ge 0
  % }
}
\chapter{Computing equations}{\label{chapter:equations}
  In this chapter we discuss how to compute
  equations for $2$-group Belyi maps
  corresponding to the $2$-group
  permutation triples
  computed in Chapter
  \ref{chapter:grouptheory}.
  As was the case for computing the permutation
  triples,
  the algorithm to compute equations follows an
  iterative approach.
  In this chapter we
  construct the $2$-group Belyi maps
  as towers of quadratic extensions
  of function fields.
  We begin in Section
  \ref{sec:numberfields}
  by discussing the analogous situation
  over number fields.
  In Section \ref{sec:algebraicfunctionfields}
  and Section \ref{sec:functionfieldextensions}
  we discuss the relevant
  background about algebraic function fields.
  The algorithms to compute
  equations for
  $2$-group Belyi maps
  (over $\FF_q$)
  are described in Section
  \ref{sec:functionfieldalgorithm},
  the implementation in characteristic zero
  is detailed in Section
  \ref{sec:characteristiczero},
  and the results of these computations
  can be found in
  Section
  \ref{sec:resultsequations}.
  \section{Quadratic extensions of number fields}{
    \label{sec:numberfields}
    By way of motivation,
    let $F$ be a number field
    and let $\ZZ_F$ denote the ring of integers
    of $F$.
    Kummer theory tells us that quadratic
    extensions of $F$ are in bijection with
    nontrivial cosets $dF^{\times 2}$
    in the quotient
    $F^\times/F^{\times 2}$;
    such a coset defines a
    quadratic extension
    $F(\sqrt{d})$.
    Conversely,
    let $F(\alpha)$ be a quadratic extension of $F$.
    The discriminant of the minimal polynomial
    of $\alpha$ defines the bijection in the other
    direction.
    % \begin{equation}
      % \label{eqn:kummer}
      % \begin{tikzpicture}[scale=1]
      %   \node at (-4,0) (extensions)
      %   {
      %     $
      %       \left\{
      %         \begin{minipage}{26ex}
      %           quadratic extensions $K/F$
      %         \end{minipage}
      %       \right\}
      %     $
      %   };
      %   \node at (4,0) (discriminants)
      %   {
      %     $
      %       \left\{
      %         \begin{minipage}{26ex}
      %           $d\in F^\times/F^{\times 2}$
      %           with
      %           $d\not\in F^{\times 2}$
      %         \end{minipage}
      %       \right\}
      %     $
      %   };
      %   \draw[<->] (extensions) to node [above] {$\sim$} node {} (discriminants);
      %   % \draw[<->] (dessins) to node [above,rotate=35] {$\sim$} node {} (fields);
      % \end{tikzpicture}
    % \end{equation}
    \par
    Let $\Pl(F)$ denote the set of places of $F$
    and let $S_\infty$ denote the Archimedean
    places.
    For $v\in\Pl(F)\setminus S_\infty$
    let $\mathfrak{p}_v$
    be the prime ideal of $\ZZ_F$ corresponding
    to $v$.
    Let $S\subset\Pl(F)\setminus S_\infty$
    be a finite set of nonarchimedean places,
    and further suppose that each place of $S$
    has odd order residue field.
    We aim to answer the following question.
    \begin{question}
      \label{ques:construct}
      How do we construct a
      quadratic extension of
      $F$ ramified at $\mathfrak{p}_v$
      for all $v\in S$
      and unramified at all
      nonarchimedean places outside of $S$?
      If so, then
      how \emph{unique} is the construction?
    \end{question}
    To formulate this question more clearly,
    let
    $
      % \label{eqn:aadivisor}
      \mathfrak{a}\colonequals
      \prod_{v\in S}\mathfrak{p}_v
    $
    encode the primes
    we want to ramify in this quadratic extension.
    There are three possibilities.
    \begin{itemize}
      \item
        It is possible that no such
        extension exists.
      \item
        $\mathfrak{a}=(d)$ is principal
        % we have $\mathfrak{a} = (d)$,
        and the extension
        $F(\sqrt{d})$ is a quadratic extension
        ramified exactly at each $\mathfrak{p}_v$,
        and the generator $d$ is unique
        up to multiplication
        by a unit in $\ZZ_F^{\times 2}$.
      \item
        If $\mathfrak{a}$ is not principal,
        it is possible there
        exists a fractional ideal
        $\mathfrak{b}$ such that
        $\mathfrak{a}\mathfrak{b}^2=(d)$.
        In this case,
        we can again construct the extension
        $F(\sqrt{d})$ with the prescribed
        ramification,
        but $d$ is only unique up to
        the ideal $\mathfrak{b}$ used
        to construct it.
    \end{itemize}
    Let us consider the last case more closely.
    Let $\Cl_F$ denote the class group of $F$
    and for a fractional ideal
    $\mathfrak{c}$ let $[\mathfrak{c}]$
    denote its ideal class in $\Cl_F$.
    The equation $\mathfrak{a}\mathfrak{b}^2=(d)$
    means that $[\mathfrak{a}] = [\mathfrak{b}^{-2}]$
    so that $[\mathfrak{a}]\in\Cl_F^2$.
    Moreover, if we take an element
    $[\mathfrak{c}]$
    of order $2$
    in $\Cl_F$,
    then
    \begin{equation}
      \label{eqn:twotorsionintheclassgroup}
      [\mathfrak{a}\mathfrak{b}^2]
      % =[\mathfrak{a}\mathfrak{b}^2][(1)]
      % =[\mathfrak{a}\mathfrak{b}^2]
      =[\mathfrak{a}\mathfrak{b}^2][\mathfrak{c}^2]
      =[\mathfrak{a}(\mathfrak{bc})^2].
    \end{equation}
    Thus,
    in the case where $\mathfrak{a}$ is not
    principal,
    but there exists $\mathfrak{b}$ with
    $\mathfrak{a}\mathfrak{b}^2$ principal,
    we have
    $[\mathfrak{a}]\in\Cl_F^2$ and
    $[\mathfrak{b}]$ is unique up to
    multiplication by $[\mathfrak{c}]\in\Cl_F[2]$.
    % an order $2$ element in $\Cl_F$.
    \par
    We can now formulate our precise goal.
    Given $\mathfrak{a}$
    (encoding ramification data),
    find $\mathfrak{b}^2$ and $d$ such that
    $\mathfrak{a}\mathfrak{b}^2=(d)$.
    In the following sections we will rephrase this
    problem in the function field setting.
  }
  \section{Curves and algebraic function fields}{
    \label{sec:algebraicfunctionfields}
    In this section we summarize the
    setting in which the algorithms of this
    chapter are stated.
    There are many comprehensive resources
    on this topic such as
    \cite[I.6]{hartshorne},
    \cite{stick},
    and
    \cite{rosen}.
    \par
    First, let $K$ be a perfect field.
    \begin{definition}
      \label{def:algfuncfield}
      An \defi{algebraic function field
      in one variable over $K$}
      is a field extension $F$
      over $K$
      of transcendence degree $1$.
      That is,
      there exists $x\in F$ such that
      $x$ is transcendental over $K$
      and $[F:K(x)]$ is finite.
    \end{definition}
    \begin{definition}
      \label{def:exactconstantfield}
      We say the $K$ is the
      \defi{constant field}
      of $F$.
      The \defi{exact constant field}
      of $F$ is the algebraic closure
      of $K$ in $F$.
    \end{definition}
    \begin{remark}
      \label{rmk:constantfields}
      In theory we can assume that
      $K$ is the exact constant field of $F$,
      but in practice for $K$ a number field
      or $\FF_q$ we try to work with
      constant fields that are as simple as
      possible.
    \end{remark}
    \begin{example}
      \label{exm:functionfieldofcurve}
      Let $X$ be an irreducible affine plane curve
      defined by the defining equation
      $f(x,y)=0$ with
      $f\in K[x,y]$.
      Then \defi{the function field of $X$},
      denoted by $K(X)$, is defined to be
      the field of fractions of the coordinate
      ring
      $\frac{K[x,y]}{(f(x,y))}$
      of $X$.
    \end{example}
    \begin{definition}
      \label{def:functionfieldplace}
      A \defi{place} $P$ of $F$ is the
      maximal ideal of some discrete
      valuation ring $\OO_P$ of $F$.
      We denote the valuation on $F$
      corresponding to $P$ by $\ord_P$.
      The set of places of $F$
      is denoted $\Pl(F)$.
      The \defi{degree} of a place $P$,
      denoted $\deg(P)$,
      is the index
      $[\OO_P/P:K]$
      of the
      \defi{residue class field}
      as an extension of $K$.
    \end{definition}
    \begin{definition}
      \label{def:divisorclassgroup}
      The \defi{divisor class group}
      of $F$, denoted $\Div(F)$,
      is the free abelian group
      generated by the places of $F$.
      A \defi{divisor}
      $D\in\Div(F)$ is represented by
      a formal sum of places
      $D = \sum_{P\in\Pl(F)}a_PP$
      with $a_P\in\ZZ$ for all $P$
      and $a_P=0$ for all but finitely many $P$.
      We define $\ord_P(D)$ to be the
      coefficient $a_P$ in the representation of
      $D$.
    \end{definition}
    \begin{definition}
      \label{def:suppordeg}
      The \defi{support} of a divisor
      $D=\sum_{P\in\Pl(F)}a_PP$,
      denoted $\supp(D)$,
      is $\{P\in\Pl(F):a_P\neq 0\}$.
      The \defi{degree} of $D$
      is defined to be
      $\deg(D)\colonequals\sum_{P\in\Pl(F)}
      a_P\deg(P)$.
      The subgroup of $\Div(F)$
      consisting of the set of
      \defi{degree zero divisors of $F$}
      is denoted by $\Div^0(F)$.
    \end{definition}
    \begin{definition}
      \label{def:principaldivisors}
      The image of the map
      $\ddiv\colon F^\times\to\Div(F)$
      defined by
      \begin{equation}
        \label{eqn:divf}
        \ddiv(f)=
        \sum_{P\in\Pl(F)}\ord_P(f)P
      \end{equation}
      is the subgroup of
      \defi{principal divisors}
      of $F$
      and denoted
      $\Princ(F)$.
      Two divisors $D_1,D_2\in\Div(F)$
      are \defi{linearly equivalent}
      if $D_1-D_2\in\Princ(F)$.
    \end{definition}
    \begin{definition}
      \label{def:pic}
      The \defi{Picard group} of $F$
      is defined by
      $\Pic(F)\colonequals\Div(F)/\Princ(F)$.
      The \defi{Jacobian} of $F$
      is defined by
      $\Pic^0(F)\colonequals\Div^0(F)/\Princ(F)$.
    \end{definition}
    \begin{definition}
      \label{def:effectivedivisor}
      There is a partial order on
      $\Div(F)$
      defined by
      $D_1\geq D_2$ if and only if
      $\ord_P(D_1)\geq\ord_P(D_2)$
      for all $P\in\Pl(F)$.
      We say that $D\in\Div(F)$ is
      \defi{effective}
      if $D\geq 0$.
    \end{definition}
    \begin{definition}
      \label{def:riemannroch}
      The \defi{Riemann-Roch space}
      of a divisor $D\in\Div(F)$ is defined by
      $\mathscr{L}(D)
      \colonequals
      \{
        f\in F : \ddiv(f)+D\geq 0
      \}
      \cup
      \{0\}
      $.
    \end{definition}
    Now that we have some of the basic definitions
    of algebraic function fields,
    we also need to 
    introduce some terminology
    concerning extensions of algebraic
    function fields.
    \begin{definition}
      \label{def:functionfieldextensions}
      Let $F,F'$ be algebraic function
      fields over constant fields $K,K'$
      respectively
      and suppose that
      $F\subseteq F'$ and $K\subseteq K'$.
      When these conditions are satisfied
      we say that
      $F'$ is an
      \defi{algebraic function field extension}
      of $F$.
    \end{definition}
    Every place $P'$ of $F'$ \defi{lies above}
    a unique place $P = F\cap P'$ of $F$.
    Every place $P$ of $F$ \defi{lies below}
    finitely many places $P'$ of $F'$.
    We denote a place $P'$ above $P$
    by $P'|P$.
    When $P'|P$ we can view $\OO_{P'}$
    as a free $\OO_P$-module
    of rank $[F':F]$
    and
    $\OO_P=\OO_{P'}\cap F$.
    \par
    We now summarize the
    fundamental identity
    from algebraic number theory
    in the function field setting.
    Let $F'$ over $K'$ be an extension
    of $F$ over $K$
    and let $P'$ be a place of $F'$ above
    $P\in\Pl(F)$.
    There is a unique positive integer
    denoted $e(P'|P)$
    such that
    $\ord_{P'}(f) = e(P'|P)\ord_P(f)$
    for all $f\in F$.
    The positive integer
    $e(P'|P)$ is called the
    \defi{ramification index of $P'|P$}.
    The \defi{residue degree},
    denoted $f(P'|P)$
    is defined to be the index
    $[\OO_{P'}/P':\OO_{P}/P]$
    which makes sense after embedding
    $\OO_P/P$ into $\OO_{P'}/P'$.
    The \defi{fundamental identity} is then
    given by the equation
    \begin{equation}
      \label{eqn:fundamentalidentity}
      [F':F]=
      \sum_{P'|P}e(P'|P)f(P'|P).
    \end{equation}
    \par
    We now summarize some necessary facts about
    extending the field of constants of
    and algebraic function field.
    \begin{definition}
      \label{def:constantfieldextension}
      An extension $F'$ (with constants $K'$)
      of $F$ (with constants $K$)
      is a \defi{constant field extension}
      if $F'= FK'$.
    \end{definition}
    Constant field extensions are one way in
    which the function field setting differs
    from the number field setting.
    When $F'=FK'$ is a constant field extension
    of $F$ over $K$, there are
    several observations to make.
    First, the relative degree over
    the rational function field does not change,
    that is,
    $[F:K(x)] = [F':K'(x)]$ for
    all $x\in F\setminus K$.
    Second,
    no places of $F$ ramify in $F'$.
    Lastly,
    define the \defi{conorm map} by
    \begin{equation}
      \label{eqn:conorm}
      \con_{F'|F}(P)
      \colonequals\sum_{P'|P}e(P'|P)P'\in\Div(F').
    \end{equation}
    The conorm map extends to a homomorphism
    on divisor, principal divisors,
    and hence on divisor classes.
    Since constant field extensions
    are unramified,
    the conorm map
    induces an injection
    $\Pic(F)\hookrightarrow\Pic(F')$.
    \par
    We conclude this section by proving a
    lemma we will need later in this chapter.
    \begin{lemma}
      \label{lem:kummerramification}
      % rosen Prop 10.3
      Let $aF^{\times 2}$ be a nontrivial
      coset of $F^\times/F^{\times 2}$
      and consider the extension
      $K\colonequals F(\sqrt{a})$.
      Then a prime $P$ of $F$ is ramified
      in $K$ if and only if
      $\ord_P(a)$ is odd.
    \end{lemma}
    \begin{proof}
      Since $a$ is not a square in $F$,
      the extension is quadratic.
      Suppose $\ord_P(a)$ is odd
      and let $\mathfrak{p}$ be
      a place above $P$ in $K$.
      Then we have
      \begin{equation}
        \label{eqn:sqrtsquared}
        2\ord_{\mathfrak{p}}(\sqrt{a})
        = \ord_{\mathfrak{p}}(a)
        = e(\mathfrak{p}/P)\ord_P(a).
      \end{equation}
      Since $\ord_P(a)$ is odd,
      \eqref{eqn:sqrtsquared}
      implies that $2$ divides
      $e(\mathfrak{p}/P)$ so that
      $P$ is ramified in $K$.
      Moreover, this says that
      $e(\mathfrak{p}/P) = 2$.
      \par
      For the converse
      we check ramification locally
      at the place $P$.
      Suppose $e\colonequals\ord_P(a)$ is even.
      Choose a uniformizer $t$ at $P$
      and let $b = a/t^{(e/2)}$.
      Then $F(\sqrt{a}) = F(\sqrt{b})$,
      and $\ord_P(b) = 0$ implies
      that $F(\sqrt{b}) = F(\sqrt{a})$
      is unramified at $P$.
      \par
      For a more general proof
      of this in
      arbitrary Kummer extensions
      see \cite[Proposition 10.3]{rosen}.
    \end{proof}
    % \mm{
    %   I haven't quite decided how I want to organize
    %   this section, so you can omit reading this for
    %   now.
    %   Instead, here is a list of things
    %   that I need to explain in this section.
    %   \begin{itemize}
    %     \item
    %       Places of an absolute extension
    %     \item
    %       Places of a relative extension
    %     \item
    %       ramification in
    %       $F/\FF_q(x)$
    %     \item
    %       ramification in
    %       $F(\sqrt{f})/F$
    %     \item
    %       $\Div(F),\Pic(F),\Pic^0(F)$
    %       (with sensitivity to constants)
    %     \item
    %       $\mathscr{L}(D)$
    %       for $D\in\Div(F)$
    %   \end{itemize}
    %   anything to add?
    % }
    % Before describing the function field
    % analog to
    % Section \ref{sec:numberfields},
    % we provide some background material on
    % algebraic function fields.
    % Let $\FF_q$ be a finite field of
    % characteristic $p\neq 2$
    % and let $\phi\colon X\to\PP^1$
    % be a degree $d$ morphism of
    % regular complete irreducible curves
    % over $\FF_q$.
    % Let $\FF_q(x)$ be the rational function
    % field (the field of fractions
    % of the polynomial ring $\FF_q[x]$)
    % and $F$ the degree $d$ field extension
    % of $\FF_q(x)$ corresponding to $\phi$.
    % Let $U_0$ and $U_\infty$ be the open sets
    % on $\PP_1$ so that
    % the ring of regular functions on $U_0$
    % is $\FF_q[x]$ and the ring of
    % regular functions on $U_\infty$ is
    % $\FF_q[1/x]$.
    % Let $\{V_0,V_\infty\}$ be the open cover
    % of $X$ obtained by pulling back $U_0$ and
    % $U_\infty$ via $\phi$.
    % Define $R_0$ and $R_\infty$ to be the ring
    % of regular functions on $V_0$ and $V_\infty$
    % respectively.
    % Let $\wt{R_0}$ and $\wt{R_\infty}$
    % denote the normalizations of $R_0$ and
    % $R_\infty$ respectively.
    % \par
    % Alternatively,
    % we can think of $\wt{R_0}$ as the integral
    % closure of $\FF_q[x]$ in $F$
    % and $\wt{R_\infty}$ as the integral closure
    % of $\mathcal{O}_\infty$ in $F$
    % where
    % $\mathcal{O}_\infty$ denotes the valuation
    % ring of $\FF_q(x)$ determined by the
    % degree.
    % \mm{decide on above perspective or below...}
    % Let $F$ be a function field
    % with field of constants the
    % finite field $\FF_q$
    % of characteristic $p\neq 2$.
    % \begin{definition}
    %   \label{def:valuationring}
    %   A \defi{valuation ring} of $F$
    %   is a ring $\mathcal{O}$
    %   strictly contained in $F$
    %   and strictly containing $\FF_q$
    %   such that for every $a\in F$
    %   either $a\in\mathcal{O}$
    %   or $a^{-1}\in\mathcal{O}$.
    % \end{definition}
    % \begin{definition}
    %   \label{def:places}
    %   A \defi{place} of $F$ is a maximal ideal of
    %   some valuation ring $\mathcal{O}$ of $F$.
    %   Let $\Pl(F)$ denote the set of places of $F$.
    % \end{definition}
    % Every valuation ring $\mathcal{O}$
    % turns out to be a local ring uniquely
    % determined by its maximal ideal $P$,
    % so it is customary to write
    % $\mathcal{O}_P$ in place of $\mathcal{O}$.
    % $\mathcal{O}_P$ is a
    % discrete valuation ring with
    % maximal ideal $P=t\mathcal{O}$,
    % and therefore
    % defines a map $v_P\colon F\to\ZZ\cup\infty$
    % as follows.
    % Every $a\in F^\times$
    % can be expressed as $a=ut^n$
    % where $n\in\ZZ$ and $u\in\mathcal{O}_P^\times$.
    % This expression is unique up to the unit $u$.
    % A place of $F$ therefore defines a function
    % \begin{align}
    %   \label{eqn:vP}
    %   v_P(a) =
    %   \begin{cases}
    %     n&\text{ if }a=ut^n\\
    %     \infty&\text{ if }a=0
    %   \end{cases}
    % \end{align}
    % which induces a nonarchimedean absolute value
    % on $F$.
    % We can use this function to recharacterize
    % the valuation ring as follows.
    % \begin{align}
    %   \label{eqn:valuationringfromabs}
    %   \begin{split}
    %     \mathcal{O}_P
    %     &= \{a\in F : v_P(a)\geq 0\}\\
    %     \mathcal{O}_P^\times
    %     &= \{a\in F : v_P(a)=0\}\\
    %     P
    %     &=\{a\in F : v_P(a)>0\}
    %   \end{split}
    % \end{align}
    % \begin{definition}
    %   \label{def:residueclassfield}
    %   Let $P\in\Pl(F)$.
    %   The field $\mathcal{O}_P/P$ is called the
    %   \defi{residue class field} of $P$.
    %   The field $\mathcal{O}_P/P$ is a finite
    %   extension of $\FF_q$
    %   and we define the
    %   \defi{degree of a place $P$}
    %   to be the index
    %   $[\mathcal{O}_P/P:\FF_q]$.
    % \end{definition}
    % \mm{define extensions K/F, places above places, and ramification of places}
    % \mm{divisors, principal divisors, and Picard}
  }
  \section{Quadratic extensions of function fields}{
    \label{sec:functionfieldextensions}
    We now address two tasks concerning
    quadratic extensions of function fields
    that we need for the algorithms
    in Section \ref{sec:functionfieldalgorithm}.
    \par
    The first task is the problem
    (analogous to the problem in Section
    \ref{sec:numberfields})
    of
    finding a quadratic extension $F(\sqrt{f})/F$
    with ramification
    (in the relative extension)
    prescribed by
    $R\in\Div(F)$.
    By Proposition
    \ref{lem:kummerramification},
    we can take all nonzero coefficients
    of $R$ to have absolute value $1$.
    As was the case for number fields,
    there are three possibilities.
    \par
    First, it could be the case that no such
    extension exists in which case
    there is nothing to do.
    The other easy case occurs when $R$ is
    a principal divisor so that
    $R = \ddiv(f)$ for some $f\in F^\times$.
    In this case, the extension
    $F(\sqrt{f})$ has the desired ramification
    determined by $R$.
    \par
    The last case occurs when $R$
    is not principal, but there exists
    $D\in\Div(F)$ with $R-2D=\ddiv(f)$
    for some $f\in F$.
    By
    Proposition \ref{lem:kummerramification},
    the extension $F(\sqrt{f})/F$
    will be ramified precisely at the
    places in the
    support of
    $R$.
    For $D\in\Div(F)$,
    let $[D]$ denote the class of $D$
    in $\Pic(F)$.
    Since $[R-2D]=0\in\Pic^0(F)$,
    we have that $R\in 2\Pic(F)$.
    Moreover,
    if we let $[T]\in\Pic^0(F)[2]$,
    then
    \begin{equation}
      \label{eqn:twotorsionpoint}
      [R-2D] = [R-2D]-[2T] = [R-2(D+T)].
    \end{equation}
    Thus,
    in the case where $R-2D$ is principal,
    we have $R\in 2\Div(F)$
    and $D$ is unique up to an order $2$
    element of $\Pic(F)$.
    The fact that we cannot determine $D$
    exactly requires us to compute
    $\Pic(F)$ to carry out the desired
    computations.
    This forces us to work over $\FF_q$
    where Picard group computations
    are implemented.
    \par
    The other task is to determine
    when the quadratic extension
    $F(\sqrt{f})$ over $F$ is Galois
    (as an absolute extension of $\FF_q(x)$)
    given that $F$ is Galois
    over $\FF_q(x)$.
    Kummer theory tells us precisely when
    such an extension is Galois in the following
    Lemma.
    \begin{lemma}
      \label{lem:isgalois}
      Let $F$ be a Galois extension of
      $\FF_q(x)$
      with Galois group $G$ and
      let $f\in F^\times/F^{\times 2}$.
      Then the quadratic extension
      $F(\sqrt{f})$ is Galois
      as an absolute extension of
      $\FF_q(x)$ if and only if
      $\sigma(f)/f$ is a square
      in $F$
      for every $\sigma\in G$.
    \end{lemma}
    We can now formulate these concepts
    into an algorithm over $\FF_q$.
  }
  \section{An algorithm over $\FF_q$}{
    \label{sec:functionfieldalgorithm}
    Let $F$ be function field
    with field of constants
    $\FF_q$ with $q=p^r$ and $p\neq 2$.
    Let $\FF_q(x)$ denote the rational
    function field in the variable $x$.
    \begin{definition}
      \label{def:tamebelyimap}
      A \defi{tame Belyi map}
      over a field $K$
      is a branched cover
      $\phi\colon X\to\PP_K^1$
      over $K$ unramified away from
      $\{0,1,\infty\}$
      such that the characteristic of $K$
      does not divide the degree of $\phi$.
    \end{definition}
    % \begin{definition}
      % \label{def:tamebelyimap}
      % A \defi{tame Belyi map}
      % over $\FF_q$ is a tame extension of function
      % fields
      % $\FF_q(x)\hookrightarrow F$
      % with $[F:\FF_q(x)]$ coprime to $p$
      % unramified outside of all
      % places above $\{0,1,\infty\}$.
    % \end{definition}
    A tame Belyi map $\phi\colon X\to\PP_K^1$
    corresponds to an extension of function fields
    $K(\PP^1)\hookrightarrow K(X)$,
    unramified away from $\{0,1,\infty\}$,
    with the characteristic of $K$ not dividing
    $[K(X):K(\PP^1)]$.
    \begin{definition}
      \label{def:galoistamebelyi}
      A \defi{Galois tame Belyi map}
      is a tame Belyi map $\phi\colon X\to\PP_K^1$
      that is also Galois
      (see Definition \ref{def:galoisbelyi}).
    \end{definition}
    \begin{definition}
      \label{def:pgroupbelyimapmodq}
      A \defi{$2$-group Belyi map modulo $q$}
      is a Galois tame Belyi map
      $\phi\colon X\to\PP_{\FF_q}^1$
      with $\deg(\phi)$ a power of $2$.
      % is a Galois extension of
      % function fields
      % $\FF_q(x)\hookrightarrow F$ with
      % $[F:\FF_q(x)]=2^m$
      % unramified outside of
      % all places above $\{0,1,\infty\}$.
    \end{definition}
    A $2$-group Belyi map modulo $q$ corresponds
    to a function field extension
    $\FF_q(x)\hookrightarrow F$ with
    $[F:\FF_q(x)]$ a power of $2$
    unramified outside of all places above
    $\{0,1,\infty\}$.
    We will refer to $2$-group Belyi maps modulo $q$
    by their corresponding function fields.
    \begin{remark}
      \label{rmk:tamebelyi}
      % A Belyi map
      % $\phi\colon X\to\PP^1$ over a
      % field of characteristic $p$
      % where $p\nmid\Mon(\phi)$
      % is called \defi{tame}.
      The theory of tame Belyi maps is
      similar to the theory in characteristic zero.
    \end{remark}
    % \begin{definition}\label{def:belyiisomodq}
      % Two Belyi maps
      % $\FF_q(x)\hookrightarrow F_1$ and
      % $\FF_q(x)\hookrightarrow F_2$,
      % are \defi{isomorphic}
      % if there exists an isomorphism
      % $\varphi\colon F_1\to F_2$
      % of function fields over
      % $\FF_q(x)$
      % such that the diagram
      % \begin{equation}
      %   \label{eqn:belyiisomodq}
      %   \begin{tikzcd}
      %     F_1\arrow{rr}{\varphi}
      %     \arrow[-]{dr}[swap]{}
      %     &
      %     &F_2\arrow[-]{dl}{}\\
      %     &\FF_q(x)
      %   \end{tikzcd}
      % \end{equation}
      % commutes.
    % \end{definition}
    % \mm{
    %   Theorem that establishes precisely
    %   the connection between
    %   $2$-group Belyi maps
    %   and $2$-group Belyi maps mod $q$.
    % }
    We now describe the algorithms
    to iteratively compute $2$-group Belyi maps
    modulo $q$.
    The basic idea is to compute a tower
    of quadratic extensions
    by extracting square roots
    to work our way up the tower.
    Since we are concerned with Galois
    Belyi maps,
    we want to make sure
    that each intermediate field
    is Galois as an absolute extension
    of $\FF_q(x)$.
    To start,
    we describe
    the degree $2$ Belyi maps.
    % Algorithm
    % \ref{alg:degree2FF}
    % details the degree $2$ case,
    % and
    % Algorithm
    % \ref{alg:getcandidates}
    % describes the iteration.
    \begin{lemma}
      \label{lem:degree2FF}
      % Let the notation be
      % as above in the beginning of
      % Section
      % \ref{sec:functionfieldalgorithm}.
      % Let $d=2$.
      % \mm{degree 2 Belyi maps mod q up to iso}
      Let $K$ be a field of characteristic not equal to $2$.
      The three degree $2$ Belyi maps
      up to isomorphism
      are
      \begin{equation}
        \label{eqn:degree2FF}
        F_{(1,2,2)} =
        \frac{K(x)[y]}{(y^2 + x - 1)},\;
        F_{(2,1,2)} =
        \frac{K(x)[y]}{(y^2 - x)},
        \text{ and }
        F_{(2,2,1)} =
        \frac{K(x)[y]}{(y^2 - x^2 + x)}.
      \end{equation}
      In particular,
      if we replace $K$ with $\FF_q$ with $q$ odd,
      then these are the three degree $2$
      $2$-group Belyi maps
      modulo $q$ up to isomorphism.
    \end{lemma}
    \begin{proof}
      All three degree $2$ passports
      $(0,\ZZ/2\ZZ,(1,2,2))$,
      $(0,\ZZ/2\ZZ,(2,1,2))$, and
      \newline
      $(0,\ZZ/2\ZZ,(2,2,1))$
      have size $1$ by
      Theorem
      \ref{thm:isoclasses}
      and Theorem
      \ref{thm:passports}.
      Since each field is a
      $2$-group Belyi map
      with passport specified
      by its subscript
      with no two isomorphic,
      this is an exhaustive list.
    \end{proof}
    \begin{remark}
      \label{rmk:degree2FFlax}
      All three $2$-group Belyi maps
      in Lemma
      \ref{lem:degree2FF}
      are lax isomorphic.
    \end{remark}
    Next, we discuss the algorithms
    to test when a quadratic extension
    is Galois (over the rational function field).
    \begin{alg}[IsGalois]
      \label{alg:isgalois}
      % Let the notation be
      % as above in the beginning of
      % Section
      % \ref{sec:functionfieldalgorithm}.
      \,
      \newline
      \textbf{Input}:
      \begin{itemize}
        \item
          $F$ a Galois extension of
          $\FF_q(x)$
        \item
          $\Gal(F\,|\,\FF_q(x))$
          explicitly given
          as automorphisms of $F$
        \item
          $f\in F$
      \end{itemize}
      \textbf{Output}:
      \textsf{True} if the quadratic extension
      $F(\sqrt{f})$ of $F$
      is a Galois extension over $\FF_q(x)$
      and \textsf{False} otherwise
      \begin{enumerate}
        \item
          For each generator
          $\sigma$ of $\Gal(F\,|\,\FF_q(x))$
          test if $\sigma(f)/f$ is a square
          in $F^\times$.
        \item
          If $\sigma(f)/f\in F^{\times 2}$
          for all generators $\sigma$,
          then return \textsf{True}
          otherwise return \textsf{False}.
      \end{enumerate}
    \end{alg}
    \begin{proof}[Proof of correctness]
      The correctness of this algorithm
      follows from Kummer theory
      as discussed in
      Lemma \ref{lem:isgalois}.
      It suffices to test on generators
      since the property of being a square
      is multiplicative.
    \end{proof}
    \begin{alg}[IsGaloisOverExtension]
      \label{alg:isgaloisoverextension}
      % Let the notation be
      % as above in the beginning of
      % Section
      % \ref{sec:functionfieldalgorithm}.
      \,
      \newline
      \textbf{Input}:
      \begin{itemize}
        \item
          $F$ a Galois extension of
          $\FF_q(x)$
        \item
          $\Gal(F\,|\,\FF_q(x))$
          explicitly given
          as automorphisms of $F$
        \item
          $f\in F$
      \end{itemize}
      Let $F'$ be the function field $F$
      with the constant field
      extended
      from $\FF_q$ to $\FF_{q^2}$.
      \textbf{Output}:
      \textsf{True} if the quadratic extension
      $F(\sqrt{f})$ of $F$
      is a Galois extension over $\FF_{q^m}(x)$
      after extending the field of constants
      from $\FF_q$ to $\FF_{q^m}$
      (for some positive integer $m$)
      and \textsf{False} otherwise
      \begin{enumerate}
        \item
          For each generator
          $\sigma$ of $\Gal(F'\,|\,\FF_{q^2}(x))$
          test if $\sigma(f)/f$ is a square
          in $F'$.
        \item
          If $\sigma(f)/f$ is a square in $F'$
          for all generators $\sigma$,
          then return \textsf{True}
          otherwise return \textsf{False}.
      \end{enumerate}
    \end{alg}
    \begin{proof}[Proof of correctness]
      The proof is similar to the previous
      algorithm.
      The proof that
      it is sufficient to check if
      elements are square
      over $\FF_{q^2}$
      can be found in
      \cite[Corollary 3.7.4]{stick}.
    \end{proof}
    % \begin{lemma}
    %   \label{lem:q2isgoodenough}
    %   Let $g\in F^{\times 2}$
    %   where $F$ is
    %   an algebraic function
    %   field over $\FFqal(x)$
    %   and $q$ is odd.
    %   Then $g$ is also a square in
    %   some algebraic function
    %   field over $\FF_{q^2}(x)$.
    % \end{lemma}
    % \mm{$q^2$ is good enough to
    % lift auts but not for field of definition}
    The next algorithm details the process
    of finding
    the appropriate
    candidate function to obtain
    a quadratic extension by extracting
    a square root.
    \begin{alg}[GetCandidateFunctions]
      \label{alg:getcandidates}
      % Let the notation be
      % as above in the beginning of
      % Section
      % \ref{sec:functionfieldalgorithm}.
      \,
      \newline
      \textbf{Input}:
      \begin{itemize}
        \item 
          $F$
          a $2$-group Belyi map modulo $q$
          of degree $d=2^m$
          corresponding to a $2$-group
          permutation triple $\sigma$
        \item
          A passport
          $\mathcal{P}=(\wt{G},(a,b,c))$
          with $\wt{G}$ a $2$-group of order
          $2d$ such that there
          exists a
          $2$-group permutation triple
          $\wt{\sigma}$ with passport
          $\mathcal{P}$
          that is a lift of
          $\sigma$
        \item
          $\Gal(F\,|\,\FF_q(x))\cong
          \langle\sigma\rangle$
          explicitly given
          as automorphisms of $F$
      \end{itemize}
      \textbf{Output}:
      A list of candidate functions
      $\{f_i\}$ with each $f_i\in F$
      such that $F(\sqrt{f_i})$ is a
      $2$-group Belyi map modulo $q$
      with passport $\mathcal{P}$.
      % \mm{list of candidate functions...}
        % $\wt{\phi}\in\wt{F}$
        % a $2$-group Belyi map modulo $q$
        % (or $q^2$)
        % of degree $2d$
        % with $\Gal(\wt{F}/\FF_q(x))
        % \cong\langle\wt{\sigma}\rangle$
        % explicitly given as
        % automorphisms of $\wt{F}$
        % such that each automorphism
        % restricts to an automorphism
        % in $\Gal(F/\FF_q(x))$ given
        % as input.
        % \mm{
        % note we can check if we need
        % $q^2$ at the very beginning
      % }
      \begin{enumerate}
        \item\label{alg:gencandidates_setup}
          For $s\in\{0,1,\infty\}$
          compute
          \begin{align}
            \label{eqn:ramcoeffs}
            r_s\colonequals
            \begin{cases}
              0&
              \text{ if }
              \order(\sigma_s)=
              \order(\wt{\sigma}_s)\\
              1&
              \text{ if }
              \order(\sigma_s)<
              \order(\wt{\sigma}_s)\\
            \end{cases}
          \end{align}
        \item\label{alg:getcandidates_getram}
          Compute
          \begin{equation}
            \label{eqn:ramdivisor}
            R\colonequals
            \sum_{s\in\{0,1,\infty\}}r_sR_s
            \in\Div(F)
          \end{equation}
          where
          $R_0,R_1,R_\infty$
          are defined to be the supports
          of
          % $\ddiv(\phi)$,
          % $\ddiv(\phi-1)$,
          % and $\ddiv(1/\phi)$
          $\ddiv(x)$,
          $\ddiv(x-1)$,
          and $\ddiv(1/x)$
          respectively.
        \item\label{alg:getcandidates_Pic}
          Compute the abelian group
          $\Pic(F)=T\oplus\ZZ$
          (with $T$ a finite abelian group)
          along with a map
          % $\psi\colon\Pic(F)\to\Div(F)$.
          $\psi\colon\Div(F)\to\Pic(F)$.
        \item\label{alg:getcandidates_PicR}
          Compute
          $[R]\colonequals\psi(R)$.
        \item
          Check that $[R]\in 2\Pic(F)$.
          If not, then return the empty set,
          otherwise continue.
        \item\label{alg:getcandidates_Pic2}
          For each $a\in\Pic(F)[2]$ compute the
          following:
          \begin{enumerate}
            \item\label{alg:getcandidates_Da}
              Let $D_a\colonequals\psi^{-1}(a+[R]/2)
              \in\Div(F)$.
            \item\label{alg:getcandidates_LDa}
              Compute $\mathscr{L}(R-2D_a)$.
            \item\label{alg:getcandidates_dimLDa}
              If $\mathscr{L}(R-2D_a)$ has dimension
              $1$, then compute
              $f_a\in F$ with $\ddiv(f_a)$
              generating $\mathscr{L}(R-2D_a)$
              and go to Step
              \ref{alg:getcandidates_isgalois}.
              Otherwise go to the next
              $a\in\Pic(F)[2]$.
            \item\label{alg:getcandidates_isgalois}
              Apply Algorithm \ref{alg:isgalois}
              to $F$,
              $\Gal(F\,|\,\FF_q(x))$,
              and $f_a$
              from Step
              \ref{alg:getcandidates_dimLDa}
              to see if $F(\sqrt{f_a})$
              generates a Galois extension.
              If $F(\sqrt{f_a})$ is Galois over
              $\FF_q(x)$ then save $f_a$
              and
              go to the next
              $a\in\Pic(F)[2]$.
              If $F(\sqrt{f_a})$ is not Galois
              over $\FF_q(x)$, then go to
              Step
              \ref{alg:getcandidates_isgaloisoverextension}.
            \item\label{alg:getcandidates_isgaloisoverextension}
              Let $F'$ be the function field
              $F$ after extending the field of
              constants $\FF_q$ to $\FF_{q^2}$.
              Apply Algorithm \ref{alg:isgaloisoverextension}
              to $F'$,
              $\Gal(F'\,|\,\FF_{q^2}(x))$,
              and $f_a$
              (viewed as an element of $F'$)
              from Step
              \ref{alg:getcandidates_dimLDa}
              to see if $F'(\sqrt{f_a})$
              generates a Galois extension.
              If $F(\sqrt{f_a})$ is Galois over
              $\FF_{q^2}(x)$ then save $f_a$.
              Go to the next
              $a\in\Pic(F)[2]$.
          \end{enumerate}
        \item\label{alg:getcandidates_overext}
          Let $S$ be the set of $f_a$
          saved in Step
          \ref{alg:getcandidates_isgalois}.
          Let $S'$ be the set of $f_a$
          saved in Step
          \ref{alg:getcandidates_isgaloisoverextension}.
        \item\label{alg:getcandidates_throwaway}
        \begin{itemize}
          \item
            If $S$ is nonempty,
            then
            for each $f_a\in S$
            compute
            $F(\sqrt{f_a})$,
            \[
              G_a\cong\Gal(F(\sqrt{f_a})\,|\,\FF_q(x)),
            \]
            and let
            $S''=
            \{f_a\in S:G_a\cong\wt{G}\}$.
          \item
            If $S$ is empty,
            then
            for each $f_a\in S'$
            compute
            $F'(\sqrt{f_a})$,
            \[
              G_a\cong\Gal(F'(\sqrt{f_a})\,|\,\FF_{q^2}(x)),
            \]
            and let
            $S''=
            \{f_a\in S':G_a\cong\wt{G}\}$.
        \end{itemize}
        \item\label{alg:getcandidates_return}
          Return the list $S''$ from
          Step
          \ref{alg:getcandidates_throwaway}.
      \end{enumerate}
      % \begin{enumerate}
        % \item
        %   \label{prescribedramification}
        %   Let $R$ be the empty set of points on $X$.
        %   For each $s\in\{0,1,\infty\}$,
        %   If the order of $\sigma_s$ is strictly less than the order of $\wt{\sigma}_s$,
        %   then append the ramification points
        %   $\{Q_{s,\tau}\}_{\tau\in\sigma_s}$
        %   (the ramification points on $X$ above $s$ corresponding to the cycles of $\sigma_s$)
        %   to $R$.
        % \item
        %   Let $K$ be a number field containing all coordinates of
        %   points in $R$
        %   (a subset of the ramification points of $\phi$).
        % % \item
        % %   Let $D = \sum_{P}n_PP$ be a degree $0$ divisor on $X$ 
        % %   with $n_P$ odd for every $P\in R$
        % %   and $n_P = 0$ for $P\not\in R$.
        % %   \mm{class group and base field}
        % \item
        %   Let $M = (R+2\ZZ R)\cap\Div^0(X)$.
        % \item
        %   \label{ramificationfulloop}
        %   For each $D\in M$ do the following:
        %   \begin{itemize}
        %     \item
        %       Compute the Riemann-Roch space $\LL(D)$.
        %     \item
        %       If $\dim\LL(D) = 1$, then
        %       compute $f\in K(X)^\times$
        %       corresponding to a generator
        %       of $\LL(D)$
        %       exit the loop
        %       and go to Step \ref{wefoundafunction}.
        %     \item
        %       If $\dim\LL(D) = 0$,
        %       then continue this loop
        %       with another choice of $D$.
        %   \end{itemize}
        % \item
        %   \label{wefoundafunction}
        %   Write $f=a/b$ with $a,b\in K[y_1,\dots,y_n]$ and construct the ideal
        %   \[
        %     \wt{I}\colonequals\langle g_1,\dots,g_k, by_{n+1}^2-a\rangle
        %   \]
        %   in $K[y_1,\dots,y_n,y_{n+1}]$.
        % \item
        %   \label{saturate}
        %   Saturate $\wt{I}$ at $\langle b\rangle$ and denote this ideal by $\sat(\wt{I})$.
        % \item
        %   Let $\wt{X}$ be the curve corresponding to $\sat(\wt{I})$ and
        %   $\wt{\phi}$ the map $(y_1,\dots,y_{n+1})\mapsto y_1$.
      % \end{enumerate}
    \end{alg}
    \begin{proof}[Proof of correctness]
      First, note that since
      we enumerated the isomorphism
      classes of $2$-group Belyi maps
      in Chapter \ref{chapter:grouptheory},
      we know the size of each passport
      $\mathcal{P}$ as input to this algorithm.
      The divisor $R$ computed in
      Step \ref{alg:getcandidates_getram}
      encodes the ramification
      required to obtain a $2$-group Belyi map
      with ramification matching the
      passport $\mathcal{P}$.
      From the discussion
      in Section
      \ref{sec:functionfieldextensions},
      $R\in 2\Div(F)$,
      and we can find all solutions to
      the equation
      \begin{equation}
        \label{eqn:solutioninPic}
        [R-2D]=[0]
      \end{equation}
      in $\Pic(F)$.
      For every element $a\in\Pic(F)[2]$
      we get a solution
      to \eqref{eqn:solutioninPic}.
      More precisely, the divisor $D_a$
      computed in Step
      \ref{alg:getcandidates_Da}
      satisfies $[R-2D_a]=[0]$,
      and all solutions to
      \eqref{eqn:solutioninPic}
      are of the form $D_a$
      for some $a\in\Pic(F)[2]$.
      Now,
      since $R-2D_a$ is principal for each $a$,
      we can find a candidate function
      $f_a\in F$ with $\ddiv(f_a) = R-2D_a$.
      After collecting the candidate functions
      $f_a$,
      we first use
      Algorithm
      \ref{alg:isgalois}
      and
      Algorithm
      \ref{alg:isgaloisoverextension}
      to eliminate $f_a$ that
      do not generate Galois extensions.
      Lastly,
      in Step
      \ref{alg:getcandidates_throwaway},
      we only keep candidate functions
      $f_a$ that generate extensions
      with Galois group isomorphic
      to the group
      $\wt{G}$ specified by the passport
      $\mathcal{P}$.
      Algorithm
      \ref{alg:isgalois}
      and Algorithm
      \ref{alg:isgaloisoverextension}
      guarantee that that no
      further constant field extension is
      required.
      % \mm{
      %   need to address why there is no reason
      %   to consider constant extension
      %   beyond $q^2$...
      %   if $F(\sqrt{f_a})$ is not Galois over
      %   $\FF_{q^2}(x)$, is it possible
      %   for $F(\sqrt{f_a})$ to be Galois
      %   after extending the constants to
      %   $\FF_{q^3}$ or higher?
      % }
    \end{proof}
    The next algorithm details
    the process of extracting a square root
    of a candidate function
    (obtained from the output of
    Algorithm \ref{alg:getcandidates})
    and lifting automorphisms.
    \begin{alg}[LiftBelyiMap]
      \label{alg:liftbelyimap}
      % Let the notation be
      % as above in the beginning of
      % Section
      % \ref{sec:functionfieldalgorithm}.
      \,
      \newline
      \textbf{Input}:
      \begin{itemize}
        \item
          The same input as in
          Algorithm
          \ref{alg:getcandidates}
        \item
          Additionally,
          a specific
          $f_a$ from the output of
          Algorithm
          \ref{alg:getcandidates}
      \end{itemize}
      \textbf{Output}:
      A $2$-group Belyi map modulo $q$
      with passport $\mathcal{P}$
      and explicit automorphisms
      identified with its Galois group
      $\wt{G}$
      \begin{enumerate}
        \item
          Compute $m_{f_a,\FF_q(x)}\in\FF_q(x)[y]$
          the minimal polynomial of $f_a$
          over $\FF_q(x)$
          and let $\alpha$ be
          a root of $m_{f_a,\FF_q(x)}(y^2)$.
          Let $\wt{F}$ denote the extension
          $\FF_q(x)(\alpha)$.
        \item
          Let $m_{\alpha,\FF_q(x)}$ be the
          minimal polynomial of $\alpha$
          over $\FF_q(x)$ and compute the set
          \begin{equation}
            \label{eqn:factorminpoly}
            R\colonequals
            \{
              r:r\text{ is a root of }
              m_{\alpha,\FF_q(x)}
              \text{ in }
              \wt{F}
            \}.
          \end{equation}
        \item
          Return the following:
          \begin{itemize}
            \item
              The absolute extension $\wt{F}$ of
              $\FF_q(x)$
            \item
              The set of field automorphisms
              $\{\tau_r:r\in R\}$
              where $\tau_r\colon\wt{F}\to\wt{F}$
              is defined by
              $\alpha\mapsto r$.
          \end{itemize}
      \end{enumerate}
    \end{alg}
    \begin{proof}[Proof of correctness]
      Since $f_a$ is obtained from
      the output of
      Algorithm
      \ref{alg:getcandidates},
      the extension
      $\wt{F}$ is Galois
      so that $m_{\alpha,\FF_q(x)}$
      has exactly $\deg(\wt{F})$ roots
      in $\wt{F}$.
      Again by Algorithm
      \ref{alg:getcandidates},
      the extension $\wt{F}$
      defines a $2$-group
      Belyi map modulo $q$
      with passport $\mathcal{P}$.
      The maps $\tau_r\colon\alpha\mapsto r$ define
      $\deg(\wt{F})$ automorphisms of $\wt{F}$
      over $\FF_q(x)$.
    \end{proof}
    % \mm{
    %   by construction the $\phi\in F$
    %   defining the Belyi map is always $x$.
    % }
    With
    Algorithm \ref{alg:getcandidates}
    and Algorithm \ref{alg:liftbelyimap}
    at our disposal,
    we can now explain how to
    compute \emph{all} $2$-group Belyi maps
    with a given passport.
    \begin{alg}[ComputePassport]
      \label{alg:computepassport}
      % Let the notation be
      % as above in the beginning of
      % Section
      % \ref{sec:functionfieldalgorithm}.
      \,
      \newline
      \textbf{Input}:
      \begin{itemize}
        \item
          A passport
          $\mathcal{P}_{\text{above}} =
          (\wt{G},(a,b,c))$
        \item
          A list of passports
          $\mathcal{P}_1,
          \dots,
          \mathcal{P}_k$
        \item
          For each
          $\mathcal{P}_i$
          a list of triples of data
          $(\sigma_i^1,F_i^1,G_i^1),
          \dots
          (\sigma_i^{\#\mathcal{P}_i},
          F_i^{\#\mathcal{P}_i},
          G_i^{\#\mathcal{P}_i})$
          with the $F_i^j$
          pairwise non-isomorphic
          and
          each
          $(\sigma_i^j,F_i^j,G_i^j)$
          satisfying the following:
          \begin{itemize}
            \item
              $\sigma_i^j$ is a $2$-group
              permutation triple with passport
              $\mathcal{P}_i$
            \item
              There exists
              a $2$-group
              permutation triple
              $\wt{\sigma}_i^j$
              with passport
              $\mathcal{P}_\text{above}$
              that is a lift
              of $\sigma_i^j$
              % for every $k$
              % There exists a list of $2$-group
              % permutation triples
              % $\wt{\sigma_i^{j,k}}$
              % with passport
              % $\mathcal{P}_\text{above}$
              % that are lifts
              % of $\sigma_i^j$
              % for every $k$
            \item
              $F_i^j$ is a $2$-group Belyi map
              modulo $q$
            \item
              $G_i^j$ is the Galois group of
              $F_i^j$ over $\FF_q(x)$
              explicitly given as
              automorphisms of $F_i^j$
          \end{itemize}
      \end{itemize}
      \textbf{Output}:
      A list of triples of data
      % $(\wt{\sigma}^1,\wt{F}^1,\wt{G}^1),
      $(\wt{F}^1,\wt{G}^1),
      \dots,
      % (\wt{\sigma}^{\#\mathcal{P}_\text{above}},
      (\wt{F}^{\#\mathcal{P}_\text{above}},
      \wt{G}^{\#\mathcal{P}_\text{above}})$
      with
      % $\wt{\sigma}^j$
      % a $2$-group permutation triple
      % with passport $\mathcal{P}$,
      $\wt{F}^j$
      a $2$-group Belyi map modulo $q'$
      (with $q'$ a power of $q$),
      $\wt{G}^j$
      the Galois group
      of $F_i^j$
      explicitly given
      as automorphisms of $\wt{F}^j$,
      and the $\wt{F}^j$ pairwise
      non-isomorphic.
      \begin{enumerate}
        \item
          \label{alg:computepassport_alldownstairs}
          Apply Algorithm
          \ref{alg:getcandidates}
          to every triple of data
          $(\sigma_i^j,F_i^j,G_i^j)$
          downstairs
          (along with the passport
          $\mathcal{P}_\text{above}$)
          to obtain a list of candidate functions
          $\mathrm{CFS}_q
          \colonequals\{f_i^{j,k}\}$ with
          $f_i^{j,k}\in F_i^j$ for each $k$.
        \item
          \label{alg:computepassport_lift}
          For each $f_i^{j,k}\in\mathrm{CFS}_q$,
          apply Algorithm
          \ref{alg:liftbelyimap}
          with input
          $(\sigma_i^j,F_i^j,G_i^j)$
          and $f_i^{j,k}$
          to obtain
          % $(\wt{\sigma_i^{j,k}},
          % $(\wt{F_i^{j,k}}, \wt{G_i^{j,k}})$
          % with
          $\wt{F_i^{j,k}}$ a $2$-group
          Belyi map modulo $q$ with
          passport $\mathcal{P}$ and
          Galois group
          $\wt{G_i^{j,k}}$.
          Let $\mathrm{BELYI}_q$
          denote the list of
          all pairs
          $(\wt{F_i^{j,k}}, \wt{G_i^{j,k}})$
          obtained in this step.
        \item
          \label{alg:computepassport_testiso}
          Test isomorphism of fields
          $\wt{F_i^{j,k}}$
          and $\wt{F_i^{j,k'}}$
          over $\FF_q(x)$
          for each pair of fields
          in $\mathrm{BELYI}_q$.
          Keep exactly one
          representative of each isomorphism
          class from $\mathrm{BELYI}_q$
          and store this data
          (including the Galois group)
          in $\mathrm{BELYIISO}_q$.
        \item
          If $\#\mathrm{BELYIISO}_q=
          \#\mathcal{P}_\text{above}$
          then return
          $\mathrm{BELYIISO}_q$.
          Otherwise,
          extend the constant field
          from $\FF_q$ to $\FF_{q^2}$
          and repeat Steps
          \ref{alg:computepassport_alldownstairs},
          \ref{alg:computepassport_lift},
          and
          \ref{alg:computepassport_testiso}
          to obtain lists
          $\mathrm{CFS}_{q^2}$,
          $\mathrm{BELYI}_{q^2}$,
          and
          $\mathrm{BELYIISO}_{q^2}$.
          Continue this process for
          $q,q^2,q^3,\dots$
          and return
          $\mathrm{BELYIISO}_{q^m}$
          for the first $m\in\{1,2,\dots\}$
          with the same cardinality
          as $\mathcal{P}_\text{above}$.
      \end{enumerate}
      % \mm{
        % Rough steps of the algorithm
        % \begin{enumerate}
        %   \item
        %     Apply Algorithm
        %     \ref{alg:getcandidates}
        %     to every triple of data downstairs
        %     (this tells us
        %     if we get any candidates mod $q$
        %     or as a twist)
        %   \item
        %     Look at all candidate functions
        %     obtained mod $q$ or as twists
        %   \item
        %     Lift using all candidate functions
        %     and Algorithm
        %     \ref{alg:liftbelyimap}
        %   \item
        %     Isomorphism test to see if we have
        %     enough distinct isomorphism classes
        %     to fill out the passport
        % \end{enumerate}
      % }
    \end{alg}
    \begin{proof}[Proof of correctness]
      This algorithm is largely bookkeeping
      and applying Algorithm
      \ref{alg:getcandidates}
      and Algorithm
      \ref{alg:liftbelyimap}.
      The triples of data downstairs
      enumerate the isomorphism classes
      of $2$-group Belyi maps modulo $q$
      in all passports that have a representative
      with a lift that has passport
      $\mathcal{P}_\text{above}$.
      \par
      It is important to note at this point
      that for every downstairs passport
      $\mathcal{P}_i$,
      we need \emph{all} $\#\mathcal{P}_i$
      triples of data
      $(\sigma_i^j,F_i^j,G_i^j)$.
      This is because
      Algorithm \ref{alg:getcandidates}
      only identifies candidate functions
      that produce $2$-group Belyi maps
      with the correct passport.
      It does not provide a way to identify
      the precise isomorphism class.
      That is why,
      in this algorithm,
      we must be content with a list of
      pairwise
      non-isomorphic Belyi maps.
      By testing isomorphisms we can ensure
      that we have a representative
      from every isomorphism class,
      but we cannot identify the permutation
      triple corresponding to a Belyi map.
      In this algorithm the permutation triples
      (obtained from the algorithms in Section
      \ref{sec:triplesalgorithm})
      are simply a bookkeeping tool.
      \par
      The one subtle point is
      explaining how to obtain
      the $q'$.
      After each round of computing
      the lists
      $\mathrm{CFS}_{q^i}$,
      $\mathrm{BELYI}_{q^i}$,
      and
      $\mathrm{BELYIISO}_{q^i}$
      it is possible that
      we failed to find all candidate functions
      over the constant field
      $\FF_{q^i}$.
      The enumeration of isomorphism classes
      in Section
      \ref{sec:triplesalgorithm}
      ensures that this process of
      extending the constant field will
      terminate,
      but does not provide an
      a priori bound on the $q'$
      required.
    \end{proof}
    Applying Algorithm
    \ref{alg:computepassport}
    to every degree $d$ passport
    allows us to enumerate all
    $2$-group Belyi maps modulo $q$
    \emph{one degree at a time}.
    Section \ref{sec:resultsequations}
    details how far we were able to push
    these computations in practice
    using \cite{twogroupdessins}.
    % \mm{
    %   maybe another wrapper on top
    %   to explain how the iteration works...
    % }
    \begin{remark}
      \label{rmk:computationalremarks}
      The algorithms in this section rely
      on the \textsf{Magma}
      implementations to compute
      class groups $\Pic(F)$
      for global function fields,
      and the implementations
      to compute Riemann-Roch spaces
      $\mathscr{L}(D)$.
      % \mm{class group, galois group,
      % RiemannRoch algorithms in Magma}
    \end{remark}
    We conclude this section
    with an example of how these
    algorithms are applied in a specific
    example.
    \begin{example}
      \label{exm:equationsinmagma}
      In this example we use the algorithms
      in this section to compute
      all three
      $2$-group Belyi maps modulo $3$
      with passport $(3,G,(4,4,4))$
      where $G = (\ZZ/4\ZZ) : (\ZZ/4\ZZ)$
      is the Galois group described
      at the following link.
      \begin{center}
        \url{http://www.lmfdb.org/GaloisGroup/16T8}
      \end{center}
      The \textsf{Magma} script can be run
      using the following command
      from the repository \cite{twogroupdessins}.
      \begin{code}
magma thesis_examples/compute_passport_16T8_444_g3.m
      \end{code}
      The source code in this file is
      as follows.
      \begin{magma}
load "config.m";
SetVerbose("TwoDBPassport", 3);
SetVerbose("TwoDB", 1);
objs := GetPassportObjects(16);
s := objs[#objs-2];
ComputeBelyiMaps(s : optimized := false);
      \end{magma}
      This example has $7$ size $1$ passports
      $\mathcal{P}_i$ as in
      Algorithm \ref{alg:computepassport}.
      Each isomorphism class downstairs
      yields a candidate Belyi map upstairs,
      and the isomorphism
      checking in
      Algorithm
      \ref{alg:computepassport}
      Step
      \ref{alg:computepassport_testiso}
      correctly identifies the
      $3$ distinct isomorphism classes
      out of $7$.
      % \mm{
      %   full example with magma code...
      %   there is a size 3 passport
      %   in degree 16 with 3 passports below
      %   and one unramified cover
      %   with ramification $(4,4,4)$...
      % }
    \end{example}
  }
  \section{An implementation over $\QQal$}{
    \label{sec:characteristiczero}
    We now discuss the situation in
    characteristic zero.
    The procedure has the same broad strokes as
    that in characteristic $p\neq 2$,
    but there is a key difference
    in the technique to
    get candidate functions
    to extract a square root of in
    Algorithm
    \ref{alg:getcandidates}.
    \par
    In characteristic zero
    there is no implementation
    to compute $\Pic(F)$
    (for general $F$).
    To show how we can get around
    this (in some cases),
    we now rewrite
    Algorithm
    \ref{alg:getcandidates}
    in the characteristic zero setting.
    \begin{alg}
      \label{alg:charzero}
      \,
      \newline
      \textbf{Input}:
      \begin{itemize}
        \item 
          $K(x)\hookrightarrow F\colonequals K(X)$
          a $2$-group Belyi map
          of degree $d=2^m$
          corresponding to a $2$-group
          permutation triple $\sigma$
        \item
          A passport
          $\mathcal{P}=(\wt{G},(a,b,c))$
          with $\wt{G}$ a $2$-group of order
          $2d$ such that there
          exists a
          $2$-group permutation triple
          $\wt{\sigma}$ with passport
          $\mathcal{P}$
          that is a lift of
          $\sigma$
        \item
          $\Gal(F\,|\,K(x))\cong
          \langle\sigma\rangle$
          explicitly given
          as automorphisms of $F$
      \end{itemize}
      \textbf{Output}:
      A list of candidate functions
      $\{f_i\}$ with each $f_i\in F$
      such that
      $K(x)\hookrightarrow F(\sqrt{f_i})$
      is a $2$-group Belyi map
      with passport $\mathcal{P}$.
      \begin{enumerate}
        \item\label{alg:charzero_setup}
          For $s\in\{0,1,\infty\}$
          compute
          \begin{align}
            \label{eqn:ramcoeffs}
            r_s\colonequals
            \begin{cases}
              0&
              \text{ if }
              \order(\sigma_s)=
              \order(\wt{\sigma}_s)\\
              1&
              \text{ if }
              \order(\sigma_s)<
              \order(\wt{\sigma}_s)\\
            \end{cases}
          \end{align}
        \item\label{alg:charzero_getram}
          Compute
          \begin{equation}
            \label{eqn:ramdivisor}
            R\colonequals
            \sum_{s\in\{0,1,\infty\}}r_sR_s
            \in\Div(F)
          \end{equation}
          where
          $R_0,R_1,R_\infty$
          are defined to be the supports
          of
          $\ddiv(x)$,
          $\ddiv(x-1)$,
          and $\ddiv(1/x)$
          respectively.
        \item
          \label{alg:charzero_alldegreezero}
          Let $M$ denote the set
          $(R+2\ZZ R)\,\cap\,\Div^0(F)$
          and for $B\in\ZZ_{\geq 1}$
          let
          \[
            M_B=
            \Big\{R+2nR : n\in 
            \{-B,-B+1,\dots,B-1,B\}\Big\}
            \cap\Div^0(F).
          \]
        \item
          For each $D\in M$ compute the
          following:
          \begin{enumerate}
            \item
              Compute $\mathscr{L}(D)$.
            \item
              If $\mathscr{L}(D)$
              has dimension
              $1$, then compute
              $f_D\in F$ with $\ddiv(f_D)$
              generating $\mathscr{L}(D)$
              and go to the next step.
              Otherwise,
              go to the next $D\in M$.
            \item\label{alg:charzero_isgal}
              Check to see if
              $F(\sqrt{f_D})$ is Galois.
              If $F(\sqrt{f_D})$ is Galois,
              then save $f_D$.
              and go to the next $D\in M$.
              If $F(\sqrt{f_D})$ is not Galois,
              then go to the next Step.
            \item\label{alg:charzero_isgalext}
              Let $F'$ be the function field
              $F$ after extending the field of
              constants to the compositum
              of the residue fields of
              all places in the support of $D$.
              Check to see if
              $F'(\sqrt{f_D})$ is Galois.
              If $F'(\sqrt{f_D})$ is Galois,
              then save $f_D$.
              Go to the next $D\in M$.
          \end{enumerate}
        \item\label{alg:charzero_overext}
          Let $S$ be the set of $f_D$
          saved in Step
          \ref{alg:charzero_isgal}.
          Let $S'$ be the set of $f_a$
          saved in Step
          \ref{alg:charzero_isgalext}.
        \item\label{alg:charzero_throwaway}
        \begin{itemize}
          \item
            If $S$ is nonempty,
            then
            for each $f_D\in S$
            compute
            $F(\sqrt{f_D})$,
            \[
              G_D\cong\Gal(F(\sqrt{f_a})\,|\,K(x)),
            \]
            and let
            $S''=
            \{f_D\in S:G_D\cong\wt{G}\}$.
          \item
            If $S$ is empty,
            then
            for each $f_D\in S'$
            compute
            $F'(\sqrt{f_D})$,
            \[
              G_D\cong\Gal(F'(\sqrt{f_D})\,|\,K(x)),
            \]
            and let
            $S''=
            \{f_D\in S':G_D\cong\wt{G}\}$.
        \end{itemize}
        \item\label{alg:charzero_return}
          Return the list $S''$ from
          Step
          \ref{alg:charzero_throwaway}.
      \end{enumerate}
    \end{alg}
    Although Algorithm
    \ref{alg:charzero}
    in characteristic zero
    resembles
    Algorithm
    \ref{alg:getcandidates} over $\FF_q$,
    the characteristic zero algorithm
    is unfortunately not guaranteed to
    find any candidate functions!
    This is due to the fact that
    we are not computing representatives
    of $\Pic^0(F)[2]$.
    \par
    Without enumerating representatives
    of $\Pic^0(F)[2]$,
    the approach in characteristic zero
    will always be ad hoc
    in the sense that in Step
    \ref{alg:charzero_alldegreezero}
    we are blindly looking at all combinations
    of points that yield the desired ramification.
    This process is guaranteed to succeed
    when $F$ has class number $1$,
    but this condition is not often satisfied
    (in fact there are only $8$ such non-rational function fields
    over $\FF_q$).
    \par
    This ad hoc approach does, however,
    allow us to compute some $2$-group
    Belyi maps in characteristic zero.
    We conclude this section by describing
    the results of these computations
    along with those in
    positive characteristic.
  }
  \section{Results of computations}{
    \label{sec:resultsequations}
    In this section we summarize
    the computations carried
    out in \cite{twogroupdessins}
    based on the algorithms
    discussed in Section
    \ref{sec:functionfieldalgorithm}
    and
    Section
    \ref{sec:characteristiczero}.
    \par
    In characteristic $3$
    we were able to compute
    all $2$-group Belyi maps modulo $3$
    up to degree $32$.
    The results of these computations can
    be accessed in \textsf{Magma}
    (with working directory the
    repository \cite{twogroupdessins})
    using
    the following code.
    \begin{magma}
load "config.m";
d := 16;
objs := [ReadTwoDBPassport(f) : f in PassportFilenames(d)];
    \end{magma}
    The information for a given passport can
    be accessed using the following code.
    \begin{magma}
s := Random(objs);
FunctionFields(s);
BelyiMaps(s);
FunctionFieldAutomorphisms(s);
    \end{magma}
    In addition to the systematic
    computation of $2$-group
    Belyi maps modulo $3$,
    we were also able to apply the
    implementation in
    Section
    \ref{sec:characteristiczero}
    to compute hundreds
    of $2$-group Belyi maps
    in characteristic zero
    with degrees up to $256$.
    \par
    We conclude this chapter
    with interesting examples
    encountered during these computations.
    First,
    in Section
    \ref{sec:labels}
    we set up some notation to
    help with the description of these
    examples.
  }
  \section{Naming conventions for database examples}{
    \label{sec:labels}
    In this section we explain the conventions
    used for filenames in
    \cite{twogroupdessins}.
    This will be useful for the rest of
    this chapter when referring to
    specific examples.
    \par
    The first point to explain is how we deal
    with isomorphism classes of $2$-group
    Belyi maps
    as opposed to passports.
    As discussed in Chapter
    \ref{chapter:grouptheory},
    isomorphism classes of $2$-group Belyi maps
    correspond to $2$-group permutation triples
    each with a unique filename of the form
    \begin{equation}
      \label{eqn:galmapfilename}
      \texttt{DNG-a,b,c-gE-H}
    \end{equation}
    where
    \begin{align}
      \label{eqn:galmapfilenameparts}
      \begin{split}
        \texttt{D} &:\text{ degree in }
        \{2,4,8,16,32,64,128,256\}\\
        \texttt{N} &:\text{ either \texttt{T}
        or \texttt{S} identifying group
        database}\\
        \texttt{G} &:\text{ a positive integer
        identifying the group}\\
        \texttt{a} &:\text{ ramification index
        of $0$ in }
        \{2,4,8,16,32,64,128,256\}\\
        \texttt{b} &:\text{ ramification index
        of $1$ in }
        \{2,4,8,16,32,64,128,256\}\\
        \texttt{c} &:\text{ ramification index
        of $\infty$ in }
        \{2,4,8,16,32,64,128,256\}\\
        \texttt{g} &:\text{ just the letter 
        \texttt{g}}\\
        \texttt{E} &:\text{ the genus in }
        \ZZ_{\geq 0}\\
        \texttt{H} &:\text{ the hash
        of the $2$-group permutation triple 
        a positive integer}
      \end{split}
    \end{align}
    For example,
    the filename
    \texttt{16T12-4,8,2-g2-1396531181}
    corresponds to a degree $64$
    Belyi map with monodromy group
    identified in the
    transitive group database
    (\texttt{T} for transitive)
    \begin{center}
      \url{http://magma.maths.usyd.edu.au/magma/handbook/text/753}
    \end{center}
    by \texttt{16T12}.
    The \texttt{4,8,2} encodes the ramification
    above $0,1,\infty$ respectively,
    the \texttt{g2} indicates that this
    Belyi map has genus $2$,
    and the \texttt{1396531181}
    is the hash of the permutation triple
    corresponding to this Belyi map.
    \par
    Another example,
    \texttt{64S7-8,8,4-g17-1653847134}
    corresponds to a degree $16$
    Belyi map with monodromy group
    identified in the
    small group database
    (\texttt{S} for small)
    \begin{center}
      \url{http://magma.maths.usyd.edu.au/magma/handbook/text/748}
    \end{center}
    by \texttt{64S7}.
    The \texttt{8,8,4} encodes the ramification
    above $0,1,\infty$ respectively,
    the \texttt{g17} indicates that this
    Belyi map has genus $17$,
    and the \texttt{1653847134}
    is the hash of the permutation triple
    corresponding to this Belyi map.
    In this example,
    the hash is necessary to distinguish
    this example from
    \begin{align*}
      &\texttt{64S7-8,8,4-g17-2483683244},\\
      &\texttt{64S7-8,8,4-g17-623082418},\\
      &\text{and }\texttt{64S7-8,8,4-g17-964508325}.
    \end{align*}
    \par
    A passport is given a similar filename.
    The only difference for a passport is that
    the hash is no longer required.
    For example,
    the Belyi maps with filenames
    in the previous paragraph
    % \begin{align*}
      % &\texttt{64S7-8,8,4-g17-1653847134}\\
      % &\texttt{64S7-8,8,4-g17-2483683244},\\
      % &\texttt{64S7-8,8,4-g17-623082418},\\
      % &\text{and }\texttt{64S7-8,8,4-g17-964508325}
    % \end{align*}
    all have the same passport
    \texttt{64S7-8,8,4-g17}.
    The hash is to distinguish between
    isomorphism classes within a passport.
    \par
    The reason for this
    distinction is that the algorithms
    for computing equations in this chapter
    are designed for passports.
    Now that we have a concise way of
    talking about example,
    we use this in the rest of the chapter
    to discuss examples.
  }
  \section{Degree $2$}{
    \label{sec:d2}
    In degree $2$,
    there are $3$
    isomorphism classes of
    $2$-group Belyi maps.
    In characteristic zero they are
    represented by
    \begin{equation}
      \label{eqn:d2}
      % F_{(1,2,2)} =
      \frac{\QQ(x)[y]}{(y^2 + x - 1)},\;
      % F_{(2,1,2)} =
      \frac{\QQ(x)[y]}{(y^2 - x)},
      \text{ and }
      % F_{(2,2,1)} =
      \frac{\QQ(x)[y]}{(y^2 - x^2 + x)}
    \end{equation}
    and each is the unique
    Belyi map with passport
    \texttt{2T1-1,2,2-g0},
    \texttt{2T1-2,1,2-g0},
    and
    \texttt{2T1-2,2,1-g0},
    respectively.
  }
  \section{Degree $4$}{
    \label{sec:d4}
    In degree $4$ there are $7$
    isomorphism classes of
    $2$-group Belyi maps
    and each has
    passport size $1$.
    \par
    All function fields are of the form
    $\QQ(x)[y]/(f(x,y))$
    with $f$ one of the polynomials in
    \eqref{eqn:d4}.
    The subscripts indicate the passport.
    \begin{align}
      \label{eqn:d4}
      \begin{split}
        f_{\texttt{4T1-1,4,4-g0}}
        &= y^4 + x - 1\\
        f_{\texttt{4T1-2,4,4-g1}}
        &= y^4 + x^3 - x^2\\
        f_{\texttt{4T1-4,1,4-g0}}
        &= y^4 - x\\
        f_{\texttt{4T1-4,2,4-g1}}
        &= y^4 - x^3 + 2x^2 - x\\
        f_{\texttt{4T1-4,4,1-g0}}
        &= y^4 - x^4 + x^3\\
        f_{\texttt{4T1-4,4,2-g1}}
        &= y^4 - x^2 + x\\
        f_{\texttt{4T2-2,2,2-g0}}
        &= y^4 + (4x - 2)y^2 + 1\\
      \end{split}
    \end{align}
    Although every function field has
    constant field $\QQ$,
    lifting the automorphisms
    requires extending the constant
    field to $\QQ(\zeta_{4})$
    in all examples except for the
    Belyi map defined by
    $f_{\texttt{4T2-2,2,2-g0}}$.
    \par
    Use the following code from
    \cite{twogroupdessins}
    to obtain a list of these
    function fields in \textsf{Magma}.
    \begin{magma}
load "config.m";
objs := [ReadTwoDBPassportChar0(f) : f in PassportFilenames(4)];
fields := [FunctionFields(s)[1] : s in objs];
    \end{magma}
  }
  \section{Degree $8$}{
    \label{sec:d8}
    In degree $8$ there are $13$
    isomorphism classes of $2$-group Belyi maps
    that have size $1$ passports
    and there are $3$ size $2$ passports.
    \par
    All function fields are of the form
    $\QQ(x)[y]/(f(x,y))$
    with $f$ one of the polynomials in
    \eqref{eqn:d8_188},
    \eqref{eqn:d8_288},
    \eqref{eqn:d8_244},
    \eqref{eqn:d8_224},
    \eqref{eqn:d8_444}, for size $1$ passports,
    or
    \eqref{eqn:d8size2},
    for size $2$ passports.
    The subscripts indicate the passport.
    \begin{align}
      \label{eqn:d8_188}
      \begin{split}
        f_{\texttt{8T1-1,8,8-g0}}
        &= y^8 + x - 1\\
        f_{\texttt{8T1-8,1,8-g0}}
        &= y^8 - x\\
        f_{\texttt{8T1-8,8,1-g0}}
        &= y^8 - x^8 + x^7
      \end{split}
    \end{align}
    \begin{align}
      \label{eqn:d8_288}
      \begin{split}
        f_{\texttt{8T1-2,8,8-g2}}
        &= y^8 + x^5 - x^4\\
        f_{\texttt{8T1-8,2,8-g2}}
        &= y^8 - x^5 + 4x^4 - 6x^3 + 4x^2 - x\\
        f_{\texttt{8T1-8,8,2-g2}}
        &= y^8 - x^4 + 3x^3 - 3x^2 + x
      \end{split}
    \end{align}
    \begin{align}
      \label{eqn:d8_224}
      \begin{split}
        f_{\texttt{8T4-2,2,4-g0}}
        &= y^8 + (4x - 2)y^4 + 1\\
        f_{\texttt{8T4-2,4,2-g0}}
        &= y^8 + (8x^4 - 16x^3 + 16x - 8)y^4 + 16x^8 - 128x^7 + 448x^6\\
        &- 896x^5 + 1120x^4 - 896x^3 + 448x^2 - 128x + 16\\
        f_{\texttt{8T4-4,2,2-g0}}
        &= y^8 + (-8x^4 + 16x^3)y^4 + 16x^8\\
      \end{split}
    \end{align}
    \begin{align}
      \label{eqn:d8_444}
      \begin{split}
        f_{\texttt{8T5-4,4,4-g2}}
        &= y^8 + (1/2x^3 - 3/2x^2 + x)y^4 + 1/16x^6 - 1/8x^5 + 1/16x^4\\
      \end{split}
    \end{align}
    \begin{align}
      \label{eqn:d8_244}
      \begin{split}
        f_{\texttt{8T2-2,4,4-g1}}
        &= y^8 + (8x - 4)y^6 + (4x - 4)y^5
        + (16x^2 - 31/2x + 11/2)y^4\\
        &+ (-8x + 8)y^3
        + (2x^2 + x + 1)y^2
        + (-15x^2 + 18x - 3)y\\
        &+ 4x^3 - 127/16x^2 + 35/8x + 9/16\\
        f_{\texttt{8T2-4,2,4-g1}}
        &= y^8 + (1/16x^3 - 1/16x^2 + 1/128x)y^4 + 1/65536x^2\\
        f_{\texttt{8T2-4,4,2-g1}}
        &= y^8 + (8x - 4)y^6 + (-1/2x^4 - 23/2x^3 + 28x^2 - 16x + 6)y^4\\
        &+ (6x^5 - 25x^4 + 27x^3 - 8x^2 + 8x - 4)y^2 + 1/16x^8 - 9/8x^7\\
        &+ 97/16x^6 - 9x^5 + 7/2x^4 + 9/2x^3 - 4x^2 + 1
      \end{split}
    \end{align}
    \begin{align}
      \label{eqn:d8size2}
      \begin{split}
        f_{\texttt{8T1-4,8,8-g3}}
        &= y^8 + x^5 - 3x^4 + 3x^3 - x^2\\
        f_{\texttt{8T1-8,4,8-g3}}
        &= y^8 - x^5 + 2x^4 - x^3\\
        f_{\texttt{8T1-8,8,4-g3}}
        &= y^8 - x^2 + x
      \end{split}
    \end{align}
    Although every function field has
    constant field $\QQ$,
    lifting the automorphisms
    requires extending the constant
    field to $\QQ(\zeta_{8})$
    in all examples except for the
    those that are covers of the
    Belyi map defined by
    $f_{\texttt{4T2-2,2,2-g0}}$
    in Section \ref{sec:d4}.
    \par
    Use the following code from
    \cite{twogroupdessins}
    to obtain a list of these
    function fields in \textsf{Magma}.
    \begin{magma}
load "config.m";
objs := [ReadTwoDBPassportChar0(f) : f in PassportFilenames(8)];
fields := [FunctionFields(s)[1] : s in objs];
    \end{magma}
    \par
    Perhaps the most interesting note to make
    in degree $8$ is that it is
    the lowest degree where the
    characteristic zero approach
    appears to fail.
    Indeed, the genus $3$ passports
    represented in
    \eqref{eqn:d8size2}
    have size $2$, but only a single
    representative Belyi map for each.
    In these examples,
    all candidate functions obtained from
    the ad hoc approach in
    Algorithm
    \ref{alg:charzero}
    yield isomorphic function fields
    over $\QQ(x)$,
    so we know we are missing a
    Belyi map in each one of these passports.
    \par
    The techniques in
    Section \ref{sec:functionfieldalgorithm},
    however,
    do manage to succeed for these
    size $2$ passports.
    Working over $\FF_3$,
    using Algorithm
    \ref{alg:computepassport},
    we obtain the
    function fields
    \begin{equation}
      \label{eqn:d8_884_FF3}
      \frac{\FF_3(x)[y]}{(y^8 + 2x^6 + 2x^5 + 2x^4 + x^3 + x^2 + x)}\quad\text{ and }\quad
      \frac{\FF_3(x)[y]}{(y^8 + x^6 + 2x^3)}
    \end{equation}
    for the passport \texttt{8T1-8,8,4-g3},
    and one can check that the
    fields in \eqref{eqn:d8_884_FF3}
    are not isomorphic.
    \par
    To obtain a list of degree $8$
    function fields corresponding to
    $2$-group Belyi maps
    in characteristic $3$,
    use the following code from
    \cite{twogroupdessins}.
    \begin{magma}
load "config.m";
objs := [ReadTwoDBPassport(f) : f in PassportFilenames(8)];
fields := [* *];
for s in objs do
  fields cat:= FunctionFields(s);
end for;
    \end{magma}
  }
  \section{Degree $16$}{
    \label{sec:d16}
    In degree $16$ there are $55$
    isomorphism classes of $2$-group Belyi maps
    with $41$ distinct passports.
    All passports are of size $1$
    except for
    Passport
    \texttt{16T8-4,4,4-g3} which has size $3$,
    Passports
    \texttt{16T1-4,16,16-g6},
    \texttt{16T1-16,4,16-g6},
    \texttt{16T1-16,16,4-g6}
    which have size $2$,
    and Passports
    \texttt{16T1-8,16,16-g7},
    \texttt{16T1-16,8,16-g7},
    \texttt{16T1-16,16,8-g7}
    which have size $4$.
    \par
    Algorithm
    \ref{alg:computepassport}
    succeeds in characteristic $3$,
    but the equations are too large to write here.
    For an illustrative example
    of Algorithm
    \ref{alg:computepassport}
    in action
    see Example
    \ref{exm:equationsinmagma}.
    To obtain a list of degree $16$
    function fields corresponding to
    $2$-group Belyi maps
    in characteristic $3$,
    use the following code from
    \cite{twogroupdessins}.
    \begin{magma}
load "config.m";
objs := [ReadTwoDBPassport(f) : f in PassportFilenames(16)];
fields := [* *];
for s in objs do
  fields cat:= FunctionFields(s);
end for;
    \end{magma}
    % we list $14$ function fields
    % corresponding to $2$-group Belyi maps
    % over $\QQ(x)$.
    % As indicated in Theorem
    % \ref{thm:laxpassports},
    % each of the $14$ Belyi maps we list below
    % is lax isomorphic to a Belyi map
    % in all $41$ passports of degree $16$.
    % \begin{align}
      % \label{eqn:d16t5_488}
      % \begin{split}
      %   f_{\texttt{}}
      %   &=\\
      % \end{split}
    % \end{align}
    % \begin{align}
      % \label{eqn:d16t6_288}
      % \begin{split}
      %   f_{\texttt{}}
      %   &=\\
      % \end{split}
    % \end{align}
    Below are some of the characteristic zero
    function fields
    corresponding to
    $2$-group Belyi maps
    computed using
    Algorithm
    \ref{alg:charzero}.
    At this point the equations tend to be
    too big to fit on a page.
    \begin{align}
      \label{eqn:d16t1_11616}
      \begin{split}
        f_{\texttt{16T1-1,16,16-g0}}
        &=y^{16} + x - 1\\
      \end{split}
    \end{align}
    \begin{align}
      \label{eqn:d16t13_228}
      \begin{split}
        f_{\texttt{16T13-2,2,8-g0}}
        &=y^{16} + (4x - 2)y^8 + 1\\
      \end{split}
    \end{align}
    % \begin{align}
      % \label{eqn:d16t5_288}
      % \begin{split}
      %   &f_{\texttt{16T6-8,2,8-g3}}=\\
      %   &y^{16}+ (-512x^7 + 512x^6 - 128x^5)y^{13}\\
      %   &+ (1/8x^3 - 1/8x^2 + 1/64x)y^{12}\\
      %   &+ (49152x^{14} - 57344x^{13} - 8192x^{12} + 36864x^{11} - 17408x^{10} + 2560x^9)y^{10}\\
      %   &+ (32x^{10} - 88x^9 + 86x^8 - 36x^7 + 11/2x^6)y^9\\
      %   &+ (536870912x^{25} - 2147483648x^{24} + 3758096384x^{23} - 3758096384x^{22} + 2348810240x^{21} - 939524096x^{20}\\
      %   &+ 234881024x^{19} - 33554432x^{18} + 2097152x^{17} + 1/256x^6 - 1/128x^5 + 5/1024x^4 - 1/1024x^3 + 3/32768x^2)y^8\\
      %   &+ (-6291456x^{20} + 18874368x^{19} - 23592960x^{18} + 15728640x^{17}\\
      %   &- 5898240x^{16} + 1179648x^{15} - 98304x^{14})y^7\\
      %   &+ (-4096x^{16} + 12672x^{15} - 15808x^{14} + 10176x^{13}\\
      %   &- 3552x^{12} + 632x^{11} - 44x^{10})y^6\\
      %   &+ (-137438953472x^{32} + 687194767360x^{31} - 1546188226560x^{30}\\
      %   &+ 2061584302080x^{29} - 1803886264320x^{28} + 1082331758592x^{27} - 450971566080x^{26} + 128849018880x^{25}\\
      %   &- 24159191040x^{24} + 2684354560x^{23} - 134217728x^{22}\\
      %   &- 1/2x^{12} + 11/8x^{11} - 23/16x^{10} + 93/128x^9 - 23/128x^8 + 9/512x^7)y^5\\
      %   &+ (33554432x^{28} - 167772160x^{27} + 524288000x^{26} - 1090519040x^{25}\\
      %   &+ 1468006400x^{24} - 1291845632x^{23} + 752353280x^{22}\\
      %   &- 288358400x^{21} + 70123520x^{20} - 9830400x^{19} + 606208x^{18}\\
      %   &+ 1/524288x^5 - 1/524288x^4 + 1/4194304x^3)y^4\\
      %   &+ (98304x^{22} - 368640x^{21} + 589824x^{20} - 522240x^{19}\\
      %   &+ 276480x^{18} - 87552x^{17} + 15360x^{16} - 1152x^{15})y^3\\
      %   &+ (13194139533312x^{39} - 85761906966528x^{38} + 257285720899584x^{37} - 471690488315904x^{36} + 589613110394880x^{35}\\
      %   &- 530651799355392x^{34} + 353767866236928x^{33} - 176883933118464x^{32} + 66331474919424x^{31}\\
      %   &- 18425409699840x^{30} + 3685081939968x^{29} - 502511173632x^{28} + 41875931136x^{27}\\
      %   &- 1610612736x^{26} + 12x^{18} - 42x^{17} + 243/4x^{16}\\
      %   &- 367/8x^{15} + 149/8x^{14} - 57/16x^{13} + 7/64x^{12} + 5/128x^{11})y^2\\
      %   &+ (8589934592*x^35 - 53687091200*x^34 + 153008209920*x^33 - 263066746880*x^32 + 304003153920*x^31 - 248839667712*x^30 + 147975045120*x^29 - 64424509440*x^28 + 20384317440*x^27 - 4571791360*x^26 + 689963008*x^25 - 62914560*x^24 + 2621440*x^23 + 1/8192*x^11 - 7/32768*x^10 + 1/8192*x^9 - 3/131072*x^8)*y + 72057594037927936*x^50 - 576460752303423488*x^49 + 2161727821137838080*x^48 - 5044031582654955520*x^47 + 8196551321814302720*x^46 - 9835861586177163264*x^45 + 9016206453995732992*x^44 - 6440147467139809280*x^43 + 3622582950266142720*x^42 - 1610036866784952320*x^41 + 563512903374733312*x^40 - 153685337284018176*x^39 + 32017778600837120*x^38 - 4925812092436480*x^37 + 527765581332480*x^36 - 35184372088832*x^35 + 1099511627776*x^34 + 1048576*x^31 - 6291456*x^30 + 17039360*x^29 - 27525120*x^28 + 29499392*x^27 - 22052864*x^26 + 11755520*x^25 - 4481024*x^24 + 1203200*x^23 - 219136*x^22 + 25088*x^21 - 1536*x^20 + 32*x^19 + 1/4294967296*x^4\\
      % \end{split}
    % \end{align}
    \begin{align}
      \label{eqn:d16t1_81616}
      \begin{split}
        f_{\texttt{16T1-8,16,16-g7}}
        &=y^{16} + x^9 - 7x^8 + 21x^7 - 35x^6 + 35x^5 - 21x^4 + 7x^3 - x^2\\
      \end{split}
    \end{align}
    % \begin{align}
    %   \label{eqn:d16t4_444}
    %   \begin{split}
    %     f_{\texttt{}}
    %     &=\\
    %   \end{split}
    % \end{align}
    % \begin{align}
    %   \label{eqn:d16t14_448}
    %   \begin{split}
    %     f_{\texttt{}}
    %     &=\\
    %   \end{split}
    % \end{align}
    \begin{align}
      \label{eqn:d16t12_248}
      \begin{split}
        f_{\texttt{16T12-4,2,8-g2}}
        &=y^{16} + (-1/8x^4 + 1/8x^3)y^{12} + (1/1024x^9 - 1/2048x^8\\
        &- 1/256x^7 + 1/256x^6)y^8 + (1/32768x^{12} - 1/32768x^{11})y^4\\
        &+ 1/16777216x^{16}
      \end{split}
    \end{align}
    \begin{align}
      \label{eqn:d16t10_244}
      \begin{split}
        &f_{\texttt{16T10-4,2,4-g1}}=\\
        &y^{16} + (-1/4x^4 + 1/2x^3 + 6x^2 + 8x - 4)y^{12} + (-5x^4 - 15x^3 + 20x^2)y^{10}\\
        &+ (3/128x^8 - 1/16x^7 + 19/16x^6 + 15x^5 - 83/4x^4\\
        &+ 15/2x^3 + 34x^2 - 16x + 6)y^8\\
        &+ (1/8x^8 - 47/8x^7 + 31/4x^6 + 6x^5 - 46x^4 + 46x^3 - 8x^2)y^6\\
        &+ (-1/1024x^{12} + 1/512x^{11} - 11/128x^{10} + 17/16x^9 - 63/64x^8\\
        &- 29/8x^7 + 113/8x^6 - 13x^5 - 63/4x^4 + 79/2x^3 - 22x^2 + 8x - 4)y^4\\
        &+ (3/256x^{12} - 23/256x^{11} + 1/64x^{10} + 9/16x^9 - 11/8x^8\\
        &+ 9/8x^7 + 7/4x^6 - 6x^5 - x^4 + 17x^3 - 12x^2)y^2\\
        &+ 1/65536x^{16} - 1/2048x^{14} + 1/512x^{13} + 3/1024x^{12} - 15/512x^{11}\\
        &+ 9/128x^{10} - 3/32x^9 + 35/128x^8 - 9/16x^7 + 3/16x^6 + 3/4x^4\\
        &+ 1/2x^3 - 2x^2 + 1
      \end{split}
    \end{align}
    % \begin{align}
    %   \label{eqn:d16t1_21616}
    %   \begin{split}
    %     f_{\texttt{}}
    %     &=\\
    %   \end{split}
    % \end{align}
    % \begin{align}
    %   \label{eqn:d16t1_41616}
    %   \begin{split}
    %     f_{\texttt{}}
    %     &=\\
    %   \end{split}
    % \end{align}
    % \begin{align}
      % \label{eqn:d16t6_488}
      % \begin{split}
      %   f_{\texttt{}}
      %   &=\\
      % \end{split}
    % \end{align}
    \begin{align}
      \label{eqn:d16t8_444}
      \begin{split}
        &f_{\texttt{16T8-4,4,4-g3}}=\\
        &y^{16} + (-1/4x^4 + 3/4x^3 - 3/4x^2 + 33/4x - 4)y^{12}\\
        &+ (-20x^3 + 40x^2 - 20x)y^{10}\\
        &+ (3/128x^8 - 9/64x^7 + 45/128x^6 + 481/32x^5 - 6899/128x^4\\
        &+ 4455/64x^3 - 2909/128x^2 - 33/4x + 6)y^8\\
        &+ (-11/2x^7 + 55/2x^6 - 55x^5 + 39x^4 + 25/2x^3 - 53/2x^2 + 8x)y^6\\
        &+ (-1/1024x^{12} + 9/1024x^{11} - 9/256x^{10} + 269/256x^9 - 3287/512x^8\\
        &+ 8991/512x^7 - 5305/256x^6 + 65/256x^5 + 15527/1024x^4\\
        &+ 3201/1024x^3 - 1135/64x^2 + 63/4x - 4)y^4\\
        &+ (-5/64x^{11} + 5/8x^{10} - 35/16x^9 + 27/8x^8 + 1/32x^7 - 65/8x^6\\
        &+ 205/16x^5 - 75/8x^4 + 987/64x^3 - 49/2x^2 + 12x)y^2\\
        &+ 1/65536x^{16} - 3/16384x^{15} + 33/32768x^{14} - 23/16384x^{13}\\
        &- 721/65536x^{12} + 549/8192x^{11} - 2009/16384x^{10} - 995/8192x^9\\
        &+ 63855/65536x^8 - 33495/16384x^7 + 78241/32768x^6 - 20131/16384x^5\\
        &- 63679/65536x^4 + 2097/1024x^3 - 157/128x^2 + 1/4x + 1
      \end{split}
    \end{align}
  }
  \section{Degree $32$}{
    \label{sec:d32}
    In degree $32$ there are $151$
    isomorphism classes of $2$-group Belyi maps
    with $96$ distinct passports.
    All passports are of size $1$
    except for the following.
    All size $8$ passports are
    listed in
    \eqref{eqn:d32size8},
    the size $6$ passport is
    listed in
    \eqref{eqn:d32size6},
    all size $4$ passports are
    listed in
    \eqref{eqn:d32size4},
    the size $3$ passport is
    listed in
    \eqref{eqn:d32size3},
    and
    all size $2$ passports are
    listed in
    \eqref{eqn:d32size2}.
    \begin{align}
      \label{eqn:d32size8}
      \texttt{32S1-16,32,32-g15},\quad
      \texttt{32S1-32,16,32-g15},\quad
      \texttt{32S1-32,32,16-g15}
      % \begin{split}
      %   &\texttt{32S1-16,32,32-g15}\\
      %   &\texttt{32S1-32,16,32-g15}\\
      %   &\texttt{32S1-32,32,16-g15}\\
      % \end{split}
    \end{align}
    \begin{align}
      \label{eqn:d32size6}
      \texttt{32S15-8,8,8-g11}
    \end{align}
    \begin{align}
      \label{eqn:d32size4}
      \texttt{32S1-8,32,32-g14},\quad
      \texttt{32S1-32,8,32-g14},\quad
      \texttt{32S1-32,32,8-g14}
      % \begin{split}
      %   &\texttt{32S1-8,32,32-g14}\\
      %   &\texttt{32S1-32,8,32-g14}\\
      %   &\texttt{32S1-32,32,8-g14}\\
      % \end{split}
    \end{align}
    \begin{align}
      \label{eqn:d32size3}
      \texttt{32S6-4,4,4-g5}
    \end{align}
    \begin{align}
      \label{eqn:d32size2}
      \begin{split}
        % 32S1-4,32,32-g12
        &\texttt{32S1-4,32,32-g12},\quad
        \texttt{32S1-32,4,32-g12},\quad
        \texttt{32S1-32,32,4-g12},\\
        % 32S10-4,4,8-g7
        &\texttt{32S10-4,4,8-g7},\quad
        \texttt{32S10-4,8,4-g7},\quad
        \texttt{32S10-8,4,4-g7},\\
        % 32S11-4,4,8-g7
        &\texttt{32S11-4,4,8-g7},\quad
        \texttt{32S11-4,8,4-g7},\quad
        \texttt{32S11-8,4,4-g7},\\
        % 32S12-4,8,8-g9
        &\texttt{32S12-4,8,8-g9},\quad
        \texttt{32S12-8,4,8-g9},\quad
        \texttt{32S12-8,8,4-g9}\\
        % 32S16-8,16,16-g13
        &\texttt{32S16-8,16,16-g13},\quad
        \texttt{32S16-16,8,16-g13},\quad
        \texttt{32S16-16,16,8-g13}\\
        % 32S17-8,16,16-g13
        &\texttt{32S17-8,16,16-g13},\quad
        \texttt{32S17-16,8,16-g13},\quad
        \texttt{32S17-16,16,8-g13}
      \end{split}
    \end{align}
    To obtain a list of degree $16$
    function fields corresponding to
    $2$-group Belyi maps
    in characteristic $3$,
    use the following code from
    \cite{twogroupdessins}.
    \begin{magma}
load "config.m";
objs := [ReadTwoDBPassport(f) : f in PassportFilenames(32)];
fields := [* *];
for s in objs do
  fields cat:= FunctionFields(s);
end for;
    \end{magma}
  }
  % \section{Example \texttt{256S100-4,4,4-g57}}{
  %   \label{sec:256S100-4,4,4-g57}
  % }
%\chapter{Background}{\label{chapter:background}
  %  \section{What is a Belyi map?}{\label{sec:belyimaps}
  %    % \subsection{Complex manifolds, Riemann surfaces, and branched covers}{\label{subsec:riemannsurfaces}
  %    %   In Section \ref{subsec:riemannsurfaces} we outline
  %    %   basic results needed to define a ($2$-group) Belyi map
  %    %   as a holomorphic map of Riemann surfaces.
  %    %   For a more detailed discussion
  %    %   see \cite{miranda, farkas}.
  %    %   \begin{definition}
  %    %     \label{def:chart}
  %    %     A \defi{chart} on a topological space $X$
  %    %     is a homeomorphism
  %    %     $\phi\colon U\to V$
  %    %     where $U$ is an open subset of $X$
  %    %     and $V$ an open subset of $\CC$.
  %    %     We say the chart is \defi{centered}
  %    %     at $p\in U$
  %    %     if $\phi(p) = 0$.
  %    %     We say that $z = \phi(x)$ for $x\in U$
  %    %     is a \defi{local coordinate on $X$}.
  %    %   \end{definition}
  %    %   \begin{definition}
  %    %     \label{def:compatible}
  %    %     Let $\phi_1\colon U_1\to V_1$
  %    %     and $\phi_2\colon U_2\to V_2$
  %    %     be charts.
  %    %     $\phi_1$ and $\phi_2$ are
  %    %     \defi{compatible}
  %    %     if they are disjoint or
  %    %     the \defi{transition map}
  %    %     \[
  %    %       \phi_2\circ\phi_1^{-1}\colon
  %    %       \phi_1(U_1\cap U_2)\to
  %    %       \phi_2(U_1\cap U_2)
  %    %     \]
  %    %     is holomorphic.
  %    %   \end{definition}
  %    %   \begin{definition}
  %    %     \label{def:atlas}
  %    %     A \defi{complex atlas}
  %    %     on $X$ is a collection of compatible
  %    %     charts that cover $X$.
  %    %   \end{definition}
  %    %   \begin{definition}
  %    %     \label{def:equivatlas}
  %    %     Two atlases $\mathscr{A}_1$ and $\mathscr{A}_2$
  %    %     are \defi{equivalent}
  %    %     if every pair of charts
  %    %     $\phi_1, \phi_2$
  %    %     with
  %    %     $\phi_1\in\mathscr{A}_1$
  %    %     and
  %    %     $\phi_2\in\mathscr{A}_2$
  %    %     are compatible.
  %    %   \end{definition}
  %    %   \begin{definition}
  %    %     \label{def:complexstructure}
  %    %     A \defi{complex structure}
  %    %     on a topological space $X$
  %    %     is an equivalence class of atlases.
  %    %     % maximal atlas
  %    %   \end{definition}
  %    %   \begin{definition}
  %    %     \label{def:riemannsurface}
  %    %     A \defi{Riemann surface}
  %    %     is a second countable, connected,
  %    %     Hausdorff topological space $X$
  %    %     equipped with a complex structure.
  %    %   \end{definition}
  %    %   \begin{example}
  %    %     \label{exm:PP1}
  %    %     Let $\PP^1_\CC$ (or simply $\PP^1$)
  %    %     denote the set of $1$-dimensional subspaces of $\CC^2$
  %    %     which we can write as
  %    %     \[
  %    %       \{[z:w] : z,w\in\CC\text{ and }zw\ne 0\}
  %    %     \]
  %    %     where $[z:w]$ denotes the $\CC$-span of $(z,w)\in\CC^2$.
  %    %     Let $U_0 = \{[z:w]\in\PP^1 : z\neq 0\}$,
  %    %     $U_1 = \{[z:w]\in\PP^1 : w\neq 0\}$,
  %    %     define $\phi_0\colon U_0\to\CC$
  %    %     by $[z:w]\mapsto \frac{w}{z}$,
  %    %     and
  %    %     define $\phi_1\colon U_1\to\CC$
  %    %     by $[z:w]\mapsto \frac{z}{w}$.
  %    %     On $V\colonequals \phi_i(U_0\cap U_1) = \CC^\times$
  %    %     we have the holomorphic transition function
  %    %     $\phi_1\circ\phi_0^{-1}\colon V\to\CC$
  %    %     defined by $z\mapsto \frac{1}{z}$.
  %    %     The atlas consisting of these two charts $\phi_0,\phi_1$
  %    %     define a complex structure on $\PP^1$
  %    %     giving it the structure of a Riemann surface.
  %    %   \end{example}
  %    %   \begin{example}
  %    %     \label{exm:planecurves}
  %    %     \mm{plane curves and local complete intersections in $\PP^n$}
  %    %   \end{example}
  %    %   \begin{definition}
  %    %     \label{def:singularities}
  %    %     A function $f\colon X\to\CC$ is
  %    %     \defi{holomorphic
  %    %       (respectively has a removable singularity,
  %    %       has a pole, has an essential singularity)
  %    %     }
  %    %     at $p\in X$ if there exists a chart
  %    %     $\phi\colon U\to V$ such that
  %    %     $f\circ\phi^{-1}$ is
  %    %     holomorphic (respectively has a removable singularity, has a pole,
  %    %     has an essential singularity)
  %    %     at $\phi(p)$.
  %    %     $f$ is \defi{holomorphic on an open set $W\subseteq X$} if
  %    %     $f$ is holomorphic at all $p\in W$.
  %    %     $f$ is \defi{meromorphic} at $p\in X$
  %    %     if $f$ is holomorphic, has a removable singularity,
  %    %     or has a pole at $p$.
  %    %     $f$ is \defi{meromorphic on an open set $W\subseteq X$}
  %    %     if
  %    %     $f$ is meromorphic at all $p\in W$.
  %    %   \end{definition}
  %    %   % O_X(W) = {f holo on W}
  %    %   \begin{definition}
  %    %     \label{def:laurentseries}
  %    %     Let $W$ be an open subset of $X$ and denote the set of
  %    %     meromorphic functions on $W$ by
  %    %     \[
  %    %       \mathcal{M}_X(W)
  %    %       =
  %    %       \{f\colon W\to\CC : f\text{ is meromorphic on $W$ }\}.
  %    %     \]
  %    %     Let $p\in W$ and
  %    %     let $f\in\mathcal{M}_X(W)$.
  %    %     Then there exists a chart $\phi$ on $W$
  %    %     with local coordinate $z$
  %    %     and
  %    %     $\phi(p) = z_0$
  %    %     such that $f\circ\phi^{-1}$ is meromorphic at $z_0$.
  %    %     Thus,
  %    %     we can write a \defi{Laurent series expansion
  %    %     for $f\circ\phi^{-1}$}
  %    %     in a neighborhood of $z_0$
  %    %     in the local coordinate $z$
  %    %     as
  %    %     \[
  %    %       (f\circ\phi^{-1})(z) = \sum_{n}c_n(z-z_0)^n.
  %    %     \]
  %    %   \end{definition}
  %    %   \begin{definition}
  %    %     \label{def:ord}
  %    %     The minimum $n$ such that $c_n\neq 0$
  %    %     in Definition \ref{def:laurentseries}
  %    %     is the \defi{order of $f$ at $p$}
  %    %     and denoted
  %    %     $\ord_p(f)$.
  %    %   \end{definition}
  %    %   \begin{definition}
  %    %     \label{ref:holomorphicmapofRS}
  %    %     $F\colon X\to Y$ is
  %    %     \defi{holomorphic}
  %    %     at $p\in X$
  %    %     if there exists
  %    %     charts
  %    %     $\phi_1\colon U_1\to\CC$
  %    %     $\phi_2\colon U_2\to\CC$
  %    %     with $p\in U_1$ and $F(p)\in U_2$
  %    %     such that
  %    %     $\phi_2\circ F\circ\phi_1^{-1}$
  %    %     is holomorphic at $\phi_1(p)$.
  %    %     Similarly,
  %    %     $F$ is \defi{holomorphic on an open set $W\subseteq X$}
  %    %     if it is holomorphic at every $p\in W$.
  %    %   \end{definition}
  %    %   \begin{definition}
  %    %     \label{def:RSautoiso}
  %    %     An \defi{isomorphism} of Riemann surfaces is
  %    %     a bijective holomorphic map $F\colon X\to Y$
  %    %     where $F^{-1}$ is holomorphic.
  %    %     An isomorphism from $X$ to $X$ is
  %    %     an \defi{automorphism}.
  %    %   \end{definition}
  %    %   \begin{example}
  %    %     \label{exm:PP1isoCCoo}
  %    %     $\PP^1$ defined in Example
  %    %     \ref{exm:PP1}
  %    %     is isomorphic to $\CC\cup \{\infty\}$
  %    %     the compactification of the complex plane
  %    %     via stereographic projection.
  %    %   \end{example}
  %    %   \begin{theorem}
  %    %     \label{thm:compactonto}
  %    %     Let $X$ be a compact Riemann surface
  %    %     and $F\colon X\to Y$
  %    %     a nonconstant holomorphic map.
  %    %     Then $Y$ is compact an $F$ is onto.
  %    %   \end{theorem}
  %    %   \begin{prop}
  %    %     \label{prop:discretefibers}
  %    %     Let $F\colon X\to Y$ be a nonconstant
  %    %     holomorphic map of Riemann surfaces.
  %    %     Then for every $y\in Y$,
  %    %     the fiber $F^{-1}(y)$ is a discrete subset of $X$.
  %    %   \end{prop}
  %    %   % meromorphic functions on X correspond to holomorphic maps to PP1 not identically oo
  %    %   \begin{theorem}
  %    %     \label{thm:localnormalform}
  %    %     Let $F\colon X\to Y$ be a nonconstant holomorphic map.
  %    %     Let $p\in X$.
  %    %     Then there exists a positive integer $m$ such that
  %    %     for all charts
  %    %     $\phi_2$ centered at $F(p)$
  %    %     there exists a chart $\phi_1$ centered at $p$
  %    %     (let $z$ be the local coordinate)
  %    %     with $(\phi_2\circ F\circ\phi_1^{-1})(z) = z^m$.
  %    %   \end{theorem}
  %    %   \begin{definition}
  %    %     \label{def:multiplicity}
  %    %     Let $F\colon X\to Y$ be a holomorphic map
  %    %     of Riemann surfaces.
  %    %     The \defi{multiplicity}
  %    %     of $F$ at $p\in X$ is denoted
  %    %     $\mult_p(F)$ and is defined to be the unique
  %    %     integer $m$ from Theorem \ref{thm:localnormalform}
  %    %     such that there exist local coordinates about $p$
  %    %     and $F(p)$ so that
  %    %     $F$ can be written as $z\mapsto z^m$.
  %    %   \end{definition}
  %    %   \begin{definition}
  %    %     \label{def:ramificationRS}
  %    %     Let $F\colon X\to Y$ be a nonconstant holomorphic
  %    %     map of Riemann surfaces.
  %    %     $p\in X$ is a \defi{ramification point}
  %    %     of $F$
  %    %     if $\mult_p(F)\geq 2$.
  %    %     $y\in X$ is a \defi{branch point} of $F$
  %    %     if $F^{-1}(y)$ contains a ramification point.
  %    %   \end{definition}
  %    %   \begin{example}
  %    %     \label{exm:planecurve}
  %    %     \mm{plane curves, p.46, hyperelliptic curves, \ldots}
  %    %   \end{example}
  %    %   \begin{definition}
  %    %     \label{def:degreemapofRS}
  %    %     The \defi{degree}
  %    %     of a nonconstant holomorphic map
  %    %     $F\colon X\to Y$
  %    %     is
  %    %     \[
  %    %       \deg(F) \colonequals
  %    %       \sum_{p\in F^{-1}(y)}\mult_p(F)
  %    %     \]
  %    %     for any $y\in Y$.
  %    %   \end{definition}
  %    %   % sum_p ord_p(f) = 0
  %    %   \begin{theorem}[Riemann-Hurwitz]
  %    %     \label{thm:riemannhurwitzforriemannsurfaces}
  %    %     Let $F\colon X\to Y$
  %    %     be a nonconstant holomorphic map of
  %    %     compact Riemann surfaces.
  %    %     Let $g(X), g(Y)$ be the topological genus
  %    %     of $X, Y$ respectively.
  %    %     Then
  %    %     \begin{equation}
  %    %       \label{eqn:riemannhurwitzforriemannsurfaces}
  %    %       2g(X)-2=
  %    %       \deg(F)(2g(Y)-2)+
  %    %       \sum_{p\in X}(\mult_p(F)-1).
  %    %     \end{equation}
  %    %   \end{theorem}
  %    %   % groups acting on Riemann surfaces
  %    %   \begin{definition}
  %    %     \label{def:coveringspace}
  %    %     A \defi{covering space}
  %    %     of a real or complex manifold $V$
  %    %     is a continuous map
  %    %     $F\colon U\to V$ such that
  %    %     the following conditions hold:
  %    %     \begin{itemize}
  %    %       \item
  %    %         $F$ is surjective
  %    %       \item
  %    %         For all $v\in V$
  %    %         there exists a neighborhood
  %    %         $W$ of $v\in V$ such that
  %    %         $F^{-1}(W)$ consists of a disjoint
  %    %         union of open sets of $U$
  %    %         $\{U_\alpha\}_{\alpha\in I}$
  %    %         with $F|_{U_\alpha}\colon U_\alpha\to W$
  %    %         a homeomorphism.
  %    %     \end{itemize}
  %    %     The cardinality of $I$ is the \defi{degree}
  %    %     of the cover.
  %    %   \end{definition}
  %    %   \begin{definition}
  %    %     \label{def:coveringspaceiso}
  %    %     Two covering spaces $U_1\to V$
  %    %     and $U_2\to V$ are
  %    %     \defi{isomorphic}
  %    %     if there exists a homeomorphism
  %    %     $U_1\to U_2$
  %    %     making the diagram
  %    %     \begin{equation}
  %    %       \label{eqn:coveringspaceiso}
  %    %       \begin{tikzcd}
  %    %         U_1\arrow{dr}\arrow{rr}{}&&U_2\arrow{dl}\\
  %    %                        &V
  %    %       \end{tikzcd}
  %    %     \end{equation}
  %    %     commute.
  %    %   \end{definition}
  %    %   \begin{prop}
  %    %     \label{prop:universalcoveringspace}
  %    %     Given a real or complex manifold $V$,
  %    %     there exists a covering space
  %    %     $F_0\colon U_0\to V$
  %    %     such that $U_0$ is simply connected.
  %    %     $F_0$ is unique up to isomorphism
  %    %     and is universal in the following sense:
  %    %     If $F\colon U\to V$ is another cover of $V$,
  %    %     then there exists
  %    %     $G\colon U_0\to V$
  %    %     such that $F_0 = F\circ G$.
  %    %   \end{prop}
  %    %   Pick a basepoint $q\in V$
  %    %   and let $\pi_1(V,q)$ denote
  %    %   the \defi{fundamental group} of $V$
  %    %   with loops based at $q$.
  %    %   Then $\pi_1(V,q)$ acts on the cover
  %    %   $F_0\colon U_0\to V$
  %    %   via path lifting.
  %    %   % iso classes of connected covers F:U->V correspond to conj classes of subgroups of pi_1(V,q)
  %    %   We now restrict to the case of finite degree covers.
  %    %   Let $F\colon U\to V$ be a degree $d$ cover
  %    %   and consider the fiber of $q$,
  %    %   $F^{-1}(q) = \{x_1,\dots,x_n\}$.
  %    %   To a loop $\gamma$ on $V$ based at $q$,
  %    %   we can lift $\gamma$ to $d$ paths
  %    %   $\wt{\gamma}_1,\dots,\wt{\gamma}_d$
  %    %   in $U$ where $\wt{\gamma}_i$
  %    %   starts at $x_i$
  %    %   and ends at $x_j$ for some $j$.
  %    %   For each $i\in \{1,\ldots,d\}$ denote the terminal
  %    %   point of $\wt{\gamma}_i$ by $x_{\sigma(i)}\in F^{-1}(q)$.
  %    %   $\sigma$ defines
  %    %   a \defi{monodromy representation}
  %    %   \begin{equation}
  %    %     \label{eqn:monodromyrep}
  %    %     \rho\colon
  %    %     \pi_1(V,q)\to S_d.
  %    %   \end{equation}
  %    %   \begin{lemma}
  %    %     \label{lem:transitive}
  %    %     Let $\rho\colon\pi_1(V,q)\to S_d$
  %    %     be the monodromy representation
  %    %     of a
  %    %     finite degree cover $F\colon U\to V$
  %    %     with $U$ connected.
  %    %     Then the image of $\rho$
  %    %     is a transitive subgroup of $S_d$.
  %    %   \end{lemma}
  %    %   \begin{definition}
  %    %     \label{def:monodromyrepofholomorphicmap}
  %    %     Let $F\colon X\to Y$ be a nonconstant holomorphic
  %    %     map of Riemann surfaces.
  %    %     Let
  %    %     \begin{align*}
  %    %       V &\colonequals Y\sm \{\text{branch points of $F$}\}\\
  %    %       U &\colonequals X\sm \{\text{preimages of branch points of $F$}\}.
  %    %     \end{align*}
  %    %     Then $F|_U\colon U\to V$ is a covering space
  %    %     and induces a monodromy representation
  %    %     which we refer to as the
  %    %     \defi{monodromy representation of $F$}.
  %    %   \end{definition}
  %    %   \begin{definition}
  %    %     \label{def:branchedcoverofriemannsurface}
  %    %     A \defi{branched cover of Riemann surfaces} is a nonconstant holomorphic map
  %    %     of Riemann surfaces
  %    %     $\phi\colon X\to Y$ where $X$ is a compact connected Riemann surface.
  %    %   \end{definition}
  %    %   Let $Y$ be a compact connected Riemann surface,
  %    %   let $B\subseteq Y$ be a finite set,
  %    %   let $V\colonequals Y\sm B$,
  %    %   and let $F\colon U\to V$ be a finite degree cover.
  %    %   Then there is a unique complex structure on $U$
  %    %   making $F$ holomorphic.
  %    %   Let $b\in B$
  %    %   and consider a neighborhood $W$ of $b$
  %    %   small enough so that $W\sm \{b\}$ is homeomorphic to a punctured disk.
  %    %   Then $F^{-1}(W\sm \{b\})$ is a finite disjoint union
  %    %   of punctured disks $\{\wt{U}_j\}_j$.
  %    %   Moreover,
  %    %   by Theorem \ref{thm:localnormalform}
  %    %   there are integers $m_j$ for each $j$ such that
  %    %   \[
  %    %     F|_{\wt{U}_j}\colon\wt{U}_j\to W\sm \{b\}
  %    %   \]
  %    %   can be written as $z\mapsto z^{m_j}$ in local coordinates.
  %    %   Extending this holomorphic map to the unpunctured disks
  %    %   for every $b\in B$ yields the following correspondence.
  %    %   \begin{prop}
  %    %     \label{prop:branchedcoverofriemannsurfaces}
  %    %     Let $Y$ be a compact Riemann surface,
  %    %     $B\subseteq Y$ a finite set,
  %    %     and $q\in Y\sm B$ a basepoint.
  %    %     Then there is a bijection of sets
  %    %     \begin{center}
  %    %       \begin{tikzpicture}[scale=1]
  %    %         \node at (-4,0) (maps)
  %    %         {
  %    %           $
  %    %             \left\{
  %    %               \begin{minipage}{30ex}
  %    %                 isomorphism classes of
  %    %                 degree $d$
  %    %                 holomorphic maps
  %    %                 $F\colon X\to Y$
  %    %                 with branch points
  %    %                 contained in $B$
  %    %               \end{minipage}
  %    %             \right\}
  %    %           $
  %    %         };
  %    %         \node at (4,0) (perms)
  %    %         {
  %    %           $
  %    %             \left\{
  %    %               \begin{minipage}{30ex}
  %    %                 group homomorphisms
  %    %                 $\rho\colon\pi_1(Y\sm B, q)\to S_d$
  %    %                 with image a transitive subgroup
  %    %                 up to conjugation in $S_d$
  %    %               \end{minipage}
  %    %             \right\}
  %    %           $
  %    %         };
  %    %         % \draw[<->] (maps) to node [above,rotate=-35] {$\sim$} node {} (perms);
  %    %         \draw[<->] (maps) to node [above] {$\sim$} node {} (perms);
  %    %       \end{tikzpicture}
  %    %     \end{center}
  %    %   \end{prop}
  %    %   If we let $Y = \PP^1$ in Proposition \ref{prop:branchedcoverofriemannsurfaces}
  %    %   and $B = \{b_1,\ldots,b_n\}\subseteq\PP^1$ we obtain
  %    %   the following correspondence:
  %    %   \begin{center}
  %    %     \begin{tikzpicture}[scale=1]
  %    %     \label{branchedcoverscorrespondtopermutations}
  %    %       \node at (-4,0) (maps)
  %    %       {
  %    %         $
  %    %           \left\{
  %    %             \begin{minipage}{30ex}
  %    %               isomorphism classes of
  %    %               degree $d$
  %    %               holomorphic maps
  %    %               $F\colon X\to \PP^1$
  %    %               with branch points
  %    %               contained in $B$
  %    %             \end{minipage}
  %    %           \right\}
  %    %         $
  %    %       };
  %    %       \node at (4,0) (perms)
  %    %       {
  %    %         $
  %    %           \left\{
  %    %             \begin{minipage}{37ex}
  %    %               $n$-tuples of permutations
  %    %               $(\sigma_1,\ldots,\sigma_n)\in S_d^n$
  %    %               generating a transitive subgroup
  %    %               with $\sigma_1\cdots\sigma_n=1$
  %    %               up to simultaneous conjugation in $S_d$
  %    %             \end{minipage}
  %    %           \right\}
  %    %         $
  %    %       };
  %    %       % \draw[<->] (maps) to node [above,rotate=-35] {$\sim$} node {} (perms);
  %    %       \draw[<->] (maps) to node [above] {$\sim$} node {} (perms);
  %    %     \end{tikzpicture}
  %    %   \end{center}
  %    %   Moreover,
  %    %   if $\sigma_i$ has cycle structure $(m_1,\dots,m_k)$,
  %    %   then there are $k$ preimages
  %    %   $u_1,\dots,u_k$ of $b_i$
  %    %   in the cover $F\colon X\to\PP^1$
  %    %   with $\mult_{u_j}(F) = m_j$ for all $j$.
  %    %   \begin{definition}
  %    %     \label{def:belyimapriemannsurface}
  %    %     A \defi{Belyi map} is a nonconstant holomorphic map of
  %    %     compact connected Riemann surfaces
  %    %     $F\colon X\to\PP^1$ with no more than $3$ branch points.
  %    %   \end{definition}
  %    %   \begin{definition}\label{def:galoiscover}
  %    %   \end{definition}
  %    % }
  %    % \subsection{Branched covering spaces}{\label{subsec:branchedcovers}
  %    %   \mm{monodromy, ramification, Galois cover, etc}
  %    %   \begin{definition}\label{def:galoiscover}
  %    %   \end{definition}
  %    % }
  %    \subsection{Algebraic curves}{\label{subsec:algebraiccurves}
  %      Let $K$ be a field isomorphic to the complex numbers,
  %      the real numbers, a number field, or a finite field.
  %      Let $\Kal$ denote an algebraic closure of $K$.
  %      For a detailed treatment see
  %      \cite[Chapters 1-2]{silverman}.
  %      %%%%%%%%%%%%AFFINE VARIETIES
  %      \begin{definition}
  %        \label{def:affinespace}
  %        \defi{Affine $n$-space} over $K$ is the
  %        set of $n$-tuples of elements in $\Kal$
  %        and denoted $\AA^n(\Kal)$ or $\AA^n$.
  %        % set of K-rational points of AA^n
  %      \end{definition}
  %      We will denote \defi{points} in $\AA^n$ by $P$.
  %      % Galois action on AA^n
  %      Let $\Kal[x_1,\dots,x_n]$ be the $n$-variable
  %      polynomial ring over $\Kal$.
  %      To each ideal $I\trianglelefteq\Kal[x_1,\dots,x_n]$
  %      we associate the following subset of $\AA^n$.
  %      \begin{equation}
  %        \label{eqn:algebraicset}
  %        V(I)\colonequals
  %        \{P\in\AA^n : f(P) = 0 \text{ for all } f\in I\}
  %      \end{equation}
  %      \begin{definition}
  %        \label{def:algebraicset}
  %        Subsets of $\AA^n$
  %        of the form $V(I)$ for some ideal $I$
  %        as in Equation \ref{eqn:algebraicset}
  %        are called
  %        \defi{affine algebraic sets}.
  %      \end{definition}
  %      Let $V$ be an affine algebraic set.
  %      To such a set we can associate
  %      the following ideal of $\Kal[x_1,\dots,x_n]$.
  %      \begin{equation}
  %        \label{eqn:idealofalgebraicset}
  %        I(V)\colonequals
  %        \{f\in\Kal[x_1,\dots,x_n]: f(P)=0 \text{ for all }f\in I\}.
  %      \end{equation}
  %      \begin{definition}
  %        \label{def:definedoverK}
  %        Let $\AA^n(K)$ denote the set of
  %        \defi{$K$-rational points} of $\AA^n$ defined by
  %        \[
  %          \AA^n(K)\colonequals
  %          \{(x_1,\dots,x_n)\in\AA^n : x_i\in K\}.
  %        \]
  %        Let $V$ be an algebraic set.
  %        We say $V$ is
  %        \defi{defined over $K$}
  %        if $I(V)$ can be generated by
  %        polynomials in $K[x_1,\dots,x_n]$.
  %        For such $V$ we can define the
  %        \defi{$K$-rational points of $V$} by
  %        \[
  %          V(K)\colonequals
  %          V\cap\AA^n(K).
  %        \]
  %      \end{definition}
  %      Let $G_K = \Gal(\Kal/K)$.
  %      Another way to characterize $V(K)$ is
  %      the points fixed under the action of $G_K$:
  %      \begin{equation}
  %        \label{eqn:galoisactiononV}
  %        V(K) =
  %        \{P\in V : P^\sigma = P\text{ for all }\sigma\in G_K\}
  %      \end{equation}
  %      \begin{definition}
  %        \label{def:affinevariety}
  %        An affine algebraic set $V$ is called
  %        an \defi{affine variety}
  %        if $I(V)\trianglelefteq\Kal[x_1,\dots,x_n]$ is a prime ideal.
  %      \end{definition}
  %      If an affine algebraic set $V$ is defined over $K$,
  %      then $V$ is an affine variety if
  %      $I(V)\trianglelefteq K[x_1,\dots,x_n]$
  %      is a prime ideal.
  %      \begin{definition}
  %        \label{def:coordinateringfunctionfield}
  %        Let $V$ be an affine variety defined over $K$.
  %        We define the
  %        \defi{affine coordinate ring}
  %        by
  %        \[
  %          K[V]
  %          \colonequals
  %          \frac{
  %            K[x_1,\dots,x_n]
  %          }{
  %          I(V)
  %          }
  %        \]
  %        and the
  %        \defi{function field of $V$}
  %        by the field of fractions of $K[V]$
  %        denoted $K(V)$.
  %        We can similarly define this construction
  %        for $\Kal[V]$ and $\Kal(V)$.
  %      \end{definition}
  %      \begin{definition}
  %        \label{def:dimension}
  %        The \defi{dimension}
  %        of an affine variety $V$
  %        is the transcendence degree of the field extension
  %        $\Kal(V)$ over $\Kal$.
  %      \end{definition}
  %      % \begin{definition}
  %      %   \label{def:nonsingular}
  %      %   Let $V$ be an affine variety of dimension $d$,
  %      %   let $P\in V$,
  %      %   and $I(V)$ generated by
  %      %   $\{f_1,\dots,f_m\}$.
  %      %   We say $V$ is \defi{nonsingular at $P$}
  %      %   if the $m\times n$ matrix of partial
  %      %   derivatives
  %      %   whose $(i,j)$-entry is
  %      %   \[
  %      %     \frac{\partial f_i}{\partial x_j}
  %      %   \]
  %      %   has rank $n-d$.
  %      % \end{definition}
  %      \begin{definition}
  %        \label{def:nonsingular}
  %        Let $V$ be a variety of dimension $d$
  %        and $P\in V$.
  %        Consider the maximal ideal
  %        \[
  %          M_P\colonequals
  %          \{f\in\Kal[x_1,\dots,x_n]:f(P)=0\}.
  %        \]
  %        The quotient $M_P/M_P^2$ is a finite dimensional
  %        vector space over $\Kal$.
  %        We say $P$ is \defi{nonsingular}
  %        if the dimension of $M_P/M_P^2$ as
  %        a vector space over $\Kal$
  %        is equal to $d$.
  %      \end{definition}
  %      \begin{definition}
  %        \label{def:localringatP}
  %        Let $V$ be an affine variety
  %        and $P\in V$.
  %        The \defi{ring of regular functions on $V$ at $P$}
  %        is defined to be the localization
  %        of $\Kal[V]$ at the maximal ideal $M_P$
  %        (denoted $\Kal[V]_P$).
  %        More explicitly we have
  %        \[
  %          \Kal[V]_P
  %          \colonequals
  %          \Kal[V]_{M_P}
  %          =
  %          \{f/g\in\Kal[V]:g(P)\neq 0\}
  %        \]
  %        so that
  %        the elements of
  %        $\Kal[V]_P$
  %        are well-defined
  %        as functions on $V$.
  %      \end{definition}
  %      %%%%%%%%%%%%PROJECTIVE VARIETIES
  %      \begin{definition}
  %        \label{def:projectivespace}
  %        \defi{Projective $n$-space}
  %        over $K$ is denoted by $\PP^n(\Kal)$
  %        or $\PP^n$
  %        and is defined to be
  %        \[
  %          \PP^n\colonequals
  %          \{(x_0,\dots,x_{n})\in\AA^{n+1}:\text{ not all $x_i=0$ }\}/\sim
  %        \]
  %        where
  %        $(x_0,\dots,x_n)\sim(x_0',\dots,x_n')$
  %        if there exists $\lambda\in(\Kal)^\times$ with
  %        $x_i=\lambda y_i$ for all $i\in \{0,\dots,n\}$.
  %        The equivalence class of $(x_0,\dots,x_n)\in\AA^{n+1}$
  %        with respect to $\sim$ is denoted by
  %        $[x_0,\dots,x_n]$ or $[x_0:\cdots:x_n]$.
  %        We call these $x_i$
  %        \defi{homogeneous coordinates}
  %        of the point in $\PP^n$.
  %        As in the affine case,
  %        we define the
  %        \defi{$K$-rational points} of $\PP^n$
  %        to be
  %        \[
  %          \PP^n(K)\colonequals
  %          \{[x_0,\dots,x_n]\in\PP^n:x_i\in K \text{ for all } i\}.
  %        \]
  %      \end{definition}
  %      \begin{definition}
  %        \label{def:minimalfieldofdefinitionPPn}
  %        Let $P\in\PP^n$ with homogeneous coordinates
  %        $[x_0,\dots,x_n]$.
  %        The \defi{minimal field of definition of $P$ over $K$}
  %        is
  %        \[
  %          K(P)\colonequals
  %          K(x_0/x_i,\dots,x_n/x_i)
  %        \]
  %        for any $i\in \{0,\dots,n\}$.
  %      \end{definition}
  %      $\PP^n(K)$ is the set of $P\in\PP^n$
  %      fixed by the action of $G_K$.
  %      On the other hand,
  %      $K(P)$ is the fixed field of the
  %      subgroup
  %      $\{\sigma\in G_K: P = P^\sigma\}$.
  %      \begin{definition}
  %        \label{def:homogeneousideal}
  %        An ideal $I\trianglelefteq\Kal[x_0,\dots,x_n]$
  %        is \defi{homogeneous}
  %        if it can be generated by homogeneous polynomials.
  %        To a homogeneous ideal $I$ we can associate a subset of
  %        $\PP^n$ as follows.
  %        \[
  %          V(I)\colonequals
  %          \{P\in\PP^n : f(P)=0\text{ for all homogeneous $f\in\Kal[x_0,\dots,x_n]$ }\}
  %        \]
  %        A \defi{projective algebraic set}
  %        is a subset of $\PP^n$ which is $V(I)$
  %        for some homogeneous ideal $I$.
  %      \end{definition}
  %      To any projective algebraic set $V$,
  %      we can associate a homogeneous ideal $I(V)$ defined by
  %      \begin{equation}
  %        \label{eqn:projectivealgebraicsetideal}
  %        I(V)
  %        \colonequals
  %        \{f\in\Kal[x_0,\dots,x_n]:f \text{ is homogeneous and }f(P) = 0\text{ for all $P\in V$ }\}
  %      \end{equation}
  %      % \begin{theorem}\label{thm:curvesandfunctionfields}
  %      %   \mm{correspondence curves and function fields}
  %      % \end{theorem}
  %      % \begin{definition}
  %      %   Let $K\subseteq\CC$ be a field.
  %      %   A \defi{branched cover of algebraic curves over $K$}
  %      %   is a finite map of curves
  %      %   $\phi\colon X\to\PP^1$ defined over $K$.
  %      % \end{definition}
  %      % \mm{enough to define good curves and function fields}
  %    }
  %    \subsection{Riemann's existence theorem}{\label{subsec:riemannsexistence}
  %      Riemann surfaces are defined in Section \ref{subsec:riemannsurfaces}.
  %      Algebraic curves are defined in Section \ref{subsec:algebraiccurves}.
  %      Here in Section \ref{subsec:riemannsexistence}
  %      we establish the connection between these objects over the complex numbers.
  %      \par
  %      Let $X$ be an algebraic curve over $\CC$.
  %      Let $\CC(t)$ denote the function field of $\PP^1$.
  %      By Theorem \ref{thm:curvesandfunctionfields},
  %      $X$ corresponds to a finite extension $L\colonequals\CC(X)$
  %      over $\CC(t)$.
  %      Let $\alpha$ be a primitive element of $L/\CC(t)$.
  %      Then there exists a polynomial
  %      \begin{equation}
  %        \label{eqn:minpoly}
  %        f(x,t) = a_0(t)+a_1(t)x+a_2(t)x^2+\cdots+a_n(t)x^n\in\CC(t)[x]
  %      \end{equation}
  %      where $f(\alpha,t) = 0$ and (after possibly clearing denominators)
  %      $a_i(t) \in\CC[t]$.
  %      The polynomial $f$ in Equation \ref{eqn:minpoly}
  %      defines a Riemann surface $X'$
  %      as a branched cover of $\PP^1$
  %      with branch points
  %      \[
  %        S \colonequals\left\{t_0\in\CC : f(x,t_0) \text{ has repeated roots }\right\}.
  %      \]
  %      Here $x$ can be viewed as a meromorphic function on $X'$
  %      and we can identify the field of meromorphic functions on $X'$
  %      with $L$.
  %      This explains how we obtain a Riemann surface from an algebraic curve.
  %      \par
  %      Suppose instead we start with a compact Riemann surface $X$.
  %      Can we reverse the above process to construct an algebraic curve?
  %      The crucial part of this process is proving that
  %      there exists a meromorphic function on $X$ that realizes $X$
  %      as a branched cover of $\PP^1$
  %      (see Theorem \ref{thm:riemannsexistence} below).
  %      Given the existence of such a function,
  %      the field of meromorphic functions on $X$ is then realized
  %      as a finite extension of the meromorphic functions on $\PP^1$.
  %      Finally,
  %      by Theorem \ref{thm:curvesandfunctionfields},
  %      this corresponds to an algebraic curve.
  %      The existence of such a function
  %      is given by
  %      Theorem \ref{thm:riemannsexistence}
  %      (Riemann's existence theorem).
  %      \begin{theorem}\label{thm:riemannsexistence}
  %        Let $X$ be a compact Riemann surface.
  %        Then there exists a meromorphic function on $X$
  %        that separates points.
  %        That is, for any set of distinct points
  %        $\{x_1,\dots,x_n\}\subset X$
  %        and any set of distinct points
  %        $\{t_1,\dots,t_n\}\subset\PP^1$
  %        there exists a meromorphic function $f$
  %        on $X$ such that $f(x_i) = t_i$ for all $i$.
  %      \end{theorem}
  %      \mm{todo: more details...other formulations}
  %    }
  %    \subsection{Belyi's theorem}{\label{subsec:belyistheorem}
  %      In Sections
  %      \ref{subsec:riemannsufraces}, \ref{subsec:algebraiccurves},
  %      and \ref{subsec:riemannsexistence}
  %      we established the equivalence between
  %      compact Riemann surfaces and algebraic curves over $\CC$.
  %      This was done, in part,
  %      using branched covers.
  %      It turns out that branched covers are the key
  %      to descending from the transcendental world to
  %      the number-theoretic world in the following sense.
  %      \begin{theorem}[Belyi's theorem \cite{belyi}]\label{thm:belyistheorem}
  %        An algebraic curve $X$ over $\CC$ can be defined over a number field
  %        if and only if there exists a branched cover
  %        $\phi\colon X\to\PP^1$ unramified outside
  %        $\{0,1,\infty\}$.
  %      \end{theorem}
  %      These remarkable covers are the main focus of this work.
  %    }
  %    \subsection{Belyi maps and Galois Belyi maps}{\label{subsec:belyimaps}
  %      We now set up the framework to discuss
  %      the main mathematical objects of interest in this work.
  %      \begin{definition}\label{def:belyimap}
  %        A \defi{Belyi map}
  %        is a branched cover
  %        of algebraic curves over $\CC$
  %        (equivalently of Riemann surfaces)
  %        $\phi\colon X \to \PP^1$
  %        that is
  %        unramified outside
  %        $\{0,1,\infty\}$.
  %      \end{definition}
  %      \begin{definition}\label{def:belyiiso}
  %        Two Belyi maps
  %        $\phi\colon X\to\PP^1$ and
  %        $\phi'\colon X'\to\PP^1$
  %        are \defi{isomorphic}
  %        if there exists an isomorphism
  %        between $X$ and $X'$
  %        such that the diagram in Figure
  %        \ref{fig:belyiiso}
  %        commutes.
  %        If instead we only insist that the isomorphism
  %        makes the diagram in Figure
  %        \ref{fig:belyilax} commute,
  %        then we say that $\phi$ and $\phi'$
  %        are \defi{lax isomorphic}.
  %        \begin{figure}[ht]
  %          \begin{center}
  %            \begin{tikzcd}
  %              X\arrow{rr}{\sim}\arrow{dr}[swap]{\phi}&&X'\arrow{dl}{\phi'}\\
  %                                                     &\PP^1
  %            \end{tikzcd}
  %          \end{center}
  %          \caption{Belyi map isomorphism}
  %          \label{fig:belyiiso}
  %        \end{figure}
  %        \begin{figure}[ht]
  %          \begin{center}
  %            \begin{tikzcd}
  %              X\arrow{r}{\sim}\arrow{d}[swap]{\phi}&X'\arrow{d}{\phi'}\\
  %              \PP^1\arrow{r}{\sim}&\PP^1
  %            \end{tikzcd}
  %          \end{center}
  %          \caption{Belyi map lax isomorphism}
  %          \label{fig:belyilax}
  %        \end{figure}
  %      \end{definition}
  %      \begin{definition}\label{def:galoisbelyi}
  %        A Belyi map $\phi\colon X\to\PP^1$
  %        is \defi{Galois}
  %        if it is Galois as a cover
  %        (see Definition \ref{def:galoiscover}).
  %        A curve $X$ that admits a Galois Belyi map
  %        is called a \defi{Galois Belyi curve}.
  %      \end{definition}
  %      \begin{prop}\label{prop:galoiscover}
  %        Let $\phi\colon X\to\PP^1$ be a Galois Belyi map
  %        and let $\CC(X)$ be the function field of $X$.
  %        Then the field extension
  %        $\CC(X)/\CC(\PP^1)$ is Galois.
  %      \end{prop}
  %      \begin{proof}
  %      \end{proof}
  %      \begin{definition}\label{def:ramificationtype}
  %        The ramification
  %        of a degree $d$ Belyi map $\phi$
  %        can be encoded with $3$ partitions of $d$
  %        denoted $(\lambda_0,\lambda_1,\lambda_\infty)$.
  %        We call this triple of partitions
  %        the \defi{ramification type} of $\phi$.
  %        When $\phi$ is Galois,
  %        according to Lemma \ref{lem:regular},
  %        the ramification type of $\phi$ can more simply be encoded by
  %        a triple of integers $(a,b,c)\in\ZZ_{\geq 1}^3$.
  %      \end{definition}
  %      Let $\phi\colon X\to\PP^1$ be a Belyi map of degree $d$.
  %      Once we label the sheets of the cover
  %      and pick a basepoint $\star\not\in\{0,1,\infty\}$,
  %      we obtain a homomorphism
  %      \begin{equation}\label{eqn:monodromy}
  %        h\colon\pi_1(\PP^1\setminus\{0,1,\infty\},\star)\to S_d
  %      \end{equation}
  %      by lifting paths around the branch points of $\phi$.
  %      \begin{definition}\label{def:monodromy}
  %        The image of $h$ in Equation \ref{eqn:monodromy}
  %        is the \defi{monodromy group} of $\phi$
  %        denoted $\Mon(\phi)$.
  %        When $\phi$ is a Galois Belyi map,
  %        we can identify $\Mon(\phi)$
  %        as the Galois group
  %        $\Gal(\CC(X)/\CC(\PP^1))$.
  %        For this reason,
  %        we may also write $\Gal(\phi)$
  %        to denote $\Mon(\phi)$ when $\phi$
  %        is Galois.
  %      \end{definition}
  %      \mm{todo: any propositions about monodromy groups can go here}
  %      \begin{definition}\label{def:Gbelyi}
  %        A \defi{$G$-Galois Belyi map}
  %        is a Galois Belyi map
  %        $\phi\colon X\to\PP^1$
  %        with monodromy group $G$
  %        equipped with an isomorphism
  %        \[
  %          i\colon G\stackrel{\sim}{\to}\Mon(\phi)\leq\Aut(X).
  %        \]
  %        An \defi{isomorphism of $G$-Galois Belyi maps}
  %        $(\phi\colon X\to\PP^1, i\colon G\to\Mon(\phi))$
  %        and
  %        $(\phi'\colon X'\to\PP^1, i'\colon G\to\Mon(\phi)$
  %        is an isomorphism $h\colon X\stackrel{\sim}{\to} X'$ such that
  %        for all $g\in G$
  %        the diagram in
  %        Figure \ref{fig:Gbelyiiso} commutes.
  %        \begin{figure}[ht]
  %          \begin{center}
  %            \begin{tikzcd}
  %              X\arrow{rr}{h}\arrow{d}[swap]{i(g)}&&X'\arrow{d}{i'(g)}\\
  %              X\arrow{rr}{h}\arrow{dr}[swap]{\phi}&&X'\arrow{dl}{\phi'}\\
  %                                                     &\PP^1
  %            \end{tikzcd}
  %          \end{center}
  %          \caption{$G$-Galois Belyi map isomorphism}
  %          \label{fig:Gbelyiiso}
  %        \end{figure}
  %        \begin{prop}\label{prop:Gauts}
  %          \mm{\cite[Prop. 3.6 ish]{triangles}}
  %        \end{prop}
  %      \end{definition}
  %    }
  %    \subsection{Permutation triples and passports}{\label{subsec:passports}
  %      % A \defi{(nice) curve} over $K$ is
  %      % a smooth, projective, geometrically connected (irreducible) scheme of finite
  %      % type over $K$ that is pure of dimension $1$.
  %      % After extension to $\CC$, a curve
  %      % may be thought of as a compact, connected Riemann surface.  A \defi{Belyi map}
  %      % over $K$ is a finite morphism $\phi\colon X \to \PP^1$ over $K$ that is
  %      % unramified outside $\{0,1,\infty\}$; we will sometimes write $(X,\phi)$ when we
  %      % want to pay special attention to the source curve $X$.  Two Belyi maps
  %      % $\phi,\phi'$ are \defi{isomorphic} if there is an isomorphism $\iota\colon X
  %      % \xrightarrow{\sim} X'$ of curves such that $\phi'\iota=\phi$.
  %      % Let $\phi\colon X\to\PP^1$ be a Belyi map over $\overline{\QQ}$ of degree
  %      % $d \in \ZZ_{\geq 1}$.
  %      % The \defi{monodromy group} of $\phi$ is the Galois group
  %      % $\Mon(\phi) \colonequals \Gal(\CC(X)\,|\,\CC(\PP^1)) \leq S_d$ of the
  %      % corresponding extension of function fields (understood as the action of the
  %      % automorphism group of the normal closure); the group $\Mon(\phi)$ may also be
  %      % obtained by lifting paths around $0,1,\infty$ to $X$.
  %      \begin{definition}
  %        \label{def:permutationtriple}
  %        A \defi{permutation triple} of degree $d \in \ZZ_{\geq 1}$ is a tuple $\sigma =
  %        (\sigma_0,\sigma_1,\sigma_\infty)\in S_d^3$ such that $\sigma_\infty \sigma_1
  %        \sigma_0 = 1$.
  %        A permutation triple is \defi{transitive} if the subgroup
  %        $\langle \sigma \rangle \leq S_d$ generated by $\sigma$ is transitive.
  %        We say
  %        that two permutation triples $\sigma,\sigma'$ are \defi{simultaneously
  %        conjugate} if there exists $\tau\in S_d$ such that
  %        \begin{equation}\label{eqn:simconj}
  %          \sigma^\tau \colonequals
  %          (\tau^{-1}\sigma_0\tau, \tau^{-1}\sigma_1\tau, \tau^{-1}\sigma_\infty\tau)
  %          = \left(\sigma'_0,\sigma'_1,\sigma'_\infty\right)
  %          = \sigma'.
  %        \end{equation}
  %        An \defi{automorphism} of a permutation triple $\sigma$ is an element of $S_d$ that
  %        simultaneously conjugates $\sigma$ to itself, i.e.,
  %        $\Aut(\sigma)=Z_{S_d}(\langle \sigma \rangle)$, the centralizer inside $S_d$.
  %      \end{definition}
  %      \begin{lemma}
  %        \label{lem:simulisom}
  %        The set of transitive permutation triples of degree $d$ up to simultaneous
  %        conjugation is in bijection with the set of Belyi maps of degree $d$ up to
  %        isomorphism.
  %      \end{lemma}
  %      \begin{proof}
  %        The correspondence is via monodromy \cite[Lemma 1.1]{KMSV}; in particular,
  %        the monodromy group of a Belyi map is (conjugate in $S_d$ to) the group
  %        generated by~$\sigma$.
  %      \end{proof}
  %      The group $G_\QQ\colonequals\Gal(\QQal/ \QQ)$ acts on Belyi maps by acting on the
  %      coefficients of a set of defining equations; under the bijection of Lemma
  %      \ref{lem:simulisom}, it thereby acts on the set of transitive permutation
  %      triples, but this action is rather mysterious.
  %      We can cut this action down to size by identifying some basic invariants, as
  %      follows.
  %      \begin{definition}
  %        \label{def:passport}
  %        A \defi{passport} consists of the data $\mathcal{P}=(g,G,\lambda)$
  %        where $g \geq 0$ is an integer, $G \leq S_d$ is a transitive subgroup, and
  %        $\lambda=(\lambda_0,\lambda_1,\lambda_\infty)$ is a tuple of partitions
  %        $\lambda_s$ of $d$ for $s=0,1,\infty$.
  %        These partitions will be also be
  %        thought of as a tuple of conjugacy classes $C=(C_0,C_1,C_\infty)$ by cycle
  %        type, so we will also write passports as $(g,G,C)$.
  %      \end{definition}
  %      \begin{definition}
  %        \label{def:passportofbelyimap}
  %        The \defi{passport} of a
  %        Belyi map $\phi\colon X \to \PP^1$ is $(g(X),\Mon(\phi),
  %        (\lambda_0,\lambda_1,\lambda_\infty))$,
  %        where $g(X)$ is the genus of $X$ and
  %        $\lambda_s$ is the partition of $d$ obtained by the ramification degrees above
  %        $s=0,1,\infty$, respectively.
  %      \end{definition}
  %      \begin{definition}
  %        \label{def:passportofpermutationtriple}
  %        The \defi{passport} of a transitive
  %        permutation triple $\sigma$ is
  %        $(g(\sigma),\langle \sigma \rangle, \lambda(\sigma))$,
  %        where (by Riemann--Hurwitz)
  %        \begin{equation}\label{eqn:riemannhurwitzfortriples}
  %          g(\sigma) \colonequals 1-d+(e(\sigma_0)+e(\sigma_1)+e(\sigma_\infty))/2
  %        \end{equation}
  %        and $e$ is the index of a permutation ($d$ minus the number of orbits), and
  %        $\lambda(\sigma)$ is the cycle type of $\sigma_s$ for $s=0,1,\infty$.
  %      \end{definition}
  %      \begin{definition}
  %        \label{def:passportsize}
  %        The
  %        \defi{size} of a passport $\mathcal{P}$ is the number of simultaneous conjugacy
  %        classes (as in \ref{eqn:simconj}) of (necessarily transitive) permutation
  %        triples $\sigma$ with passport $\mathcal{P}$.
  %      \end{definition}
  %      The action of $G_\QQ$ on Belyi maps preserves passports.
  %      Therefore, after computing equations for all Belyi maps with a given
  %      passport, we can try to identify the Galois orbits of this action.
  %      \begin{definition}
  %        \label{def:reduciblepassport}
  %        We say a passport is \defi{irreducible} if it has one
  %        $G_\QQ$-orbit and
  %        \defi{reducible} otherwise.
  %      \end{definition}
  %    }
  %    \subsection{Triangle groups}{\label{subsec:trianglegroups}
  %      \begin{definition}
  %        \label{def:geometrytype}
  %        Let $(a,b,c)\in\ZZ_{\geq 1}^3$.
  %        If $1\in(a,b,c)$, then we say the triple is \defi{degenerate}.
  %        Otherwise, we call the triple
  %        \defi{spherical},
  %        \defi{Euclidean},
  %        or \defi{hyperbolic}
  %        according to whether the value of
  %        \begin{equation}
  %          \label{eqn:eulerchar}
  %          \chi(a,b,c) = 1-\frac{1}{a}-\frac{1}{b}-\frac{1}{c}
  %        \end{equation}
  %        is negative, zero, or positive.
  %        We call this the \defi{geometry type}
  %        of the triple.
  %        We associate the \defi{geometry}
  %        \begin{equation}
  %          \label{eqn:geometrytype}
  %          H=
  %          \begin{cases}
  %            \PP^1&\chi(a,b,c)<0\\
  %            \CC&\chi(a,b,c)=0\\
  %            \mathfrak{H}&\chi(a,b,c)<0
  %          \end{cases}
  %        \end{equation}
  %        where $\mathfrak{H}$ denotes the complex upper half-plane.
  %      \end{definition}
  %      \begin{definition}
  %        \label{def:trianglegroup}
  %        For each triple $(a,b,c)$ in Definition \ref{def:geometrytype}
  %        we define the \defi{triangle group}
  %        \begin{equation}
  %          \label{eqn:trianglegroup}
  %          \Delta(a,b,c)
  %          =
  %          \langle
  %          \delta_a, \delta_b, \delta_c |
  %          \delta_a^a=\delta_b^b=\delta_c^c=\delta_c\delta_b\delta_a=1
  %          \rangle
  %        \end{equation}
  %        The \defi{geometry type}
  %        of a triangle group $\Delta(a,b,c)$
  %        is the geometry type of the triple $(a,b,c)$.
  %      \end{definition}
  %      \begin{definition}\label{def:geometrytypeofbelyimap}
  %        The \defi{geometry type} of a Galois Belyi map
  %        with ramification type $(a,b,c)$
  %        is the geometry type of $(a,b,c)$.
  %      \end{definition}
  %      \begin{definition}\label{def:geometrytypeofpermutationtriple}
  %        Let $\sigma=(\sigma_0,\sigma_1,\sigma_\infty)$ be a transitive permutation triple.
  %        Let $a,b,c$ be the orders of
  %        $\sigma_0,\sigma_1,\sigma_\infty$ respectively.
  %        The \defi{geometry type} of $\sigma$
  %        is the geometry type of $(a,b,c)$.
  %      \end{definition}
  %      The connection between Belyi maps and triangle groups
  %      of various geometry types is explained by Lemma
  %      \ref{lem:belyimapsandtrianglegroups}.
  %      \begin{lemma}
  %        \label{lem:belyimapsandtrianglegroups}
  %        The set of isomorphism classes of of degree $d$
  %        Belyi maps with ramification type $(a,b,c)$
  %        is in bijection with the set of
  %        index $d$ subgroups
  %        $\Gamma\leq\Delta(a,b,c)$
  %        up to isomorphism.
  %      \end{lemma}
  %      \begin{proof}
  %        See \cite{KMSV} for a detailed discussion.
  %      \end{proof}
  %    }
  %    \subsection{Background results on Belyi maps}{\label{subsec:backgroundresults}
  %      \begin{theorem}\label{thm:bigbijection}
  %        \mm{big bijection}
  %      \end{theorem}
  %      \begin{prop}\label{prop:galoisaction}
  %        \mm{Galois action on Belyi maps}
  %      \end{prop}
  %      \begin{prop}\label{prop:galoiscorrespondence}
  %        Galois correspondence of Belyi maps
  %      \end{prop}
  %      \begin{proof}
  %      \end{proof}
  %      \mm{\cite[1.6, 1.7]{SV}}
  %    }
  %    \subsection{Fields of moduli and fields of definition}{\label{subsec:fieldsofmodulifieldsofdefinition}
  %      Let $\Aut(\CC)$ denote the field automorphisms of $\CC$.
  %      \begin{definition}
  %        \label{def:fieldofmoduli}
  %        Let $X$ be an algebraic curve over $\CC$.
  %        The \defi{field of moduli} of $X$ is the fixed field of the
  %        field automorphisms
  %        \[
  %          \{\tau\in\Aut(\CC) : X^\tau\cong X\}
  %        \]
  %        where $\tau\in\Aut(\CC)$ acts on the defining equations of $X$.
  %        Denote this field as $M(X)$.
  %      \end{definition}
  %      \begin{definition}
  %        \label{def:fieldofmodulibelyimap}
  %        Let $\phi\colon X\to\PP^1$ be a Belyi map.
  %        The \defi{field of moduli} of $\phi$ is the fixed field of the
  %        field automorphisms
  %        \[
  %          \{\tau\in\Aut(\CC) : \phi^\tau\cong \phi\}
  %        \]
  %        where $\tau\in\Aut(\CC)$ acts on the defining equations of $\phi$
  %        and isomorphism is determined by
  %        Definition \ref{def:belyiiso}.
  %        Denote this field as $M(\phi)$.
  %      \end{definition}
  %      \begin{definition}
  %        \label{def:fieldofmoduliGbelyimap}
  %        Let $\phi\colon X\to\PP^1$ be a $G$-Galois Belyi map.
  %        The \defi{field of moduli} of $\phi$ is the fixed field of the
  %        field automorphisms
  %        \[
  %          \{\tau\in\Aut(\CC) : \phi^\tau\cong \phi\}
  %        \]
  %        where $\tau\in\Aut(\CC)$ acts on the defining equations of $\phi$
  %        and isomorphism is determined by
  %        Definition \ref{def:Gbelyi}.
  %        Denote this field as $M(\phi)$.
  %      \end{definition}
  %      \begin{theorem}\label{thm:fieldofmoduli}
  %        Let $\phi:X\to\PP^1$ be a Belyi map
  %        with passport $\mathcal{P}$.
  %        Then the degree of the field of moduli of $\phi$
  %        is bounded by the size of $\mathcal{P}$.
  %      \end{theorem}
  %      \begin{proof}
  %        \cite{SV}
  %      \end{proof}
  %      \begin{definition}
  %        \label{def:fieldofdefinition}
  %        Let $\phi\colon X\to\PP^1$ be a Belyi map.
  %        A number field $K$ is a
  %        \defi{field of definition} for $\phi$
  %        if $\phi$ and $X$ can be defined
  %        with equations over $K$.
  %        If $K$ is a field of definition for $\phi$
  %        we say $\phi$ is \defi{defined over} $K$.
  %      \end{definition}
  %      \begin{theorem}
  %        \label{thm:galoisbelyimapoverfieldofmoduli}
  %        A Galois Belyi map is defined over
  %        its field of moduli.
  %      \end{theorem}
  %      \begin{proof}
  %        \cite[Lemma 4.1]{triangles}
  %      \end{proof}
  %    }
  %  }
  %  \section{Group theory}{
  %    \subsection{Central group extensions and $H^2(G,A)$}{
  %      \begin{definition}
  %        \label{def:isoextentions}
  %      \end{definition}
  %    }
  %    \subsection{Holt's algorithm and \texttt{Magma} implementation}{
  %      \label{subsec:holtsalgorithm}
  %    }
  %    \subsection{Results on $2$-groups}{
  %      \begin{lemma}\label{lem:2groupfiltration}
  %        \mm{todo}
  %      \end{lemma}
  %    }
  %  }
  %  % \section{Jacobians of curves}{
  %  %   \subsection{Abel-Jacobi and the construction over $\CC$}{
  %  %   }
  %  %   \subsection{Algebraic construction}{
  %  %   }
  %  %   \subsection{Riemann-Roch}{
  %  %   }
  %  %   \subsection{Torsion points and torsion fields}{
  %  %   }
  %  % }
  %  % \section{Galois representations}{
  %  %   \subsection{Representations of Galois groups of number fields}{
  %  %   }
  %  %   \subsection{Representations coming from geometry}{
  %  %   }
  %  % }
%}
% \chapter{A database of $2$-group Belyi maps}{\label{chapter:database}
  %In this chapter we describe an algorithm
  %to generate $2$-group Belyi maps of a given degree.
  %The algorithm is inductive in the degree.
  %The base case in degree $1$ is discussed in Section \ref{sec:degree1}.
  %We then move on to describe the inductive step of the algorithm
  %which we describe in two parts.
  %First we discuss the algorithm to enumerate the isomorphism classes
  %using permutation triples in Section \ref{sec:isoclasses}.
  %For a discussion on the relationship between permutation triples and Belyi maps
  %see Section \ref{sec:belyimaps}.
  %Next we discuss the inductive step to produce Belyi curves and maps in Section \ref{sec:curvesandmaps}.
  %In Section \ref{sec:runtime} we give a detailed description of the running time of the algorithm.
  %Lastly,
  %in Section \ref{sec:computations},
  %we discuss the implementation and computations that we have carried out explicitly.
  %Recall the definition of a $G$-Galois Belyi map in Section \ref{sec:belyimaps}.
  %In this section we narrow our focus
  %to $G$-Galois Belyi maps with $\#G$ a power of $2$.
  %\begin{definition}\label{def:2groupbelyi}
  %  A \defi{$2$-group Belyi map} is a Galois Belyi map
  %  with monodromy group a $2$-group.
  %\end{definition}
  %\section{Degree $1$ Belyi maps}{\label{sec:degree1}
  %}
  %%\section{An algorithm to enumerate isomorphism classes of $2$-group Belyi maps}{\label{sec:isoclasses}
  %%  The algorithm we describe here is iterative.
  %%  The degree $1$ case is discussed in Section \ref{sec:degree1}.
  %%  We now set up some notation for the iteration.
  %%  \begin{notation}\label{not:triples}
  %%    First we suppose that we are given
  %%    $\sigma$ a permutation triple corresponding
  %%    to a $2$-group Belyi map $\phi:X\to\PP^1$.
  %%    \begin{definition}
  %%      \label{def:lift}
  %%      We say that a permutation triple $\wt{\sigma}$
  %%      is a \defi{degree $2$ lift} (or simply a \defi{lift})
  %%      of a permutation triple $\sigma$ if
  %%      there exists a short exact sequence of groups
  %%      as in Figure \ref{fig:lifttriple}
  %%      with $\iota(\ZZ/2\ZZ)$ contained in the center of
  %%      $\langle\wt{\sigma}\rangle$.
  %%    \end{definition}
  %%    \begin{figure}[ht]\label{fig:lifttriple}
  %%      \[
  %%        \begin{tikzcd}[ampersand replacement=\&, row sep = small]
  %%          1\arrow{r}\&\ZZ/2\ZZ\arrow{r}{\iota}\&\langle\widetilde{\sigma}\rangle\arrow{r}{\pi}\&\langle\sigma\rangle\arrow{r}\&1
  %%        \end{tikzcd}
  %%      \]
  %%      \caption{$\wt{\sigma}$ a lift of $\sigma$}
  %%    \end{figure}
  %%    In Algorithm \ref{alg:triples} below
  %%    we describe how to determine all lifts $\wt{\sigma}$ (up to isomorphism)
  %%    of a given permutation triple $\sigma$.
  %%    \begin{lemma}
  %%      \label{lem:central}
  %%      Let $\sigma$ be a permutation triple corresponding to a $2$-group Belyi map
  %%      $\phi:X\to\PP^1$ and $\wt{\sigma}$ a lift of $\sigma$
  %%      corresponding to a $2$-group Belyi map $\wt{\phi}:\wt{X}\to\PP^1$.
  %%      Then there exists a permutation triple $\wt{\sigma}'$
  %%      that is simultaneously conjugate to $\wt{\sigma}$ with
  %%      $\iota(\langle\wt{\sigma}'\rangle)$ contained in the center of
  %%      $\langle\sigma\rangle$.
  %%    \end{lemma}
  %%    \begin{proof}
  %%    \end{proof}
  %%    \begin{remark}
  %%      \label{rmk:central}
  %%      In light of Lemma \ref{lem:central},
  %%      we can restrict our attention to central extensions of
  %%      $\langle\sigma\rangle$
  %%      in Definition \ref{def:lift}.
  %%    \end{remark}
  %%  \end{notation}
  %%  \begin{alg}\label{alg:triples}
  %%    Let the notation be as described above in \ref{not:triples}.
  %%    \newline
  %%    \textbf{Input}: $\sigma=(\sigma_0,\sigma_1,\sigma_\infty)\in S_d^3$ a permutation triple
  %%    corresponding to a $2$-group Belyi map
  %%    \newline
  %%    \textbf{Output}: all lifts $\wt{\sigma}$
  %%    of $\sigma$ up to simultaneous conjugation in $S_{2d}$
  %%    \begin{enumerate}
  %%      \item
  %%        Let $G = \langle\sigma\rangle$
  %%        and compute all central extensions $\wt{G}$
  %%        sitting in the exact sequence in Figure \ref{fig:centralext}
  %%        up to isomorphism (see Definition \ref{def:isoextentions}).
  %%        For more information about the algorithms to do this
  %%        see Section \ref{subsec:holtsalgorithm}.
  %%        \begin{figure}[ht]
  %%          \[
  %%            \begin{tikzcd}[ampersand replacement=\&, row sep = small]
  %%              1\arrow{r}\&\ZZ/2\ZZ\arrow{r}{\iota}\&\wt{G}\arrow{r}{\pi}\&G\arrow{r}\&1
  %%            \end{tikzcd}
  %%          \]
  %%          \caption{$\wt{G}$ a (central) extension of $G$}
  %%          \label{fig:centralext}
  %%        \end{figure}
  %%      \item
  %%        For each extension $\wt{G}$ as in Figure \ref{fig:centralext}
  %%        from the previous step
  %%        we perform the following:
  %%        \begin{enumerate}
  %%          \item
  %%            % For $s\in\{0,1,\infty\}$,
  %%            % $\sigma_s$ has $2$ preimages under the map $\pi$
  %%            % which yields $8$ triples $\wt{\sigma} = (\wt{\sigma}_0,\wt{\sigma}_1,\wt{\sigma}_\infty)$
  %%            % with the property that $\pi(\wt{\sigma}_s) = \sigma_s$ for each $s$.
  %%            Consider the set of triples
  %%            \begin{equation}
  %%              \label{eqn:lifts}
  %%              \{
  %%                \wt{\sigma}\colonequals(\wt{\sigma}_0, \wt{\sigma}_1, \wt{\sigma}_\infty) :
  %%                \wt{\sigma}_s\in\pi^{-1}(\sigma_s)\text{ for }
  %%                s\in\{0,1,\infty\}
  %%              \}
  %%            \end{equation}
  %%            and let $\Lifts(\sigma)$ denote the set of such $\wt{\sigma}$
  %%            with the property that $\wt{\sigma}_\infty\wt{\sigma}_1\wt{\sigma}_0 = 1$
  %%            and $\langle\wt{\sigma}\rangle = \wt{G}$.
  %%          \item
  %%            For each $\wt{\sigma}\in\Lifts(\sigma)$ compute
  %%            $\order(\wt{\sigma})\colonequals(\order(\wt{\sigma}_0), \order(\wt{\sigma}_1), \order(\wt{\sigma}_\infty))\in\ZZ^3$
  %%            and sort $\Lifts(\sigma)$ according to $\order(\wt{\sigma})$.
  %%            Let
  %%            \begin{equation}
  %%              \label{eqn:liftsorders}
  %%              \Lifts(\sigma,(a,b,c))\colonequals
  %%              \{
  %%                \wt{\sigma}\in\Lifts(\sigma) :
  %%                \order(\wt{\sigma}) = (a,b,c)
  %%              \}.
  %%            \end{equation}
  %%          \item
  %%            For each set of triples $\Lifts(\sigma,(a,b,c))$
  %%            remove simultaneously conjugate triples so that
  %%            $\Lifts(\sigma,(a,b,c))$ has exactly one representative
  %%            from each simultaneous conjugacy class.
  %%            \mm{TODO: reword}
  %%        \end{enumerate}
  %%      \item
  %%        Return the union of the sets $\Lifts(\sigma,(a,b,c))$
  %%        ranging over all extensions as in Figure
  %%        \ref{fig:centralext}
  %%        and for each extension ranging over all orders
  %%        $(a,b,c)$.
  %%    \end{enumerate}
  %%  \end{alg}
  %%  \begin{proof}[Proof of correctness]
  %%    The algorithms in Step 1 are addressed in Section \ref{subsec:holtsalgorithm}.
  %%    Let $\phi:X\to\PP^1$ be the $2$-group Belyi map corresponding to $\sigma$.
  %%    By Proposition \ref{prop:galoiscorrespondence},
  %%    the groups obtained from Step 1 are precisely the
  %%    groups that can occur as monodromy groups of degree $2$ covers of $X$.
  %%    \mm{
  %%      lemma in section about extensions
  %%      (or in background about Belyi maps)
  %%      to prove
  %%      that two isomorphic extensions cannot produce
  %%      nonisomorphic Belyi maps
  %%      and that two nonisomorphic extensions cannot produce
  %%      isomorphic Belyi maps
  %%    }
  %%    In Step 2 we restrict our attention to a single extension of $G$
  %%    as in Figure \ref{fig:centralext}.
  %%    When we pullback a triple $\sigma$ under the map $\pi$,
  %%    there are $2^3=8$ preimages $\wt{\sigma}$.
  %%    Of these $8$ preimages, exactly $4$ have the property that
  %%    $\wt{\sigma}_\infty\wt{\sigma}_1\wt{\sigma}_0=1$.
  %%    Of these $4$ triples, we only take those that generate $\wt{G}$
  %%    and this makes up the set $\Lifts(\sigma)$.
  %%    In Step 2(b),
  %%    we are sorting $\Lifts(\sigma)$ by passport.
  %%    Since $2$-group Belyi maps are Galois,
  %%    the cycle structure of each $\wt{\sigma}_s\in\wt{\sigma}$
  %%    is determined by the order of $\wt{\sigma}_s$
  %%    so that sorting by order is the same as sorting by cycle structure.
  %%    \begin{remark}\label{rmk:twouniquepassports}
  %%      In fact,
  %%      even though we do not need this for the algorithm,
  %%      there are at most $2$ different passports that can occur
  %%      in $\Lifts(\sigma)$.
  %%      $2$ different passports occur when one of $\sigma_s\in\sigma$
  %%      is the identity.
  %%      If $\sigma$ does not contain an identity element,
  %%      then all triples in $\Lifts(\sigma)$ have the same passport.
  %%    \end{remark}
  %%    At this point,
  %%    we have constructed the sets $\Lifts(\sigma,(a,b,c))$.
  %%    In light of Remark \ref{rmk:twouniquepassports},
  %%    there are only $2$ possibilities:
  %%    \begin{itemize}
  %%      \item
  %%        There is only one such set $\Lifts(\sigma, (a,b,c))$
  %%        consisting of at most $4$ triples.
  %%      \item
  %%        There are $2$ sets $\Lifts(\sigma, (a,b,c))$ and $\Lifts(\sigma, (a',b',c'))$
  %%        each consisting of at most $2$ triples.
  %%    \end{itemize}
  %%    Step 2(c) is to eliminate simultaneous conjugation in each
  %%    set $\Lifts(\sigma, (a,b,c))$.
  %%    After Step 2(c) is complete,
  %%    the sets $\Lifts(\sigma, (a,b,c))$ contain exactly one permutation triple
  %%    for each isomorphism class of $2$-group Belyi map with passport determined by
  %%    $(a,b,c)$ and monodromy group $\wt{G}$ such that the diagram in
  %%    Figure \ref{fig:liftbelyimap} commutes.
  %%    \begin{figure}[ht]
  %%      \[
  %%        \begin{tikzcd}[ampersand replacement=\&]
  %%          \widetilde{X}\arrow{dd}[swap]{\widetilde{\phi}}\arrow{dr}{2}\\
  %%          \&X\arrow{dl}{\phi}\\
  %%          \PP^1
  %%        \end{tikzcd}
  %%      \]
  %%      \caption{
  %%        The permutation triples $\wt{\sigma}$ constructed in
  %%        Algorithm \ref{alg:triples}
  %%        correspond to Belyi maps $\wt{\phi}:\wt{X}\to\PP^1$
  %%        in the above diagram.
  %%      }
  %%      \label{fig:liftbelyimap}
  %%    \end{figure}
  %%    In Step 3 we collect together all sets $\Lifts(\sigma, (a,b,c))$
  %%    as we range over all possible extensions in Step 1,
  %%    and by the discussion for Step 2 yields the desired output.
  %%  \end{proof}
  %%  We now illustrate Algorithm \ref{alg:triples}
  %%  with the following example.
  %%  \begin{example}\label{exm:lift}
  %%    In this example we carry out Algorithm \ref{alg:triples} for
  %%    the degree $2$ permutation triple
  %%    $\sigma = ((1\,2),(1)(2),(1\,2))$.
  %%    Here $G = \langle\sigma\rangle\cong\ZZ/2\ZZ$.
  %%    In Step 1,
  %%    we obtain two group extensions
  %%    $\wt{G}_1\cong\ZZ/2\ZZ\times\ZZ/2\ZZ$
  %%    and
  %%    $\wt{G}_2\cong\ZZ/4\ZZ$:
  %%    \begin{figure}[ht]
  %%      \[
  %%        \begin{tikzcd}[ampersand replacement=\&, row sep = small]
  %%          1\arrow{r}\&\ZZ/2\ZZ\arrow{r}{\iota_1}\&\wt{G}_1\arrow{r}{\pi_1}\&G\arrow{r}\&1
  %%        \end{tikzcd}
  %%      \]
  %%      \[
  %%        \begin{tikzcd}[ampersand replacement=\&, row sep = small]
  %%          1\arrow{r}\&\ZZ/2\ZZ\arrow{r}{\iota_2}\&\wt{G}_2\arrow{r}{\pi_2}\&G\arrow{r}\&1
  %%        \end{tikzcd}
  %%      \]
  %%      \caption{Two extensions of $G$ in Example \ref{exm:lift}}
  %%      \label{fig:liftexampleextensions}
  %%    \end{figure}
  %%    We will consider the two extensions separately:
  %%    \begin{itemize}
  %%      \item
  %%        For $\wt{G}_1$, we have
  %%        \begin{align*}
  %%          \Lifts(\sigma) =
  %%          &\Big\{
  %%            ((1\,2)(3\,4), (1)(2)(3)(4), (1\,2)(3\,4)),
  %%            ((1\,2)(3\,4), (1\,3)(2\,4), (1\,4)(2\,3)),\\
  %%          &((1\,4)(2\,3), (1)(2)(3)(4), (1\,4)(2\,3)),
  %%            ((1\,4)(2\,3), (1\,3)(2\,4), (1\,2)(3\,4))
  %%          \Big\}
  %%        \end{align*}
  %%        Before we continue with the algorithm,
  %%        let us take a moment to explain this more closely in the following remark.
  %%        \begin{remark}\label{rem:actiononblocks}
  %%          First, note that the image of $\iota_1$ is an order $2$ subgroup of
  %%          $\wt{G}_1$.
  %%          Let $\tau\in\wt{G}_1$ denote the generator of this image.
  %%          From the perspective of branched covers,
  %%          $\tau$ is identifying $4$ sheets in a degree $4$ cover down to $2$ sheets
  %%          in a degree $2$ cover.
  %%          Elements $\wt{\sigma}$ of $\Lifts(\sigma)$ must induce a well-defined action on the
  %%          identified sheets and this action must be compatible with $\sigma$.
  %%          In this example $\tau = (1\,3)(2\,4)$ meaning that $\tau$
  %%          identifies the sheets labeled $1$ and $3$ into a single sheet
  %%          and $\tau$ identifies the sheets labeled $2$ and $4$ into a single sheet.
  %%          Another way of saying that $\wt{\sigma}$ induces a well-defined
  %%          action is that $\wt{\sigma}$ acts on the blocks
  %%          $\left\{\boxed{1\,3},\boxed{2\,4}\right\}$.
  %%          Saying that this action is compatible with $\sigma$ means that for each
  %%          $s\in\{0,1,\infty\}$ the induced action of $\wt{\sigma}_s$ on blocks
  %%          is the same as $\sigma_s$.
  %%          For
  %%          \[
  %%            \wt{\sigma} = ((1\,2)(3\,4), (1\,3)(2\,4), (1\,4)(2\,4))
  %%          \]
  %%          we have
  %%          $\wt{\sigma}_0\boxed{1\,3} = \boxed{2\,4}$
  %%          and
  %%          $\wt{\sigma}_0\boxed{2\,4} = \boxed{1\,3}$
  %%          so that the induced permutation of blocks is
  %%          \[
  %%            \left(\boxed{1\,3},\boxed{2\,4}\right)
  %%          \]
  %%          which is the same as the permutation $\sigma_0 = (1\,2)$
  %%          (as long as we identity $\boxed{1\,3}$ with $1$ and $\boxed{2\,4}$ with $2$).
  %%        \end{remark}
  %%        To finish Step 2(a)
  %%        we only take triples in $\Lifts(\sigma)$ that generate $\wt{G}_1$,
  %%        so at the end of Step 2(a) for this extension we have
  %%        \begin{align*}
  %%          \Lifts(\sigma) =
  %%          \Big\{
  %%            ((1\,2)(3\,4), (1\,3)(2\,4), (1\,4)(2\,3)),
  %%            ((1\,4)(2\,3), (1\,3)(2\,4), (1\,2)(3\,4))
  %%          \Big\}.
  %%        \end{align*}
  %%        In Step 2(b) we sort $\Lifts(\sigma)$ into passports as determined by orders of elements.
  %%        Here, all $\wt{\sigma}\in\Lifts(\sigma)$ have the same orders
  %%        (and hence belong to the same passport).
  %%        Thus we get a single set $\Lifts(\sigma,(2,2,2)) = \Lifts(\sigma)$.
  %%        Lastly,
  %%        in Step 2(c) we see that that the two triples in
  %%        $\Lifts(\sigma, (2,2,2))$ are simultaneously conjugate
  %%        (by the permutation $(2\,4)$)
  %%        and hence we remove one of the triples from $\Lifts(\sigma, (2,2,2))$.
  %%      \item
  %%        For $\wt{G}_2$, we have
  %%        \begin{align*}
  %%          \Lifts(\sigma) =
  %%          &\Big\{
  %%            ((1\,4\,3\,2), (1)(2)(3)(4), (1\,2\,3\,4)),
  %%            ((1\,2\,3\,4), (1\,3)(2\,4), (1\,2\,3\,4)),\\
  %%          &((1\,2\,3\,4),  (1)(2)(3)(4), (1\,4\,3\,2)),
  %%            ((1\,4\,3\,2), (1\,3)(2\,4), (1\,4\,3\,2))
  %%          \Big\}
  %%        \end{align*}
  %%        All $4$ of the above triples in $\Lifts(\sigma)$
  %%        generate $\wt{G}_2$, so we continue to Step 2(b) with
  %%        $\#\Lifts(\sigma) = 4$.
  %%        In Step 2(b),
  %%        we sort $\Lifts(\sigma)$ into two sets
  %%        $\Lifts(\sigma, (4,1,4))$ and $\Lifts(\sigma,(4,2,4))$
  %%        each containing $2$ triples.
  %%        In Step 2(c),
  %%        we find that the $2$ triples in $\Lifts(\sigma, (4,1,4))$
  %%        are simultaneously conjugate
  %%        (by the permutation $(2\,4)$)
  %%        and the $2$ triples in $\Lifts(\sigma, (4,2,4))$
  %%        are simultaneously conjugate
  %%        (also by the permutation $(2\,4)$),
  %%        so we remove one permutation triple from each of these sets
  %%        so that $\Lifts(\sigma, (4,1,4))$
  %%        and
  %%        $\Lifts(\sigma, (4,2,4))$ both have cardinality $1$.
  %%    \end{itemize}
  %%    In Step 3,
  %%    we return
  %%    \[
  %%      \Lifts(\sigma, (2,2,2))\cup\Lifts(\sigma, (4,1,4))\cup\Lifts(\sigma, (4,2,4))
  %%    \]
  %%    which is a set of $3$ permutation triples
  %%    each corresponding to an isomorphism class of $2$-group Belyi map
  %%    as in Figure \ref{fig:liftbelyimap}.
  %%  \end{example}
  %%  Now that we have an algorithm to find all lifts of a single
  %%  permutation triple,
  %%  the next step is to describe how to use this
  %%  to organize all isomorpism classes of $2$-group Belyi maps
  %%  of a given degree.
  %%  \begin{alg}\label{alg:alltriples}
  %%    Let the notation be as described above in \ref{not:triples}.
  %%    \newline
  %%    \textbf{Input}: $d = 2^m$ for some positive integer $m$
  %%    \newline
  %%    \textbf{Output}: a sequence of bipartite graphs
  %%    $\mathscr{G}_2, \mathscr{G}_4,\dots,\mathscr{G}_{2^m}$
  %%    where the two sets of nodes of $\mathscr{G}_{2^i}$
  %%    are
  %%    \begin{itemize}
  %%      \item
  %%        $\mathscr{G}_{2^i}^\text{above}$ :
  %%        the set of isomorphism classes of $2$-group Belyi maps
  %%        of degree $2^i$ indexed by permutation triples $\wt{\sigma}$
  %%      \item
  %%        $\mathscr{G}_{2^i}^\text{below}$ :
  %%        the set of isomorphism classes of $2$-group Belyi maps
  %%        of degree $2^{i-1}$ indexed by permutation triples $\sigma$
  %%    \end{itemize}
  %%    and there is an edge between $\wt{\sigma}$ and $\sigma$
  %%    if and only if $\wt{\sigma}$ is a lift
  %%    (as in Definition \ref{def:lift})
  %%    of $\sigma$.
  %%    % Moreover,
  %%    % for each $i$,
  %%    % we have
  %%    % $\mathscr{G}_{2^{i-1}}^\text{above} = \mathscr{G}_{2^i}^\text{below}$.
  %%    This algorithm is iterative.
  %%    For each $i=1,\dots,m$,
  %%    we use $\mathscr{G}_{2^i}^\text{below}$ to
  %%    compute $\mathscr{G}_{2^i}^\text{above}$
  %%    and then we define
  %%    \[
  %%      \mathscr{G}_{2^{i+1}}^\text{below}\colonequals\mathscr{G}_{2^i}^\text{above}
  %%    \]
  %%    and continue the process.
  %%    \begin{enumerate}
  %%      \item
  %%        To begin the iteration we
  %%        let $\mathscr{G}_2^\text{below}$ = $\{\sigma\}$
  %%        where $\sigma = ((1),(1),(1))\in S_1^3$
  %%        corresponds to the degree $1$ Belyi map.
  %%      \item
  %%        Now suppose we have computed $\mathscr{G}_{2^i}^\text{below}$.
  %%        We compute $\mathscr{G}_{2^i}^\text{above}$ as follows:
  %%        \begin{enumerate}
  %%          \item
  %%            Apply Algorithm \ref{alg:triples} to every
  %%            $\sigma\in\mathscr{G}_{2^i}^\text{below}$
  %%            to obtain
  %%            $\#\mathscr{G}_{2^i}^\text{below}$
  %%            sets
  %%            $\Lifts(\sigma)$.
  %%            As a word of caution,
  %%            the notation $\Lifts(\sigma)$ has
  %%            a different meaning here than in
  %%            Algorithm \ref{alg:triples}.
  %%            Here $\Lifts(\sigma)$ is the set of
  %%            lifts of $\sigma$ up to simultaneous
  %%            conjugation.
  %%            Let
  %%            \[
  %%              \mathscr{G}_{2^i}^\text{above}
  %%              \colonequals
  %%              \bigcup_{\sigma\in\mathscr{G}_{2^i}^\text{below}}
  %%              \Lifts(\sigma)
  %%            \]
  %%            and place an edge 
  %%            of $\mathscr{G}_{2^i}$
  %%            between
  %%            $\wt{\sigma}\in\mathscr{G}_{2^i}^\text{above}$
  %%            and
  %%            $\sigma\in\mathscr{G}_{2^i}^\text{below}$
  %%            if and only if $\wt{\sigma}\in\Lifts(\sigma)$.
  %%          \item
  %%            Consider all pairs $(\wt{\sigma},\wt{\sigma}')\in\mathscr{G}_{2^i}^\text{above}$
  %%            and for each pair
  %%            test if $\wt{\sigma}$ is simultaneously conjugate to
  %%            $\wt{\sigma}'$ in $S_{2^i}$.
  %%            If the pair is simultaneously conjugate,
  %%            then combine the nodes $\wt{\sigma}$ and $\wt{\sigma}'$
  %%            into a single node (take either triple)
  %%            and combine the edge sets of $\wt{\sigma}$
  %%            and $\wt{\sigma}'$ to be the edge set of the new node.
  %%          \item
  %%            Return the resulting bipartite graph as
  %%            $\mathscr{G}_{2^i}$.
  %%          \item
  %%            If $i<m$, then let
  %%            $\mathscr{G}_{2^{i+1}}^\text{below}\colonequals\mathscr{G}_{2^i}^\text{above}$
  %%            and repeat Step 2 with $i+1$.
  %%            If $i=m$,
  %%            then return the sequence of bipartite graphs
  %%            $\mathscr{G}_2,\mathscr{G}_4,\dots,\mathscr{G}_{2^m}$.
  %%        \end{enumerate}
  %%    \end{enumerate}
  %%  \end{alg}
  %%  \begin{proof}[Proof of correctness]
  %%    % first use 2-groups and Galois correspondence to see we get all iso classes
  %%    We first address the claim that
  %%    every $2$-group Belyi map $\phi:X\to\PP^1$ of degree $2^i$
  %%    is represented by a permutation triple in
  %%    $\mathcal{G}_{2^i}^\text{above}$.
  %%    Let $G$ be the monodromy group of $\phi$.
  %%    Since $\#G=2^i$,
  %%    by Lemma \ref{lem:2groupfiltration},
  %%    there exists a normal tower of groups
  %%    \begin{equation}\label{eqn:2groupfiltration}
  %%      G_0\trianglelefteq G_1\trianglelefteq\dots
  %%      \trianglelefteq G_i
  %%    \end{equation}
  %%    where $G_0 = \{1\}$, $G_i = G$,
  %%    and each consecutive quotient is isomorphic
  %%    to $\ZZ/2\ZZ$.
  %%    By the Galois correspondence,
  %%    Proposition \ref{prop:galoiscorrespondence},
  %%    this normal tower of groups corresponds to
  %%    the diagram in Figure \ref{fig:2groupfiltration}.
  %%    \begin{figure}[ht]
  %%      \begin{center}
  %%        \begin{tikzpicture}[scale=0.3]
  %%          \node at (0,0) (X0) {$X_0 = \PP^1$};
  %%          \node [above left=of X0] (X1) {$X_1$};
  %%          \node [above =of X1] (X2) {$X_2$};
  %%          \node [above =of X2] (dots) {$\vdots$};
  %%          \node [above =of dots] (Xr) {$X_i=X$};
  %%          %\draw[<->] (X1) to node [above,rotate=-35] {$\sim$} node {} (X0);
  %%          \draw[->] (X1) to node [below] {$2$} node [right] {$\phi_1$} (X0);
  %%          \draw[->] (X2) to node [left] {$2$} node {} (X1);
  %%          \draw[->, bend left] (X2) to node [below, right] {$\phi_2$} node {} (X0);
  %%          \draw[->] (dots) to node [left] {$2$} node {} (X2);
  %%          \draw[->] (Xr) to node [left] {$2$} node {} (dots);
  %%          \draw[->, bend left] (Xr) to node [below, right] {$\phi_i=\phi$} node {} (X0);
  %%          %\node at (8,0) (KX0) {$K(X_1) = K(\PP^1)$};
  %%          %\node [above left=of KX0] (KX1) {$K(X_2)$};
  %%          %\node [above =of KX1] (KX2) {$K(X_3)$};
  %%          %\node [above =of KX2] (dots) {$\vdots$};
  %%          %\node [above =of dots] (KXr) {$K(X_r)$};
  %%          %%\draw[<->] (X1) to node [above,rotate=-35] {$\sim$} node {} (X0);
  %%          %\draw[-] (KX1) to node [below] {$2$} node {} (KX0);
  %%          %\draw[-] (KX2) to node [left] {$2$} node {} (KX1);
  %%          %\draw[-, bend left] (KX2) to node [below, left] {} node {} (KX0);
  %%          %\draw[-] (dots) to node [left] {$2$} node {} (KX2);
  %%          %\draw[-] (KXr) to node [left] {$2$} node {} (dots);
  %%          %\draw[-, bend left] (KXr) to node [below, left] {} node {} (KX0);
  %%        \end{tikzpicture}
  %%      \end{center}
  %%      \caption{
  %%        A $2$-group Belyi map $\phi$
  %%        as a sequence of degree $2$ covers.
  %%        For $j\in\{1,\dots,i\}$,
  %%        $\phi$ factors through
  %%        a degree $2^j$ Belyi map denoted $\phi_j$.
  %%      }
  %%      \label{fig:2groupfiltration}
  %%    \end{figure}
  %%    Let $\sigma_j$ be the permutation triple corresponding to
  %%    $\phi_j$ in Figure \ref{fig:2groupfiltration}.
  %%    Applying Algorithm \ref{alg:triples} to $\sigma_j$
  %%    we obtain $\sigma_{j+1}$ as a lift of $\sigma_j$
  %%    so that the permutation triple corresponding to $\phi$
  %%    appears in $\mathscr{G}_{2^i}^\text{above}$.
  %%    This shows that every $2$-group Belyi map
  %%    of degree $2^i$ is represented by at least one node in
  %%    $\mathscr{G}_{2^i}$.
  %%    We now claim that every $2$-group Belyi map of degree $2^i$
  %%    is represented by exactly one node in $\mathscr{G}_{2^i}$.
  %%    % then show why only one in each iso class
  %%    Since we are applying Algorithm \ref{alg:triples}
  %%    to every permutation triple in $\mathscr{G}_{2^i}^\text{below}$,
  %%    it is possible that in Step 2(a),
  %%    the set of permutation triples in $\mathscr{G}_{2^i}^\text{above}$
  %%    has simultaneously conjugate triples
  %%    which arise when a degree $2^i$ Belyi map
  %%    is a degree $2$ cover of more than one
  %%    nonisomorphic Belyi map of degree $2^{i-1}$.
  %%    In Step 2(b),
  %%    we combine permutation triples in $\mathscr{G}_{2^i}^\text{above}$
  %%    that are simultaneously conjugate by taking a single
  %%    permutation triple to represent this isomorphism class
  %%    of $2$-group Belyi map.
  %%    Note that in Step 2(b)
  %%    we never remove any edges in the graph
  %%    $\mathscr{G}_{2^i}$.
  %%    It follows from Step 2(b)
  %%    that $\mathscr{G}_{2^i}^\text{above}$ has at most
  %%    one node for each
  %%    $2$-group Belyi map
  %%    isomorphism class of degree $2^i$.
  %%  \end{proof}
  %%  \begin{theorem}\label{thm:isoclasses}
  %%    The following table lists the number of isomorphism classes of
  %%    $2$-group Belyi maps of degree $d$ for $d$ up to $256$.
  %%    \[
  %%      \begin{tabular}{|c||c|c|c|c|c|c|c|c|}
  %%        \hline
  %%        $d$ & $2$ & $4$ & $8$ & $16$ & $32$ & $64$ & $128$ & $256$\\
  %%        \hline
  %%        \# & $$ & $$ & $$ & $$ & $$ & $$ & $$ & $$\\
  %%        \hline
  %%      \end{tabular}
  %%    \]
  %%  \end{theorem}
  %%  \begin{proof}
  %%    Apply Algorithm \ref{alg:alltriples}.
  %%    \mm{maybe go up to 512 or 1024 and include source code link to implementation}
  %%  \end{proof}
  %%  \begin{alg}\label{alg:allpassports}
  %%    We use Algorithm \ref{alg:alltriples}
  %%    to count the number of Passports of $2$-group Belyi maps
  %%    of a given degree.
  %%    \mm{todo}
  %%  \end{alg}
  %%  \begin{theorem}\label{thm:passports}
  %%    The following table lists the number of passports of
  %%    $2$-group Belyi maps of degree $d$ for $d$ up to $256$.
  %%    \[
  %%      \begin{tabular}{|c||c|c|c|c|c|c|c|c|}
  %%        \hline
  %%        $d$ & $2$ & $4$ & $8$ & $16$ & $32$ & $64$ & $128$ & $256$\\
  %%        \hline
  %%        \# passports & $3$ & $7$ & $16$ & $41$ & $96$ & $267$ & $834$ & $2893$\\
  %%        \hline
  %%      \end{tabular}
  %%    \]
  %%  \end{theorem}
  %%  \begin{proof}
  %%    Apply Algorithm \ref{alg:allpassports}.
  %%  \end{proof}
  %%}
  %\section{An algorithm to compute $2$-group Belyi curves and maps}{\label{sec:curvesandmaps}
  %  The algorithm we describe here is iterative.
  %  The degree $1$ case is discussed in Section \ref{sec:degree1}.
  %  We now set up some notation for the iteration.
  %  \begin{notation}\label{not:maps}
  %    First suppose
  %    we are given the following data:
  %    \begin{itemize}
  %      \item
  %        $X\subset\PP^n_K$ defined over a number field $K$
  %        with coordinates $x_0,\dots,x_n$
  %        cut out by the equations $\{h_i=0\}_i$ with $h_i\in K[x_0,\dots,x_n]$
  %      \item
  %        $\phi:X\to\PP^1$ a $2$-group Belyi map of degree $d=2^n$
  %        given by $\phi([x_0:\dots:x_n])=[x_0:x_1]$
  %        with monodromy group $G = \langle\sigma\rangle$
  %        (necessarily a $2$-group)
  %        with $\sigma$ a permutation triple corresponding to $\phi$
  %      \item
  %        For $s\in\{0,1,\infty\}$ and $\tau$ a cycle of $\sigma_s\in\sigma$,
  %        denote the ramification point above $s$ corresponding to
  %        $\tau$ by $Q_{s,\tau}$
  %      \item
  %        $Y\subset\AA^n_K$ the affine patch of $X$ with $x_0\neq 0$
  %        with coordinates $(y_1,\dots,y_n)$ where $y_i = x_i/x_0$
  %        cut out by the equations $\{g_i=0\}_i$ with $g_i\in K[y_1,\dots,y_n]$
  %        so that $\phi:Y\to\AA^1$ is given by $\phi(y_1,\dots,y_n) = y_1$
  %      \item
  %        $\wt{\sigma}$ as in the output of Algorithm \ref{alg:triples}
  %        applied to the input $\sigma$
  %    \end{itemize}
  %    Algorithm \ref{alg:lift} below
  %    describes how to lift the degree $d$ Belyi map $\phi$
  %    to a degree $2d$ Belyi map $\wt{\phi}$ with ramification prescribed by $\wt{\sigma}$
  %    (also see Figure \ref{fig:lift}).
  %  \end{notation}
  %  \begin{figure}[ht]
  %    \label{fig:lift}
  %    \[
  %      \begin{tikzcd}[ampersand replacement=\&]
  %        \widetilde{X}\arrow{dd}[swap]{\widetilde{\phi}}\arrow{dr}{\psi}\\
  %        % \widetilde{X}\arrow{dd}[swap]{\widetilde{\phi}}\arrow{dr}{\text{degree 2}}\\
  %        \&X\arrow{dl}{\phi}\\
  %        \PP^1
  %      \end{tikzcd}
  %    \]
  %    \caption{
  %    Algorithm \ref{alg:lift} describes how to construct $\wt{\phi}$
  %      corresponding to a permutation triple $\wt{\sigma}$
  %      from a given $2$-group Belyi map $\phi$.
  %      % In Algorithm \ref{alg:lift} we construct $\wt{\phi}$
  %      % from a given $2$-group Belyi map $\phi$.
  %    }
  %  \end{figure}
  %  \begin{lemma}\label{lem:1dim}
  %    Let $D$ be a degree $0$ divisor on $X$.
  %    Then $\ddim\LL(D) \leq 1$.
  %  \end{lemma}
  %  \begin{proof}
  %    Suppose $\ddeg D = 0$, and
  %    Let $f,g\in\LL(D)\setminus\{0\}$.
  %    Write $D=D_0-D_\infty$ with $D_0,D_\infty\geq 0$.
  %    Since $f,g\in\LL(D)$, we have
  %    $\ddiv f,\ddiv g\geq D_0-D_\infty$.
  %    In particular,
  %    $f/g\in K^\times$.
  %  \end{proof}
  %  \begin{definition}
  %    \label{def:ramificationful}
  %    Let $\phi\colon X\to\PP^1$ be a $2$-group Belyi map.
  %    Let $\ddiv\phi = D_0-D_\infty$
  %    and $\ddiv(\phi-1) = D_1 - D'_\infty$
  %    with $D_0, D_1, D_\infty, D'_\infty$ effective.
  %    For $s\in\{0,1,\infty\}$ let
  %    \[
  %      R_s \subseteq \supp D_s
  %    \]
  %    and $R\colonequals R_0+R_1+R_\infty$.
  %    Let $K$ be a number field
  %    containing the coordinates of
  %    all ramification points in $\supp R$
  %    and
  %    let
  %    \begin{equation}
  %      \label{eqn:ramificationful}
  %      M = (R+2\ZZ R)\cap\Div^0(X),
  %    \end{equation}
  %    and consider the map
  %    $M\to\Pic^0(X)(K)$.
  %    If this map has nontrivial kernel we say that
  %    $\phi$ is
  %    \defi{fully ramified for the ramification divisor $R$}.
  %  \end{definition}
  %  \begin{alg}\label{alg:lift}
  %    Let the notation be as described above in \ref{not:maps}.
  %    \newline
  %    \textbf{Input}:
  %    \begin{itemize}
  %      \item 
  %        $\phi:X\to\PP^1$ a $2$-group Belyi map
  %      \item
  %        $\wt{\sigma}$ a permutation triple which is a lift
  %        of $\sigma$ a permutation triple corresponding to $\phi$
  %      \item
  %        Suppose $\phi$ is fully ramified for
  %        $R$ in Step \ref{prescribedramification}
  %    \end{itemize}
  %    \textbf{Output}: A model over a number field $K$ for the Belyi map
  %    $\wt{\phi}:\wt{X}\to\PP^1$
  %    with monodromy $\wt{\sigma}$.
  %    \begin{enumerate}
  %      \item
  %        \label{prescribedramification}
  %        Let $R$ be the empty set of points on $X$.
  %        For each $s\in\{0,1,\infty\}$,
  %        If the order of $\sigma_s$ is strictly less than the order of $\wt{\sigma}_s$,
  %        then append the ramification points
  %        $\{Q_{s,\tau}\}_{\tau\in\sigma_s}$
  %        (the ramification points on $X$ above $s$ corresponding to the cycles of $\sigma_s$)
  %        to $R$.
  %      \item
  %        Let $K$ be a number field containing all coordinates of
  %        points in $R$
  %        (a subset of the ramification points of $\phi$).
  %      % \item
  %      %   Let $D = \sum_{P}n_PP$ be a degree $0$ divisor on $X$ 
  %      %   with $n_P$ odd for every $P\in R$
  %      %   and $n_P = 0$ for $P\not\in R$.
  %      %   \mm{class group and base field}
  %      \item
  %        Let $M = (R+2\ZZ R)\cap\Div^0(X)$.
  %      \item
  %        \label{ramificationfulloop}
  %        For each $D\in M$ do the following:
  %        \begin{itemize}
  %          \item
  %            Compute the Riemann-Roch space $\LL(D)$.
  %          \item
  %            If $\dim\LL(D) = 1$, then
  %            compute $f\in K(X)^\times$
  %            corresponding to a generator
  %            of $\LL(D)$
  %            exit the loop
  %            and go to Step \ref{wefoundafunction}.
  %          \item
  %            If $\dim\LL(D) = 0$,
  %            then continue this loop
  %            with another choice of $D$.
  %        \end{itemize}
  %      \item
  %        \label{wefoundafunction}
  %        Write $f=a/b$ with $a,b\in K[y_1,\dots,y_n]$ and construct the ideal
  %        \[
  %          \wt{I}\colonequals\langle g_1,\dots,g_k, by_{n+1}^2-a\rangle
  %        \]
  %        in $K[y_1,\dots,y_n,y_{n+1}]$.
  %      \item
  %        \label{saturate}
  %        Saturate $\wt{I}$ at $\langle b\rangle$ and denote this ideal by $\sat(\wt{I})$.
  %      \item
  %        Let $\wt{X}$ be the curve corresponding to $\sat(\wt{I})$ and
  %        $\wt{\phi}$ the map $(y_1,\dots,y_{n+1})\mapsto y_1$.
  %    \end{enumerate}
  %  \end{alg}
  %  \begin{proof}[Proof of correctness]
  %    By Algorithm \ref{alg:triples},
  %    there exists a $2$-group Belyi map $\wt{\phi}:\wt{X}\to\PP^1$
  %    with ramification according to $\wt{\sigma}$.
  %    Since $\wt{\phi}$ is Galois,
  %    the ramification behavior above each $s\in\{0,1,\infty\}$
  %    is constant
  %    (i.e. for a fixed $s$, all $Q_{s,\tau}$ are either unramified or
  %    ramified to order $2$).
  %    This ensures that the set $R$ constructed in Step $1$
  %    is precisely the set of ramification values of $\psi$
  %    (in Figure \ref{fig:lift}).
  %    Now that we have the ramification points,
  %    we can construct the new Belyi map and curve
  %    by extracting a square root in the function field.
  %    More precisely,
  %    again by Algorithm \ref{alg:triples},
  %    there exists $\wt{X}$ and a number field $K$
  %    with $K(\wt{X}) = K(X,\sqrt{f})$
  %    where $f\in K(X)^\times/K(X)^{\times 2}$
  %    and
  %    \begin{equation}\label{eqn:classgroup}
  %      \ddiv f= \sum_{Q_{s,\tau}\in R} Q_{s,\tau}+2D_\epsilon
  %      \in\frac{\Div^0(X)}{2\Div^0(X)}
  %    \end{equation}
  %    Since $\phi$ is fully ramified for $R$,
  %    there is a $D\in M$ such that
  %    $f\in\LL(D)$
  %    will be obtained in Step \ref{ramificationfulloop}.
  %    In Step \ref{wefoundafunction}
  %    we start with the ideal of $X$
  %    and add a new equation (using an extra variable)
  %    corresponding to extracting the square root of $f$.
  %    This is our candidate ideal for $\wt{X}$,
  %    but this process may introduce extra components.
  %    To eliminate these components, we saturate the ideal
  %    in Step \ref{saturate}.
  %    By construction,
  %    the projection map to the first (affine)
  %    coordinate
  %    is the desired Belyi map
  %    with Belyi curve $\wt{X}$.
  %  \end{proof}
  %  \begin{remark}
  %    \label{rmk:fullyramifiedinpractice}
  %    The condition that $\phi$
  %    is fully ramified is required
  %    to avoid a potentially infinite loop in
  %    Step \ref{ramificationfulloop}.
  %    Testing this condition is only implemented over
  %    a finite field,
  %    so in practice we simply search for candidate divisors
  %    in $M$ without testing if $\phi$ is fully ramified.
  %    This appears to work well in practice,
  %    and has been used to carry out the explicit computations
  %    in Section \ref{sec:computations}.
  %  \end{remark}
  %  \begin{remark}
  %    \label{rmk:basefield}
  %    Another important aspect of this process is the choice of $K$.
  %    In Algorithm \ref{alg:lift},
  %    we try to keep the degree of $K$ as small as possible.
  %    Adjoining all coordinates of ramification points
  %    can lead to high degree extensions
  %    which are not feasible in practice.
  %    We choose to obtain the Belyi curve
  %    over a subfield when possible.
  %  \end{remark}
  %}
  %\section{Running time analysis}{\label{sec:runtime}
  %}
  %\section{Explicit computations}{\label{sec:computations}
  %  \mm{link to database, code, and some tables}
  %}
% } 
% \chapter{Classifying low genus and hyperelliptic $2$-group Belyi maps}{\label{chapter:classify}
%   In this chapter we organize some results on $2$-group Belyi maps
%   with low genus.
%   The conditions that need to be satisfied for a general Belyi map
%   to be a $2$-group Belyi map are quite stringent.
%   This allows us to give a clear picture of the story
%   in these special cases.
%   % \section{Remarks on Galois Belyi maps}{\label{sec:galoisbelyimaps}
%     % We summarize some of the results on Galois Belyi maps
%     % that we use for $2$-group Belyi maps.
%     % A great deal is known about Galois Belyi maps (regular dessins) in general
%     % (see \mm{TODO: sources}).
%     % \begin{lemma}\label{lem:regular}
%     %   Let $\sigma$ be a degree $d$ permutation triple corresponding to
%     %   $\phi\colon X\to\PP^1$ a Galois Belyi map with monodromy group $G$
%     %   and
%     %   $m_s$ be the order of $\sigma_s$ for $s\in\{0,1,\infty\}$.
%     %   Then $\sigma_s$ consists of $d/m_s$ many $m_s$-cycles.
%     %   In particular,
%     %   for a $2$-group Belyi map,
%     %   $m_s$ and $\#G$ are powers of $2$.
%     % \end{lemma}
%     % \begin{proof}
%     % \end{proof}
%     % In light of Lemma \ref{lem:regular},
%     % we get a refined version of Riemann-Hurwitz for Galois Belyi maps.
%     % \begin{prop}\label{prop:riemannhurwitzgalois}
%     %   Let $\sigma$ be a degree $d$ permutation triple corresponding to
%     %   $\phi\colon X\to\PP^1$ a Galois Belyi map with monodromy group $G$.
%     %   Let $a,b,c$ be the orders of $\sigma_0,\sigma_1,\sigma_\infty$
%     %   respectively.
%     %   Then
%     %   \begin{equation}\label{eqn:riemannhurwitzgalois}
%     %     g(X) = 1+\frac{\#G}{2}\left(1-\frac{1}{a}-\frac{1}{b}-\frac{1}{c}\right).
%     %   \end{equation}
%     % \end{prop}
%     % \begin{proof}
%     % \end{proof}
%   % }
%   \section{Genus $0$}{\label{sec:genuszero}
%     Let $\phi:X\to\PP^1$ be a $2$-group Belyi map where $X$ has genus $0$.
%     Proposition \ref{prop:riemannhurwitzgalois} immediately restricts the
%     possibilities for ramification indices.
%     \begin{prop}\label{prop:genuszeroramification}
%       A $2$-group Belyi map of genus $0$ with monodromy group $G$
%       has the following possibilities for ramification indices:
%       \begin{itemize}
%         \item degenerate: $(1,\#G, \#G)$, $(\#G, 1, \#G)$, $(\#G, \#G, 1)$
%         \item
%           dihedral:
%           $\left(\frac{\#G}{2}, 2, 2\right)$,
%           $\left(2, \frac{\#G}{2}, 2\right)$,
%           $\left(2, 2, \frac{\#G}{2}\right)$
%       \end{itemize}
%     \end{prop}
%     \begin{proof}
%       Let $a,b,c$ be the ramification indices of the Belyi map.
%       Then by Lemma \ref{lem:regular},
%       $a,b,c,\#G$ are all positive powers of $2$.
%       Without loss of generality we may assume $a\le b\le c$.
%       The proof is by cases.
%       For $g(X)=0$, Proposition \ref{prop:riemannhurwitzgalois}
%       yields
%       \begin{equation}\label{eqn:riemannhurwitzgenuszero}
%         \frac{\#G}{2}\left(1-\frac{1}{a}-\frac{1}{b}-\frac{1}{c}\right) = -1.
%       \end{equation}
%       \begin{itemize}
%         \item[\underline{$a=1$}:]
%           If $a=1$, then
%           \eqref{eqn:riemannhurwitzgenuszero}
%           becomes
%           $\frac{1}{b}+\frac{1}{c}=\frac{2}{\#G}$.
%           \begin{itemize}
%             \item[\underline{$b=1$}:]
%               If $a=b=1$, then
%               \eqref{eqn:riemannhurwitzgenuszero}
%               implies $a=b=c=\#G=1$.
%             \item[\underline{$b\ge 2$}:]
%               If $a=1$ and $b\geq 2$,
%               then we can let $b=2^m$ and $c=2^n$ with $m\le n$.
%               In this case
%               \eqref{eqn:riemannhurwitzgenuszero}
%               becomes
%               \[
%                 \frac{1}{2^m}+\frac{1}{2^n} = \frac{2}{\#G}
%                 \implies
%                 \#G\left(2^{n-m}+1\right) = 2^{n+1}.
%               \]
%               Since $\#G$ is a power of $2$, we must have
%               $2^{n-m}+1\in\{1,2\}$ which only occurs when $m=n$.
%               Therefore $m=n$ which implies $b=c=\#G$.
%           \end{itemize}
%         \item[\underline{$a=2$}:]
%           If $a=2$, then
%           \eqref{eqn:riemannhurwitzgenuszero} becomes
%           \[
%             \frac{2}{\#G} = \frac{1}{b}+\frac{1}{c}-\frac{1}{2}.
%           \]
%           \begin{itemize}
%             \item[\underline{$b=2$}:]
%               If $a=2$ and $b=2$, then
%               \eqref{eqn:riemannhurwitzgenuszero} implies
%               $c=\frac{\#G}{2}$.
%             \item[\underline{$b\ge 4$}:]
%               If $a=2$ and $b,c\geq 4$, then
%               \eqref{eqn:riemannhurwitzgenuszero} implies
%               \[
%                 \frac{2}{\#G} = \frac{1}{b}+\frac{1}{c}-\frac{1}{2}
%                 \implies
%                 \frac{2}{\#G}\le 0
%               \]
%               which cannot occur.
%           \end{itemize}
%         \item[\underline{$a\geq 4$}:]
%           If $a,b,c\geq 4$, then
%           \eqref{eqn:riemannhurwitzgenuszero} becomes
%           \[
%             \frac{2}{\#G} = \frac{1}{b}+\frac{1}{c} - \left(1-\frac{1}{a}\right).
%           \]
%           But
%           $\left(1-\frac{1}{a}\right)\ge\frac{3}{4}$
%           and
%           $\frac{1}{b}+\frac{1}{c}\le\frac{1}{2}$
%           imply that
%           $\frac{2}{\#G}<0$
%           which cannot occur.
%       \end{itemize}
%       In summary there are $2$ possibilities:
%       \begin{itemize}
%         \item $a=1$ and $b=c=\#G$
%         \item $a=2$, $b=2$, and $c=\frac{\#G}{2}$
%       \end{itemize}
%       By reordering the ramification indices
%       we obtain the possibilities in Proposition \ref{prop:genuszeroramification}.
%     \end{proof}
%     In particular,
%     from
%     Proposition \ref{prop:genuszeroramification}
%     we see that all genus $0$ $2$-group Belyi maps
%     are degenerate or spherical dihedral.
%     The explicit maps in these cases are well understood
%     \mm{TODO: cite}\cite{turkishbelyithesis}.
%     We summarize with Proposition \ref{prop:turkishbelyithesis}.
%     \begin{prop}\label{prop:turkishbelyithesis}
%       Every possible ramification type in
%       Proposition \ref{prop:genuszeroramification}
%       corresponds to exactly one Belyi map up to isomorphism.
%       Moreover, the equations for these maps have simple formulas
%       given below.
%       In the formulas below,
%       we use the notation from Proposition \ref{prop:genuszeroramification}
%       for ramification types and write a Belyi map
%       $\phi:\PP^1\to\PP^1$ with monodromy $G$
%       as a rational function in the coordinate $x$
%       on an affine patch of the domain of $\phi$.
%       \begin{itemize}
%         \item
%           $(1,1,1)$
%           \[
%             \phi(x) = x
%           \]
%         \item
%           $(1,\#G, \#G)$, $\#G\ge 2$
%           \[
%             \phi(x) = 1-x^{\#G}
%           \]
%         \item
%           $(\#G, 1, \#G)$, $\#G\ge 2$
%           \[
%             \phi(x) = x^{\#G}
%           \]
%         \item
%           $(\#G, \#G, 1)$, $\#G\ge 2$
%           \[
%             \phi(x) = \frac{x^{\#G}}{x^{\#G}-1}
%           \]
%         \item
%           $(2, 2, 2)$, $\#G = 2$
%           \[
%             \phi(x) = -\left(\frac{x(x-1)}{x-\frac{1}{2}}\right)^2
%           \]
%         \item
%           $(2, 2, \frac{\#G}{2})$, $\#G\ge 4$
%           \[
%             \phi(x) = -\frac{1}{4}\left(x^{\#G/2}+\frac{1}{x^{\#G/2}}\right)
%             +\frac{1}{2}
%           \]
%         \item
%           $(2, \frac{\#G}{2}, 2)$, $\#G\ge 4$
%           \[
%             \phi(x) = 1-\frac{1}{1-\left(-\frac{1}{4}\left(x^{\#G/2}+\frac{1}{x^{\#G/2}}\right)
%             +\frac{1}{2}\right)}
%           \]
%         \item
%           $(\frac{\#G}{2}, 2, 2)$, $\#G\ge 4$
%           \[
%             \phi(x) = \frac{1}{-\frac{1}{4}\left(x^{\#G/2}+\frac{1}{x^{\#G/2}}\right)
%             +\frac{1}{2}}
%           \]
%       \end{itemize}
%     \end{prop}
%     \begin{proof}
%       We first address the correctness of the equations.
%       For the ramification triples containing $1$,
%       the equations are all lax isomorphic to one of the form
%       \begin{equation}\label{eqn:squaringmap}
%         \phi(x) = x^{\#G}
%       \end{equation}
%       for the ramification triple
%       $(\#G, 1, \#G)$.
%       The rational function $\phi$ in
%       \eqref{eqn:squaringmap}
%       has a root of multiplicity $\#G$ at $0$,
%       a pole of multiplicity $\#G$ at $\infty$,
%       and $\#G$ unique preimages above $1$.
%       The Belyi maps for ramification triples
%       $(1,\#G, \#G)$ and $(\#G, \#G, 1)$
%       are lax isomorphic to $\phi$ in
%       \eqref{eqn:squaringmap}
%       and similarly have the correct ramification of this
%       degenerate Belyi map.
%       \par
%       For the other ramification triples, we focus on
%       the triple $(2,2,\frac{\#G}{2})$.
%       The equation for this map is a modification
%       (pointed out to me by Sam Schiavone)
%       of the dihedral Belyi map
%       \begin{equation}\label{eqn:dihedral}
%         \phi(x) = x^d+\frac{1}{x^d}
%       \end{equation}
%       in \cite[Example 5.1.2]{turkishbelyithesis}.
%       The other dihedral maps are then lax isomorphic to
%       (the modification of) the map in
%       \eqref{eqn:dihedral}.
%       \par
%       To show that there is at most one Belyi map in each of the above cases,
%       we refer to Algorithm \ref{alg:triples}.
%       \mm{todo}
%     \end{proof}
%   }
%   \section{Genus $1$}{\label{sec:genusone}
%     Let $\phi\colon X\to\PP^1$ be a $2$-group Belyi map
%     where $X$ has genus $1$.
%     Let $(a,b,c)$ be the ramification indices of $\phi$
%     with $a\leq b\leq c$.
%     From Proposition \ref{prop:riemannhurwitzgalois},
%     we have that
%     \begin{equation}\label{eqn:genusoneramification}
%       % \frac{\#G}{2}\left(1-\frac{1}{a}-\frac{1}{b}-\frac{1}{c}\right) = 0
%       % \implies
%       % \left(1-\frac{1}{a}-\frac{1}{b}-\frac{1}{c}\right) = 0.
%       \frac{1}{a}+\frac{1}{b}+\frac{1}{c} = 0.
%     \end{equation}
%     Since $a,b,c$ are powers of $2$,
%     the only solution to
%     \eqref{eqn:genusoneramification}
%     is $a=2$ and $b=c=4$.
%     We summarize this discussion in
%     Proposition \ref{prop:genusoneramification}.
%     \begin{prop}\label{prop:genusoneramification}
%       The only possible ramification indices for
%       a $2$-group Belyi map of genus $1$
%       are $(2,4,4)$, $(4,2,4)$, or $(4,4,2)$.
%     \end{prop}
%     As was the case in genus $0$,
%     all ramification triples in
%     Proposition \ref{prop:genusoneramification}
%     have corresponding Belyi maps.
%     However, as we see in
%     Proposition \ref{prop:genusonemaps},
%     these genus $1$ Belyi maps occur in infinite families.
%     \begin{prop}\label{prop:genusonemaps}
%       Let $(a,b,c)$ be a ramification triple in
%       Proposition \ref{prop:genusoneramification}
%       and let $d = 2^m$ for $m\in\ZZ_{\geq 2}$.
%       Then there exists exactly one
%       degree $d$ $2$-group Belyi map up to isomorpism
%       with ramification $(a,b,c)$.
%       Moreover, the equations for these maps have simple formulas
%       which are described below.
%       In these equations let $E$ be the elliptic curve
%       with $j$-invariant $1728$ given by the Weierstrass equation
%       \[
%         E\colon y^2 = x^3+x.
%       \]
%       Every degree $4$ Belyi map below is of the form
%       $\phi\colon E\to\PP^1$ where $\phi$
%       (written as an element of the function field of $E$)
%       is one of the following:
%       \begin{align}\label{eqn:d4g1}
%         \begin{split}
%           \phi_{(2,4,4)} &= \frac{x^2+1}{x^2}\\
%           \phi_{(4,2,4)} &= \phi_{(2,4,4)} - 1=-\frac{1}{x^2}\\
%           \phi_{(4,4,2)} &= \frac{1}{\phi_{(2,4,4)}} = \frac{x^2}{x^2+1}
%         \end{split}
%       \end{align}
%       Every degree $d$ Belyi map
%       for $d\geq 8$ is of the form
%       \[
%         E\stackrel{\psi}{\to}E\stackrel{\phi}{\to}\PP^1
%       \]
%       where $\phi$ is a degree $4$ genus $1$ Belyi map
%       and $\psi$ is degree $d/4$ isogeny of $E$.
%       Moreover,
%       if we let $\alpha\colon E\to E$ be defined by
%       \begin{equation}\label{eqn:alpha}
%         (x,y)\mapsto
%         \left(
%           (1+\sqrt{-1})^{-2}\left(x+\frac{1}{x}\right),
%           (1+\sqrt{-1})^{-3}y\left(1-\frac{1}{x^2}\right)
%         \right)
%       \end{equation}
%       then $\psi$ is the map $\alpha$ composed with itself
%       $d/8$ times.
%     \end{prop}
%     \begin{proof}
%       For a proof that these are the only such $2$-group Belyi maps
%       we used \cite[Lemma 3.5]{triangles}.
%       This can also be seen from Algorithm \ref{alg:alltriples}.
%       The degree $4$ Belyi maps are all lax isomorphic
%       to the degree $4$ genus $1$ Belyi map with
%       ramification indices $(4,4,2)$ in
%       \cite{belyidb}.
%       For degree $d$ with $d\geq 8$
%       let $\phi$ be one of the degree $4$ maps in
%       \eqref{eqn:d4g1}.
%       We then precompose
%       $\phi$
%       with $\alpha\cdots\alpha$ ($d/8$ times)
%       where $\alpha$ is the
%       degree $2$ endomorphism of $E$
%       found in
%       \cite[Proposition 2.3.1]{advancedsilverman}.
%       Since isogenies are unramified
%       in characteristic $0$
%       (see \cite[Chapter III, Theorem 4.10]{silverman})
%       the composition $\phi\alpha^{d/8}$
%       is a degree $d$ Belyi map with the same ramification type as $\phi$.
%     \end{proof}
%   }
%   \section{Hyperelliptic}{\label{sec:hyperelliptic}
%     \begin{definition}\label{def:hyperelliptic}
%       Let $\phi\colon X\to\PP^1$ be a Belyi map
%       of genus $\geq 2$.
%       We say a Belyi map $\phi$ is \defi{hyperelliptic}
%       if $X$ is a hyperelliptic curve.
%       A hyperelliptic curve $X$ over $\CC$
%       is defined by having
%       an element $\iota\in\Aut(X)$
%       such that
%       the quotient map
%       $X\to X/\langle\iota\rangle$
%       is a degree $2$ map to $\PP^1$.
%       This element $\iota$ is known as the
%       \defi{hyperelliptic involution}.
%     \end{definition}
%     Let $\phi\colon X\to\PP^1$ be a hyperelliptic
%     $2$-group Belyi map with monodromy group
%     $H\leq G\colonequals\Aut(X)$,
%     and hyperelliptic involution $\iota\in\Aut(X)$.
%     \begin{lemma}\label{lem:hypinvolutioncentral}
%       % $G\leq\Aut(X)$ and $\iota$ is central in $G$.
%       $\langle\iota\rangle\trianglelefteq\Aut(X)$
%     \end{lemma}
%     \begin{proof}
%     \end{proof}
%     \begin{definition}\label{def:reducedautomorphismgroup}
%       The
%       \defi{reduced automorphism group} of $X$
%       is the quotient group
%       $G_{\rred}\colonequals G/\langle\iota\rangle$.
%     \end{definition}
%     % \begin{lemma}\label{lem:normalsubgroup}
%     %   $H\trianglelefteq\Aut(X)$.
%     % \end{lemma}
%     From Lemma \ref{lem:hypinvolutioncentral}
%     and the Galois condition on $\phi$,
%     we obtain the diagram in Figure \ref{fig:hyperellipticbelyi}.
%     \begin{figure}[ht]
%       \[
%         \begin{tikzcd}
%           &X\arrow[swap, Rightarrow]{dl}{\phi}\arrow[Rightarrow]{dr}{\langle\iota\rangle}
%           \arrow[Rightarrow]{dd}{\psi}\\
%           \PP^1\simeq X/H\arrow{dr}&&\PP^1\simeq X/\langle\iota\rangle\arrow[Rightarrow]{dl}{G/\langle\iota\rangle}\\
%                          &\PP^1\simeq X/G
%         \end{tikzcd}
%       \]
%       \caption{Galois theory for a hyperelliptic Belyi map}
%       \label{fig:hyperellipticbelyi}
%     \end{figure}
%     \begin{prop}\label{prop:liftbelyi}
%       Let $\phi$ and $\psi$ be the maps shown in
%       Figure \ref{fig:hyperellipticbelyi}.
%       If $\phi$ is a Belyi map,
%       then $\psi$ is a Belyi map.
%     \end{prop}
%     \begin{proof}
%       By Theorem \ref{thm:bigbijection},
%       $\phi$ corresponds to
%       a normal inclusion of triangle groups
%       $
%         \Delta_1\trianglelefteq\Delta_H
%       $
%       and the map $X / H \to X / G$
%       corresponds to an inclusion of Fuchsian groups
%       \begin{equation}\label{eqn:fuchsianinclusion}
%         \Delta_H\leq\Gamma.
%       \end{equation}
%       By a result in \cite[Page 36]{singerman},
%       the inclusion of a triangle group $\Delta_H$ in
%       a Fuchsian group $\Gamma$
%       as in \eqref{eqn:fuchsianinclusion}
%       implies that $\Gamma$ is a triangle group
%       which we denote $\Delta_G$.
%       Now we have the
%       (normal by Lemma \ref{lem:hypinvolutioncentral})
%       inclusion
%       $\Delta_1\trianglelefteq\Delta_G$
%       which (again by Theorem \ref{thm:bigbijection})
%       implies that $\psi$ is a Belyi map.
%     \end{proof}
%     Proposition \ref{prop:liftbelyi}
%     reduces the classification of these hyperelliptic $2$-group Belyi maps
%     to the situation on the right side of the diagram in
%     Figure \ref{fig:hyperellipticbelyi}.
%     The possibilities for $G_{\rred}$
%     in this setting are known
%     (see \cite[\S 1.1]{dolgachev}).
%     Moreover, since $G$ is a
%     $2$-group (\mm{$G$ only contains a $2$-group\ldots}),
%     the only possibilities for $G_{\rred}$
%     are cyclic or dihedral of order $\#G / 2$.
%     $G$ is then an extension of $G_{\rred}$
%     by $\iota$
%     (an element of order $2$ generating a normal subgroup of $G$).
%     Such groups are classified in
%     \cite{hyperelliptic} which we summarize in the
%     following theorem.
%     \begin{theorem}
%       \label{thm:hyperelliptic}
%       Let $G$ be the full automorphism group of a
%       $2$-group Belyi curve.
%       Let $\#G_{\rred} = 2^n$.
%       Then $G$ is isomorphic to one of the following groups:
%       \begin{itemize}
%         \item
%           $\ZZ/2^{n+1}\ZZ$
%         \item
%           $\ZZ / 2^n\ZZ\times\ZZ / 2\ZZ$
%         \item
%           $D_{2^{n+1}}$
%         \item
%           $D_{2^n}\times\ZZ / 2\ZZ$
%       \end{itemize}
%       where $D_m$ denotes the dihedral group of order $m$.
%     \end{theorem}
%     \begin{proof}
%       \cite[Theorem 2.1]{hyperelliptic}.
%     \end{proof}
%     \mm{
%       you get a genus zero phi0 : Belyi map
%       PP1 $\to$ PP1 and the degree 2 map on top must be ramified,
%       corresponding to the hyperelliptic involution, can only be ramified
%       along the preimages of ramification points of phi0, and in a
%       group-invariant way, so that should really give you the equations as
%       well.
%     }
%     \newline
%     \mm{
%       maybe write down explicit maps for g=2,3
%     }
%       %   \begin{theorem}\label{thm:hyperelliptic}
%       %     Let $\phi:X\to\PP^1$ be a hyperelliptic $2$-group Belyi map
%       %     with monodromy group $G$.
%       %     Let $\iota$ denote the hyperelliptic involution in $\Aut(X)$.
%       %     Then $\overline{G}\colonequals G/\langle\iota\rangle$
%       %     is a subgroup of $\Aut(\PP^1)\cong\PGL_2(\CC)$.
%       %     (see Figure \ref{fig:hyperelliptic}).
%       %     In particular,
%       %     $\overline{G}$ is either cyclic, dihedral, or exceptional.
%       %     \begin{figure}[ht]
%       %       \[
%       %         \begin{tikzcd}[ampersand replacement=\&]
%       %           X\arrow{dd}[swap]{G}\arrow{dr}{\langle\iota\rangle}\\
%       %           \&\PP^1\arrow{dl}{G/\langle\iota\rangle}\\
%       %           \PP^1
%       %         \end{tikzcd}
%       %       \]
%       %       \caption{
%       %         A hyperelliptic $2$-group Belyi map
%       %         with monodromy group $G$
%       %         factors through a degree $2$ map to $\PP^1$
%       %         and an automorphism of $\PP^1$.
%       %       }
%       %       \label{fig:hyperelliptic}
%       %     \end{figure}
%       %   \end{theorem}
%   }
%     % We start Section \ref{sec:classification} by giving a complete description of the geometry type
%     % (see Definition \ref{def:geometrytype}) of a
%     % $2$-group Belyi map in Proposition \ref{prop:geometrytype}
%     % and finish Section \ref{sec:classification}
%     % with a classification of hyperelliptic
%     % $2$-group Belyi maps in Theorem \ref{thm:hyperelliptic}.
%     % \begin{prop}\label{prop:geometrytype}
%     %   Let $\phi:X\to\PP^1$ be a $2$-group Belyi map
%     %   corresponding to a permutation triple $\sigma$
%     %   with $(a,b,c)$ the triple of orders of permutations in $\sigma$.
%     %   Then the geometry type of $\phi$ is determined by $(a,b,c)$
%     %   and is one of the following:
%     %   \begin{enumerate}
%     %     \item[(degenerate)]
%     %       $\phi$ is degenerate if $1\in(a,b,c)$.
%     %     \item[(spherical)]
%     %       $\phi$ is spherical if $(a,b,c)$ is of the form
%     %       $(2,2,c)$, $(2,b,2)$, or $(a,2,2)$.
%     %     \item[(Euclidean)]
%     %       $\phi$ is Euclidean if $(a,b,c)$ is of the form
%     %       $(2,4,4)$, $(4,2,4)$, or $(4,4,2)$.
%     %     \item[(hyperbolic)]
%     %       $\phi$ is hyperbolic (with genus at least $2$) in all other cases.
%     %   \end{enumerate}
%     % \end{prop}
%     % \begin{proof}
%     % \end{proof}
%     % \subsection{Classifying spherical $2$-group Belyi maps}{\label{subsec:elliptic}}{
%     % }
%     % \subsection{Classifying genus $1$ $2$-group Belyi maps}{\label{subsec:elliptic}}{
%     % }
%     % \subsection{Classifying hyperelliptic $2$-group Belyi maps}{\label{subsec:hyperelliptic}}{
%       % \begin{definition}
%       %   Let $K$ be a number field with algebraic closure $\overline{K}$.
%       %   Let $X$ be an irreducible smooth projective curve over $K$.
%       %   $X$ is \defi{hyperelliptic} if there exists a degree $2$ map
%       %   $\iota:X\to\PP^1$ (defined over $\overline{K}$).
%       %   The degree $2$ map $\iota$ is called the
%       %   \defi{hyperelliptic involution}.
%       % \end{definition}
%     %   \begin{definition}
%     %     A \defi{hyperelliptic Belyi map} is a Belyi map $\phi:X\to\PP^1$
%     %     where $X$ is a hyperelliptic curve.
%     %   \end{definition}
%       %   \begin{lemma}\label{lem:hypinvolutioncentral}
%       %     Let $\phi:X\to\PP^1$ be a hyperelliptic $2$-group Belyi map
%       %     with monodromy group $G$.
%       %     Let $\iota$ denote the hyperelliptic involution in $\Aut(X)$.
%       %     Then
%       %     $G\leq\Aut(X)$ and $\iota$ is central in $G$.
%       %   \end{lemma}
%     %   \begin{proof}
%     %   \end{proof}
%       %   \begin{theorem}\label{thm:hyperelliptic}
%       %     Let $\phi:X\to\PP^1$ be a hyperelliptic $2$-group Belyi map
%       %     with monodromy group $G$.
%       %     Let $\iota$ denote the hyperelliptic involution in $\Aut(X)$.
%       %     Then $\overline{G}\colonequals G/\langle\iota\rangle$
%       %     is a subgroup of $\Aut(\PP^1)\cong\PGL_2(\CC)$.
%       %     (see Figure \ref{fig:hyperelliptic}).
%       %     In particular,
%       %     $\overline{G}$ is either cyclic, dihedral, or exceptional.
%       %     \begin{figure}[ht]
%       %       \[
%       %         \begin{tikzcd}[ampersand replacement=\&]
%       %           X\arrow{dd}[swap]{G}\arrow{dr}{\langle\iota\rangle}\\
%       %           \&\PP^1\arrow{dl}{G/\langle\iota\rangle}\\
%       %           \PP^1
%       %         \end{tikzcd}
%       %       \]
%       %       \caption{
%       %         A hyperelliptic $2$-group Belyi map
%       %         with monodromy group $G$
%       %         factors through a degree $2$ map to $\PP^1$
%       %         and an automorphism of $\PP^1$.
%       %       }
%       %       \label{fig:hyperelliptic}
%       %     \end{figure}
%       %   \end{theorem}
%     %   \begin{proof}
%     %   \end{proof}
%     % }
% }
% \chapter{Gross's conjecture for $p=2$}{\label{chapter:grossconjecture}
%   We begin this chapter with Theorem \ref{thm:beckmann} which provides
%   the arithmetic motivation to study $2$-group Belyi maps.
%   We then detail past results on Gross's conjecture in Section
%   \ref{sec:grossconjecturepast}
%   and finish with some discussion on $2$-group Belyi maps
%   in relation to the $p=2$ case of Gross's conjecture.
%   \section{Beckmann's theorem}{\label{sec:beckmann}
%     In this Section we state Beckmann's theorem
%     for Belyi maps over $\CC$
%     from
%     $1989$ which can be found in \cite{beckmann}.
%     We then adapt Theorem \ref{thm:beckmann}
%     to our particular situation in
%     Corollary \ref{cor:beckmann}.
%     \begin{theorem}\label{thm:beckmann}
%       Let $\phi:X\to\PP^1$ be a Belyi map with
%       monodromy group $G$
%       and suppose $p$ does not divide $\# G$.
%       Then there exists a number field $M$
%       with the following properties:
%       \begin{itemize}
%         \item
%           $p$ is unramified in $M$
%         \item
%           the Belyi map $\phi$ is defined over $M$
%         \item
%           the Belyi curve $X$ is defined over $M$
%         \item
%           $X$ has good reduction at all primes $\mathfrak{p}$
%           of $M$ above $p$
%       \end{itemize}
%     \end{theorem}
%     \begin{proof}
%       \cite{beckmann}
%     \end{proof}
%     \begin{corr}\label{cor:beckmann}
%       Let $\phi:X\to\PP^1$ be a $2$-group Belyi map.
%       Then there exists a smooth projective model for $X$
%       with good reduction away from $p=2$.
%     \end{corr}
%     \begin{proof}
%     \end{proof}
%   }
%   \section{Past results on Gross's conjecture}{\label{sec:grossconjecturepast}
%   }
%   \section{A nonsolvable Galois number field ramified only at $2$}{\label{sec:numberfieldramifiedat2}
%   }
% }

\chapter{Future work}{\label{chapter:future}
  As is typically the case with any work of mathematics,
  there is more to investigate about
  $2$-group Belyi maps.
  \section{Implementations}{\label{sec:futureimplementation}
    One task is to compute more examples
    and optimize the implementations to
    aid in computing permutation triples
    and equations.
    In particular, there are two possible improvements
    to the implementations used here
    that could aid in this process.
    \begin{enumerate}
      \item
        Implement a way to lift Belyi maps
        over $\FF_q$ to characteristic zero.
      \item
        Take advantage of the iterative structure
        of these Belyi maps in the computation
        of Picard groups.
    \end{enumerate}
    Another task is to extend these
    techniques to deal with $p$-group Belyi maps
    for $p\geq 3$
    and non-Galois Belyi maps.
  }
  \section{Applications}{\label{sec:futureapplications}
    Besides pushing the computations and generalizing
    the implementations,
    there is the more interesting question of
    how to apply these computations.
    Chapter \ref{chapter:fieldsofdefinition},
    for example,
    provides evidence that
    $2$-group Belyi maps appear
    to have small refined passports.
    This might suggest that the moduli fields
    of $2$-group Belyi maps
    are
    \emph{close to being abelian}.
    With Gross's conjecture in mind,
    we pose the following question.
    \begin{question}
      \label{ques:solvable}
      Are the moduli fields of
      $2$-group Belyi maps always solvable?
    \end{question}
    More work is needed to conjecture on this question,
    but either answer is interesting.
    Since the moduli field of a $2$-group
    Belyi map is ramified only at $2$,
    a nonsolvable moduli field would be
    an example of a nonsolvable number field
    ramified only at $2$.
    To illustrate the type of computation that
    would be helpful in answering
    Question
    \ref{ques:solvable},
    consider the permutation
    triple in
    \eqref{eqn:bigrefined}
    provided by
    David P. Roberts.
    \begin{align}
      \label{eqn:bigrefined}
      \begin{split}
        \sigma_0^a &= (1, 20, 16, 28, 7, 21, 11, 32, 4, 17, 14, 26, 5, 24, 9, 29)\\
        \sigma_0^b &= (2, 19, 15, 27, 8, 22, 12, 31, 3, 18, 13, 25, 6, 23, 10, 30)\\
        \sigma_1^a &= (1, 11, 3, 10)(2, 12, 4, 9)(5, 14, 8, 15, 6, 13, 7, 16)\\
        \sigma_1^b &= (17, 27, 23, 29, 19, 26, 21, 31, 18, 28, 24, 30, 20, 25, 22, 32)\\
        \sigma_0 &= \sigma_0^a\sigma_0^b\\
        \sigma_1 &= \sigma_1^a\sigma_1^b\\
        \sigma_\infty &= (\sigma_1\sigma_0)^{-1}
      \end{split}
    \end{align}
    The triple defined in
    \eqref{eqn:bigrefined}
    is a non-Galois permutation triple
    corresponding to a
    $2$-group Belyi map of degree
    $2^{20}$
    with refined passport size $16$.
    Modifying our implementations to compute equations
    for these non-Galois triples could potentially
    shed light on an answer to
    Question \ref{ques:solvable}.
    \par
    Another interesting application of these computations would be
    to use the iterative structure of
    $2$-group Belyi maps to explain
    the splitting of passports.
    \par
    Lastly,
    even if all moduli fields
    of $2$-group Belyi maps end up being
    solvable,
    there is still the possibility of
    finding a nonsolvable field by
    computing $2$-torsion fields
    as described in
    Section \ref{sec:motivation}.
    Torsion fields could be computed
    using techniques in
    \cite{edgar, mascot}.
    Torsion fields could also be computed
    using Christian Neurohr's implementation
    of Riemann surfaces in \cite{magma}
    which will be available in a future release.
  }
  % higher degree over FF3
  % non-Galois setting
  % char 0 algorithm
  % p-Group Belyi for p odd
  % compute torsion fields
  %
}


%%%%%%%%%%%%%%%%%%%%%%%%%%%%%%%%%%%%%%%%%%%%%%%%%%
%Backend of thesis (references and the like)
%%%%%%%%%%%%%%%%%%%%%%%%%%%%%%%%%%%%%%%%%%%%%%%%%%
\backmatter

%Add your bibliography filename and uncomment these lines
\nocite{*}
\bibliographystyle{amsplain}
\addcontentsline{toc}{chapter}{References}
\bibliography{references}

%Uncomment if you want an index

%\printindex

\end{document}
